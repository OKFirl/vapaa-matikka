Suoran yhtälö

Opi perusteet

\begin{tehtava}
Ratkaise suoran $y=3x+1$ nollakohta.
\begin{vastaus}
$x=-\frac{1}{3}$
\end{vastaus}
\end{tehtava}

\begin{tehtava}
Ratkaise suorien $y=-5x+3$ ja $y=2x-17$ leikkauspiste.
\begin{vastaus}
$(\frac{20}{7},-\frac{79}{70})$
\end{vastaus}
\end{tehtava}

\begin{tehtava}
Mikä on x-akselin suuntaisen suoran, joka kulkee pisteen $(1,3)$ kautta, yhtälö?
\begin{vastaus}
$y=3$
\end{vastaus}
\end{tehtava}

\begin{tehtava}
Mikä on suoran $y=3,14x-10$ yhtälö?
\begin{enumerate}[a)]
\item vakiotermi
\item kulmakerroin?
\end{enumerate}
\begin{vastaus}
a)$-10$ b) $3,14$
\end{vastaus}
\end{tehtava}

\begin{tehtava}
Suora kulkee pisteiden $(2,1)$ ja $(5,9)$ kautta. Määritä kulmakerroin.
\begin{vastaus}
Kulmakerroin on $\frac{8}{3}$
\end{vastaus}
\end{tehtava}

\begin{tehtava}
Piirrä suora $y=9x-1$.
\begin{vastaus}
puuttuu
\end{vastaus}
\end{tehtava}


Hallitse kokonaisuus

\begin{tehtava}
Ratkaise suorien $y=-x+2$ ja $y=2x-4$ leikkauspiste.
\begin{vastaus}
$(2,0)$
\end{vastaus}
\end{tehtava}

\begin{tehtava}
Määritä
\begin{enumerate}[a)]
\item x-akselin suuntaisen suoran
\item y-akselin suuntaisen suoran kulmakerroin?
\end{enumerate}
\begin{vastaus}
a) $0$ b) ei määritelty (ääretön)
\end{vastaus}
\end{tehtava}

\begin{tehtava}
Piirrä suora $y=-2x+3$.
\begin{vastaus}
puuttuu
\end{vastaus}
\end{tehtava}

\begin{tehtava}
Ratkaise $\frac{y}{2}=\frac{x}{2}+2$ nollakohta.
\begin{vastaus}
$(-4,0)$
\end{vastaus}
\end{tehtava}

\begin{tehtava}
Ratkaise suoran $6=-60x+600y$ nollakohta. 
\begin{vastaus}
$x=-\frac{1}{10}$
\end{vastaus}
\end{tehtava}

\begin{tehtava}
Missä pistessä suora $y=\frac{16x}{25}+\frac{36}{49}$ 
\begin{enumerate}[a)]
\item leikkaa x-akselin
\item leikkaa y-akselin?
\end{enumerate}
\begin{vastaus}
a)$(-\frac{225}{196},0)$ b) $(0,\frac{36}{49})$
\end{vastaus}
\end{tehtava}

\begin{tehtava}
Ratkaise suorien $y=-\frac{2}{5}$ ja $3y=18x+20$ leikkauspiste.
\begin{vastaus}
$(-\frac{53}{45},-\frac{2}{5})$
\end{vastaus}
\end{tehtava}

\begin{tehtava}
Ratkaise suoran $16y-9x=-5y-11x+27$ nollakohta. 
\begin{vastaus}
$(11,0)$
\end{vastaus}
\end{tehtava}

Suoran yhtälön muut muodot

Opi perusteet

\begin{tehtava}
Mikä on suoran yhtälö normaalimuodossa?
\begin{enumerate}[a)]
\item $y=-15x+2$
\item $2y=11x+7$
\item $2y+5x-8=13y-6x-8$
\end{enumerate}
\begin{vastaus}
a)$15x+y-2=0$ b) $11x-2y+7=0$ c) $-11x+11y=0$
\end{vastaus}
\end{tehtava}


\begin{tehtava}
Suoran kulmakerroin on $\frac{1}{2}$ ja suora kulkee pisteen
\begin{enumerate}[a)]
\item $(-12,4)$
\item $(3,9)$. Mikä on suoran yhtälö?
\end{enumerate}
\begin{vastaus}
a)$y=\frac{1}{2}x+10$ b) $y=\frac{1}{2}x+\frac{15}{2}$
\end{vastaus}
\end{tehtava}

\begin{tehtava}
Mikä on pisteiden
\begin{enumerate}[a)]
\item $(1,-2)$ ja $(3,1)$
\item $(0,0)$ ja $(-4,4)$ kautta kulkevan suoran yhtälö?
\end{enumerate}
\begin{vastaus}
a)$y=\frac{3}{2}x-\frac{7}{2}$ b) $y=-x$
\end{vastaus}
\end{tehtava}

Hallitse kokonaisuus

\begin{tehtava}
Tutki ovatko pisteet  
\begin{enumerate}[a)]
\item $(1,-5)$, $(4,-23)$ja $(4,-239)$
\item $(7,3)$, $(-2,10)$ ja $(-3,90)$ samalla suoralla?
\end{enumerate}
\begin{vastaus}
a) kyllä b) ei
\end{vastaus}
\end{tehtava}

\begin{tehtava}
Määritä luku $t$ niin, että pisteet $(-t+3,-4)$, $(6,t-5)$ ja $(5,-4)$ ovat samalla suoralla.
\begin{vastaus}
$t=-2$ tai $t=1$
\end{vastaus}
\end{tehtava}
