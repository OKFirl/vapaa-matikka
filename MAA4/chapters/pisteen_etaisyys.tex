\section{Pisteen etäisyys suorasta}

\laatikko{
KIRJOITA TÄHÄN LUKUUN

\begin{itemize}
\item pisteen etäisyyden suorasta laskeminen yhdenmuotoisilla
kolmioilla
\item pisteen etäisyys suorasta -kaava: $d=\frac{|ax_0+by_0+c|}{\sqrt{a^2+b^2}}$
\item sovelluksia
\item kaavan todistuksen voi laittaa tähän osioon tai liitteeksi,
käytetään yhdenmuotoisia kolmioita
\end{itemize}

KIITOS!}

Pisteen etäisyydellä suorasta tarkoitetaan pisteen ja mielivaltaisen suoran pisteen pienintä mahdollista etäisyyttä.
Jos tunnetaan jokin vaaka- tai pystysuora suora ja jokin koordinaatiston piste, kyseisen pisteen etäisyys annetusta suorasta on helppo määrittää.

\begin{minipage}{0.45\textwidth}
\begin{kuva}
    kuvaaja.pohja(-2, 3, -1, 5, korkeus = 4, nimiX = "$x$", nimiY = "$y$", ruudukko = True)
    with palautin():
        vari("lightgray")
        kuvaaja.piirraParametri("1.7", "t", a = 1, b = 3.3)
    kuvaaja.piirra("1", nimi = "$y=1$", suunta = 0)
    kuvaaja.piste((1.7, 3.3), "$(1,7; 3,3)$", 95)
\end{kuva}

Etäisyys on $3,3-1=2,3$.
\end{minipage}
\begin{minipage}{0.45\textwidth}
\begin{kuva}
    kuvaaja.pohja(-2, 3, -1, 5, korkeus = 4, nimiX = "$x$", nimiY = "$y$", ruudukko = True)
    with palautin():
        vari("lightgray")
        kuvaaja.piirraParametri("t", "3.3", a = -1, b = 1.7)
    kuvaaja.piirraParametri("-1", "t", a = -1, b = 5, nimi = "$x = -1$", suunta = 90)
    kuvaaja.piste((1.7, 3.3), "$(1,7; 3,3)$", 90)
\end{kuva}

Etäisyys on $1,7-(-1)=2,7$.
\end{minipage}

Jos suora on kalteva, etäisyyden määrittäminen ei ole näin suoraviivaisesti.
Seuraavaksi tutustutaan kahteen tapaan tämän pulman ratkaisemiseksi.

\subsection*{Pisteen etäisyys suorasta yhdenmuotoisten kolmioiden avulla}

Tarkastellaan esimerkkinä suoraa $l$, jonka normaalimuotoinen yhtälö on $3x-4y -12=0$.
Selvitetään pisteen $P=(8, 5)$ etäisyys suorasta $l$.

\begin{kuva}
    kuvaaja.pohja(-1, 10, -4, 5, korkeus = 4, nimiX = "$x$", nimiY = "$y$", ruudukko = True)
    with palautin():
        vari("lightgray")
        kuvaaja.piirraParametri("8+0.75*t", "5-t", a = 0, b = 1.28)
    kuvaaja.piirra(".75*x-3", nimi = "$l$", kohta = 2, suunta = -45)    
    kuvaaja.piste((8,5), "$P$", -135)
\end{kuva}

%    kuvaaja.piirra(".75*x-3", nimi = "$3x -4y- 12=0$", kohta = 2, suunta = -45)    


Kuvaan on merkitty suorakulmaiset kolmiot $OAB$ ja $PQR$.
Etäisyys, jonka haluamme selvittää, on kolmion $PQR$ sivun $r$ pituus.
Tehtävä ratkeaa, kun huomataan, että kolmiot $OAB$ ja $PQR$ ovat yhdenmuotoisia.
Tämä johtuu siitä, että molemmat ovat suorakulmaisia ja lisäksi kulmat $OAB$ ja $PRQ$ ovat samankokoiset (ks. kuva alla).

[EI OLE VALMIS, KUVIA JA TEKSTISELITYSTÄ VOISI VIELÄ MIETTIÄ]

%    kuvaaja.pohja(-1, 10, -4, 5)
\begin{kuva}
    kuvaaja.piirra(".75*x-3", a = -1, b = 10)
    kuvaaja.piirraParametri("0", "t", a = -3, b = 0)
    kuvaaja.piirraParametri("t", "0", a = 0, b = 4)
    kuvaaja.piirraParametri("8", "t", a = 3, b = 5)
    kuvaaja.piirraParametri("8+0.75*t", "5-t", a = 0, b = 1.28)
    kuvaaja.piste((8,5), "$P$", 180)
    kuvaaja.piste((4,0), "$A$", -45)
    kuvaaja.piste((0,-3), "$B$", -45)
    kuvaaja.piste((0,0), "$O$", 135)
    kuvaaja.piste((8.96, 3.72), "$Q$", -45)
    kuvaaja.piste((8,3), "$R$", -45)
\end{kuva}

Kolmion $OAB$ sivut selviävät, kun ratkaistaan, missä pisteissä suora leikkaa $x$- ja $y$-akselit.
Asettamalla suoran yhtälössä $x=0$ saadaan
\[
3x-4\cdot 0=12, \quad \text{josta} \quad x=\frac{12}{3}=4.
\]
Pisteen $A$ koordinaatit ovat siis $(4, 0)$. Toisaalta kun $x=0$, saadaan
\[
3\cdot 0-4\cdot y=12, \quad \text{josta} \quad y=-\frac{12}{4}=-3.
\]
Pisteen $B$ koordinaatit ovat siis $(0, 3)$. Nyt tunnetaan sivut $a=4$ ja $b=3$, ja lisäksi Pythagoraan lauseen perusteella
\[
c=\sqrt{a^2+b^2}=\sqrt{4^2+3^2}=\sqrt{25}=5.
\]

Koska kolmiot $OAB$ ja $PQR$ ovat yhdenmuotoiset, saadaan verranto
\[
\frac{r}{q}=\frac{a}{c}.
\]
Tunnemme jo sivut $a$ ja $c$, joten enää on selvitettävä sivu $q$. Tämä on sama kuin pisteiden $P$ ja $R$ välinen etäisyys.

Pisteen $R$ $x$-koordinaatti on sama kuin pisteen $P$, eli 8. Koska $R$ on suoralla $l$, sen $y$-koordinaatti saadaan suoran yhtälöstä:
\begin{align*}
3\cdot 8-4y & =12 \\
-4y & =12-3\cdot 8 \\
-4y & =-12 \\
y & =3. \\
\end{align*}
Nyt siis $R=(8, 3)$. Pisteiden $P$ ja $R$ välinen etäisyys on siis $5-3=2$, ja tämä on sivun $q$ pituus.

Kun verrantoon $\dfrac{r}{q}=\dfrac{a}{c}$ sijoitetaan tunnetut sivujen pituudet, saadaan
\begin{align*}
\frac{r}{2} & =\frac{4}{5} \quad \ppalkki \cdot 2 \\[3pt]
r & =\frac{8}{5}.
\end{align*}
Siispä pisteen $P$ etäisyys suorasta $l$ on $\dfrac{8}{5}$.

\subsection*{Pisteen etäisyys suorasta kaavan avulla}

Edellä esitetystä tavasta laskea pisteen etäisyys suorasta voidaan johtaa myös kaava.
Jos suoran yhtälö on annettu normaalimuodossa $Ax+By+C=0$ ja pisteen koordinaatit ovat $(x_0, y_0)$, etäisyys $d$ saadaan seuraavasta kaavasta.
\laatikko[Pisteen etäisyys suorasta]{
\[
d=\frac{|Ax_0+By_0+C|}{\sqrt{A^2+B^2}}
\]
}
Kaavan johtaminen esitetään liitteessä. (VAI TÄSSÄ?)

\begin{esimerkki} Lasketaan aiemman esimerkin pisteen $P=(8, 5)$ etäisyys suorasta $l$, jonka normaalimuotoinen yhtälö on $3x-4y-12=0$.
\begin{esimratk}
Käytetään kaavaa, jolloin $A=3$, $B=-4$ ja $C=12$, sekä $x_0=8$ ja $y_0=5$. Kaavan mukaan etäisyys on
\[
d=\frac{|Ax_0+By_0+C|}{\sqrt{A^2+B^2}}
=\frac{|3\cdot 8-4\cdot 5-12|}{\sqrt{3^2+(-4)^2}}
=\frac{|24-20-12|}{\sqrt{9+16}}=\frac{|-8|}{\sqrt{25}}
=\frac{8}{5}.
\]
\end{esimratk}
\begin{esimvast}
Etäisyys on $\dfrac{8}{5}$.
\end{esimvast}
\end{esimerkki}

\begin{esimerkki} Etsitään ne pisteet, joiden $x$-koordinaatti on 6 ja joiden etäisyys suorasta $-6x+8y+3=0$.
\begin{esimratk}
Piste, jonka $x$-koordinaatti on 6, on muotoa $(6, y)$. Sijoitetaan tämä etäisyyden kaavaan ja sievennetään.
Nyt $A=-6$, $B=8$, $C=3$, $x_0=6$ ja $y_0=y$.
\[
d=\frac{|-6\cdot 6+8y+3|}{\sqrt{(-6)^2+8^2}}
=\frac{|-36+8y+3|}{\sqrt{36+64}}
=\frac{|8y-33|}{\sqrt{100}}
=\frac{|8y-33|}{10}.
\]
Tehtävänannon mukaan etäisyyden pitäisi olla $d=6$. Tästä saadaan yhtälö
\[
\frac{|8y-33|}{10}=6 \quad \text{eli} \quad |8y-33|=60.
\]
Tämä itseisarvoyhtälö ratkeaa jakautumalla kahteen tapaukseen:
\begin{align*}
8y-33 & =60 & &\text{tai} & 8y-33 & =-60 \\
8y & =99 & & & 8y & =-27 \\
y & =\frac{99}{8} & & & y & =-\frac{27}{8}.
\end{align*}
\end{esimratk}
\begin{esimvast}
Pisteet ovat $\bigl(6, \frac{99}{8}\bigr)$ ja $\bigl(6, -\frac{27}{8}\bigr)$.
\end{esimvast}
\end{esimerkki}



\subsection*{Pisteen etäisyys suorasta, kaavan todistus}

Lasketaan pisteen $P = (x_0, y_0)$ etäisyys suorasta $l$: $ax+by+c=0$.

Olkoon $Q$ suoralla $l$ siten, että $l$ ja $PQ$ ovat kohtisuorassa. Huomataan, että jos piste $R$ on suoralla $l$, Pythagoraan lauseen mukaan
\[
PR^2 = PQ^2+QR^2 \geq PQ^2,
\]
jolloin myös $PR \geq PQ$. Siis $PQ$ on määritelmän nojalla pisteen $P$ etäisyys suorasta $l$.

Suorat $PQ$ ja $l$ ovat kohtisuorassa. Jos $l$ ei ole $x$-, eikä $y$-akselin suuntainen, eli $a,b \neq 0$ sen kulmakerroin on $-\frac{a}{b}$. Koska suoran ja normaalin kulmakerrointen tulo on $-1$, suoran $PQ$ kulmakerroin on
\[
-\frac{1}{\frac{-a}{b}} = \frac{b}{a}.
\]
Se kulkee lisäksi pisteen $(x_0,y_0)$ kautta, joten sen yhtälö on
\[
y-y_0 = \frac{b}{a}(x-x0)
\]
tai normaalimuodossa
\[
bx-ay+ay_0-bx_0 = 0.
\]
Toisaalta, jos $a = 0$, suora $PQ$ on muotoa $x+C$, jollain reaaliluvulla $C$; koska se lisäksi kulkee pisteen $P$ kautta, sen on oltava edellistä muotoa. Sama päättely voidaan toistaa kun $b = 0$, jolloin nähdään, että kaava normaalille pätee myös jos $a = 0$ tai $b = 0$.

Koska $Q$ kuuluu suoralle $l$ ja sen normaalille, sen koordinaattien on toteutettava yhtälöpari
\[
\left\{    
    \begin{array}{rcl}
        ax_q + by_q + c &=&0 \\
        bx_q-ay_q+ay_0-bx_0 &=& 0 \\
    \end{array}
    \right.
\]
Yhtälöpari voidaan ratkaista yhteenlaskumenetelmällä kertomalla ylempi yhtälö $a$:lla ja alempi $b$:llä.
\[
\left\{    
    \begin{array}{rcl}
        a^2x_q + aby_q + ac &=&0 \\
        b^2x_q-aby_q+aby_0-b^2x_0 &=& 0 \\
    \end{array}
    \right.
\]
ja laskemalla yhtälöt puolittain yhteen
\begin{align*}
 a^2x_q + aby_q + ac + b^2x_q-aby_q+aby_0-b^2x_0 &= 0 \\
 (a^2+b^2)x_q &= -ac-aby_0+b^2x0 \\
 x_q = \frac{-ac-aby_0+b^2x0}{(a^2+b^2)}.
\end{align*}
Vastaavasti
\[
y_q = \frac{-bc+a^2y_0-abx_0}{(a^2+b^2)}.
\]
Nyt
\begin{align*}
PQ &= \sqrt{(x_q-x_0)^2+(y_q-y0)^2} \\
&= \sqrt{\Big(\frac{-ac-aby_0+b^2x_0-(a^2+b^2)x_0}{a^2+b^2}\Big)^2+\Big(\frac{-bc+a^2y_0-abx_0-(a^2+b^2)y_0}{a^2+b^2}\Big)^2} \\
& = \frac{\sqrt{(-ac-aby_0-a^2x_0)^2+(-bc-abx_0-b^2 y_0)^2}}{a^2+b^2} \\
& = \frac{\sqrt{(a(-c-by_0-ax_0))^2+(b(-c-ax_0-by_0))^2}}{a^2+b^2} \\
& = \frac{\sqrt{(a^2+b^2)(-c-by_0-ax_0)^2}}{a^2+b^2} \\
& = \frac{\sqrt{(a^2+b^2)}}{a^2+b^2}\sqrt{(-c-by_0-ax_0)^2} \\
& = \frac{|ax_0+by_0+c|}{\sqrt{a^2+b^2}}.
\end{align*}
\begin{tehtavasivu}

\subsubsection*{Opi perusteet}

\subsubsection*{Hallitse kokonaisuus}

\subsubsection*{Sekalaisia tehtäviä}

TÄHÄN TEHTÄVIÄ SIJOITTAMISTA ODOTTAMAAN

\end{tehtavasivu}