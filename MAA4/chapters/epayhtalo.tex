\section{Itseisarvoepäyhtälöt}

\laatikko{
KIRJOITA TÄHÄN LUKUUN

\begin{itemize}
\item miten ratkaistaan epäyhtälöitä tyyliin
\item $|x|<a$,  $|x|>a$
\item $|f(x)|<a$, $|f(x)|<|g(x)|$, $|f(x)|<g(x)$
\end{itemize}

KIITOS!}

\begin{esimerkki}
Ratkaise epäyhtälö $|x|<3$.

\begin{esimratk}
On löydettävä kaikki sellaiset luvut, joiden itseisarvo on pienempi kuin $3$. Esimerkiksi luvut $2$ ja $-1$ käyvät, mutta luvut $4$ ja $-5,5$ eivät käy. Epäyhtälön ratkaisuja ovat täsmälleen ne luvut, joiden etäisyys nollasta on pienempi kuin $3$.

(Tähän kuva.)

Siten ratkaisu on $-3<x<3$.
\end{esimratk}

\begin{esimvast}
$-3<x<3$
\end{esimvast}


\end{esimerkki}

\begin{esimerkki}
Ratkaise epäyhtälö $|x|>5$.

\begin{esimratk}
On löydettävä kaikki sellaiset luvut, joiden itseisarvo on suurempi kuin $5$. Esimerkiksi luvut $2$ ja $-4,5$ eivät ole epäyhtälön ratkaisuja, mutta luvut $5,5$ ja $-6$ ovat. Epäyhtälön ratkaisuja ovat täsmälleen ne luvut, joiden etäisyys nollasta on suurempi kuin $5$.
 
(Tähän kuva.)

Nyt ratkaisu on ilmoitettava kahdessa osassa: $x<-5$ tai $x>5$.
\end{esimratk}

\begin{esimvast}
$x<-5$ tai $x>5$
\end{esimvast}
\end{esimerkki}

\begin{esimerkki}
Ratkaise epäyhtälö $|x+4|<2$.

\begin{esimratk}
Nyt luvun $x+4$ itseisarvo on pienempi kuin kaksi, joten tiedetään, että $-2<x+4<2$. Nyt saadaan ratkaistavaksi epäyhtälöt $-2<x+4$ ja $x+4<2$. Ratkaistaan nämä kaksi yhtälöä erikseen:

\begin{align*}
-2&<x+4 & \ppalkki{+2} \\
0&<x+6 & \ppalkki{-6} \\
-6&<x &
\end{align*}

\begin{align*}
x+4&<2 & \ppalkki{-4} \\
x&<-2 & \ppalkki{-6}
\end{align*}

Kun vastaukset yhdistetään saadaan ratkaisuksi $-6<x<-2$.

(Tähän kuva.)

\end{esimratk}

\begin{esimvast}
$-6<x<-2$.
\end{esimvast}
\end{esimerkki}


\subsection*{Tehtäviä}

\begin{tehtavasivu}

\subsubsection*{Opi perusteet}

\subsubsection*{Hallitse kokonaisuus}

\subsubsection*{Sekalaisia tehtäviä}

TÄHÄN TEHTÄVIÄ SIJOITTAMISTA ODOTTAMAAN

\begin{tehtava}
Ratkaise seuraavat epäyhtälöt.
	\begin{alakohdat}
		\alakohta{$|x|<6$}
		\alakohta{$|x|>10$}
		\alakohta{$|x|<1,6$}
	\end{alakohdat}
	\begin{vastaus}
		\begin{alakohdat}
			\alakohta{$-6<x<6$}
			\alakohta{$x<-10$ tai $x>10$}
			\alakohta{$-1,6<x<1,6$}
		\end{alakohdat}
	\end{vastaus}
\end{tehtava}

\begin{tehtava}
Ratkaise seuraavat epäyhtälöt.
	\begin{alakohdat}
		\alakohta{$|x|\le 6$}
		\alakohta{$|x|>-3$}
	\end{alakohdat}
	\begin{vastaus}
		\begin{alakohdat}
			\alakohta{$-6 \le x \le 6$}
			\alakohta{ei ratkaisua}
		\end{alakohdat}
	\end{vastaus}
\end{tehtava}

\begin{tehtava}
Ratkaise seuraavat epäyhtälöt.
	\begin{alakohdat}
		\alakohta{$|x+6|>3$}
		\alakohta{$|x-5|<2$}
		\alakohta{$|x^2+1| \ge 3$}
	\end{alakohdat}
	\begin{vastaus}
		\begin{alakohdat}
			\alakohta{$x<-9$ tai $x>-3$}
			\alakohta{$3<x<7$}
			\alakohta{$x<\sqrt{-2}$ tai $x>\sqrt{2}$}
		\end{alakohdat}
	\end{vastaus}
\end{tehtava}

\begin{tehtava}
Ratkaise epäyhtälö $|x^2+1| \ge 3$.
	\begin{vastaus}
          $x<\sqrt{-2}$ tai $x>\sqrt{2}$
	\end{vastaus}
\end{tehtava}

\end{tehtavasivu}
