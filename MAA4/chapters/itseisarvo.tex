\section{Itseisarvo}


\laatikko{
KIRJOITA TÄHÄN LUKUUN

\begin{itemize}
\item itseisarvon määritelmä (paloittain määritelty muoto)
\item geometrinen tulkinta lukusuoralla
\item itseisarvoyhtälöistä tyyppiä $|f(x)|=a$, $|f(x)|=|g(x)|$,
$|f(x)|=g(x)$
\end{itemize}

KIITOS!}


\qrlinkki{http://opetus.tv/maa/maa4/itseisarvo/}{Opetus.tv: Itseisarvolauseke ja itseisarvon ominaisuuksia}


% Rumasti toteutettu, mutta toistaiseksi toimii.
% Korjatkaa, jos joku saa siistimmin aikaiseksi.
\begin{lukusuora}{-5}{5}{10}
	\lukusuorapiste{0}{$0$}
	\lukusuorapiste{3}{}
	\lukusuorapiste{-3}{}
	\lukusuoraalanimi{3.1}{$a$}
	\lukusuoraalanimi{-3.3}{$-a$}
	\lukusuoranimi{1.5}{$\overbrace{\hspace{27 mm}}^{|a|}$}
	\lukusuoranimi{-1.5}{$\overbrace{\hspace{27 mm}}^{|-a|}$}
	
	\lukusuorapiste{2}{}
	\lukusuorapiste{-2}{}
	\lukusuoraalanimi{-2.3}{$-2$}
	\lukusuoraalanimi{2.1}{$2$}
	\lukusuoraalanimi{1}{$\underbrace{\hspace{18 mm}}_{|2|=2}$}
	\lukusuoraalanimi{-1}{$\underbrace{\hspace{18 mm}}_{|-2|=2}$}
\end{lukusuora}


Itseisarvo voidaan tulkita lukusuoralla pisteen etäisyytenä nollasta.
Epänegatiivisen luvun itseisarvo on luku itse ja
negatiivisen luvun itseisarvo on luvun vastaluku.

\laatikko{
	\textbf{Itseisarvon määritelmä}\\
	$|x|= \begin{cases}
			x, & \kun x \geq 0 \\
			-x, & \kun x < 0
	\end{cases}$
}

Itseisarvo on funktio
\[
||: \mathbb{R} \rightarrow \mathbb{R}, \;
|x| = \begin{cases}
		x, & \kun x \geq 0 \\
		-x, & \kun x < 0.
	\end{cases}
\]

\begin{esimerkki}
Esitä ilman itseisarvomerkkejä
	\begin{alakohdat}
		\alakohta{$|3-\pi|$}
		\alakohta{$|x-3|$}
	\end{alakohdat}
	\textbf{Ratkaisu}
	\begin{alakohdat}
		\alakohta{Koska $3-\pi\approx-0,14<0$, niin $|3-\pi|=-(3-\pi)=\pi-3$}
		\alakohta{Koska $x-3\geq 0$, kun $x\geq3$, niin\\
			$|x-3|= \begin{cases}
				x-3, & \kun x \geq 3 \\
				-x+3, & \kun x < 3
			\end{cases}$
		}
	\end{alakohdat}
\end{esimerkki}

\vspace{10 mm}

\laatikko{
\textbf{Itseisarvon ominaisuuksia}\\ \\
	\begin{tabular}{l l}
		$|a|\geq0$ & Itseisarvo on aina epänegatiivinen \\
		$|a|=|-a|$ & Luvun ja sen vastaluvun itseisarvot ovat yhtäsuuret \\
		$|a|^2=a^2$ & Luvun itseisarvon neliö on yhtäsuuri kuin luvun neliö \\
		$|ab|=|a||b|$ & Tulon itseisarvo \\
		$\Bigl|\dfrac{a}{b}\Bigr|=\dfrac{|a|}{|b|}$ & Osamäärän itseisarvo

	\end{tabular}
}

\begin{esimerkki}
	Esitä lauseke $|x^2-16|+3x$ ilman itseisarvomerkkejä.\\
	\textbf{Ratkaisu} \\
	Tutkitaan ensin lausekkeen $x^2-16$ merkit.
	
	\begin{align*}
		\text{Nollakohdat: } \qquad	x^2-16 &=0 \\
			x^2 &=16 \\
			x &=\pm\sqrt{16} \\
			x &=\pm 4
	\end{align*}
	
	Lausekkeen $x^2-16$ kuvaaja on ylöspäin aukeneva paraabeli, joka leikkaa x-akselin kohdissa $-4$ ja $4$.
	
% Käyttö:
% \begin{lukusuora}{-1}{1}{10}
% \lukusuoraparaabeli{0.2}{0.8}{1}
% \lukusuoravalisa{0.2}{0.8}{x}{y}
% \lukusuoravaliaa{-0.5}{-0.3}{a}{b}
% \lukusuorapiste{0}{$x^2+y^2$}
% \lukusuoranimi{0.5}{$+$}
% \lukusuoraalanimi{0.05}{$-$}
% \end{lukusuora}
% 
% luo lukusuoran joka piirretään 10 pituisena, käyttää sisäisesti
% koordinaatteja -1..1, ja jolla näytetään väli [0.2, 0.8[ ja
% väli ]-0.5, -0.3[ nimillä [x, y[ ja ]a, b[. Näyttää pisteen
% 0 nimellä $x^2+y^2$ ja piirtää paraabelin jonka nollakohdat ovat
% 0.2 ja 0.8 ja huippu 1. Lisäksi se laittaa +-merkin suoran yläpuolelle
% kohtaan 0.5 ja --merkin suoran alapuolelle kohtaan 0.05.
%
% Mielivaltaisia kuvaajia saa lukusuoralle lukusuorakuvaaja-komennolla.
% Nuolia lukujen välille saa lukusuoranuoli-komennolla.


\begin{lukusuora}{-10}{10}{8}
	\lukusuoraparaabeli{-4}{4}{-1.5}
	\lukusuorapiste{-4}{$-4$}
	\lukusuorapiste{4}{$4$}
	\lukusuoranimi{-6}{$+$}
	\lukusuoranimi{6}{$+$}
	\lukusuoraalanimi{0}{$-$}
\end{lukusuora}

Kun $-4<x<4$, niin $x^2-16<0$. Tällöin
\[
|x^2-16|+3x = -(x^2-16)+3x = -x^2+16+3x=-x^2+3x+16.
\]
Kun $x\leq-4$ tai $x\geq4$, on $x^2-16\geq0$, ja tällöin
\[
|x^2-16|+3x =x^2-16+3x=x^2+3x-16
\]

Siis:\\

			$|x^2-16|+3x=
			\begin{cases}
				-x^2+3x+16, & \kun -4<x<4 \\
				x^2+3x-16, & \kun x\leq -4 \tai x\geq 4
			\end{cases}$

\end{esimerkki}

\subsubsection*{Itseisarvo yhtälö}


Itseisarvoyhtälössä muuttuja on itseisarvomerkkien sisällä. Itseisarvoyhtälö voidaan ratkaista esittämällä yhtälö ilman itseisarvo merkkejä hyödyntäen itseisarvon määritelmää ja ominaisuuksia, ja ratkaisemalla saadut yhtälöt.

\laatikko{
\textbf{Yhtälö: \quad $|f(x)|=a$}\\
Jos $a\geq0$, niin
\[
	f(x)=a  \quad \tai \quad f(x)=-a
\]
Jos $a<0$, niin yhtälöllä $|f(x)|=a$ ei ole ratkaisuja.

}

\begin{esimerkki}

Ratkaise yhtälö
	\begin{alakohdat}
		\alakohta{ $|x|=3$}
		\alakohta{ $|x|=-2$}
	\end{alakohdat}
	\textbf{Ratkaisu} \\
	\begin{alakohdat}
		\alakohta{Ainoastaan lukujen $3$ ja $-3$ itseisarvot ovat 3, joten $x=3$ tai $x=-3$. Siis $x=\pm 3$}
		\alakohta{Itseisarvo ei voi olla negatiivinen, joten yhtälöllä ei ole ratkaisua.}
	\end{alakohdat}

\end{esimerkki}


\begin{esimerkki}
Ratkaise yhtälö $|3x-4|=2$. \\
\textbf{Ratkaisu}\\
Yhtälön ratkaisut saadaan, kun luvun $3x-4$ etäisyys nollasta on 2, eli jos (ja vain jos) luku on $2$ tai $-2$. Saadaan:
	\begin{align*}
		3x-4&=2    &\tai  \qquad    \qquad      3x-4&=-2 \\
		3x&=6      &     \qquad      3x&= 2 \\
		x&=2       &     \qquad          x&=\frac{2}{3}
	\end{align*}
Ratkaisun oikeellisuuden voi tarkistaa sijoittamalla saadut ratkaisut alkupäiseen yhtälöön.
\[|3\cdot2-4|=|6-4|=|2|=2 \quad \text{ ja } \quad \vert3\cdot\dfrac{2}{3}-4\vert=|2-4|=|-2|=2\text{.}\]

\textbf{Vastaus: } $x=2$  tai  $x=\dfrac{2}{3}$.



\end{esimerkki}




\begin{tehtavasivu}

\subsubsection*{Opi perusteet}

\subsubsection*{Hallitse kokonaisuus}

\subsubsection*{Sekalaisia tehtäviä}

TÄHÄN TEHTÄVIÄ SIJOITTAMISTA ODOTTAMAAN

Tehtävissä 1--n esitä lauseke ilman itseisarvomerkkejä.

\begin{tehtava}
	\begin{alakohdat}
		\alakohta{$|\pi-2^2|$}
		\alakohta{$|2x-6|$}
		\alakohta{$x+|6-3x|$}
	\end{alakohdat}
	\begin{vastaus}
		\begin{alakohdat}
			\alakohta{$-(\pi-4)=-\pi+4=4-\pi$}
			\alakohta{$\begin{cases}
					-2x+6, & \jos x<3 \\
					2x-6, & \jos x \geq 3
				\end{cases}$}
			\alakohta{$\begin{cases}
					x+(6-3x), & \jos 6-3x \geq0 \\
					x-(6-3x), & \jos 6-3x <0 
				\end{cases}\\
				=\begin{cases}
					x+6-3x, & \jos -3x \geq-6 \\
					x-6+3x, & \jos -3x <-6 
				\end{cases}\\
				=\begin{cases}
					-2x+6, & \jos x \leq2 \\
					4x-6, & \jos x >2 
				\end{cases}$}
		\end{alakohdat}
	\end{vastaus}
\end{tehtava}

\begin{tehtava}
	\begin{alakohdat}
		\alakohta{$2x-x|2-x|$}
		\alakohta{$|x^2+3|$}
		\alakohta{$|x^2-4|$}
	\end{alakohdat}
	\begin{vastaus}
		\begin{alakohdat}
			\alakohta{$\begin{cases}
					x^2, & \jos x \leq2 \\
					-x^2+4x, & \jos x>2 
				\end{cases}$}
			\alakohta{$x^2+3$}
			\alakohta{$\begin{cases}
					x^2-4, & \jos x \leq -2 \tai x \geq 2 \\
					-x^2+4, & \jos -2<x<2 
				\end{cases}$}
		\end{alakohdat}
	\end{vastaus}
\end{tehtava}

\begin{tehtava}
	\begin{alakohdat}
		\alakohta{$(x-1)|4x^2+4x+1|$}
		\alakohta{$|-3x^2+4x-2|$}
	\end{alakohdat}
	\begin{vastaus}
		\begin{alakohdat}
			\alakohta{$4x^3-3x-1$}
			\alakohta{$3x^2-4x+2$}
		\end{alakohdat}
	\end{vastaus}
\end{tehtava}

\begin{tehtava}
	\begin{alakohdat}
		\alakohta{$3|9-x^2|-2|x^2-9|$, kun $x\leq-3$}
		\alakohta{$\dfrac{|-x^2+4x+5|}{|x^2-5x|}$, kun $x>5$}
	\end{alakohdat}
	\begin{vastaus}
		\begin{alakohdat}
			\alakohta{$x^2-9$ (vinkki: vastalukujen itseisarvot ovat yhtä suuret)}
			%
			\alakohta{$\frac{x+1}{x}$}
		\end{alakohdat}
	\end{vastaus}
\end{tehtava}

\end{tehtavasivu}