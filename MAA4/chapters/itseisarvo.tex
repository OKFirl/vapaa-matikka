\section{Itseisarvo}


\laatikko{
KIRJOITA TÄHÄN LUKUUN

\begin{itemize}
\item itseisarvoyhtälöistä tyyppiä
$|f(x)|=g(x)$ (tarvitaanko omaksi luvukseen? - ei kuulu OPSiin mutta hyvä osata - nyt vain yksi esimerkki)
\end{itemize}

KIITOS!}


\qrlinkki{http://opetus.tv/maa/maa4/itseisarvo/}{Opetus.tv: Itseisarvolauseke ja itseisarvon ominaisuuksia}


% Rumasti toteutettu, mutta toistaiseksi toimii.
% Korjatkaa, jos joku saa siistimmin aikaiseksi.
\begin{lukusuora}{-5}{5}{10}
	\lukusuorapiste{0}{$0$}
	\lukusuorapiste{3}{}
	\lukusuorapiste{-3}{}
	\lukusuoraalanimi{3.1}{$a$}
	\lukusuoraalanimi{-3.3}{$-a$}
	\lukusuoranimi{1.5}{$\overbrace{\hspace{27 mm}}^{|a|}$}
	\lukusuoranimi{-1.5}{$\overbrace{\hspace{27 mm}}^{|-a|}$}
	
	\lukusuorapiste{2}{}
	\lukusuorapiste{-2}{}
	\lukusuoraalanimi{-2.3}{$-2$}
	\lukusuoraalanimi{2.1}{$2$}
	\lukusuoraalanimi{1}{$\underbrace{\hspace{18 mm}}_{|2|=2}$}
	\lukusuoraalanimi{-1}{$\underbrace{\hspace{18 mm}}_{|-2|=2}$}
\end{lukusuora}


Itseisarvo voidaan tulkita lukusuoralla pisteen etäisyytenä nollasta.
Epänegatiivisen luvun itseisarvo on luku itse ja
negatiivisen luvun itseisarvo on luvun vastaluku.

\laatikko{
\textbf{Itseisarvon määritelmä}

\[
|x|=\begin{cases}
	x, & \kun x \geq 0 \\
	-x, & \kun x < 0
\end{cases} 
\]
}

% TÄMÄ oli minusta tällaisenaan hyvin epäselvä (Jokke) eikä kovin tarpeellinen
% Itseisarvo on funktio
% \[
% ||: \R \rightarrow \R, \;
% |x| = \begin{cases}
% 		x, & \kun x \geq 0 \\
% 		-x, & \kun x < 0.
% 	\end{cases}
% \]

\begin{esimerkki}
Esitä ilman itseisarvomerkkejä
\begin{alakohdat}
	\alakohta{$|3-\pi|$}
	\alakohta{$|x-3|$}
\end{alakohdat}

\begin{esimratk}
\begin{alakohdat}
\alakohta{Koska $3-\pi\approx-0,14<0$, niin $|3-\pi|=-(3-\pi)=\pi-3$}
\alakohta{Koska $x-3\geq 0$, kun $x\geq3$, niin
\[
|x-3|= \begin{cases}
x-3, & \kun x \geq 3 \\
-x+3, & \kun x < 3
\end{cases}
\]
}
\end{alakohdat}
\end{esimratk}
\end{esimerkki}

\laatikko{
\textbf{Itseisarvon ominaisuuksia}

\begin{tabular}{ll}
	$|a|\geq0$ & Itseisarvo on aina epänegatiivinen \\
	$|a|=|-a|$ & Luvun ja sen vastaluvun itseisarvot ovat yhtäsuuret \\
	$|a|^2=a^2$ & Luvun itseisarvon neliö on yhtäsuuri kuin luvun neliö \\
	$|ab|=|a||b|$ & Tulon itseisarvo on tekijöiden itseisarvojen tulo \\
	$\Bigl|\dfrac{a}{b}\Bigr|=\dfrac{|a|}{|b|}$ & Osamäärän itseisarvo on tekijöiden itseisarvojen osamäärä
\end{tabular}
}

\begin{esimerkki}
Esitä lauseke $|x^2-16|+3x$ ilman itseisarvomerkkejä.

\begin{esimratk}
Tutkitaan ensin lausekkeen $x^2-16$ merkit. Aloitetaan etsimällä lausekkeen nollakohdat:

\begin{align*}
x^2-16 & =0 \\
x^2 & =16 \\
x & =\pm\sqrt{16} \\
x & =\pm 4
\end{align*}

Lausekkeen $x^2-16$ kuvaaja on ylöspäin aukeneva paraabeli, joka leikkaa $x$-akselin kohdissa $-4$ ja $4$.
	
% Käyttö:
% \begin{lukusuora}{-1}{1}{10}
% \lukusuoraparaabeli{0.2}{0.8}{1}
% \lukusuoravalisa{0.2}{0.8}{x}{y}
% \lukusuoravaliaa{-0.5}{-0.3}{a}{b}
% \lukusuorapiste{0}{$x^2+y^2$}
% \lukusuoranimi{0.5}{$+$}
% \lukusuoraalanimi{0.05}{$-$}
% \end{lukusuora}
% 
% luo lukusuoran joka piirretään 10 pituisena, käyttää sisäisesti
% koordinaatteja -1..1, ja jolla näytetään väli [0.2, 0.8[ ja
% väli ]-0.5, -0.3[ nimillä [x, y[ ja ]a, b[. Näyttää pisteen
% 0 nimellä $x^2+y^2$ ja piirtää paraabelin jonka nollakohdat ovat
% 0.2 ja 0.8 ja huippu 1. Lisäksi se laittaa +-merkin suoran yläpuolelle
% kohtaan 0.5 ja --merkin suoran alapuolelle kohtaan 0.05.
%
% Mielivaltaisia kuvaajia saa lukusuoralle lukusuorakuvaaja-komennolla.
% Nuolia lukujen välille saa lukusuoranuoli-komennolla.


\begin{lukusuora}{-10}{10}{8}
	\lukusuoraparaabeli{-4}{4}{-1.5}
	\lukusuorapiste{-4}{$-4$}
	\lukusuorapiste{4}{$4$}
	\lukusuoranimi{-6}{$+$}
	\lukusuoranimi{6}{$+$}
	\lukusuoraalanimi{0}{$-$}
\end{lukusuora}

Kun $-4<x<4$, niin $x^2-16<0$. Tällöin
\[
|x^2-16|+3x = -(x^2-16)+3x = -x^2+16+3x=-x^2+3x+16.
\]
Kun $x\leq-4$ tai $x\geq4$, on $x^2-16\geq0$, ja tällöin
\[
|x^2-16|+3x =x^2-16+3x=x^2+3x-16.
\]
\end{esimratk}

\begin{esimvast}
\[
|x^2-16|+3x=
\begin{cases}
-x^2+3x+16, & \kun -4<x<4 \\
x^2+3x-16, & \text{kun $x\leq -4$ tai $x\geq 4$}
\end{cases}
\]
\end{esimvast}
\end{esimerkki}

\subsection*{Yhtälö $|f(x)|=a$}

\qrlinkki{http://opetus.tv/maa/maa4/itseisarvoyhtalo/}{Opetus.tv: Itseisarvoyhtälö}

Itseisarvoyhtälössä muuttuja on itseisarvomerkkien sisällä. Itseisarvoyhtälö voidaan ratkaista esittämällä yhtälö ensin ilman itseisarvomerkkejä. Tässä voi hyödyntää itseisarvon määritelmää ja ominaisuuksia. Sen jälkeen ratkaistaan saadut yhtälöt tavalliseen tapaan.

\laatikko{
\textbf{Yhtälö:} $|f(x)|=a$

\textbf{Ratkaisu:} Jos $a\geq0$, niin
\[
f(x)=a  \quad \tai \quad f(x)=-a
\]
Jos $a<0$, niin yhtälöllä ei ole ratkaisuja.
}

\begin{esimerkki}

Ratkaise yhtälö
\begin{alakohdat}
	\alakohta{ $|x|=3$}
	\alakohta{ $|x|=-2$}
\end{alakohdat}

\begin{esimratk}
\begin{alakohdat}
	\alakohta{Ainoastaan lukujen $3$ ja $-3$ itseisarvot ovat 3, joten $x=3$ tai $x=-3$. Siis $x=\pm 3$}
	\alakohta{Itseisarvo ei voi olla negatiivinen, joten yhtälöllä ei ole ratkaisua.}
\end{alakohdat}
\end{esimratk}
\end{esimerkki}

\begin{esimerkki}
Ratkaise yhtälö $|3x-4|=2$.

\begin{esimratk}
Yhtälön toteutuu, kun luvun $3x-4$ etäisyys nollasta on 2, eli ainoastaan jos luku on $2$ tai $-2$. Saadaan:
\begin{align*}
3x-4 & =2 & & \tai & 3x-4 & =-2 \\
3x & =6 && & 3x & = 2 \\
x & =2 && & x & =\frac{2}{3}
\end{align*}
Ratkaisun oikeellisuuden voi tarkistaa sijoittamalla saadut ratkaisut alkupäiseen yhtälöön.
\[
|3\cdot2-4|=|6-4|=|2|=2 \quad \text{ja} \quad
\left|3\cdot\frac{2}{3}-4\right|=|2-4|=|-2|=2.
\]
\end{esimratk}

\begin{esimvast}
$x=2$ tai $x=\dfrac{2}{3}$.
\end{esimvast}
\end{esimerkki}

\begin{esimerkki}
Ratkaise yhtälö $3x=3+|2x-3|$. (YO S82/1)

\begin{esimratk}
Siirretään aluksi termejä niin, että itseisarvolauseke jää yksin omalle puolelleen:
\begin{align*}
3x & =3+|2x-3|  \\
3x-3 & =\underbrace{|2x-3|}_{\geq 0}
\end{align*}
Koska $|2x-3|\geq0$ aina, niin on oltava myös $3x-3\geq0$. Tällöin
\begin{align*}
	3x-3&\geq0 \\
	3x&\geq3 \\
	x&\geq1
\end{align*}
Itseisarvoyhtälöstä saadaan:
\begin{align*}
	2x-3&=3x-3   & &\tai & 2x-3 & =-(3x-3) \\
	-x&=0        && & 2x-3&=-3x+3 \\
	x&=0         && & 5x&=6 \\
	\text{ei kelpaa, } & \text{oltava $x\geq1$}  && & x&=\frac{6}{5}, \text{ kelpaa}
\end{align*}
\end{esimratk}

\begin{esimvast}
$x=\dfrac{6}{5}=1\dfrac{1}{5}$.
\end{esimvast}
\end{esimerkki}

Edellisen esimerkin tilannetta $3x-3=|2x-3|$ voidaan tarkastella myös piirtämällä kuvaajat $y=3x-3$ ja $y=|2x-3|$. Lausekkeet ovat yhtäsuuret, kun niiden saamat $y$:n arvot ovat yhtäsuuret, siis kohdissa, joissa kuvaajat leikkaavat toisensa.

\begin{kuva}
    kuvaaja.pohja(-1.5, 4.5, -2, 3, nimiX = "$x$", nimiY = "$y$")
    piste((1.2, 0.6), "", 180)
    vari("red")
    kuvaaja.piirra("3*x-3", nimi = "$y=3x - 3$", suunta = 0, kohta=0.7)
    vari("blue")
    kuvaaja.piirra("abs(2*x-3)", nimi = "$y=|2x - 3|$", kohta = 2.2)
\end{kuva}

Huomaamme, että lausekkeet saavat saman arvon kohdassa, jossa \mbox{$x=\frac{6}{5}$}. Esimerkkiä ratkaistaessa vastaan tullut epäkelpo ratkaisu $x=0$ olisi vastannut suoran $y=2x-3$ (ilman itseisarvomerkintää) leikkauspistettä suoran $y=3x-3$ kanssa, eli tilannetta, jossa sininen suora olisi kulkenut myös $x$-akselin alapuolella.

\subsection*{Yhtälö $|f(x)|=|g(x)|$}

Merkintä $|a|=|b|$ tarkoittaa, että luvut $a$ ja $b$ ovat lukusuoralla yhtä kaukana nollasta. Tällöin lukujen $a$ ja $b$ täytyy olla joko samat tai toistensa vastaluvut.

\laatikko{
$|f(x)|=|g(x)| \quad \ekvi \quad f(x)=g(x) \quad \tai \quad f(x)=-g(x)$.}

Tällaisen yhtälön voi ratkaista myös neliöön korottamalla ominaisuuden $|a|^2=a^2$ avulla. Jos yhtälön molemmat puolet ovat ei-negatiivisia, niin yhtälön yhtäsuuruus säilyy neliöön korotettaessa.

\laatikko {$|f(x)|=|g(x)| \quad \ekvi \quad f(x)^2=g(x)^2$.}

\begin{esimerkki}
Ratkaise yhtälö $|2x-4|=|3-x|$.

\begin{esimratk}
\textbf{(Tapa 1)} Lukujen $2x-4$ ja $3-x$ tulee olla yhtäsuuret tai toistensa vastaluvut.
\begin{align*}
2x-4 & =3-x & &\tai & 2x-4 & =-(3-x) \\
3x & =7 && & 2x-4 & =-3+x \\
x & =\frac{7}{3} && & x & =1
\end{align*}
\end{esimratk}

\begin{esimvast}
$x=\dfrac{7}{3}$ tai $x=1$.
\end{esimvast}

\begin{esimratk}
\textbf{(Tapa 2)} Yhtälön molemmat puolet voidaan korottaa neliöön.
\begin{align*}
\underbrace{|2x-4|}_{\geq0} &= \underbrace{|3-x|}_{\geq0}    \\
|2x-4|^2 &= |3-x|^2   \\
(2x-4)^2 &= (3-x)^2   \\
4x^2-16x+16 &= 9 -6x + x^2   \\
3x^2-10x+7 &= 0   \\
x &= \frac{10\pm\sqrt{(-10)^2-4\cdot 3\cdot 7}}{2\cdot 3}   \\
x &= \frac{10\pm\sqrt{16}}{6}   \\		
x &= \frac{10\pm4}{6}     \\
x=\frac{14}{6}=\frac{7}{3} \quad  &\tai \quad x=\frac{6}{6}=1 \\
\end{align*}
\end{esimratk}

\begin{esimvast}
$x=\dfrac{7}{3}$ tai $x=1$.
\end{esimvast}
\end{esimerkki} 

\begin{esimerkki}
Ratkaise yhtälö $|2x-4|=|3-x|+2x$.

\begin{esimratk}
Huomataan, etteivät edellisen esimerkin ratkaisutavat toimi, sillä yhtälö ei ole muotoa $|f(x)|=|g(x)|$. Neliöön korotuskaan ei toimi, sillä yhtälön oikea puoli voi saada negatiivisia arvoja. Ratkaistaan yhtälö poistamalla itseisarvomerkit ja ratkaisemalla syntyvät yhtälöt alueittain. Ratkaisun apuna voidaan hyödyntää merkkikaaviota, johon merkitsemme itseisarvojen sisäisten lausekkeiden saamat merkit kullakin välillä.

\begin{lukusuora}{0}{4}{4}
	\lukusuorakuvaaja{2*x-4}
	\lukusuoranimi{1}{$2x-4$}
	\lukusuoraalanimi{1}{$-$}
	\lukusuoranimi{3}{$+$}
	\lukusuorapiste{2}{$2$}
\end{lukusuora}
\begin{lukusuora}{1}{5}{4}
	\lukusuorakuvaaja{3-x}
	\lukusuoraalanimi{2}{$3-x$}
	\lukusuoranimi{2}{$+$}
	\lukusuoraalanimi{4}{$-$}
	\lukusuorapiste{3}{$3$}
\end{lukusuora}

\begin{center}
\begin{merkkikaavio}{2}
\merkkikaavioKohta{$2$}
\merkkikaavioKohta{$3$}

	\merkkikaavioFunktio{$2x-4$}
	\merkkikaavioMerkki{$-$}
	\merkkikaavioMerkki{$+$}
	\merkkikaavioMerkki{$+$}

\merkkikaavioUusirivi
	\merkkikaavioFunktio{$3-x$}
	\merkkikaavioMerkki{$+$}
	\merkkikaavioMerkki{$+$}
	\merkkikaavioMerkki{$-$}

\end{merkkikaavio}
\end{center}

Ratkaistaan yhtälö alueittain.
\begin{itemize}
\item[(1.)] Jos $x<2$, niin saadaan:
\begin{align*}
|\underbrace{2x-4}_{<0}|&=|\underbrace{3-x}_{>0}|+2x \quad \ppalkki x<2  \\
-(2x-4)&=(3-x)+2x \\
-2x+4&=3-x+2x \\
-3x &= -1 \\
x &= \frac{1}{3} \quad \text{(kelpaa)}
\end{align*}

\item[(2.)] Jos $2\leq x\leq 3$, niin saadaan:
\begin{align*}
	|\underbrace{2x-4}_{\geq0}|&=|\underbrace{3-x}_{\geq0}|+2x \quad \ppalkki 2\leq x\leq 3  \\
	(2x-4)&=(3-x)+2x \\
	2x-4&=3-x+2x \\
	x &= 7 \quad \text{(ei kelpaa, sillä ei ole välillä $2\leq x\leq 3$)}
\end{align*}

\item[(3.)] Jos $x>3$, niin saadaan:
\begin{align*}
	|\underbrace{2x-4}_{>0}|&=|\underbrace{3-x}_{<0}|+2x \quad \ppalkki x>3  \\
	(2x-4)&=-(3-x)+2x \\
	2x-4&=-3+x+2x \\
	-x &= 1 \\
	x &= -1  \quad \text{(ei kelpaa, sillä ei ole välillä $x>3$)}
\end{align*}
\end{itemize}
\end{esimratk}
\textbf{Vastaus: }\quad $x=\dfrac{1}{3}$.

\end{esimerkki}

\begin{tehtavasivu}

\subsubsection*{Opi perusteet}

\subsubsection*{Hallitse kokonaisuus}

\subsubsection*{Sekalaisia tehtäviä}

TÄHÄN TEHTÄVIÄ SIJOITTAMISTA ODOTTAMAAN

Tehtävissä 1--n esitä lauseke ilman itseisarvomerkkejä.

\begin{tehtava}
	\begin{alakohdat}
		\alakohta{$|\pi-2^2|$}
		\alakohta{$|2x-6|$}
		\alakohta{$x+|6-3x|$}
	\end{alakohdat}
	\begin{vastaus}
		\begin{alakohdat}
			\alakohta{$-(\pi-4)=-\pi+4=4-\pi$}
			\alakohta{$\begin{cases}
					-2x+6, & \jos x<3 \\
					2x-6, & \jos x \geq 3
				\end{cases}$}
			\alakohta{$\begin{cases}
					x+(6-3x), & \jos 6-3x \geq0 \\
					x-(6-3x), & \jos 6-3x <0 
				\end{cases}\\
				=\begin{cases}
					x+6-3x, & \jos -3x \geq-6 \\
					x-6+3x, & \jos -3x <-6 
				\end{cases}\\
				=\begin{cases}
					-2x+6, & \jos x \leq2 \\
					4x-6, & \jos x >2 
				\end{cases}$}
		\end{alakohdat}
	\end{vastaus}
\end{tehtava}

\begin{tehtava}
	\begin{alakohdat}
		\alakohta{$2x-x|2-x|$}
		\alakohta{$|x^2+3|$}
		\alakohta{$|x^2-4|$}
	\end{alakohdat}
	\begin{vastaus}
		\begin{alakohdat}
			\alakohta{$\begin{cases}
					x^2, & \jos x \leq2 \\
					-x^2+4x, & \jos x>2 
				\end{cases}$}
			\alakohta{$x^2+3$}
			\alakohta{$\begin{cases}
					x^2-4, & \jos x \leq -2 \tai x \geq 2 \\
					-x^2+4, & \jos -2<x<2 
				\end{cases}$}
		\end{alakohdat}
	\end{vastaus}
\end{tehtava}

\begin{tehtava}
	\begin{alakohdat}
		\alakohta{$(x-1)|4x^2+4x+1|$}
		\alakohta{$|-3x^2+4x-2|$}
	\end{alakohdat}
	\begin{vastaus}
		\begin{alakohdat}
			\alakohta{$4x^3-3x-1$}
			\alakohta{$3x^2-4x+2$}
		\end{alakohdat}
	\end{vastaus}
\end{tehtava}

\begin{tehtava}
	\begin{alakohdat}
		\alakohta{$3|9-x^2|-2|x^2-9|$, kun $x\leq-3$}
		\alakohta{$\dfrac{|-x^2+4x+5|}{|x^2-5x|}$, kun $x>5$}
	\end{alakohdat}
	\begin{vastaus}
		\begin{alakohdat}
			\alakohta{$x^2-9$ (vinkki: vastalukujen itseisarvot ovat yhtä suuret)}
			%
			\alakohta{$\frac{x+1}{x}$}
		\end{alakohdat}
	\end{vastaus}
\end{tehtava}

\begin{tehtava}
Ratkaise yhtälö.
	\begin{alakohdat}
		\alakohta{$|x-2|=2$}
		\alakohta{$|3x|=4$}
		\alakohta{$|5x+7|=0$}
		\alakohta{$|-4x+2|+1=0$}
	\end{alakohdat}
	\begin{vastaus}
		\begin{alakohdat}
			\alakohta{$x=0$ tai $x=4$}
			\alakohta{$x=\frac{4}{3}$ tai $x=-\frac{4}{3}$}
			\alakohta{$x=-\frac{7}{5}$}
			\alakohta{ei ratkaisuja}
		\end{alakohdat}
	\end{vastaus}
\end{tehtava}

\begin{tehtava}
Ratkaise yhtälö.
	\begin{alakohdat}
		\alakohta{$|x-2|=|3x|$}
		\alakohta{$|3x|=|5x+7|$}
		\alakohta{$|5x+6|=|5x+4|$}
		\alakohta{$|-2x+2|=|x^2+2x+6|$}
	\end{alakohdat}
	\begin{vastaus}
		\begin{alakohdat}
			\alakohta{$x=-1$ tai $x=\frac{1}{2}$}
			\alakohta{$x=-\frac{7}{2}$ tai $x=-\frac{7}{8}$}
			\alakohta{$x=1$}
			\alakohta{$x=-2$}
		\end{alakohdat}
	\end{vastaus}
\end{tehtava}

\end{tehtavasivu}
