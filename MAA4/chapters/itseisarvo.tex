\section{Itseisarvo}


\laatikko{
KIRJOITA TÄHÄN LUKUUN

\begin{itemize}
\item itseisarvon määritelmä (paloittain määritelty muoto)
\item geometrinen tulkinta lukusuoralla
\item itseisarvoyhtälöistä tyyppiä $|f(x)|=a$, $|f(x)|=|g(x)|$,
$|f(x)|=g(x)$
\end{itemize}

KIITOS!}


\qrlinkki{http://opetus.tv/maa/maa4/itseisarvo/}{Opetus.tv: Itseisarvolauseke ja itseisarvon ominaisuuksia}


% Rumasti toteutettu, mutta toistaiseksi toimii.
% Korjatkaa, jos joku saa siistimmin aikaiseksi.
\begin{lukusuora}{-5}{5}{10}
	\lukusuorapiste{0}{$0$}
	\lukusuorapiste{3}{}
	\lukusuorapiste{-3}{}
	\lukusuoraalanimi{3.1}{$a$}
	\lukusuoraalanimi{-3.3}{$-a$}
	\lukusuoranimi{1.5}{$\overbrace{\hspace{27 mm}}^{|a|}$}
	\lukusuoranimi{-1.5}{$\overbrace{\hspace{27 mm}}^{|-a|}$}
	
	\lukusuorapiste{2}{}
	\lukusuorapiste{-2}{}
	\lukusuoraalanimi{-2.3}{$-2$}
	\lukusuoraalanimi{2.1}{$2$}
	\lukusuoraalanimi{1}{$\underbrace{\hspace{18 mm}}_{|2|=2}$}
	\lukusuoraalanimi{-1}{$\underbrace{\hspace{18 mm}}_{|-2|=2}$}
\end{lukusuora}



Itseisarvo kuvaa lukusuoralla pisteen etäisyyttä nollasta. Positiivisen luvun itseisarvo on luku itse ja negatiivisen luvun itseisarvo on luvun vastaluku. Nollan itseisarvo on nolla.


\laatikko{\textbf{Itseisarvon määritelmä}\\
$|x|= \left\{ \begin{array}{rcl}
		x & , kun & x \geq 0 \\
		-x & , kun & x < 0
		\end{array}\right.$}

\begin{esimerkki}
Esitä ilman itseisarvomerkkejä
	\begin{alakohdat}
		\alakohta{$|3-\pi|$}
		\alakohta{$|x-3|$}
	\end{alakohdat}
	\textbf{Ratkaisu}
	\begin{alakohdat}
		\alakohta{Koska $3-\pi\approx-0,14<0$, niin $|3-\pi|=-(3-\pi)=\pi-3$}
		\alakohta{Koska $x-3\geq 0$, kun $x\geq3$, niin\\
			$|x-3|= \left \{ \begin{array}{rcl}
			x-3 & , kun & x\geq3 \\
			-x+3 & , kun & x<3
			\end{array}\right.$
		}
	\end{alakohdat}
\end{esimerkki}

\vspace{10 mm}

\laatikko{
\textbf{Itseisarvon ominaisuuksia}\\ \\
	\begin{tabular}{l l}
		$|a|\geq0$ & Itseisarvo on aina ei-negatiivinen \\
		$|a|=|-a|$ & Luvun ja sen vastaluvun itseisarvot ovat yhtäsuuret \\
		$|a|^2=a^2$ & Luvun itseisarvon neliö on yhtäsuuri kuin luvun neliö \\
		$|ab|=|a||b|$ & Tulon itseisarvo \\
		$|\frac{a}{b}|=\frac{|a|}{|b|}$ & Osamaarän itseisarvo

	\end{tabular}
}

\begin{esimerkki}
	Esitä lauseke $|x^2-16|+3x$ ilman itseisarvomerkkejä.\\
	\textbf{Ratkaisu} \\
	Tutkitaan ensin lausekkeen $x^2-16$ merkit.
	
	\begin{align*}
		\text{Nollakohdat: } \qquad	x^2-16 &=0 \\
			x^2 &=16 \\
			x &=\pm\sqrt{16} \\
			x &=\pm 4
	\end{align*}
	
	Lausekkeen $x^2-16$ kuvaaja on ylöspäin aukeneva paraabeli, joka leikkaa x-akselin kohdissa $-4$ ja $4$.
	
% Käyttö:
% \begin{lukusuora}{-1}{1}{10}
% \lukusuoraparaabeli{0.2}{0.8}{1}
% \lukusuoravalisa{0.2}{0.8}{x}{y}
% \lukusuoravaliaa{-0.5}{-0.3}{a}{b}
% \lukusuorapiste{0}{$x^2+y^2$}
% \lukusuoranimi{0.5}{$+$}
% \lukusuoraalanimi{0.05}{$-$}
% \end{lukusuora}
% 
% luo lukusuoran joka piirretään 10 pituisena, käyttää sisäisesti
% koordinaatteja -1..1, ja jolla näytetään väli [0.2, 0.8[ ja
% väli ]-0.5, -0.3[ nimillä [x, y[ ja ]a, b[. Näyttää pisteen
% 0 nimellä $x^2+y^2$ ja piirtää paraabelin jonka nollakohdat ovat
% 0.2 ja 0.8 ja huippu 1. Lisäksi se laittaa +-merkin suoran yläpuolelle
% kohtaan 0.5 ja --merkin suoran alapuolelle kohtaan 0.05.
%
% Mielivaltaisia kuvaajia saa lukusuoralle lukusuorakuvaaja-komennolla.
% Nuolia lukujen välille saa lukusuoranuoli-komennolla.


\begin{lukusuora}{-10}{10}{8}
	\lukusuoraparaabeli{-4}{4}{-1.5}
	\lukusuorapiste{-4}{$-4$}
	\lukusuorapiste{4}{$4$}
	\lukusuoranimi{-6}{$+$}
	\lukusuoranimi{6}{$+$}
	\lukusuoraalanimi{0}{$-$}
\end{lukusuora}

Kun $-4<x<4$, niin $x^2-16<0$. Tällöin \\
 $|x^2-16|+3x = -(x^2-16)+3x = -x^2+16+3x=-x^2+3x+16$. \\
 \\
 Kun $x\leq-4$ tai $x\geq4$, on $x^2-16\geq0$ ja tällöin\\
 $|x^2-16|+3x =x^2-16+3x=x^2+3x-16$ \\
 \\
 Siis:
			$|x^2-16|+3x= \left \{ \begin{array}{rcl}
			-x^2+3x+16 & , kun & -4<x<4 \\
			x^2+3x-16 & , kun & x\leq-4 \ \ tai \ \ x\geq4
			\end{array}\right.$

\end{esimerkki}

\section{Tehtäviä}
Tehtävissä 1 - n esitä lauseke ilman itseisarvomerkkejä.

\begin{tehtava}
	\begin{alakohdat}
		\alakohta{$|\pi-2^2|$}
		\alakohta{$|2x-6|$}
		\alakohta{$x+|6-3x|$}
	\end{alakohdat}
	\begin{vastaus}
		\begin{alakohdat}
			\alakohta{$-(\pi-4)=-\pi+4=4-\pi$}
			\alakohta{$\left \{ \begin{array}{rcl}
					-2x+6, & jos & x<3 \\
					2x-6, & jos & x \geq 3
				\end{array}\right.$}
			\alakohta{$\left \{ \begin{array}{rcl}
					x+(6-3x), & jos & 6-3x \geq0 \\
					x-(6-3x), & jos & 6-3x <0 
				\end{array}\right.
				=\left \{ \begin{array}{rcl}
					x+6-3x, & jos & -3x \geq-6 \\
					x-6+3x, & jos & -3x <-6 
				\end{array}\right.\\
				=\left\{\begin{array}{rcl}
					-2x+6, & jos & x \leq2 \\
					4x-6, & jos & x >2 
				\end{array}\right.$}
		\end{alakohdat}
	\end{vastaus}
\end{tehtava}

\begin{tehtava}
	\begin{alakohdat}
		\alakohta{$2x-x|2-x|$}
		\alakohta{$|x^2+3|$}
		\alakohta{$|x^2-4|$}
	\end{alakohdat}
	\begin{vastaus}
		\begin{alakohdat}
			\alakohta{$\left \{ \begin{array}{l c l}
					\ \ \ x^2, & jos & x \leq2 \\
					-x^2+4x, & jos & x>2 
				\end{array}\right.$}
			\alakohta{$x^2+3$}
			\alakohta{$\left \{ \begin{array}{rcl}
					x^2-4, & jos & x \leq-2 \ \ tai\ \ x\geq2 \\
					-x^2+4, & jos & -2<x<2 
				\end{array}\right.$}
		\end{alakohdat}
	\end{vastaus}
\end{tehtava}

\begin{tehtava}
	\begin{alakohdat}
		\alakohta{$(x-1)|4x^2+4x+1|$}
		\alakohta{$|-3x^2+4x-2|$}
	\end{alakohdat}
	\begin{vastaus}
		\begin{alakohdat}
			\alakohta{$4x^3-3x-1$}
			\alakohta{$3x^2-4x+2$}
		\end{alakohdat}
	\end{vastaus}
\end{tehtava}

\begin{tehtava}
	\begin{alakohdat}
		\alakohta{$3|9-x^2|-2|x^2-9|,\ \ kun\ \ x\leq-3$}
		\alakohta{$\frac{|-x^2+4x+5|}{|x^2-5x|},\ \ kun\ \ x>5$}
	\end{alakohdat}
	\begin{vastaus}
		\begin{alakohdat}
			\alakohta{$x^2-9$ (vinkki: vastalukujen itseisarvot ovat yhtä suuret)}
			%
			\alakohta{$\frac{x+1}{x}$}
		\end{alakohdat}
	\end{vastaus}
\end{tehtava}