\section{Suoran yhtälö}

\laatikko{
KIRJOITA TÄHÄN LUKUUN

\begin{itemize}
\item suoran yhtälö muodossa $y=kx+b$
\item kulmakertoimen ja vakiotermin merkitys
\item pysty- ja vaakasuoran suoran yhtälö
\item suorien leikkauspisteen ratkaiseminen
\item suoran nollakohdan ratkaiseminen
\end{itemize}

KIITOS!}

Alla on kolme kuvaa, joihin on seuraavia yhtälöitä vastaavat kuvaajat:
\begin{align*}
y & =x-1 \\
y & =2x \\
\text{ja} \quad y & =-x+2
\end{align*}
Kuten huomataan, kaikki kolme kuvaajaa ovat suoria.

[TÄHÄN KUVAAJAT]

\laatikko{
Yhtälön
\[
y=kx+b
\]
määräämä kuvaaja on suora. Lukua $k$ nimitetään suoran \termi{kulmakerroin}{kulmakertoimeksi} ja lukua $b$ \termi{vakiotermi}{vakiotermiksi}.
}

\begin{esimerkki} Piirretään koordinaatistoon yhtälön $y=2x-3$ kuvaaja. Koska kyseessä on suoran yhtälö, riittää löytää kaksi pistettä, joiden kautta suora kulkee.
Valitaan esimerkiksi pisteet, joiden $x$-koordinaatit ovat $0$ ja $2$. Ensimmäisen $y$-koordinaatti on
\[
y=2\cdot 0-3=-3
\]
ja toisen
\[
y=2\cdot 3-3=6-3=3.
\]
Suoran pisteet ovat siis $(0,-3)$ ja $(2,3)$. Piirretään nämä koordinaatistoon ja vedetään niiden kautta suora.

[TÄHÄN KUVA SUORASTA]
\end{esimerkki}

\subsubsection*{Kulmakertoimen tulkinta}

Suoran kulmakerroin kertoo, miten jyrkästi suora nousee tai laskee. Tarkastellaan alla olevaa suoraa, jonka yhtälö on $y=2x$.

[TÄHÄN KUVA]

Valitaan suoralta kaksi pistettä,
$A=(1,2)$ ja $B=(2,4)$. Siirryttäessä pisteestä $A$ pisteeseen $B$ $x$-koordinaatin arvo kasvaa yhdellä ja $y$-koordinaatin arvo kahdella. Saadaan suhde
\[
\frac{\text{$y$-koordinaatin muutos}}{\text{$x$-koordinaatin muutos}}=\frac{2}{1}=2.
\]
Jos nyt valitaan suoralta jotkin toiset pisteet, esimerkiksi $D=(-1,-2)$ ja $E=(5,10)$, voidaan laskea samalla tavalla
\[
\frac{\text{$y$-koordinaatin muutos}}{\text{$x$-koordinaatin muutos}}=\frac{10-(-2)}{5-(-1)}=\frac{12}{6}=2.
\]
Huomataan, että yllä laskettu suhde on aina sama pisteistä riippumatta. Tämä johtuu siitä, että kuvan kolmiot $ABC$ ja $DEF$ ovat yhdenmuotoisia.
Suhde on lisäksi sama kuin suoran yhtälössä esiintyvä kulmakerroin.

\begin{esimerkki} Määritä alla olevien suorien kulmakertoimet.

[TÄHÄN KAKSI KUVAA SUORISTA: NOUSEVA JA LASKEVA]
\begin{esimratk} Valitaan suoralta kaksi mielivaltaista pistettä ja lasketaan $y$-koordinaatin muutoksen suhde $x$-koordinaatin muutokseen.
Ensimmäiseltä suoralta valitaan vaikkapa pisteet ??? ja ???. Nyt kulmakertoimeksi tulee
\[
\frac{\text{$y$-koordinaatin muutos}}{\text{$x$-koordinaatin muutos}}=\frac{???}{???}=\frac{???}{???}=???.
\]
Toiselta suoralta valitaan pisteet ??? ja ???. Nyt täytyy olla tarkkana etumerkkien kanssa:
\[
\frac{\text{$y$-koordinaatin muutos}}{\text{$x$-koordinaatin muutos}}=\frac{???}{???}=\frac{???}{???}=???.
\]
jotain
\end{esimratk}
\begin{esimvast}
Ensimmäisen suoran kulmakerroin on $???$ ja toisen $???$.
\end{esimvast}
\end{esimerkki}

Kulmakerroin kertoo suoran suunnasta: mitä suurempi kulmakerroin, sitä jyrkemmin suora nousee koordinaatistossa oikealle päin.
Jos kulmakerroin on negatiivinen, suora on laskeva. Vaakasuoran suoran kulmakerroin on 0.

\subsubsection*{Vakiotermin tulkinta}

Yllä nähtiin, että suora $y=2x$ kulkee origon kautta. Seuraavassa kuvassa on suoran $y=2x+1$ kuvaaja. Se saadaan nostamalla suoraa $y=2x$ yhden yksikön verran ylöspäin.

[TÄHÄN KUVA SUORASTA y=2x+1]

Tarkastellaan suoralla $y=kx+b$ olevaa pistettä, jonka $x$-koordinaatti on 0.
Tämä piste sijaitsee $y$-akselilla. Toisin sanoen se on suoran ja $y$-akselin leikkauspisteessä.
Sen $y$-koordinaatti saadaan laskemalla
\[
y=k\cdot 0+b=b.
\]
Pisteen $y$-koordinaatti on siis $b$, eli suoran yhtälön vakiotermi. Vakiotermi siis ilmaisee, missä kohtaa suora leikkaa $y$-akselin.
Alla on esimerkkejä erilaisista vakiotermeistä.

[TÄHÄN KUVIA ERILAISISTA VAKIOTERMEISTÄ]

\subsubsection*{Suoran yhtälön määrittäminen}

Suoran yhtälö voidaan määrittää kuvasta laskemalla suoran kulmakerroin sekä vakiotermi.

\begin{esimerkki} Mikä on alla olevan kuvan suoran yhtälö?
[TÄHÄN KUVA JOSTAKIN SUORASTA]
\begin{esimratk}
Aloitetaan määrittämällä kulmakerroin. Valitaan suoralta pisteet $???$ ja $???$. Kulmakertoimeksi tulee
\[
k=\frac{???}{???}=\frac{???}{???}=???.
\]
\end{esimratk}
Vakiotermi saadaan kohdasta, jossa suora leikkaa $y$-akselin. Tuossa kohdassa $y$-koordinaatti on ???. Vakiotermi on siis $b=???$.
\begin{esimvast}
Suoran yhtälö on $y=???$.
\end{esimvast}
\end{esimerkki}

Kun kulmakerroin on 0, suora on vaakasuora. Sen yhtälö on siis muotoa
\laatikko{
\[
y=b.
\]
}
Toisaalta pystysuoralla suoralla ei ole kulmakerrointa lainkaan. Sen yhtälöä ei voi ilmaista muodossa $y=\dots$, vaan sillä on yhtälö
\laatikko{
\[
x=a.
\]
}
Tässä $a$ on sen pisteen $x$-koordinaatti, jossa suora leikkaa $x$-akselin.
