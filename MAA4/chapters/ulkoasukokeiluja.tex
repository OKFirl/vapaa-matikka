 
\section{Ulkoasukokeiluja}

%\definecolor{vapaa_matikka_vari_3}{cmyk}{0.2,0.27,0.0,0.0}
\definecolor{ParisGreen}{RGB}{80,200,120}
%\definecolor{PersianGreen}{RGB}{0,166,147}
%\definecolor{darkKhaki}{RGB}{189,183,107}
%\definecolor{cornflowerBlue}{RGB}{154,206,235}
%\definecolor{thulianPink}{RGB}{222,111,161}
%\definecolor{olivine}{RGB}{154,185,115}
%\definecolor{battleshipGray}{RGB}{132,132,130}

\pgfdeclarehorizontalshading{laatikkotaustaC}{100bp}{color(0bp)=(ParisGreen); color(50bp)=(ParisGreen); color(100bp)=(ParisGreen!50)}
%\pgfdeclarehorizontalshading{laatikkotausta}{100bp}{color(0bp)=(white); color(50bp)=(vapaa_matikka_vari_3); color(100bp)=(vapaa_matikka_vari_3!50)}
\pgfdeclarehorizontalshading{laatikkotaustaE}{195mm}{color(0mm)=(ParisGreen); color(97mm)=(ParisGreen); color(195mm)=(vapaa_matikka_vari_3)}
\pgfdeclarehorizontalshading{laatikkotaustaF}{100bp}{color(0bp)=(white); color(25bp)=(white); color(75bp)=(vapaa_matikka_vari_3); color(100bp)=(vapaa_matikka_vari_3)}
\colorlet{mycolor}{green}
\pgfdeclarehorizontalshading[mycolor]{myshadingB}{1cm}{rgb(0cm)=(1,0,0); color(1cm)=(white); color(2cm)=(vapaa_matikka_vari_3)}

\mdfdefinestyle{laatikkotyyliQ}{
  usetwoside=false,
  leftmargin=-46mm,
  rightmargin=-22mm,
  innerleftmargin=0mm,
  innerrightmargin=0mm,
  innertopmargin=0mm,
  innerbottommargin=0mm,
}

\mdfdefinestyle{laatikkotyyliB}{
%  userdefinedwidth=166mm % FIXME: controversial
  usetwoside=true,
  innermargin=15mm,
  outermargin=0mm,
  innerleftmargin=36mm,
  innerrightmargin=12mm,
  middlelinewidth=0pt,
  apptotikzsetting={\tikzset{mdfbackground/.append style = {shading = laatikkotausta}}},
}

\mdfdefinestyle{laatikkotyyliC}{
  userdefinedwidth=127mm,
  leftmargin=0pt,
  innermargin=0pt,
  innerleftmargin=0,
  rightmargin=0pt,
  innerrightmargin=0pt,
  middlelinewidth=0pt,
  apptotikzsetting={\tikzset{mdfbackground/.append style = {shading = laatikkotaustaC}}},
  innertopmargin=0mm,
  innerbottommargin=0mm,
  needspace=3\baselineskip,
}

\mdfdefinestyle{laatikkotyyliW}{
  usetwoside=false,
  leftmargin=-46mm,
  rightmargin=-22mm,
  innerleftmargin=46mm,
  innerrightmargin=22mm,
  innertopmargin=8pt,
  innerbottommargin=8pt,
  hidealllines=true,
  apptotikzsetting={\tikzset{mdfbackground/.append style = {shading = laatikkotaustaF}}},
  needspace=3\baselineskip,
}

\newcommand{\laatikkoB}[1]{%
%\begin{\widepar}
\ifthispageodd
{
  \begin{mdframed}[style=laatikkotyyliB]
    #1
  \end{mdframed}
}
{
  \begin{mdframed}[style=laatikkotyyliC]
    #1
  \end{mdframed}
}
%\end{widepar}
}


\newcommand{\laatikkoE}[1]{%
%\begin{\widepar}
  \begin{mdframed}[style=laatikkotyyliW]
	#1
  \end{mdframed}
%\end{widepar}
}

\laatikko{
	\begin{tabular}{l l}
		$|a|\geq0$ & Itseisarvo on aina ei-negatiivinen \\
		$|a|=|-a|$ & Luvun ja sen vastaluvun itseisarvot ovat yhtäsuuret \\
		$|a|^2=a^2$ & Luvun itseisarvon neliö on yhtäsuuri kuin luvun neliö \\
		$|\frac{a}{b}|=\frac{|a|}{|b|}$ & Osamaarän itseisarvo on itseisarvojen osamäärä

	\end{tabular}
}

\newpage

\section{Ulkoasukokeiluja 2}

\laatikko[Itseisarvon ominaisuuksia]{
	\begin{tabular}{l l}
		$|a|\geq0$ & Itseisarvo on aina ei-negatiivinen \\
		$|a|=|-a|$ & Luvun ja sen vastaluvun itseisarvot ovat yhtäsuuret \\
		$|a|^2=a^2$ & Luvun itseisarvon neliö on yhtäsuuri kuin luvun neliö \\
		$|\frac{a}{b}|=\frac{|a|}{|b|}$ & Osamaarän itseisarvo on itseisarvojen osamäärä

	\end{tabular}
}

\begin{tehtavasivu}
\begin{tehtava}
    Ratkaise
    \begin{enumerate}[a)]
        \item $x^2 - 2x - 3 = 0$
        \item $-x^2 - 6x - 5 = 0$
        \item $x + 2x^2 - 6= 0$
        \item $1 + x + 3x^2= 0$.
    \end{enumerate}
    \begin{vastaus}
        \begin{enumerate}[a)]
            \item $x = 3 \; \tai x = -1$
            \item $x = -5 \; \tai x = -1$
            \item $x = -1 + \sqrt{2} \; \tai x = -1 - \sqrt{2}$
            \item Ei ratkaisuja.
        \end{enumerate}
    \end{vastaus}
\end{tehtava}

\begin{tehtava}
    Ratkaise
    \begin{enumerate}[a)]
        \item $9x^2 - 12x + 4 = 0$
        \item $x^2 + 2x = -4$
        \item $4x^2 = 12x - 8$
        \item $3x^2 - 13x + 50 = -2x^2 + 17x + 5$.
    \end{enumerate}
    \begin{vastaus}
        \begin{enumerate}[a)]
            \item $x = \dfrac{2}{3}$
            \item $x = -2$
            \item $x = 1$ tai $x = 2$
            \item $x = 3$
        \end{enumerate}
    \end{vastaus}
\end{tehtava}

\begin{tehtava}
    Tasaisesti kiihtyvässä liikkeessä on voimassa kaavat $v = v_0 + at$ ja $s = v_0t + \dfrac{1}{2}at^2$, missä $v$ on loppunopeus, $v_0$ alkunopeus, $a$ kiihtyvyys, $t$ aika ja $s$ siirtymä.
		\begin{enumerate}[a)]
            \item Auton nopeus on $72$~km/h. Auto pysäytetään jarruttamalla tasaisesti. Se pysähtyy $10$ sekunnissa. Laske jarrutusmatka.
            \item Kivi heitetään suoraan alas $50$ metriä syvään rotkoon nopeudella $3,0$~m/s. Kuinka monen sekunnin kuluttua se kohtaa rotkon pohjan?
        \end{enumerate}
    \begin{vastaus}
        \begin{enumerate}[a)]
            \item Jarrutusmatka on $100$ metriä.
            \item Noin $2,9$ sekunnin kuluttua.
        \end{enumerate}
    \end{vastaus}
\end{tehtava}

\begin{tehtava}
    Kahden luvun summa on $8$ ja tulo $15$. Määritä luvut.
    \begin{vastaus}
		Luvut ovat $3$ ja $5$.
    \end{vastaus}
\end{tehtava}

\begin{tehtava}
    Suorakulmaisen muotoisen alueen piiri on $34$~m ja pinta-ala $60$~m$^2$. Selvitä alueen mitat.
    \begin{vastaus}
		Alueen toinen sivu on $5$ m ja toinen $12$ m.
    \end{vastaus}
\end{tehtava}

\begin{tehtava}
    Kultaisessa leikkauksessa jana on jaettu siten, että pidemmän osan suhde lyhyempään on sama kuin koko janan suhde pidempään osaan. Tämä suhde ei riipu koko janan pituudesta ja sitä merkitään yleensä kreikkalaisella aakkosella fii eli $\varphi$. Kultaista leikkausta on taiteessa kautta aikojen pidetty ''jumalallisena suhteena''.
		\begin{enumerate}[a)]
            \item Laske kultaiseen leikkauksen suhteen $\varphi$ tarkka arvo ja likiarvo.
            \item Napa jakaa ihmisvartalon pituussuunnassa kultaisen leikkauksen suhteessa. Millä korkeudella napa on $170$~cm pitkällä ihmisellä?
        \end{enumerate}
    \begin{vastaus}
        \begin{enumerate}[a)]
            \item $ \frac{1}{\varphi} = \dfrac{\sqrt{5}-1}{2} \approx 0,618$
            \item Noin $105,1$ cm korkeudella.
        \end{enumerate}
    \end{vastaus}
\end{tehtava}

\begin{tehtava}
(K93/T5) Ratkaise yhtälö
        $\frac{2x+a^2-3a}{x-1}=a$ vakion $a$ kaikilla reaaliarvoilla.
\begin{vastaus}
        \begin{enumerate}
         \item{$x=a$, jos $a \neq 2$ ja $a \neq 1$}
         \item{$x\neq 1$, jos $a=2$}
         \item{ei ratkaisua, jos $a=1$}
        \end{enumerate}
    \end{vastaus}
\end{tehtava}

\begin{tehtava}
(K94/T2a) Polynomin $P(x)=ax^3-31x^2+1$ eräs nollakohta on $x=1$. Määritä $a$ ja ratkaise tämän jälkeen $P(x)=0$.
\begin{vastaus}
      $a=30$ yhtälön ratkaisut ovat $1$, $\frac{1}{5}$ ja $-\frac{1}{6}$.
    \end{vastaus}
\end{tehtava}

\begin{tehtava}
(K96/T2b) Yhtälössä $x^2-2ax+2a-1=0$ korvataan luku $a$ luvulla $a+1$. Miten muuttuvat yhtälön juuret?
\begin{vastaus}
     Toinen kasvaa kahdella ja toinen ei muutu.
    \end{vastaus}
\end{tehtava}

\begin{tehtava} % HANKALA!
	Ratkaise yhtälö $(x^2-2)^6=(x^2+4x+4)^3$.
	\begin{vastaus}
		$x=-1$, $x=0$ tai $x=\frac{1 \pm \sqrt{17}}{2}$
	\end{vastaus}
\end{tehtava}

\begin{tehtava}
    Ratkaise
    \begin{enumerate}[a)]
        \item $9x^2 - 12x + 4 = 0$
        \item $x^2 + 2x = -4$
        \item $4x^2 = 12x - 8$
        \item $3x^2 - 13x + 50 = -2x^2 + 17x + 5$.
    \end{enumerate}
    \begin{vastaus}
        \begin{enumerate}[a)]
            \item $x = \dfrac{2}{3}$
            \item $x = -2$
            \item $x = 1$ tai $x = 2$
            \item $x = 3$
        \end{enumerate}
    \end{vastaus}
\end{tehtava}

\begin{tehtava}
    Tasaisesti kiihtyvässä liikkeessä on voimassa kaavat $v = v_0 + at$ ja $s = v_0t + \dfrac{1}{2}at^2$, missä $v$ on loppunopeus, $v_0$ alkunopeus, $a$ kiihtyvyys, $t$ aika ja $s$ siirtymä.
		\begin{enumerate}[a)]
            \item Auton nopeus on $72$~km/h. Auto pysäytetään jarruttamalla tasaisesti. Se pysähtyy $10$ sekunnissa. Laske jarrutusmatka.
            \item Kivi heitetään suoraan alas $50$ metriä syvään rotkoon nopeudella $3,0$~m/s. Kuinka monen sekunnin kuluttua se kohtaa rotkon pohjan?
        \end{enumerate}
    \begin{vastaus}
        \begin{enumerate}[a)]
            \item Jarrutusmatka on $100$ metriä.
            \item Noin $2,9$ sekunnin kuluttua.
        \end{enumerate}
    \end{vastaus}
\end{tehtava}

\begin{tehtava}
    Kahden luvun summa on $8$ ja tulo $15$. Määritä luvut.
    \begin{vastaus}
		Luvut ovat $3$ ja $5$.
    \end{vastaus}
\end{tehtava}

\begin{tehtava}
    Suorakulmaisen muotoisen alueen piiri on $34$~m ja pinta-ala $60$~m$^2$. Selvitä alueen mitat.
    \begin{vastaus}
		Alueen toinen sivu on $5$ m ja toinen $12$ m.
    \end{vastaus}
\end{tehtava}

\begin{tehtava}
    Kultaisessa leikkauksessa jana on jaettu siten, että pidemmän osan suhde lyhyempään on sama kuin koko janan suhde pidempään osaan. Tämä suhde ei riipu koko janan pituudesta ja sitä merkitään yleensä kreikkalaisella aakkosella fii eli $\varphi$. Kultaista leikkausta on taiteessa kautta aikojen pidetty ''jumalallisena suhteena''.
		\begin{enumerate}[a)]
            \item Laske kultaiseen leikkauksen suhteen $\varphi$ tarkka arvo ja likiarvo.
            \item Napa jakaa ihmisvartalon pituussuunnassa kultaisen leikkauksen suhteessa. Millä korkeudella napa on $170$~cm pitkällä ihmisellä?
        \end{enumerate}
    \begin{vastaus}
        \begin{enumerate}[a)]
            \item $ \varphi = \dfrac{\sqrt{5}-1}{2} \approx 0,618$
            \item Noin $105,1$ cm korkeudella.
        \end{enumerate}
    \end{vastaus}
\end{tehtava}

\begin{tehtava}
(K93/T5) Ratkaise yhtälö
        $\frac{2x+a^2-3a}{x-1}=a$ vakion $a$ kaikilla reaaliarvoilla.
\begin{vastaus}
        \begin{enumerate}
         \item{$x=a$, jos $a \neq 2$ ja $a \neq 1$}
         \item{$x\neq 1$, jos $a=2$}
         \item{ei ratkaisua, jos $a=1$}
        \end{enumerate}
    \end{vastaus}
\end{tehtava}

\begin{tehtava}
(K94/T2a) Polynomin $P(x)=ax^3-31x^2+1$ eräs nollakohta on $x=1$. Määritä $a$ ja ratkaise tämän jälkeen $P(x)=0$.
\begin{vastaus}
      $a=30$ yhtälön ratkaisut ovat $1$, $\frac{1}{5}$ ja $-\frac{1}{6}$.
    \end{vastaus}
\end{tehtava}

\end{tehtavasivu}