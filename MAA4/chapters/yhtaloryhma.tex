\section{Lineaariset yhtälöryhmät} % FIXME: siirrä?

\laatikko{
KIRJOITA TÄHÄN LUKUUN

\begin{itemize}
\item mikä yhtälöryhmä on
\item miten ratkaistaan yhtälöpari (sijoitus, yhteenlaskumenetelmä)
\item että ratkaisuja voi olla yksi, nolla tai äärettömän monta
\item miten useamman tuntemattoman yhtälöryhmä ratkaistaan
\end{itemize}

KIITOS!}

\subsection*{Yhtälöryhmä}

\termi{yhtälöryhmä}{Yhtälöryhmällä} tarkoitetaan useaa yhtälöä, joiden täytyy
päteä samanaikaisesti. Englannin kielessä yhtälöryhmä tunnetaankin nimellä
\textit{simultaneous equations} eli samanaikaiset yhtälöt.
\footnote{Termi \textit{system of equations} on vähintäänkin yhtä yleinen.}
Yhtälöryhmän (mahdollisten) ratkaisujen tulee siis toteuttaa kaikki yhtälöt.

Tässä luvussa käsitellään yhtälöryhmiä, joissa kaikki yhtälöt ovat ensimmäistä
astetta. Tällaisia yhtälöryhmiä kutsutaan \termi{lineaarinen yhtälöryhmä}{lineaarisiksi yhtälöryhmiksi}.
Lisäksi myöhemmin kirjassa käsitellään erikoistapauksia toisen asteen yhtälöryhmistä
ratkaistaessa kahden ympyrän tai ympyrän ja suoran leikkauspisteitä.

\subsection*{Lineaarinen yhtälöpari}

Yksinkertaisin mielenkiintoinen yhtälöryhmä on
\termi{lineaarinen yhtälöpari}{lineaarinen yhtälöpari}.
Lineaarisessa yhtälöparissa on kaksi ensimmäisen asteen yhtälöä.
Lineaarinen yhtälöpari voidaan esittää monella tapaa. Tässä
kirjassa käytämme pääasiallisesti ns. normaalimuotoa.

\laatikko{
\textbf{Normaalimuotoinen lineaarinen yhtälöpari}

\begin{align*}
a_1x+b_1y+c_1 &= 0 \\
a_2x+b_2x+c_2 &= 0
\end{align*}

$a_1, a_2, b_1, b_2, c_1, c_2 \in \mathbb{R}$

Lineaarisen yhtälöparin ratkaisu on pari $(x, y) \in \mathbb{R}^2$, joka toteuttaa molemmat yhtälöt.
}

Aiemmin on todettu, että normaalimuotoinen ensimmäisen asteen yhtälö voidaan tulkita suorana
$(x, y)$-tasossa. Näin ollen lineaariselle yhtälöparille on geometrinen tulkinta: sen
ratkaisut ovat ne tason pisteet, joissa yhtälöitä vastaavat
suorat leikkaavat. Näitä voi olla
$0$ (suorat ovat yhdensuuntaiset, mutta eivät sama suora),
$1$ (suorat eivät ole yhdensuuntaiset) tai
äärettömän monta (suorat ovat sama suora).

\laatikko{
Lineaarisella yhtälöparilla voi olla joko $0$, $1$ tai äärettömän monta ratkaisua.
}

\subsection*{Lineaarisen yhtälöparin ratkaisumenetelmiä}

Lineaarisia yhtälöpareja ratkotaan pääasiallisesti kahdella menetelmällä.
Nämä menetelmät ovat \termi{sijoitusmenetelmä}{sijoitusmenetelmä} ja
\termi{yhteenlaskumenetelmä}{yhteenlaskumenetelmä}.

% tähän menetelmien käytöstä

\subsection*{Lineaarinen yhtälöryhmä}

Lineaarisen yhtälöparin ajatus yleistyy mihin tahansa määrään ensimmäisen asteen yhtälöitä.
Tällöin puhutaan lineaarisista yhtälöryhmistä.\footnote{Myös lineaarinen yhtälöpari on lineaarinen yhtälöryhmä.}
Myös geometrinen tulkinta yleistyy, mutta se on vaikeampi hahmottaa kuin lineaarisen yhtälöparin tapauksessa:
kolmen yhtälön lineaarinen yhtälöryhmä vastaa kolmen $(x, y)$-tason leikkauspisteitä
kolmiulotteisessa $(x, y, z)$-avaruudessa jne.

Tässä tarkastellaan lähinnä kolmen yhtälön lineaarisia yhtälöryhmiä. Neljän yhtälön
lineaarisista yhtälöryhmistä esitetään joitakin helppoja esimerkkejä. Yleisesti ottaen yhtälöryhmiä
ei ratkaista tällä kurssilla esitetyin keinoin, vaan likimääräisesti tietokoneella käyttäen numeerista 
matriisilaskentaa, joka ei kuulu lukion oppimäärään.

\subsection*{Lineaarisen yhtälöryhmän ratkaisumenetelmiä}

% tähän menetelmistä

\begin{tehtavasivu}

\begin{tehtava}
    Ratkaise yhtälöpari.
    \begin{align*}
        x+y+1 &= 0 \\
        x+2y+1 &=0
    \end{align*}
    \begin{vastaus}
        $x = -1, \, y = 0$
    \end{vastaus}
\end{tehtava}

\begin{tehtava}
    Ratkaise yhtälöpari.
    \begin{align*}
        2x+5y+1 &= 0 \\
        2x+2y+7 &=0
    \end{align*}
    \begin{vastaus}
        $x = -\frac{11}{2}, \, y = 2$
    \end{vastaus}
\end{tehtava}

\begin{tehtava}
    Ratkaise yhtälöpari. $t \in \mathbb{R}$ on vapaa parametri, joka saa sisältyä vastaukseen.
    \begin{align*}
        x+2y-t-1 &= 0 \\
        x+y+t^2 &=0
    \end{align*}
    \begin{vastaus}
        $x = -2t^2-t-1, \, y = t^2+t+1$
    \end{vastaus}
\end{tehtava}

\begin{tehtava}
    Ratkaise yhtälöryhmä.
    \begin{align*}
        x+2y+1 &= 0 \\
        x+2z+3 &=0 \\
        y+2z+5 &=0
    \end{align*}
    \begin{vastaus}
        $x = 1, \, y = -1, \, z = -2$
    \end{vastaus}
\end{tehtava}

\begin{tehtava}
    Ratkaise yhtälöryhmä.
    \begin{align*}
        x+y+z+8 &= 0 \\
        x+y+6 &=0 \\
        x+z-70 &=0
    \end{align*}
    \begin{vastaus}
        $x = 72, \, y = -78, \, z = -2$
    \end{vastaus}
\end{tehtava}

\begin{tehtava}
    Ratkaise yhtälöryhmä.
    \begin{align*}
        x+y+2z+12 &= 0 \\
        2x+2y+3z+1 &=0 \\
        3x-4 &=0
    \end{align*}
    \begin{vastaus}
        $x = \frac{4}{3}, \, y = \frac{98}{3}, \, z = -23$
    \end{vastaus}
\end{tehtava}

\begin{tehtava}
    Ratkaise yhtälöryhmä.
    \begin{align*}
        2x+3y+5z+8 &= 0 \\
        3x+5y+8z &=0 \\
        x+y-1 &=0
    \end{align*}
    \begin{vastaus}
        $x = -\frac{63}{2}, \, y = \frac{65}{2}, \, z = -\frac{17}{2}$
    \end{vastaus}
\end{tehtava}

\end{tehtavasivu}
