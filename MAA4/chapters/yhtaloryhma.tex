\section{Yhtälöryhmät} % FIXME: siirrä

\laatikko{
KIRJOITA TÄHÄN LUKUUN

\begin{itemize}
\item mikä yhtälöryhmä on
\item miten ratkaistaan yhtälöpari (sijoitus, yhteenlaskumenetelmä)
\item että ratkaisuja voi olla yksi, nolla tai äärettömän monta
\item miten useamman tuntemattoman yhtälöryhmä ratkaistaan
\end{itemize}

KIITOS!}

\subsection*{Lineaarinen yhtälöpari}

Yksinkertaisin tapaus yhtälöryhmistä on lineaarinen yhtälöpari, jossa on
kaksi ensimmäisen asteen yhtälöä. Käyttäen normaalimuotoa:

\laatikko{
Lineaarisen yhtälöparin yleinen muoto:

\begin{align*}
a_1x+b_1y+c_1 &= 0 \\
a_2x+b_2x+c_2 &= 0
\end{align*}

$a_1, a_2, b_1, b_2, c_1, c_2 \in \mathbb{R}$

Lineaarisen yhtälöparin ratkaisu on pari $(x, y) \in \mathbb{R}^2$, joka toteuttaa molemmat yhtälöt.
}

Aiemmin on todettu, että normaalimuotoinen ensimmäisen asteen yhtälö voidaan tulkita suorana
$(x, y)$-tasossa. Näin ollen lineaariselle yhtälöparille on geometrinen tulkinta: sen
ratkaisut ovat ne tason pisteet, joissa yhtälöitä vastaavat
suorat leikkaavat. Näitä voi olla $0$ (suorat ovat yhdensuuntaiset,
mutta eivät sama suora), $1$ (suorat eivät ole yhdensuuntaiset) tai äärettömän monta (suorat ovat sama suora).

\laatikko{
Lineaarisella yhtälöparilla voi olla joko $0$, $1$ tai äärettömän monta ratkaisua.
}

\subsection*{Lineaarinen yhtälöryhmä}

Lineaarisen yhtälöparin ajatus yleistyy mihin tahansa määrään ensimmäisen asteen yhtälöitä.
Tällöin puhutaan lineaarisista yhtälöryhmistä. Myös geometrinen tulkinta yleistyy, mutta se on vaikeampi
visualisoida paperille: lineaarinen kolmen yhtälön ryhmä vastaa kolmen $(x, y)$-tason leikkauspisteitä
kolmiulotteisessa $(x, y, z)$-avaruudessa jne.
