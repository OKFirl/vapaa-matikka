\section{Suorien keskinäinen asema}

\laatikko{
KIRJOITA TÄHÄN LUKUUN

\begin{itemize}
\item sama kulmakerroin --> yhdensuuntaiset tai sama suora
\item eri kulmakerroin --> yksi leikkauspiste, kytkentä yhtälöpareihin
\item suoran ja normaalin kulmakertoimet, $k_1k_2=-1$.
\end{itemize}

KIITOS!}

Piirretään koordinaatistoon kaksi eri suoraa ja tutkitaan niiden leikkauspisteitä.
On kaksi eri vaihtoehtoa:
\begin{enumerate}
 \item Suorat eivät leikkaa.
 \item Suorat leikkaavat täsmälleen yhdessä pisteessä.
\end{enumerate}

Ensimmäisessä tapauksessa suorilla on sama kulmakerroin. Sanotaan, että ne ovat yhdensuuntaiset.

\begin{kuva}
    kuvaaja.pohja(-3, 3, -3, 3, nimiX = "$x$", nimiY = "$y$")
    kuvaaja.piirra("2*x+1", nimi = "$y=2x+1$")
    kuvaaja.piirra("2*x-3", nimi = "$y=2x-3$")
\end{kuva}

Jos suorat ovat pystysuoria, ei niiden kulmakerrointa ole määritelty. Myös tällöin suorat ovat yhdensuuntaisia.

\begin{kuva}
    kuvaaja.pohja(-3, 3, -2, 2, nimiX = "$x$", nimiY = "$y$")
    kuvaaja.piirraParametri("1", "t", -2, 2, nimi = "$x=-1$")
    kuvaaja.piirraParametri("-2", "t", -2, 2, nimi = "$x=2$")
\end{kuva}

\laatikko{
Jos kahdella suoralla on sama kulmakerroin tai molemmat suorat ovat pystysuoria, ovat suorat yhdensuuntaiset.
}

Jos suorilla on eri kulmakerroin, on niillä täsmälleen yksi leikkauspiste. 

\begin{kuva}
    kuvaaja.pohja(-3, 3, -3, 3, nimiX = "$x$", nimiY = "$y$")
    kuvaaja.piirra("2*x+1", nimi = "$y=2x+1$")
    kuvaaja.piirra("-3*x+4", nimi = "$y=-3x+4$")
\end{kuva}

Kulmakertoimien perusteella voidaan myös päätellä, ovatko suorat toisiaan vastaan kohtisuorassa.

\begin{kuva}
    kuvaaja.pohja(-2, 4, -2, 4, nimiX = "$x$", nimiY = "$y$")
    kuvaaja.piirra("2*x+1", nimi = "$y=2x+1$")
    kuvaaja.piirra("-0.5*x+2", nimi = "$y=-(1/2)x+2$")
\end{kuva}

(Tähän kuvaan voisi hahmotella kolmiot, joiden avulla suorien kulmakertoimet nähdään kuvasta.)

\laatikko{
Kaksi suoraa ovat toisiaan vastaan kohtisuorassa täsmälleen silloin, jos niiden kulmakertoimien tulo on $-1$
}


\begin{tehtavasivu}

\subsubsection*{Opi perusteet}

\begin {tehtava}
Mikä on suoran $y=-\frac{1}{4}x+\pi$
\begin{enumerate} [a)]
\item kulmakerroin
\item normaalin kulmakerroin?
\end{enumerate}
\begin {vastaus}
a)$-\frac{1}{4}$  b) $4$ 
\end {vastaus}
\end {tehtava}

\begin {tehtava}
Ovatko suorat 
\begin{enumerate} [a)]
\item $y=3x+16$ ja $y=-3x-4$
\item $6y=7x+3$ ja $x=10y-9$ yhdensuuntaiset?
\end{enumerate}
\begin {vastaus}
a) ei b) ei
\end {vastaus}
\end {tehtava}

\begin {tehtava}
Muodosta suoralle $y=-9x+13$ normaali pisteen $(2,5)$ kautta.
\begin {vastaus}
$y=\frac{x}{9}+\frac{43}{9}$
\end {vastaus}
\end {tehtava}

\subsubsection*{Hallitse kokonaisuus}

\begin {tehtava}
Laske suorien leikkauspiste.
\begin{enumerate} [a)]
\item $y=4x+\sqrt{2}$ ja $y=-8x+64$
\item $x=\frac{y}{4}-8$ ja $2x=-y+6x+7$
\end{enumerate}
\begin {vastaus}
a) $\frac{64-\sqrt{2}}{12}$ b) ovat yhdensuuntaiset
\end {vastaus}
\end {tehtava}

\begin {tehtava}
Minna asuu $(x,y)$-koordinaatiston pisteessä $(-2,1)$. Hän kävelee matkan $4\sqrt{2}$ pitkin suoraa $y=x+3$ ja tulee risteykseen, jossa kääntyy $90$ astetta vasemmalle. Minna kävelee vielä matkan $8\sqrt{2}$ suoraan ja saapuu Lassin ovelle. Missä Lassi asuu ja mikä on hänen asuintiensä yhtälö?
\begin {vastaus}
Lassi asuu joko pisteessä $(-6,11)$, yhtälö tällöin $y=-x+3$ tai pisteessä $(2,-13)$, yhtälö $y=-x-11$. Vastaus riippuu siitä, kumpaan suuntaan Minna lähti alussa kävelemään.
\end {vastaus}
\end {tehtava}

\begin {tehtava}
Olkoon $A=(a+3,2)$ ja $B=(a^2,-a)$. Määritä vakio $a$ niin, että pisteiden $A$ ja $B$ kautta kulkeva suora on
\begin{enumerate} [a)]
\item vaakasuora
\item pystysuora?
\end{enumerate}
\begin {vastaus}
a) $a=-2$ b) $a=\frac{1\pm\sqrt{13}}{2} $
\end {vastaus}
\end {tehtava}

\subsubsection*{Sekalaisia tehtäviä}

TÄHÄN TEHTÄVIÄ SIJOITTAMISTA ODOTTAMAAN

\end{tehtavasivu}

