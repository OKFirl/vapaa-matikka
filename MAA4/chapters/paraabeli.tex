\section{Paraabeli}

\laatikko{
KIRJOITA TÄHÄN LUKUUN

\begin{itemize}
\item käyrän $y = ax^2+by+c$ kuvaaja on paraabeli
%%%terminologia? kuvaaja - funktion kuvaaja - käyrä ovatko samoja eivät?
\item mainitaan geometrinen määritelmä
\item paraabelin yhtälön huippumuoto $y-y_0=a(x-x_0)^2$
\end{itemize}

KIITOS!}

\emph{Paraabeli} on tason niiden pisteiden joukko, joiden etäisyys kiinteästä pisteestä, \emph{polttopisteestä} on sama kuin etäisyys kiinteästä suorasta, \emph{johtosuorasta}.

%%%%%%MAA2, luku 3.1 Toisen asteen polynomifunktio
Kussilla 2 mainittiin, että toisen asteen polynomifunktion kuvaaja on paraabeli. Nämä kuvaajat olivat muotoa $y=ax^2+bx+c$ olevia käyriä, joissa $a$ määräsi paraabelin aukeamissuunnan. Jos $a<0$ paraabeli aukeaa alaspäin ja jos $a>0$ ylöspäin.

\begin{kuva}
    kuvaaja.pohja(-1.5, 3.5, -0.5, 2.5, korkeus = 4, nimiX = "$x$", nimiY = "$y$", ruudukko = True)
    kuvaaja.piirra("0.5*x**2-x+0.25", a = -1.5, b = 3.5, nimi = "$y= 0,5x^2-x+0,25$", kohta = (3.2,2.1), suunta = (-1, 1))
\end{kuva}

\begin{kuva}
    kuvaaja.pohja(-1.5, 3.5, -0.5, 2.5, korkeus = 4, nimiX = "$x$", nimiY = "$y$", ruudukko = True)
    kuvaaja.piirra("-0.5*x**2+x+1.75", a = -1.5, b = 3.5, nimi = "$y= -0,5x^2+x+1,75$", kohta = (3.2,-0.5), suunta = (-1, 1))
\end{kuva}


\begin{esimerkki}
Määritä pistejoukon yhtälö, jolla on seuraava ominaisuus: Jokainen pistejoukon piste on yhtä etäällä pisteestä $(0, 3)$ ja suorasta $y=-3$
\begin{esimratk}
Pisteen $P=(x, y)$ etäisyys annetusta pisteestä on
\[
\sqrt{(x-0)^2+(y-3)^2}=\sqrt{x^2+(y-3)^2}
\]
Pisteen $P$ etäisyys annetusta suorasta on pisteen ja suoran $y$-koordinaattien erotuksen itseisarvo
\[
|y-(-3)| = |y+3| 
\]
Merkitään nämä etäisyydet yhtäsuuriksi ja ratkaistaan saatu yhtälö $y$:n suhteen.
\begin{align*}
|y+3| & = \sqrt{x^2+(y-3)^2} &&\ppalkki \text{neliöönkorotus, kumpikin puoli $>0$}\\
(y+3)^2  &= x^2+(y-3)^2 \\
y^2+6y+9 &=  x^2+y^2-6y+9\\
12y &= x^2 &&\ppalkki : 12\\
y &= \frac{1}{12}x^2
\end{align*}

Pistejoukon yhtälö on $y=\frac{1}{12}x^2$.

\end{esimratk}
\end{esimerkki}


%%% FIX ME ONKO HUIPPUMUOTOINEN HYVÄ TERMI? %%%%%%%%%%%
\subsection{Paraabelin huippumuotoinen yhtälö}

Kaikki muotoa $y=ax^2+bx+c$ olevat paraabelit voidaan ilmoittaa paraabelin huipun $(x_0, y_0)$ avulla seuraavasti.

\[
y-y_0 = a(x-x_0)^2
\]

%%%%% FIX ME Mikä on tämän alaluvun suhde MAA1-kirjan liitteenä olevaan alalukuun "Toisen asteen polynomin kuvaaja"


\begin{esimerkki}
Mitkä ovat paraabelin $y=3x^2-12x+13$ huipun koordinaatit?
\begin{esimratk}
Paraabelin huipun koordinaatit nähdään suoraan huippumuotoiseksi muutetusta yhtälöstä. 

\begin{align*}
y-y_0 &= a(x-x_0)^2 \\
y       &= a(x^2-2x_0x+x_0^2)+y_0\\
y       &= ax^2-2ax_0x+(ax_0^2+y_0) &&\ppalkki a=3\\
y       &= 3x^2-6x_0x+(3x_0^2+y_0)
\end{align*}



Merkitään $x$:n kertoimet ja vakiotermit samoiksi.

\begin{align*}
&\begin{cases}
-6x_0=-12 \\
3x_0^2+y_0 =13
\end{cases}\\
&\begin{cases}
x_0=2 \\
y_0 =1
\end{cases}
\end{align*}

Paraabelin huippu on pisteessä $(2,1)$.

\end{esimratk}
\end{esimerkki}



\begin{tehtavasivu}

\subsubsection*{Opi perusteet}

\subsubsection*{Hallitse kokonaisuus}

\subsubsection*{Sekalaisia tehtäviä}

TÄHÄN TEHTÄVIÄ SIJOITTAMISTA ODOTTAMAAN

\end{tehtavasivu}