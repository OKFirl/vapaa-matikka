\section{Paraabeli}

\laatikko{
KIRJOITA TÄHÄN LUKUUN

\begin{itemize}
\item käyrän $y = ax^2+by+c$ kuvaaja on paraabeli
%%%terminologia? kuvaaja - funktion kuvaaja - käyrä ovatko samoja eivät?
\item mainitaan geometrinen määritelmä
\item paraabelin yhtälön huippumuoto $y-y_0=a(x-x_0)^2$
\end{itemize}

KIITOS!}

\emph{Paraabeli} on tason niiden pisteiden joukko, joiden etäisyys kiinteästä pisteestä, \emph{polttopisteestä} on sama kuin etäisyys kiinteästä suorasta, \emph{johtosuorasta}.

%%%%%%MAA2, luku 3.1 Toisen asteen polynomifunktio
Kussilla 2 mainittiin, että toisen asteen polynomifunktion kuvaaja on paraabeli. Nämä kuvaajat olivat muotoa $y=ax^2+bx+c$ olevia käyriä, joissa $a$ määräsi paraabelin aukeamissuunnan. Jos $a<0$ paraabeli aukeaa alaspäin ja jos $a>0$ ylöspäin.

\begin{kuva}
    kuvaaja.pohja(-1.5, 3.5, -0.5, 2.5, korkeus = 4, nimiX = "$x$", nimiY = "$y$", ruudukko = True)
    kuvaaja.piirra("0.5*x**2-x+0.25", a = -1.5, b = 3.5, nimi = "$y= 0,5x^2-x+0,25$", kohta = (3.2,2.1), suunta = -135)
\end{kuva}

\begin{kuva}
    kuvaaja.pohja(-1.5, 3.5, -0.5, 2.5, korkeus = 4, nimiX = "$x$", nimiY = "$y$", ruudukko = True)
    kuvaaja.piirra("-0.5*x**2+x+1.75", a = -1.5, b = 3.5, nimi = "$y= -0,5x^2+x+1,75$", kohta = (3.2,-0.5), suunta = -135)
\end{kuva}

\begin{esimerkki}
Määritä pistejoukon yhtälö, jolla on seuraava ominaisuus: Jokainen pistejoukon piste on yhtä etäällä pisteestä $(0, 3)$ ja suorasta $y=-3$
\begin{esimratk}
Pisteen $P=(x, y)$ etäisyys annetusta pisteestä on
\[
\sqrt{(x-0)^2+(y-3)^2}=\sqrt{x^2+(y-3)^2}
\]
Pisteen $P$ etäisyys annetusta suorasta on pisteen ja suoran $y$-koordinaattien erotuksen itseisarvo
\[
|y-(-3)| = |y+3| 
\]
Merkitään nämä etäisyydet yhtäsuuriksi ja ratkaistaan saatu yhtälö $y$:n suhteen.
\begin{align*}
|y+3| & = \sqrt{x^2+(y-3)^2} &&\ppalkki \text{neliöönkorotus, kumpikin puoli $>0$}\\
(y+3)^2  &= x^2+(y-3)^2 \\
y^2+6y+9 &=  x^2+y^2-6y+9\\
12y &= x^2 &&\ppalkki : 12\\
y &= \frac{1}{12}x^2
\end{align*}

Pistejoukon yhtälö on $y=\frac{1}{12}x^2$.

\end{esimratk}
\end{esimerkki}

%%% FIX ME ONKO HUIPPUMUOTOINEN HYVÄ TERMI? %%%%%%%%%%%
\subsection{Paraabelin huippumuotoinen yhtälö}

Kaikki muotoa $y=ax^2+bx+c$ olevat paraabelit voidaan ilmoittaa paraabelin huipun $(x_0, y_0)$ avulla seuraavasti.

\[
y-y_0 = a(x-x_0)^2
\]

%%%%% FIX ME Mikä on tämän alaluvun suhde MAA1-kirjan liitteenä olevaan alalukuun "Toisen asteen polynomin kuvaaja"

\begin{esimerkki}
Mitkä ovat paraabelin $y=3x^2-12x+13$ huipun koordinaatit?
\begin{esimratk}
Paraabelin huipun koordinaatit näkisi suoraan huippumuotoiseksi muutetusta yhtälöstä. Helpompaa lienee kuitenkin muuttaa huippumuoitoinen yhtälö $y$:n suhteen ratkaistuksi ja merkitä yhtälöiden kertoimet samoiksi, jolloin saadaan selville $x_0$ ja $y_0$.

\begin{align*}
y-y_0 &= a(x-x_0)^2 \\
y       &= a(x^2-2x_0x+x_0^2)+y_0\\
y       &= ax^2-2ax_0x+(ax_0^2+y_0) &&\ppalkki a=3\\
y       &= 3x^2-6x_0x+(3x_0^2+y_0)
\end{align*}

Merkitään $x$:n kertoimet ja vakiotermit samoiksi.

\begin{align*}
&\begin{cases}
-6x_0=-12 \\
3x_0^2+y_0 =13
\end{cases}\\
&\begin{cases}
x_0=2 \\
y_0 =1
\end{cases}
\end{align*}

Paraabelin huippu on pisteessä $(2, 1)$.

\end{esimratk}
\end{esimerkki}

\begin{tehtavasivu}

\subsubsection*{Opi perusteet}

\subsubsection*{Hallitse kokonaisuus}

\subsubsection*{Sekalaisia tehtäviä}

TÄHÄN TEHTÄVIÄ SIJOITTAMISTA ODOTTAMAAN

\begin{tehtava}
Millaisella käyrällä ovat ympyröiden keskipisteet, kun ympyrät kulkevat pisteen $(0, 0)$ kautta ja sivuavat suoraa $x=4$?
\begin{vastaus}
%keskipiste (x, y)
% x< 4 
% kulkee origon kautta, joten (0-x)^2+(0-y)^2=r^2
% etäisyys suorasta |x-4| = r
$x=-\frac{1}{8}y^2+2$
\end{vastaus}
\end{tehtava}

\begin{tehtava}
Mikä on sen käyrän yhtälö, jonka kukin piste on yhtä etäällä suorasta $y=0$ ja pisteestä $(-3, 1)$.
\begin{vastaus}
% http://www.wolframalpha.com/input/?i=+sqrt%28%28-3-x%29%5E2+%2B+%281-y%29%5E2%29%3D+abs%280-y%29
$y = \frac{1}{2}x^2+3 x+5$
\end{vastaus}
\end{tehtava}

\begin{tehtava}
%%% ONKO JOHTOSUORA NIIN OLEELLINEN KÄSITE, ETTÄ KÄYTETÄÄN TEHTÄVISSÄ?
%% tehtävänhän voi kirjoittaa ilman tuota termiä
Mikä on sen paraabelin yhtälö, jonka polttopiste on $(2, 3)$ ja johtosuora $y=1$?
\begin{vastaus}
% http://www.wolframalpha.com/input/?i=+sqrt%28%282-x%29%5E2+%2B+%283-y%29%5E2%29%3D+abs%281-y%29
$y = \frac{1}{4}x^2-x+3$
\end{vastaus}
\end{tehtava}

\begin{tehtava}
Mitkä ovat suoran $x+y=6$ ja paraabelin $y=4x^2-3x$ leikkauspisteet?
\begin{vastaus}
% http://www.wolframalpha.com/input/?i=y%3D4x%5E2-3x%2C+x%2By%3D6
$x = -1$, $ y = 7$ tai $x = \frac{3}{2}$, $y = \frac{9}{2}$
\end{vastaus}
\end{tehtava}

%%%%%TÄMÄ JA SEURAAVA PITÄISI SIIRTÄÄ VASEMMALLE/OIKEALLE AUKEAVIEN PARAABELIEN JÄLKEEN?
\begin{tehtava}
Millaisella käyrällä ovat ympyröiden keskipisteet, kun ympyrät kulkevat pisteen $(0, 0)$ kautta ja sivuavat suoraa $x=4$?
\begin{vastaus}
%keskipiste (x, y)
% x< 4 
% kulkee origon kautta, joten (0-x)^2+(0-y)^2=r^2
% etäisyys suorasta |x-4| = r
$x=-\frac{1}{8}y^2+2$
\end{vastaus}
\end{tehtava}

\begin{tehtava}
Kuinka monta leikkauspistettä voi olla paraabelilla ja
\begin{enumerate}[a)]
\item suoralla,
\item ympyrällä,
\item toisella paraabelilla?
\end{enumerate}
\begin{vastaus}
\begin{enumerate}[a)]
\item 0--2
\item 0--4
\item 0--4
\end{enumerate}
\end{vastaus}
\end{tehtava}

\begin{tehtava}
Määritä ne paraabelin $y=x^2-1$ pisteet, jotka ovat yhtä kaukana pisteistä $(4, 4)$ ja $(4, 2)$?
\begin{vastaus}
%suoran y=3 ja paraabelin leikkauspisteet
$x=-2$, $y=3$ ja $x=2$, $y=3$
\end{vastaus}
\end{tehtava}

\begin{tehtava}
Määritä ne paraabelin $y=x^2-1$ pisteet, jotka ovat yhtä kaukana pisteistä $(4, 4)$ ja $(3, 3)$?
\begin{vastaus}
%pisteiden keskipisteen (3,5; 3,5) kautta kulkeva suora y=-x+7
%suoran  ja paraabelin leikkauspistee
% http://www.wolframalpha.com/input/?i=y%3Dx%5E2-1%2C+y%3D-x%2B7
$x = -\frac{1+\sqrt{33}}{2}$,   $y = \frac{15+\sqrt{33}}{2}$ tai $x = -\frac{\sqrt{33}-1}{2}$,   $y = \frac{15-\sqrt{33}}{2}$
\end{vastaus}
\end{tehtava}

\begin{tehtava}
Ratkaise paraabelien $y=5(x-2)^2$  ja $y=-x^2+2x+4$ leikkauspisteet?
\begin{vastaus}
%tulee toisen asteen yhtälö ratkaistavaksi
% http://www.wolframalpha.com/input/?i=y%3D5%28x-2%29%5E2%2C+y%3D-x%5E2%2B2x%2B4
$(1, 5)$ ja $(\frac{8}{3}, \frac{20}{9})$
\end{vastaus}
\end{tehtava}

\begin{tehtava}
%tehtävä helpottuu, koska vakiotermin saa suoraan
Määritä sen ylöspäin aukeavan paraabelin yhtälö, joka kulkee pisteiden $(-5, 6)$, $(0, -4)$ ja $(1, 0)$ kautta.
\begin{vastaus}
% http://www.wolframalpha.com/input/?i=y%3D+x%5E2%2B3x-4+at+x%3D%7B-5%2C+0%2C+1%7D
$y= x^2+3x-4$
\end{vastaus}
\end{tehtava}

\begin{tehtava}
Määritä sen alaspäin aukeavan paraabelin yhtälö, joka kulkee pisteiden $(-5, 6)$, $(0, -4)$ ja $(1, 0)$ kautta.
\begin{vastaus}
% http://www.wolframalpha.com/input/?i=y%3D+x%5E2%2B3x-4+at+x%3D%7B-5%2C+0%2C+1%7D
$y= x^2+3x-4$
\end{vastaus}
\end{tehtava}

\end{tehtavasivu}

