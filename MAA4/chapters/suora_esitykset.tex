\section{Suoran yhtälön muut muodot}

\laatikko{
KIRJOITA TÄHÄN LUKUUN

\begin{itemize}
\item suoran yhtälö normaalimuodossa $ax+by+c=0$
\item suoran yhtälö muodossa $y-y_0=k(x-x_0)$, suoran yhtälön muodostaminen pisteiden avulla
\end{itemize}

KIITOS!}

\subsection*{Valeorigomuoto} % tämä ei ole vakiintunut termi!!!

Toisinaan suoran yhtälöä on helpompaa tarkastella muodossa $y-y_0=k(x-x_0)$.
Tämän voidaan ajatella olevan origon kautta kulkeva suora, jos origo olisi
pisteessä $(x_0, y_0)$. Valeorigomuoto on kätevin silloin, kun tiedämme suoran
kulmakertoimen ja yhden pisteen, jonka kautta suora kulkee.

\begin{esimerkki}
    Esimerkkejä valeorigomuodon käytöstä:
    \begin{enumerate}[a)]
        \item Suoran kulmakerroin on $5$ ja suora kulkee pisteen $(3,0)$ kautta.
        \[y-y_0 = k(x-x_0) \Leftrightarrow y-0 = 5(x-3) \Leftrightarrow y = 5x-15\]
        \item Suoran kulmakerroin on $4$ ja suora kulkee pisteen $(5,7)$ kautta.
        \[y-y_0 = k(x-x_0) \Leftrightarrow y-7 = 4(x-5) \Leftrightarrow y-7 = 4x-20 \Leftrightarrow y = 4x-13\]
    \end{enumerate}
\end{esimerkki}

\subsection*{Normaalimuoto}

Suoran yhtälön kanonisin muoto on normaalimuoto tai yleinen muoto $ax+by+c=0$.
