\section{Suoran yhtälön muut muodot}

\laatikko{
KIRJOITA TÄHÄN LUKUUN

\begin{itemize}
\item suoran yhtälö normaalimuodossa $ax+by+c=0$
\item suoran yhtälö muodossa $y-y_0=k(x-x_0)$, suoran yhtälön muodostaminen pisteiden avulla
\end{itemize}

KIITOS!}

\subsection*{Valeorigomuoto} % tämä ei ole vakiintunut termi!!!

Toisinaan suoran yhtälöä on helpompaa tarkastella muodossa $y-y_0=k(x-x_0)$.
Tämän voidaan ajatella olevan origon kautta kulkeva suora, jos origo olisi
pisteessä $(x_0, y_0)$. Valeorigomuoto on kätevin silloin, kun tiedämme suoran
kulmakertoimen ja yhden pisteen, jonka kautta suora kulkee.

\begin{esimerkki}
    Esimerkkejä valeorigomuodon käytöstä:
    \begin{enumerate}[a)]
        \item Suoran kulmakerroin on $5$ ja suora kulkee pisteen $(3,0)$ kautta.
        \[y-y_0 = k(x-x_0) \ekvi y-0 = 5(x-3) \ekvi y = 5x-15\]
        \item Suoran kulmakerroin on $4$ ja suora kulkee pisteen $(5,7)$ kautta.
        \[y-y_0 = k(x-x_0) \ekvi y-7 = 4(x-5) \ekvi y-7 = 4x-20 \ekvi y = 4x-13\]
    \end{enumerate}
\end{esimerkki}

\subsection*{Normaalimuoto}

Suoran yhtälön kanonisin muoto on normaalimuoto tai yleinen muoto $ax+by+c=0$.

\begin{tehtavasivu}

\subsubsection*{Opi perusteet}

\begin{tehtava}
Mikä on suoran yhtälö normaalimuodossa?
\begin{enumerate}[a)]
\item $y=-15x+2$
\item $2y=11x+7$
\item $2y+5x-8=13y-6x-8$
\end{enumerate}
\begin{vastaus}
a)$15x+y-2=0$ b) $11x-2y+7=0$ c) $-11x+11y=0$
\end{vastaus}
\end{tehtava}


\begin{tehtava}
Suoran kulmakerroin on $\frac{1}{2}$ ja suora kulkee pisteen
\begin{enumerate}[a)]
\item $(-12,4)$
\item $(3,9)$. Mikä on suoran yhtälö?
\end{enumerate}
\begin{vastaus}
a)$y=\frac{1}{2}x+10$ b) $y=\frac{1}{2}x+\frac{15}{2}$
\end{vastaus}
\end{tehtava}

\begin{tehtava}
Mikä on pisteiden
\begin{enumerate}[a)]
\item $(1,-2)$ ja $(3,1)$
\item $(0,0)$ ja $(-4,4)$ kautta kulkevan suoran yhtälö?
\end{enumerate}
\begin{vastaus}
a)$y=\frac{3}{2}x-\frac{7}{2}$ b) $y=-x$
\end{vastaus}
\end{tehtava}

\subsubsection*{Hallitse kokonaisuus}

\begin{tehtava}
Tutki ovatko pisteet  
\begin{enumerate}[a)]
\item $(1,-5)$, $(4,-23)$ja $(4,-239)$
\item $(7,3)$, $(-2,10)$ ja $(-3,90)$ samalla suoralla?
\end{enumerate}
\begin{vastaus}
a) kyllä b) ei
\end{vastaus}
\end{tehtava}

\begin{tehtava}
Määritä luku $t$ niin, että pisteet $(-t+3,-4)$, $(6,t-5)$ ja $(5,-4)$ ovat samalla suoralla.
\begin{vastaus}
$t=-2$ tai $t=1$
\end{vastaus}
\end{tehtava}

\subsubsection*{Sekalaisia tehtäviä}

TÄHÄN TEHTÄVIÄ SIJOITTAMISTA ODOTTAMAAN

\begin {tehtava}
Suora kulkee pisteiden $(3,4)$ ja $(\sqrt{3},1)$ kautta. Määritä suoran kulmakerroin.
\begin {vastaus}
$\frac{\sqrt{3}-1}{\sqrt{3}}$
\end {vastaus}
\end {tehtava}

\end{tehtavasivu}
