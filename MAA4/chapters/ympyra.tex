\section{Ympyrä}

\laatikko{
KIRJOITA TÄHÄN LUKUUN

\begin{itemize}
\item ympyrän määritelmä ja siitä seuraava yhtälö,
origokeskinen ensin
\item muodon $x^2 + y^2 +ax +by +c=0$ täydentäminen neliöksi
ja ympyrän keskipisteen ja säteen selvittäminen siitä
\end{itemize}

KIITOS!}

\begin{kuva}
    kuvaaja.pohja(-3.5, 3.5, -3.5, 3.5, korkeus = 4, nimiX = "$x$", nimiY = "$y$", ruudukko = True)
    kuvaaja.piirraParametri("3*cos(t)", "3*sin(t)", a = 0, b = 2*pi)
    piste((3*cos(0.75), 3*sin(0.75)), "(x, y)", -120)
\end{kuva}

Kuvaan on piirretty käyrä, jonka pisteiden etäisyys origosta on 3. Huomataan, että näin muodostuva käyrä on ympyrä. Piste $(x,y)$ on ympyrällä täsmälleen silloin, jos sen etäisyys origosta on 3. Toisin sanoen täytyy päteä $\sqrt{x^2+y^2}=3$. Kun yhtälön molemmat puolet korotetaan vielä toiseen potenssiin, saadaan $x^2+y^2=9$. Ympyrän yhtälö on siis $x^2+y^2=9$.

\termi{ympyrä}{Ympyrä} muodostuu pisteistä, jotka ovat vakioetäisyydellä jostakin kiinteästä pisteestä, \termi{keskipiste}{keskipisteestä}. Tätä vakioetäisyyttä kutsutaan ympyrän \termi{säde}{säteeksi}.

Johdetaan yhtälö ympyrälle, jonka keskipiste on $(x_0,y_0)$ ja säde $r$. Piste $(x,y)$ on ympyrälllä täsmälleen silloin, jos sen etäisyys pisteestä $(x_0,y_0)$ on $r$. Luvun ?? perusteella pisteiden $(x,y)$ ja $(x_0,y_0)$ välinen etäisyys on $\sqrt{(x-x_0)^2+(y-y_0)^2}$. Tuloksena on siis yhtälö
\[
\sqrt{(x-x_0)^2+(y-y_0)^2}=r.
\]
Koska säde $r$ ei voi olla negatiivinen, voidaan yhtälön molemmat puolet korottaa toiseen potenssiin ja saadaan yhtäpitävä yhtälö
\[
(x-x_0)^2+(y-y_0)^2=r^2.
\]

%Jos erityisesti $(x_{0},y_{0})= (0,0)$, eli ympyrän keskipiste on origo, saa yhtälö muodon

%\[
%x^{2}+y^{2} = r^{2}.
%\]

\laatikko{
Jos ympyrän keskipiste on $(x_{0},y_{0})$ ja säde $r$, ympyrän yhtälö on
\[
(x-x_{0})^{2}+(y-y_{0})^{2} = r^{2}.
\]
}

Jos edellä säde $r$ on nolla, yhtälön toteuttaa vain piste $(x_{0},y_{0})$. Nollasäteinen ympyrä onkin pelkkä piste.

%\begin{esimerkki}
%Ympyrän keskipiste on $(-4,1)$ ja säde $5$. Määritä ympyrän yhtälö ja hahmottele ympyrä koordinaatistoon.
%\begin{esimratk}
%Ympyrän yhtälö saadaan käyttämällä edellä annettua kaavaa. Nyt $x_0=-4$, $y_0=1$ ja $r=5$. Ympyrän yhtälöksi saadaan $(x-(-4))^2+(y-1)^2=25$ eli
%\[
%(x+4)^2+(y-1)^2=25.
%\]
%\end{esimratk}
%\begin{esimvast}
%Ympyrän yhtälö on $(x+4)^2+(y-1)^2=25$.
%\end{esimvast}
%\end{esimerkki}

%Tähän kuva ympyrästä.

%Edellisen esimerkin ympyrän yhtälö voidaan kirjoittaa myös toisenlaisessa muodossa.
%\begin{align*}
%(x+4)^2+(y-1)^2&=25 \\
%x^2+8x+16+y^2-2y+1&=25 \\
%x^2+y^2+8x-2y-8&=0.
%\end{align*}

\begin{esimerkki}
Piirrä kuva ympyrästä, jonka yhtälö on
\[
(x-1)^2+(x+1)^2=4.
\]
\begin{esimratk}
Muokataan ympyrän yhtälöä niin, että keskipiste ja säde näkyvät suoraan:
\[
(x-1)^2+(x-(-1))^2=2^2.
\]
Tästä nähdään, että keskipiste on $(1,-1)$ ja säde 2. Nyt kuva on helppo piirtää.

TÄHÄN TARVITAAN KUVA.
\end{esimratk}
\end{esimerkki}

%\begin{esimerkki}
%Ympyrän $\Gamma_{1}$\footnote{sdf} keskipiste on $(3,-4)$ ja säde $\sqrt{2}$, ja ympyrän $\Gamma_{2}$ keskipiste origo ja säde 1. Määritä ympyröiden yhtälöt. Hahmottele ympyrät koordinaatistoon.

%\begin{esimratk}
%Edellisen mukaan ympyrän $\Gamma_{1}$ yhtälö on
%\[
%(x-3)^{2}+(y-(-4))^{2} = (\sqrt{2})^{2}
%\]
%eli sievennettynä
%\[
%(x-3)^{2}+(y+4)^{2} = 2.
%\]
%$\Gamma_{2}$:n yhtälö saadaan vastaavasti:
%\[
%x^{2}+y^{2} = 1.
%\]
%Edelliselle yksikkösäteiselle origokeskiselle ympyrälle on vakiintunut nimitys \emph{yksikköympyrä}.

%\end{esimratk}
%\end{esimerkki}

Aina keskipiste ja säde eivät näy ympyrän yhtälöstä suoraan.
Esimerkiksi yhtälö
\[
x^2+6x+y^2-4y=3
\]
on erään ympyrän yhtälö.
Tämä nähdään täydentämällä summattavat $x^2+6x$ ja $y^2-4y$ neliöiksi.

Neliöksi täydentäminen opittiin kurssissa MAA2, mutta kerrataan se tässä vielä.
Aloitetaan lausekkeesta $x^2+6x$. Se on lähes sama kuin binomin $x+3$ neliö, sillä $(x+3)^2=x^2+6x+9$. Ainoastaan vakiotermit poikkeavat toisistaan ja tämän voi korvata lisäämällä ympyrän yhtälön molemmille puolille luvun $9$: 
\begin{align*}
x^2+6x+y^2-4y &= 3 && \ppalkki +9 \\
(x^2+6x+9)+(y^2-4y) &= 12 && \\
(x+3)^2+(y^2-4y) &= 12.&& 
\end{align*}

Siirrytään sitten tarkastelemaan summattavaa $y^2-4y$. Se on puolestaan melkein binomin $y-2$ neliö, sillä $(y-2)^2=y^2-4y+4$. Binomin neliö saadaan näkyviin lisäämällä yhtälön molemmille puolille luku $4$:
\begin{align*}
(x+3)^2+(y^2-4y) &= 12 && \ppalkki +4\\
(x+3)^2+(y^2-4y+4) &= 16 && \\
(x+3)^2+(y-2)^2 &= 16 && \\
(x+3)^2+(y-2)^2 &= 4^2.&& 
\end{align*}

Nyt huomataan, että kyseessä on $(-3,2)$-keskisen $4$-säteisen ympyrän yhtälö.

\begin{esimerkki}
Ympyrän yhtälö on $x^2-8x+y^2+5y+3=0$. Määritä ympyrän keskipiste ja säde.
\begin{esimratk}
Yhtälö muuttuu muotoon $x^2-8x+y^2+5y=-3$. Suoritetaan sitten neliöksi täydentäminen:
\begin{align*}
x^2-8x+y^2+5y&=-3 && \ppalkki +16\\
x^2-8x+16+y^2+5y&=-3+16 && \\
(x^2-4)^2+y^2+5y&=13 && \ppalkki +\frac{25}{4}\\
(x^2-4)^2+y^2+5y+\frac{25}{4}&=\frac{77}{4} && \ppalkki +\frac{25}{4}\\
(x^2-4)^2+\left(y+\frac{5}{4}\right)^2&=\frac{102}{4} && \\
(x^2-4)^2+\left(y+\frac{5}{4}\right)^2&=\frac{51}{2} && 
\end{align*}
Nähdään, että keskipiste on $(4,-5/4)$ ja säde $\sqrt{51/2}$.
\end{esimratk}
\begin{esimvast}
Keskipiste on $(4,-5/4)$ ja säde $\sqrt{51/2}$.
\end{esimvast}
\end{esimerkki}

\begin{esimerkki}
Onko yhtälö $x^2-4x+y^2+2y+6=0$ ympyrän yhtälö?
\begin{esimratk}
Suoritetaan neliöksitäydennys:
\begin{align*}
x^2-4x+y^2+2y+6&=0 && \ppalkki -6\\
x^2-4x+y^2+2y&=-6 && \ppalkki +4\\
x^2-4x+4+y^2+2y&=-2 && \\
(x^2-2)^2+y^2+2y&=-2 && \ppalkki +1\\
(x^2-2)^2+y^2+2y+1&=-1 && \\
(x^2-2)^2+(y+1)^2&=-1. &&
\end{align*}
Nyt nähdään, että kyseessä ei voi olla ympyrän yhtälö, sillä säteeksi tulisi $\sqrt{-1}$.
\end{esimratk}
\begin{esimvast}
Kyseessä ei ole ympyrän yhtälö.
\end{esimvast}
\end{esimerkki}

Edellä tehty neliöön korotus voidaan voidaan suorittaa yleisesti muotoa
\[
x^2+ax+y^2+by+c = 0
\]
oleville yhtälöille. Täydentämällä $x^2+ax$ ja $y^2+by$ neliöiksi saadaan
\begin{align*}
x^2+ax+y^2+by+c &= 0 && \ppalkki +\frac{a^2}{4}+\frac{b^2}{4}-c \\
\Big(x^2+ax+\frac{a^2}{4}\Big)+\Big(y^2+by+\frac{b^2}{4}\Big) &= \frac{a^2}{4}+\frac{b^2}{4}-c  \\
\Big(x+\frac{a}{2}\Big)^2+\Big(y+\frac{b}{2}\Big)^2 &= \frac{a^2}{4}+\frac{b^2}{4}-c
\end{align*}
Jos yhtälön oikea puoli eli $\frac{a^2}{4}+\frac{b^2}{4}-c \geq 0$ yhtälö kuvaa $\sqrt{\frac{a^2}{4}+\frac{b^2}{4}-c}$-säteistä $(-\frac{a}{2},-\frac{b}{2})$-keskistä ympyrää. Jos oikea puoli on nolla, yhtälö kuvaa vastaavaa pistettä. Jos se on negatiivinen, yhtälön vasen puoli on aina positiivinen, joten yhtälön toteuttavia lukupareja ei ole. Yhtälö ei siis kuvaa mitään käyrää.

%\laatikko{Yhtälöt muotoa
%\[
%x^2+ax+y^2+by+c = 0
%\]
%kuvaavat ympyrää, pistettä tai tyhjää joukkoa.}

\begin{tehtavasivu}

\paragraph*{Opi perusteet}

\paragraph*{Hallitse kokonaisuus}

\paragraph*{Sekalaisia tehtäviä}

TÄHÄN TEHTÄVIÄ SIJOITTAMISTA ODOTTAMAAN

\begin{tehtava}
Ympyrän kespiste on $(0,0)$ ja säde $5$. Muodosta ympyrän yhtälö.
\begin{vastaus}
$x^2+y^2=5$
\end{vastaus}
\end{tehtava}

\begin{tehtava}
Määritä keskipiste ja säde.
\begin{alakohdat}
  \alakohta{$(x-3)^2+(y+7)^2=12$}
	\alakohta{$x^2+y^2=49$}
\end{alakohdat}
\begin{vastaus}
\begin{alakohdat}
	\alakohta{keskpiste $(3,-7)$, säde $2\sqrt{3}$}
	\alakohta{keskipiste $(0,0)$, säde $7$}
\end{alakohdat}
\end{vastaus}
\end{tehtava}

\begin{tehtava}
Määritä keskipiste ja säde.
\begin{alakohdat}
	\alakohta{$x^2+y^2-10x+16y+72=0$}
	\alakohta{$x^2+y^2+8x-22y+129=0$}
\end{alakohdat}
\begin{vastaus}
\begin{alakohdat}
	\alakohta{keskpiste $(5,-8)$, säde $\sqrt{17}$}
	\alakohta{keskipiste $(-4,11)$, säde $2\sqrt{2}$}
\end{alakohdat}
\end{vastaus}
\end{tehtava}

\begin{tehtava}
Määritä ympyrän $(x+10)^2+y^2=2$ keskipiste ja säde ja ratkaise ympyrän yhtälöstä $y$. 
\begin{vastaus}
keskipiste $(-10,0)$, säde $\sqrt{2}$, $y=\pm\sqrt{2-(x+10)^2}$ 
\end{vastaus}
\end{tehtava}

\begin{tehtava}
Ympyrän keskipiste on origo ja säde $3$. Onko piste 
\begin{alakohdat}
	\alakohta{$(10,-2)$}
	\alakohta{$(-3,0)$}
	\alakohta{$(2,\sqrt{5})$ ympyrän kehällä?}
\end{alakohdat}
\begin{vastaus}
\begin{alakohdat}
	\alakohta{ei!}
	\alakohta{joo!}
	\alakohta{joo!}
\end{alakohdat}
\end{vastaus}
\end{tehtava}

\begin{tehtava}
Määritä $k$ niin, että lauseke $(x-3)^2+(y+3)^2=k$ on
\begin{alakohdat}
	\alakohta{ympyrä}
	\alakohta{$\sqrt{7}$-säteinen ympyrä}
	\alakohta{origon kautta kulkeva ympyrä?}
\end{alakohdat}
\begin{vastaus}
\begin{alakohdat}
	\alakohta{$k>0$}
	\alakohta{$k=7$}
	\alakohta{$k=18$}
\end{alakohdat}
\end{vastaus}
\end{tehtava}

\begin{tehtava}
Tutki, mitä yhtälöiden kuvaajat esittävät.
\begin{alakohdat}
	\alakohta{$x^2+y^2-6x+4y+4=0$}
	\alakohta{$x^2+y^2+14x-6y+10=0$}
\end{alakohdat}
\begin{vastaus}
\begin{alakohdat}
	\alakohta{ympyrä}
	\alakohta{piste}
\end{alakohdat}
\end{vastaus}
\end{tehtava}

\begin{tehtava}
Määritä ympyrän keskipiste ja säde.
\begin{alakohdat}
	\alakohta{$(x+t)^2+(y+u)^2=k, k>0$}
	\alakohta{$(x+2)^2+(y-7)^2=-8$}
\end{alakohdat}
\begin{vastaus}
\begin{alakohdat}
	\alakohta{keskipiste $(-t,-u)$, säde  $\sqrt{k}$}
	\alakohta{ei ole ympyrä}
\end{alakohdat}
\end{vastaus}
\end{tehtava}

\begin{tehtava}
Ympyrä sivuaa $y$-akselia pisteessä $(0,-1)$ ja kulkee pisteen $(3,2)$ kautta. Mikä on ympyrän yhtälö?
\begin{vastaus}
$(x-3)^2+(y+1)^2=9$
\end{vastaus}
\end{tehtava}

\begin{tehtava}
Ympyrä kulkee pisteiden $(1,6), (-2,5)$ ja $(5,4)$ kautta. Mikä on ympyrän yhtälö?
\begin{vastaus}
$(x-1)^2+(y-1)^2=16$
\end{vastaus}
\end{tehtava}

\begin{tehtava}
Jana, jonka pituus on $t$ liikkuu koordinaatistossa siten, että sen toinen pää on $x$-akselilla ja toinen $y$-akselilla. Mitä käyrää pitkin liikkuu janan keskipiste?
\begin{vastaus}
$x^2+y^2=\frac{1}{4}t^2$
\end{vastaus}
\end{tehtava}

\end{tehtavasivu}