\section{Ympyrä}

\laatikko{
KIRJOITA TÄHÄN LUKUUN

\begin{itemize}
\item ympyrän määritelmä ja siitä seuraava yhtälö,
origokeskinen ensin
\item muodon $x^2 + y^2 +ax +by +c=0$ täydentäminen neliöksi
ja ympyrän keskipisteen ja säteen selvittäminen siitä
\end{itemize}

KIITOS!}

\begin{tehtavasivu}

\paragraph*{Opi perusteet}

\paragraph*{Hallitse kokonaisuus}

\paragraph*{Sekalaisia tehtäviä}

TÄHÄN TEHTÄVIÄ SIJOITTAMISTA ODOTTAMAAN

\begin{tehtava}
Ympyrän kespiste on $(0,0)$ ja säde $5$. Muodosta ympyrän yhtälö.
\begin{vastaus}
$x^2+y^2=5$
\end{vastaus}
\end{tehtava}

\begin{tehtava}
Märitä keskipiste ja säde.
\begin{alakohdat}
  \alakohta{$(x-3)^2+(y+7)^2=12$}
	\alakohta{$x^2+y^2=49$}
\end{alakohdat}
\begin{vastaus}
\begin{alakohdat}
	\alakohta{keskpiste $(3,-7)$, säde $2\sqrt{3}$}
	\alakohta{keskipiste $(0,0)$, säde $7$}
\end{alakohdat}
\end{vastaus}
\end{tehtava}

\begin{tehtava}
Märitä keskipiste ja säde.
\begin{alakohdat}
	\alakohta{$x^2+y^2-10x+16y+72=0$}
	\alakohta{$x^2+y^2+8x-22y+129=0$}
\end{alakohdat}
\begin{vastaus}
\begin{alakohdat}
	\alakohta{keskpiste $(5,-8)$, säde $\sqrt{17}$}
	\alakohta{keskipiste $(-4,11)$, säde $2\sqrt{2}$}
\end{alakohdat}
\end{vastaus}
\end{tehtava}

\begin{tehtava}
Määritä ympyrän $(x+10)^2+y^2=2$ keskipiste ja säde ja ratkaise ympyrän yhtälöstä $y$. 
\begin{vastaus}
keskipiste $(-10,0)$, säde $\sqrt{2}$, $y=\pm\sqrt{2-(x+10)^2}$ 
\end{vastaus}
\end{tehtava}

\begin{tehtava}
Ympyrän keskipiste on origo ja säde $3$. Onko piste 
\begin{alakohdat}
	\alakohta{$(10,-2)$}
	\alakohta{$(-3,0)$}
	\alakohta{$(2,\sqrt{5})$ ympyrän kehällä?}
\end{alakohdat}
\begin{vastaus}
\begin{alakohdat}
	\alakohta{ei!}
	\alakohta{joo!}
	\alakohta{joo!}
\end{alakohdat}
\end{vastaus}
\end{tehtava}

\begin{tehtava}
Määritä $k$ niin, että lauseke $(x-3)^2+(y+3)^2=k$ on
\begin{alakohdat}
	\alakohta{ympyrä}
	\alakohta{$\sqrt{7}$-säteinen ympyrä}
	\alakohta{origon kautta kulkeva ympyrä?}
\end{alakohdat}
\begin{vastaus}
\begin{alakohdat}
	\alakohta{$k>0$}
	\alakohta{$k=7$}
	\alakohta{$k=18$}
\end{alakohdat}
\end{vastaus}
\end{tehtava}

\begin{tehtava}
Tutki, mitä yhtälöiden kuvaajat esittävät.
\begin{alakohdat}
	\alakohta{$x^2+y^2-6x+4y+4=0$}
	\alakohta{$x^2+y^2+14x-6y+10=0$}
\end{alakohdat}
\begin{vastaus}
\begin{alakohdat}
	\alakohta{ympyrä}
	\alakohta{piste}
\end{alakohdat}
\end{vastaus}
\end{tehtava}

\begin{tehtava}
Määritä ympyrän keskipiste ja säde.
\begin{alakohdat}
	\alakohta{$(x+t)^2+(y+u)^2=k, k>0$}
	\alakohta{$(x+2)^2+(y-7)^2=-8$}
\end{alakohdat}
\begin{vastaus}
\begin{alakohdat}
	\alakohta{keskipiste $(-t,-u)$, säde  $\sqrt{k}$}
	\alakohta{ei ole ympyrä}
\end{alakohdat}
\end{vastaus}
\end{tehtava}

\begin{tehtava}
Ympyrä sivuaa $y$-akselia pisteessä $(0,-1)$ ja kulkee pisteen $(3,2)$ kautta. Mikä on ympyrän yhtälö?
\begin{vastaus}
$(x-3)^2+(y+1)^2=9$
\end{vastaus}
\end{tehtava}

\begin{tehtava}
Ympyrä kulkee pisteiden $(1,6), (-2,5)$ ja $(5,4)$ kautta. Mikä on ympyrän yhtälö?
\begin{vastaus}
$(x-1)^2+(y-1)^2=16$
\end{vastaus}
\end{tehtava}

\begin{tehtava}
Jana, jonka pituus on $t$ liikkuu koordinaatistossa siten, että sen toinen pää on $x$-akselilla ja toinen $y$-akselilla. Mitä käyrää pitkin liikkuu janan keskipiste?
\begin{vastaus}
$x^2+y^2=\frac{1}{4}t^2$
\end{vastaus}
\end{tehtava}

\end{tehtavasivu}