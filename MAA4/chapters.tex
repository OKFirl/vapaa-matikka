\chapter{Esitietoja}
	\section{Itseisarvo}


\laatikko{
KIRJOITA TÄHÄN LUKUUN

\begin{itemize}
\item itseisarvon määritelmä (paloittain määritelty muoto)
\item geometrinen tulkinta lukusuoralla
\item itseisarvoyhtälöistä tyyppiä $|f(x)|=a$, $|f(x)|=|g(x)|$,
$|f(x)|=g(x)$
\end{itemize}

KIITOS!}

Itseisarvo kuvaa lukusuoralla luvun etäisyyttä nollasta. Positiivisen luvun itseisarvo on luku itse ja negatiivisen luvun itseisarvo on luvun vastaluku. Nollan itseisarvo on nolla.


\laatikko{$|x|= \left\{ \begin{array}{rcl}
		x & , kun & x \geq 0 \\
		-x & , kun & x < 0
		\end{array}\right.$}

\begin{esimerkki}
Laske
	\begin{alakohdat}
		\alakohta{$|3-\pi|$}
		\alakohta{$|x-3|$}
	\end{alakohdat}
	\textbf{Ratkaisu}
	\begin{alakohdat}
		\alakohta{Koska $3-\pi\approx-0,14<0$, niin $|3-\pi|=-(3-\pi)=\pi-3$}
		\alakohta{Koska $x-3\geq 0$, kun $x\geq3$, niin\\
			$|x-3|= \left \{ \begin{array}{rcl}
			x-3 & , kun & x\geq3 \\
			3-x & , kun & x<3
			\end{array}\right.$
		}
	\end{alakohdat}
\end{esimerkki}


	\section{Itseisarvoepäyhtälöt}

\laatikko{
KIRJOITA TÄHÄN LUKUUN

\begin{itemize}
\item miten ratkaistaan epäyhtälöitä tyyliin
\item $|x|<a$,  $|x|>a$
\item $|f(x)|<a$, $|f(x)|<|g(x)|$, $|f(x)|<g(x)$
\end{itemize}

KIITOS!}

\begin{esimerkki}
Ratkaise epäyhtälö $|x|<3$.

\begin{esimratk}
On löydettävä kaikki sellaiset luvut, joiden itseisarvo on pienempi kuin $3$. Esimerkiksi luvut $2$ ja $-1$ käyvät, mutta luvut $4$ ja $-5,5$ eivät käy. Epäyhtälön ratkaisuja ovat täsmälleen ne luvut, joiden etäisyys nollasta on pienempi kuin $3$.

(Tähän kuva.)

Siten ratkaisu on $-3<x<3$.
\end{esimratk}

\begin{esimvast}
$-3<x<3$
\end{esimvast}


\end{esimerkki}

\begin{esimerkki}
Ratkaise epäyhtälö $|x|>5$.

\begin{esimratk}
On löydettävä kaikki sellaiset luvut, joiden itseisarvo on suurempi kuin $5$. Esimerkiksi luvut $2$ ja $-4,5$ eivät ole epäyhtälön ratkaisuja, mutta luvut $5,5$ ja $-6$ ovat. Epäyhtälön ratkaisuja ovat täsmälleen ne luvut, joiden etäisyys nollasta on suurempi kuin $5$.
 
(Tähän kuva.)

Nyt ratkaisu on ilmoitettava kahdessa osassa: $x<-5$ tai $x>5$.
\end{esimratk}

\begin{esimvast}
$x<-5$ tai $x>5$
\end{esimvast}
\end{esimerkki}

\begin{esimerkki}
Ratkaise epäyhtälö $|x+4|<2$.

\begin{esimratk}
Nyt luvun $x+4$ itseisarvo on pienempi kuin kaksi, joten tiedetään, että $-2<x+4<2$. Nyt saadaan ratkaistavaksi epäyhtälöt $-2<x+4$ ja $x+4<2$. Ratkaistaan nämä kaksi yhtälöä erikseen:

\begin{align*}
-2&<x+4 & \ppalkki{+2} \\
0&<x+6 & \ppalkki{-6} \\
-6&<x &
\end{align*}

\begin{align*}
x+4&<2 & \ppalkki{-4} \\
x&<-2 & \ppalkki{-6}
\end{align*}

Kun vastaukset yhdistetään saadaan ratkaisuksi $-6<x<-2$.

(Tähän kuva.)

\end{esimratk}

\begin{esimvast}
$-6<x<-2$.
\end{esimvast}
\end{esimerkki}


\subsection*{Tehtäviä}

\begin{tehtavasivu}

\subsubsection*{Opi perusteet}

\begin{tehtava}
Ratkaise seuraavat epäyhtälöt.
	\begin{alakohdat}
		\alakohta{$|x|<6$}
		\alakohta{$|x|>10$}
		\alakohta{$|x|<1,6$}
	\end{alakohdat}
	\begin{vastaus}
		\begin{alakohdat}
			\alakohta{$-6<x<6$}
			\alakohta{$x<-10$ tai $x>10$}
			\alakohta{$-1,6<x<1,6$}
		\end{alakohdat}
	\end{vastaus}
\end{tehtava}

\begin{tehtava}
Ratkaise seuraavat epäyhtälöt.
	\begin{alakohdat}
		\alakohta{$|x+6|>3$}
		\alakohta{$|x-5|<2$}
	\end{alakohdat}
	\begin{vastaus}
		\begin{alakohdat}
			\alakohta{$x<-9$ tai $x>-3$}
			\alakohta{$3<x<7$}
		\end{alakohdat}
	\end{vastaus}
\end{tehtava}

\subsubsection*{Hallitse kokonaisuus}

\subsubsection*{Sekalaisia tehtäviä}

TÄHÄN TEHTÄVIÄ SIJOITTAMISTA ODOTTAMAAN

\begin{tehtava}
Ratkaise seuraavat epäyhtälöt.
	\begin{alakohdat}
		\alakohta{$|x|\le 6$}
		\alakohta{$|x|>-3$}
	\end{alakohdat}
	\begin{vastaus}
		\begin{alakohdat}
			\alakohta{$-6 \le x \le 6$}
			\alakohta{ei ratkaisua}
		\end{alakohdat}
	\end{vastaus}
\end{tehtava}


\begin{tehtava}
Ratkaise epäyhtälö $|x^2+1| \ge 3$.
	\begin{vastaus}
          $x<\sqrt{-2}$ tai $x>\sqrt{2}$
	\end{vastaus}
\end{tehtava}

\end{tehtavasivu}

	% itseisarvoepäyhtälöt
	\section{Yhtälöryhmät}

\laatikko{
KIRJOITA TÄHÄN LUKUUN

\begin{itemize}
\item mikä yhtälöryhmä on
\item miten ratkaistaan yhtälöpari (sijoitus, yhteenlaskumenetelmä)
\item että ratkaisuja voi olla yksi, nolla tai äärettömän monta
\item miten useamman tuntemattoman yhtälöryhmä ratkaistaan
\end{itemize}

KIITOS!}
	% sijoitusmenetelmä
	% yhtälöiden laskeminen yhteen
	% ratkaisujen määrä
	\section{Koordinaatisto ja yhtälön kuvaaja}

\laatikko{
KIRJOITA TÄHÄN LUKUUN

\begin{itemize}
\item ihan lyhyt koordinaatistokertaus
\item kahden pisteen välinen etäisyys (pysty-tai vaakasuoraan helpolla, Pythagoraan lauseella yleensä)
\item esimerkkejä käyrän yhtälöistä, esim. suora, paraabeli, kartesiuksen lehti
\end{itemize}

KIITOS!}
	% yleistä käyristä, esim. Kartesiuksen lehdestä jotain
	% kahden pisteen välinen etäisyys (Pythagoraan lauseella)

\chapter{Suorat}
	\section{Suoran yhtälö}

\laatikko{
KIRJOITA TÄHÄN LUKUUN

\begin{itemize}
\item suoran yhtälö muodossa $y=kx+b$
\item kulmakertoimen ja vakiotermin merkitys
\item pysty- ja vaakasuoran suoran yhtälö
\item suorien leikkauspisteen/suoran ja $x$-akselin leikkauspisteen ratkaiseminen
\end{itemize}

KIITOS!}

Alla on kolme kuvaa, joihin on seuraavia yhtälöitä vastaavat kuvaajat:
\begin{align*}
y & =x-1 \\
y & =2x \\
\text{ja} \quad y & =-x+2
\end{align*}
Kuten huomataan, kaikki kolme kuvaajaa ovat suoria.

\begin{kuva}
    kuvaaja.pohja(-3.5, 3.5, -3.5, 3.5, korkeus = 4, nimiX = "$x$", nimiY = "$y$", ruudukko = True)
    kuvaaja.piirra("x-1", nimi = "$y=x-1$")
    kuvaaja.piirra("2*x", nimi = "$y=2x$", suunta = 45)
    kuvaaja.piirra("-x+2", nimi = "$y=-x+2$", kohta = -0.6, suunta = -135)
\end{kuva}

\laatikko{
Yhtälön
\[
y=kx+b
\]
määräämä kuvaaja on suora. Lukua $k$ nimitetään suoran \termi{kulmakerroin}{kulmakertoimeksi} ja lukua $b$ \termi{vakiotermi}{vakiotermiksi}.
}


%%%%%%FIX ME Pitäisikö tässä olla kolme pistettä, kun käytännössä aika usein kolmas olisi hyvä, koska huolimattomuusvirhe

\begin{esimerkki} Piirretään koordinaatistoon yhtälön $y=2x-3$ kuvaaja. Koska kyseessä on suoran yhtälö, riittää löytää kaksi pistettä, joiden kautta suora kulkee.
Valitaan esimerkiksi pisteet, joiden $x$-koordinaatit ovat $0$ ja $2$. Ensimmäisen $y$-koordinaatti on
\[
y=2\cdot 0-3=-3
\]
ja toisen
\[
y=2\cdot 2-3=4-3=1.
\]
Suoran pisteet ovat siis $(0, -3)$ ja $(2, 1)$. Piirretään nämä koordinaatistoon ja vedetään niiden kautta suora.

\begin{kuva}
    kuvaaja.pohja(-1.5, 3.5, -3.5, 3.5, korkeus = 4, nimiX = "$x$", nimiY = "$y$", ruudukko = True)
    kuvaaja.piirra("2*x-3", nimi = "$y=2x-3$")
    piste((0, -3))
    piste((2, 1))
\end{kuva}

\end{esimerkki}

\subsubsection*{Kulmakertoimen tulkinta}

Suoran kulmakerroin kertoo, miten jyrkästi suora nousee tai laskee. Tarkastellaan alla olevaa suoraa, jonka yhtälö on $y=2x$.

\begin{kuva}
    kuvaaja.pohja(-1, 3, -1, 5, korkeus = 4, nimiX = "$x$", nimiY = "$y$", ruudukko = True)
    kuvaaja.piirra("2*x", nimi = "$y=2x$")
    piste((1, 2))
    piste((2, 4))
\end{kuva}


Valitaan suoralta kaksi pistettä,
$A=(1, 2)$ ja $B=(2, 4)$. Siirryttäessä pisteestä $A$ pisteeseen $B$ $x$-koordinaatin arvo kasvaa yhdellä ja $y$-koordinaatin arvo kahdella. Saadaan suhde
\[
\frac{\text{$y$-koordinaatin muutos}}{\text{$x$-koordinaatin muutos}}=\frac{2}{1}=2.
\]
Jos nyt valitaan suoralta jotkin toiset pisteet, esimerkiksi $D=(-1, -2)$ ja $E=(5, 10)$, voidaan laskea samalla tavalla
\[
\frac{\text{$y$-koordinaatin muutos}}{\text{$x$-koordinaatin muutos}}=\frac{10-(-2)}{5-(-1)}=\frac{12}{6}=2.
\]
Huomataan, että yllä laskettu suhde on aina sama pisteistä riippumatta. Tämä johtuu siitä, että kuvan kolmiot $ABC$ ja $DEF$ ovat yhdenmuotoisia.
Suhde on lisäksi sama kuin suoran yhtälössä esiintyvä kulmakerroin.

\begin{esimerkki} Määritä alla olevien suorien kulmakertoimet.

%%% pitäisikö pisteet olla piirrettyinä kuvaan?

\begin{kuva}
    kuvaaja.pohja(-2.5, 3, -1, 5, korkeus = 4, nimiX = "$x$", nimiY = "$y$", ruudukko = True)
    kuvaaja.piirra("3*x+1")
    kuvaaja.piirra("-0.5*x+3")
    piste((0, 1))
    piste((1, 4))
    piste((-2, 4))
    piste((2, 2))
\end{kuva}

\begin{esimratk} Valitaan suorilta kaksi mielivaltaista pistettä ja lasketaan $y$-koordinaatin muutoksen suhde $x$-koordinaatin muutokseen.
Ensimmäiseltä suoralta valitaan vaikkapa pisteet $(0, 2)$ ja $(1, 4)$. Nyt kulmakertoimeksi tulee
\[
\frac{\text{$y$-koordinaatin muutos}}{\text{$x$-koordinaatin muutos}}=\frac{4-1}{1-0}=\frac{3}{1}=3.
\]
Toiselta suoralta valitaan pisteet $(-2, 4)$ ja $(2, 2)$. Nyt täytyy olla tarkkana etumerkkien kanssa:
\[
\frac{\text{$y$-koordinaatin muutos}}{\text{$x$-koordinaatin muutos}}=\frac{2-4}{2-(-2)}=\frac{-2}{4}=-\frac{1}{2}.
\]
jotain
\end{esimratk}
\begin{esimvast}
Ensimmäisen suoran kulmakerroin on $3$ ja toisen $-\frac{1}{2}$.
\end{esimvast}
\end{esimerkki}

Kulmakerroin kertoo suoran suunnasta: mitä suurempi kulmakerroin, sitä jyrkemmin suora nousee koordinaatistossa oikealle päin.
Jos kulmakerroin on negatiivinen, suora on laskeva. Vaakasuoran suoran kulmakerroin on 0.

\subsubsection*{Vakiotermin tulkinta}

Yllä nähtiin, että suora $y=2x$ kulkee origon kautta. Seuraavassa kuvassa on suoran $y=2x+1$ kuvaaja. Se saadaan nostamalla suoraa $y=2x$ yhden yksikön verran ylöspäin.

\begin{kuva}
    kuvaaja.pohja(-1, 3, -1, 5, korkeus = 4, nimiX = "$x$", nimiY = "$y$", ruudukko = True)
    kuvaaja.piirra("2*x+1", nimi = "$y=2x+1$", suunta = 45)
    vari("lightgray")
    kuvaaja.piirra("2*x")
\end{kuva}

Tarkastellaan suoralla $y=kx+b$ olevaa pistettä, jonka $x$-koordinaatti on 0.
Tämä piste sijaitsee $y$-akselilla. Toisin sanoen se on suoran ja $y$-akselin leikkauspisteessä.
Sen $y$-koordinaatti saadaan laskemalla
\[
y=k\cdot 0+b=b.
\]
Pisteen $y$-koordinaatti on siis $b$, eli suoran yhtälön vakiotermi. Vakiotermi siis ilmaisee, missä kohtaa suora leikkaa $y$-akselin.
Alla on esimerkkejä erilaisista vakiotermeistä.

[TÄHÄN KUVIA ERILAISISTA VAKIOTERMEISTÄ]

\subsubsection*{Suoran yhtälön määrittäminen}

Suoran yhtälö voidaan määrittää kuvasta laskemalla suoran kulmakerroin sekä vakiotermi.

\begin{esimerkki} Mikä on alla olevan kuvan suoran yhtälö?

\begin{kuva}
    kuvaaja.pohja(-1, 4.5, -0.5, 3.5, korkeus = 4, nimiX = "$x$", nimiY = "$y$", ruudukko = True)
    kuvaaja.piirra("0.5*x+1")
    piste((0, 1), "(0, 1)", 135)
    piste((4, 3), "(4, 3)", 135)
\end{kuva}


\begin{esimratk}
Aloitetaan määrittämällä kulmakerroin. Valitaan suoralta pisteet $(0, 1)$ ja $(4, 3)$. Kulmakertoimeksi tulee
\[
k=\frac{3-1}{4-0}=\frac{2}{4}=\frac{1}{2}\text{.}
\]
\end{esimratk}
Vakiotermi saadaan kohdasta, jossa suora leikkaa $y$-akselin. Tuossa kohdassa $y$-koordinaatti on 1. Vakiotermi on siis $b=1$.
\begin{esimvast}
Suoran yhtälö on $y=\frac{1}{2}x+1$.
\end{esimvast}
\end{esimerkki}

Kun kulmakerroin on 0, suora on vaakasuora. Sen yhtälö on siis muotoa
\laatikko[vaakasuora suora]{
\[
y=b.
\]
}
Toisaalta pystysuoralla suoralla ei ole kulmakerrointa lainkaan. Sen yhtälöä ei voi ilmaista muodossa $y=\dots$, vaan sillä on yhtälö
\laatikko[pystysuora suora]{
\[
x=a.
\]
}
Tässä $a$ on sen pisteen $x$-koordinaatti, jossa suora leikkaa $x$-akselin.

\begin{tehtavasivu}

\subsubsection*{Opi perusteet}

\begin{tehtava}
Ratkaise suoran $y=3x+1$ nollakohta.
\begin{vastaus}
$x=-\frac{1}{3}$
\end{vastaus}
\end{tehtava}

\begin{tehtava}
Ratkaise suorien $y=-5x+3$ ja $y=2x-17$ leikkauspiste.
\begin{vastaus}
% http://www.wolframalpha.com/input/?i=y%3D-5x%2B3%2C+y%3D2x-17
$(\frac{20}{7}, -\frac{79}{70})$
\end{vastaus}
\end{tehtava}

\begin{tehtava}
Mikä on $x$-akselin suuntaisen suoran, joka kulkee pisteen $(1, 3)$ kautta, yhtälö?
\begin{vastaus}
$y=3$
\end{vastaus}
\end{tehtava}

\begin{tehtava}
Mikä on suoran $y=3,14x-10$ yhtälön
\begin{enumerate}[a)]
\item vakiotermi,
\item kulmakerroin?
\end{enumerate}
\begin{vastaus}
a)$-10$ b) $3,14$
\end{vastaus}
\end{tehtava}

\begin{tehtava}
Suora kulkee pisteiden $(2, 1)$ ja $(5, 9)$ kautta. Määritä kulmakerroin.
\begin{vastaus}
Kulmakerroin on $\frac{8}{3}$
\end{vastaus}
\end{tehtava}

\begin{tehtava}
Piirrä suora $y=9x-1$.
\begin{vastaus}
puuttuu
\end{vastaus}
\end{tehtava}

\subsubsection*{Hallitse kokonaisuus}

\begin{tehtava}
Ratkaise suorien $y=-x+2$ ja $y=2x-4$ leikkauspiste.
\begin{vastaus}
$(2, 0)$
\end{vastaus}
\end{tehtava}

\begin{tehtava}
Määritä
\begin{enumerate}[a)]
\item $x$-akselin suuntaisen suoran,
\item $y$-akselin suuntaisen suoran kulmakerroin?
\end{enumerate}
\begin{vastaus}
a) $0$ b) ei määritelty %(ääretön)
\end{vastaus}
\end{tehtava}

\begin{tehtava}
Piirrä suora $y=-2x+3$.
\begin{vastaus}
puuttuu
\end{vastaus}
\end{tehtava}

\begin{tehtava}
Ratkaise $\frac{y}{2}=\frac{x}{2}+2$ nollakohta.
\begin{vastaus}
$(-4, 0)$
\end{vastaus}
\end{tehtava}

\begin{tehtava}
Ratkaise suoran $6=-60x+600y$ nollakohta.
\begin{vastaus}
$x=-\frac{1}{10}$
\end{vastaus}
\end{tehtava}

\begin{tehtava}
Missä pistessä suora $y=\frac{16x}{25}+\frac{36}{49}$
\begin{enumerate}[a)]
\item leikkaa $x$-akselin,
\item leikkaa $y$-akselin?
\end{enumerate}
\begin{vastaus}
a)$(-\frac{225}{196}, 0)$ b) $(0, \frac{36}{49})$
\end{vastaus}
\end{tehtava}

\begin{tehtava}
Ratkaise suorien $y=-\frac{2}{5}$ ja $3y=18x+20$ leikkauspiste.
\begin{vastaus}
$(-\frac{53}{45}, -\frac{2}{5})$
\end{vastaus}
\end{tehtava}

\begin{tehtava}
Ratkaise suoran $16y-9x=-5y-11x+27$ nollakohta.
\begin{vastaus}
$(11, 0)$
\end{vastaus}
\end{tehtava}

\subsubsection*{Sekalaisia tehtäviä}

LAITA TEHTÄVÄT TÄHÄN, JOS ET OLE VARMA VAIKEUSASTEESTA TAI TEHTÄVÄ
EI TÄLLÄ HETKELLÄ SOVI MUKAAN

\end{tehtavasivu}

	% ratkaistu muoto y = kx + b, kulmakerroin ja vakiotermi
	% nollakohdat ja leikkauspisteet
	% vaaka- ja pystysuorat
	\section{Suoran yhtälön muut muodot}

\laatikko{
KIRJOITA TÄHÄN LUKUUN

\begin{itemize}
\item suoran yhtälö normaalimuodossa $ax+by+c=0$
\item suoran yhtälö muodossa $y-y_0=k(x-x_0)$, suoran yhtälön muodostaminen pisteiden avulla
\end{itemize}

KIITOS!}

\subsection*{Valeorigomuoto} % tämä ei ole vakiintunut termi!!!

Toisinaan suoran yhtälöä on helpompaa tarkastella muodossa $y-y_0=k(x-x_0)$.
Tämän voidaan ajatella olevan origon kautta kulkeva suora, jos origo olisi
pisteessä $(x_0, y_0)$. Valeorigomuoto on kätevin silloin, kun tiedämme suoran
kulmakertoimen ja yhden pisteen, jonka kautta suora kulkee.

\begin{esimerkki}
    Esimerkkejä valeorigomuodon käytöstä:
    \begin{enumerate}[a)]
        \item Suoran kulmakerroin on $5$ ja suora kulkee pisteen $(3,0)$ kautta.
        \[y-y_0 = k(x-x_0) \ekvi y-0 = 5(x-3) \ekvi y = 5x-15\]
        \item Suoran kulmakerroin on $4$ ja suora kulkee pisteen $(5,7)$ kautta.
        \[y-y_0 = k(x-x_0) \ekvi y-7 = 4(x-5) \ekvi y-7 = 4x-20 \ekvi y = 4x-13\]
    \end{enumerate}
\end{esimerkki}

\subsection*{Normaalimuoto}

Suoran yhtälön kanonisin muoto on normaalimuoto tai yleinen muoto $ax+by+c=0$.

\begin{tehtavasivu}

\subsubsection*{Opi perusteet}

\begin{tehtava}
Mikä on suoran yhtälö normaalimuodossa?
\begin{enumerate}[a)]
\item $y=-15x+2$
\item $2y=11x+7$
\item $2y+5x-8=13y-6x-8$
\end{enumerate}
\begin{vastaus}
a)$15x+y-2=0$ b) $11x-2y+7=0$ c) $-11x+11y=0$
\end{vastaus}
\end{tehtava}


\begin{tehtava}
Suoran kulmakerroin on $\frac{1}{2}$ ja suora kulkee pisteen
\begin{enumerate}[a)]
\item $(-12,4)$
\item $(3,9)$. Mikä on suoran yhtälö?
\end{enumerate}
\begin{vastaus}
a)$y=\frac{1}{2}x+10$ b) $y=\frac{1}{2}x+\frac{15}{2}$
\end{vastaus}
\end{tehtava}

\begin{tehtava}
Mikä on pisteiden
\begin{enumerate}[a)]
\item $(1,-2)$ ja $(3,1)$
\item $(0,0)$ ja $(-4,4)$ kautta kulkevan suoran yhtälö?
\end{enumerate}
\begin{vastaus}
a)$y=\frac{3}{2}x-\frac{7}{2}$ b) $y=-x$
\end{vastaus}
\end{tehtava}

\subsubsection*{Hallitse kokonaisuus}

\begin{tehtava}
Tutki ovatko pisteet  
\begin{enumerate}[a)]
\item $(1,-5)$, $(4,-23)$ja $(4,-239)$
\item $(7,3)$, $(-2,10)$ ja $(-3,90)$ samalla suoralla?
\end{enumerate}
\begin{vastaus}
a) kyllä b) ei
\end{vastaus}
\end{tehtava}

\begin{tehtava}
Määritä luku $t$ niin, että pisteet $(-t+3,-4)$, $(6,t-5)$ ja $(5,-4)$ ovat samalla suoralla.
\begin{vastaus}
$t=-2$ tai $t=1$
\end{vastaus}
\end{tehtava}

\subsubsection*{Sekalaisia tehtäviä}

TÄHÄN TEHTÄVIÄ SIJOITTAMISTA ODOTTAMAAN

\begin {tehtava}
Suora kulkee pisteiden $(3,4)$ ja $(\sqrt{3},1)$ kautta. Määritä suoran kulmakerroin.
\begin {vastaus}
$\frac{\sqrt{3}-1}{\sqrt{3}}$
\end {vastaus}
\end {tehtava}

\end{tehtavasivu}

	% esitys y-y_0=k(x-x_0)
	% esitys ax + by + c = 0 (normaalimuoto)
	\section{Suorien keskinäinen asema}

\laatikko{
KIRJOITA TÄHÄN LUKUUN

\begin{itemize}
\item sama kulmakerroin --> yhdensuuntaiset tai sama suora
\item eri kulmakerroin --> yksi leikkauspiste, kytkentä yhtälöpareihin
\item suoran ja normaalin kulmakertoimet, $k_1k_2=-1$.
\end{itemize}

KIITOS!}

Piirretään koordinaatistoon kaksi eri suoraa ja tutkitaan niiden leikkauspisteitä.
On kaksi eri vaihtoehtoa:
\begin{enumerate}
 \item Suorat eivät leikkaa.
 \item Suorat leikkaavat täsmälleen yhdessä pisteessä.
\end{enumerate}

Ensimmäisessä tapauksessa suorilla on sama kulmakerroin. Sanotaan, että ne ovat yhdensuuntaiset.

\begin{kuva}
    kuvaaja.pohja(-3, 3, -3, 3, nimiX = "$x$", nimiY = "$y$")
    kuvaaja.piirra("2*x+1", nimi = "$y=2x+1$")
    kuvaaja.piirra("2*x-3", nimi = "$y=2x-3$")
\end{kuva}

Jos suorat ovat pystysuoria, ei niiden kulmakerrointa ole määritelty. Myös tällöin suorat ovat yhdensuuntaisia.

\begin{kuva}
    kuvaaja.pohja(-3, 3, -2, 2, nimiX = "$x$", nimiY = "$y$")
    kuvaaja.piirraParametri("1", "t", -2, 2, nimi = "$x=-1$")
    kuvaaja.piirraParametri("-2", "t", -2, 2, nimi = "$x=2$")
\end{kuva}

\laatikko{
Jos kahdella suoralla on sama kulmakerroin tai molemmat suorat ovat pystysuoria, ovat suorat yhdensuuntaiset.
}

Jos suorilla on eri kulmakerroin, on niillä täsmälleen yksi leikkauspiste. 

\begin{kuva}
    kuvaaja.pohja(-3, 3, -3, 3, nimiX = "$x$", nimiY = "$y$")
    kuvaaja.piirra("2*x+1", nimi = "$y=2x+1$")
    kuvaaja.piirra("-3*x+4", nimi = "$y=-3x+4$")
\end{kuva}

Kulmakertoimien perusteella voidaan myös päätellä, ovatko suorat toisiaan vastaan kohtisuorassa.

\begin{kuva}
    kuvaaja.pohja(-2, 4, -2, 4, nimiX = "$x$", nimiY = "$y$")
    kuvaaja.piirra("2*x+1", nimi = "$y=2x+1$")
    kuvaaja.piirra("-0.5*x+2", nimi = "$y=-(1/2)x+2$")
\end{kuva}

(Tähän kuvaan voisi hahmotella kolmiot, joiden avulla suorien kulmakertoimet nähdään kuvasta.)

\laatikko{
Kaksi suoraa ovat toisiaan vastaan kohtisuorassa täsmälleen silloin, jos niiden kulmakertoimien tulo on $-1$
}


\begin{tehtavasivu}

\subsubsection*{Opi perusteet}

\begin {tehtava}
Mikä on suoran $y=-\frac{1}{4}x+\pi$
\begin{enumerate} [a)]
\item kulmakerroin
\item normaalin kulmakerroin?
\end{enumerate}
\begin {vastaus}
a)$-\frac{1}{4}$  b) $4$ 
\end {vastaus}
\end {tehtava}

\begin {tehtava}
Ovatko suorat 
\begin{enumerate} [a)]
\item $y=3x+16$ ja $y=-3x-4$
\item $6y=7x+3$ ja $x=10y-9$ yhdensuuntaiset?
\end{enumerate}
\begin {vastaus}
a) ei b) ei
\end {vastaus}
\end {tehtava}

\begin {tehtava}
Muodosta suoralle $y=-9x+13$ normaali pisteen $(2, 5)$ kautta.
\begin {vastaus}
$y=\frac{x}{9}+\frac{43}{9}$
\end {vastaus}
\end {tehtava}

\subsubsection*{Hallitse kokonaisuus}

\begin {tehtava}
Laske suorien leikkauspiste.
\begin{enumerate} [a)]
\item $y=4x+\sqrt{2}$ ja $y=-8x+64$
\item $x=\frac{y}{4}-8$ ja $2x=-y+6x+7$
\end{enumerate}
\begin {vastaus}
a) $\frac{64-\sqrt{2}}{12}$ b) ovat yhdensuuntaiset
\end {vastaus}
\end {tehtava}

\begin {tehtava}
Minna asuu $(x, y)$-koordinaatiston pisteessä $(-2, 1)$. Hän kävelee matkan $4\sqrt{2}$ pitkin suoraa $y=x+3$ ja tulee risteykseen, jossa kääntyy $90$ astetta vasemmalle. Minna kävelee vielä matkan $8\sqrt{2}$ suoraan ja saapuu Lassin ovelle. Missä Lassi asuu ja mikä on hänen asuintiensä yhtälö?
\begin {vastaus}
Lassi asuu joko pisteessä $(-6, 11)$, yhtälö tällöin $y=-x+3$ tai pisteessä $(2, -13)$, yhtälö $y=-x-11$. Vastaus riippuu siitä, kumpaan suuntaan Minna lähti alussa kävelemään.
\end {vastaus}
\end {tehtava}

\begin {tehtava}
Olkoon $A=(a+3, 2)$ ja $B=(a^2, -a)$. Määritä vakio $a$ niin, että pisteiden $A$ ja $B$ kautta kulkeva suora on
\begin{enumerate} [a)]
\item vaakasuora
\item pystysuora?
\end{enumerate}
\begin {vastaus}
a) $a=-2$ b) $a=\frac{1\pm\sqrt{13}}{2} $
\end {vastaus}
\end {tehtava}

\subsubsection*{Sekalaisia tehtäviä}

TÄHÄN TEHTÄVIÄ SIJOITTAMISTA ODOTTAMAAN

\end{tehtavasivu}


	% suorien keskinäinen asema, yhdensuuntaiset suorat
	% suoralle ja sen normaalille k_1 * k_2 = -1
	\section{Pisteen etäisyys suorasta}

\laatikko{
KIRJOITA TÄHÄN LUKUUN

\begin{itemize}
\item pisteen etäisyyden suorasta laskeminen yhdenmuotoisilla
kolmioilla
\item pisteen etäisyys suorasta -kaava: $d=\frac{|ax_0+by_0+c|}{\sqrt{a^2+b^2}}$
\item sovelluksia
\item kaavan todistuksen voi laittaa tähän osioon tai liitteeksi,
käytetään yhdenmuotoisia kolmioita
\end{itemize}

KIITOS!}

Pisteen etäisyydellä suorasta tarkoitetaan pisteen ja mielivaltaisen suoran pisteen pienintä mahdollista etäisyyttä.
Jos tunnetaan jokin vaaka- tai pystysuora suora ja jokin koordinaatiston piste, kyseisen pisteen etäisyys annetusta suorasta on helppo määrittää.

\begin{minipage}{0.45\textwidth}
\begin{kuva}
    kuvaaja.pohja(-2, 3, -1, 5, korkeus = 4, nimiX = "$x$", nimiY = "$y$", ruudukko = True)
    with palautin():
        vari("lightgray")
        kuvaaja.piirraParametri("1.7", "t", a = 1, b = 3.3)
    kuvaaja.piirra("1", nimi = "$y=1$", suunta = 0)
    kuvaaja.piste((1.7, 3.3), "$(1,7; 3,3)$", 95)
\end{kuva}

Etäisyys on $3,3-1=2,3$.
\end{minipage}
\begin{minipage}{0.45\textwidth}
\begin{kuva}
    kuvaaja.pohja(-2, 3, -1, 5, korkeus = 4, nimiX = "$x$", nimiY = "$y$", ruudukko = True)
    with palautin():
        vari("lightgray")
        kuvaaja.piirraParametri("t", "3.3", a = -1, b = 1.7)
    kuvaaja.piirraParametri("-1", "t", a = -1, b = 5, nimi = "$x = -1$", suunta = 90)
    kuvaaja.piste((1.7, 3.3), "$(1,7; 3,3)$", 90)
\end{kuva}

Etäisyys on $1,7-(-1)=2,7$.
\end{minipage}

Jos suora on kalteva, etäisyyden määrittäminen ei ole näin suoraviivaisesti.
Seuraavaksi tutustutaan kahteen tapaan tämän pulman ratkaisemiseksi.

\subsection*{Pisteen etäisyys suorasta yhdenmuotoisten kolmioiden avulla}

Tarkastellaan esimerkkinä suoraa $l$, jonka normaalimuotoinen yhtälö on $3x-4y -12=0$.
Selvitetään pisteen $P=(8, 5)$ etäisyys suorasta $l$.

\begin{kuva}
    kuvaaja.pohja(-1, 10, -4, 5, korkeus = 4, nimiX = "$x$", nimiY = "$y$", ruudukko = True)
    with palautin():
        vari("lightgray")
        kuvaaja.piirraParametri("8+0.75*t", "5-t", a = 0, b = 1.28)
    kuvaaja.piirra(".75*x-3", nimi = "$l$", kohta = 2, suunta = -45)    
    kuvaaja.piste((8,5), "$P$", -135)
\end{kuva}

%    kuvaaja.piirra(".75*x-3", nimi = "$3x -4y- 12=0$", kohta = 2, suunta = -45)    


Kuvaan on merkitty suorakulmaiset kolmiot $OAB$ ja $PQR$.
Etäisyys, jonka haluamme selvittää, on kolmion $PQR$ sivun $r$ pituus.
Tehtävä ratkeaa, kun huomataan, että kolmiot $OAB$ ja $PQR$ ovat yhdenmuotoisia.
Tämä johtuu siitä, että molemmat ovat suorakulmaisia ja lisäksi kulmat $OAB$ ja $PRQ$ ovat samankokoiset (ks. kuva alla).

[EI OLE VALMIS, KUVIA JA TEKSTISELITYSTÄ VOISI VIELÄ MIETTIÄ]

%    kuvaaja.pohja(-1, 10, -4, 5)
\begin{kuva}
    kuvaaja.piirra(".75*x-3", a = -1, b = 10)
    kuvaaja.piirraParametri("0", "t", a = -3, b = 0)
    kuvaaja.piirraParametri("t", "0", a = 0, b = 4)
    kuvaaja.piirraParametri("8", "t", a = 3, b = 5)
    kuvaaja.piirraParametri("8+0.75*t", "5-t", a = 0, b = 1.28)
    kuvaaja.piste((8,5), "$P$", 180)
    kuvaaja.piste((4,0), "$A$", -45)
    kuvaaja.piste((0,-3), "$B$", -45)
    kuvaaja.piste((0,0), "$O$", 135)
    kuvaaja.piste((8.96, 3.72), "$Q$", -45)
    kuvaaja.piste((8,3), "$R$", -45)
\end{kuva}

Kolmion $OAB$ sivut selviävät, kun ratkaistaan, missä pisteissä suora leikkaa $x$- ja $y$-akselit.
Asettamalla suoran yhtälössä $x=0$ suoran yhtälössä
\[
3x-4\cdot 0=12, \quad \text{josta} \quad x=\frac{12}{3}=4.
\]
Pisteen $A$ koordinaatit ovat siis $(4, 0)$. Toisaalta kun $x=0$, saadaan
\[
3\cdot 0-4\cdot y=12, \quad \text{josta} \quad y=-\frac{12}{4}=-3.
\]
Pisteen $B$ koordinaatit ovat siis $(0, 3)$. Nyt tunnetaan sivut $a=4$ ja $b=3$, ja lisäksi Pythagoraan lauseen perusteella
\[
c=\sqrt{a^2+b^2}=\sqrt{4^2+3^2}=\sqrt{25}=5.
\]

Koska kolmiot $OAB$ ja $PQR$ ovat yhdenmuotoiset, saadaan verranto
\[
\frac{r}{q}=\frac{a}{c}.
\]
Tunnemme jo sivut $a$ ja $c$, joten enää on selvitettävä sivu $q$. Tämä on sama kuin pisteiden $P$ ja $R$ välinen etäisyys.

Pisteen $R$ $x$-koordinaatti on sama kuin pisteen $P$, eli 8. Koska $R$ on suoralla $l$, sen $y$-koordinaatti saadaan suoran yhtälöstä:
\begin{align*}
3\cdot 8-4y & =12 \\
-4y & =12-3\cdot 8 \\
-4y & =-12 \\
y & =3. \\
\end{align*}
Nyt siis $R=(8, 3)$. Pisteiden $P$ ja $R$ välinen etäisyys on siis $5-3=2$, ja tämä on sivun $q$ pituus.

Kun verrantoon $\dfrac{r}{q}=\dfrac{a}{c}$ sijoitetaan tunnetut sivujen pituudet, saadaan
\begin{align*}
\frac{r}{2} & =\frac{4}{5} \quad \ppalkki \cdot 2 \\[3pt]
r & =\frac{8}{5}.
\end{align*}
Siispä pisteen $P$ etäisyys suorasta $l$ on $\dfrac{8}{5}$.

\subsection*{Pisteen etäisyys suorasta kaavan avulla}

Edellä esitetystä tavasta laskea pisteen etäisyys suorasta voidaan johtaa myös kaava.
Jos suoran yhtälö on annettu normaalimuodossa $Ax+By+C=0$ ja pisteen koordinaatit ovat $(x_0, y_0)$, etäisyys $d$ saadaan seuraavasta kaavasta.
\laatikko[pisteen etäisyys suorasta]{
\[
d=\frac{|Ax_0+By_0+C|}{\sqrt{A^2+B^2}}
\]
}
Kaavan johtaminen esitetään liitteessä. (VAI TÄSSÄ?)

\begin{esimerkki} Lasketaan aiemman esimerkin pisteen $P=(8, 5)$ etäisyys suorasta $l$, jonka normaalimuotoinen yhtälö on $3x-4y-12=0$.
\begin{esimratk}
Käytetään kaavaa, jolloin $A=3$, $B=-4$ ja $C=12$, sekä $x_0=8$ ja $y_0=5$. Kaavan mukaan etäisyys on
\[
d=\frac{|Ax_0+By_0+C|}{\sqrt{A^2+B^2}}
=\frac{|3\cdot 8-4\cdot 5-12|}{\sqrt{3^2+(-4)^2}}
=\frac{|24-20-12|}{\sqrt{9+16}}=\frac{|-8|}{\sqrt{25}}
=\frac{8}{5}.
\]
\end{esimratk}
\begin{esimvast}
Etäisyys on $\dfrac{8}{5}$.
\end{esimvast}
\end{esimerkki}

\begin{esimerkki} Etsitään ne pisteet, joiden $x$-koordinaatti on 6 ja joiden etäisyys suorasta $-6x+8y+3=0$.
\begin{esimratk}
Piste, jonka $x$-koordinaatti on 6, on muotoa $(6, y)$. Sijoitetaan tämä etäisyyden kaavaan ja sievennetään.
Nyt $A=-6$, $B=8$, $C=3$, $x_0=6$ ja $y_0=y$.
\[
d=\frac{|-6\cdot 6+8y+3|}{\sqrt{(-6)^2+8^2}}
=\frac{|-36+8y+3|}{\sqrt{36+64}}
=\frac{|8y-33|}{\sqrt{100}}
=\frac{|8y-33|}{10}.
\]
Tehtävänannon mukaan etäisyyden pitäisi olla $d=6$. Tästä saadaan yhtälö
\[
\frac{|8y-33|}{10}=6 \quad \text{eli} \quad |8y-33|=60.
\]
Tämä itseisarvoyhtälö ratkeaa jakautumalla kahteen tapaukseen:
\begin{align*}
8y-33 & =60 & &\text{tai} & 8y-33 & =-60 \\
8y & =99 & & & 8y & =-27 \\
y & =\frac{99}{8} & & & y & =-\frac{27}{8}.
\end{align*}
\end{esimratk}
\begin{esimvast}
Pisteet ovat $\bigl(6, \frac{99}{8}\bigr)$ ja $\bigl(6, -\frac{27}{8}\bigr)$.
\end{esimvast}
\end{esimerkki}



\subsection*{Pisteen etäisyys suorasta, kaavan todistus}

Lasketaan pisteen $P = (x_0, y_0)$ etäisyys suorasta $l$: $ax+by+c=0$.

Olkoon $Q$ suoralla $l$ siten, että $l$ ja $PQ$ ovat kohtisuorassa. Huomataan, että jos piste $R$ on suoralla $l$, Pythagoraan lauseen mukaan
\[
PR^2 = PQ^2+QR^2 \geq PQ^2,
\]
jolloin myös $PR \geq PQ$. Siis $PQ$ on määritelmän nojalla pisteen $P$ etäisyys suorasta $l$.

Suorat $PQ$ ja $l$ ovat kohtisuorassa. Jos $l$ ei ole $x$-, eikä $y$-akselin suuntainen, eli $a,b \neq 0$ sen kulmakerroin on $-\frac{a}{b}$. Koska suoran ja normaalin kulmakerrointen tulo on $-1$, suoran $PQ$ kulmakerroin on
\[
-\frac{1}{\frac{-a}{b}} = \frac{b}{a}.
\]
Se kulkee lisäksi pisteen $(x_0,y_0)$ kautta, joten sen yhtälö on
\[
y-y_0 = \frac{b}{a}(x-x0)
\]
tai normaalimuodossa
\[
bx-ay+ay_0-bx_0 = 0.
\]
Toisaalta, jos $a = 0$, suora $PQ$ on muotoa $x+C$, jollain reaaliluvulla $C$; koska se lisäksi kulkee pisteen $P$ kautta, sen on oltava edellistä muotoa. Sama päättely voidaan toistaa kun $b = 0$, jolloin nähdään, että kaava normaalille pätee myös jos $a = 0$ tai $b = 0$.

Koska $Q$ kuuluu suoralle $l$ ja sen normaalille, sen koordinaattien on toteutettava yhtälöpari
\[
\left\{    
    \begin{array}{rcl}
        ax_q + by_q + c &=&0 \\
        bx_q-ay_q+ay_0-bx_0 &=& 0 \\
    \end{array}
    \right.
\]
Yhtälöpari voidaan ratkaista yhtäänlaskumenetelmällä kertomalla ylempi yhtälö $a$:lla ja alempi $b$:llä.
\[
\left\{    
    \begin{array}{rcl}
        a^2x_q + aby_q + ac &=&0 \\
        b^2x_q-aby_q+aby_0-b^2x_0 &=& 0 \\
    \end{array}
    \right.
\]
ja laskemalle yhtälöt yhtälöt yhteen
\begin{align*}
 a^2x_q + aby_q + ac + b^2x_q-aby_q+aby_0-b^2x_0 &= 0 \\
 (a^2+b^2)x_q &= -ac-aby_0+b^2x0 \\
 x_q = \frac{-ac-aby_0+b^2x0}{(a^2+b^2)}.
\end{align*}
Vastaavasti
\[
y_q = \frac{-bc+a^2y_0-abx_0}{(a^2+b^2)}.
\]
Mutta nythän
\begin{align*}
PQ &= \sqrt{(x_q-x_0)^2+(y_q-y0)^2} \\
&= \sqrt{\Big(\frac{-ac-aby_0+b^2x_0-(a^2+b^2)x_0}{a^2+b^2}\Big)^2+\Big(\frac{-bc+a^2y_0-abx_0-(a^2+b^2)y_0}{a^2+b^2}\Big)^2} \\
& = \frac{\sqrt{(-ac-aby_0-a^2x_0)^2+(-bc-abx_0-b^2)y_0)^2}}{a^2+b^2} \\
& = \frac{\sqrt{(a(-c-by_0-ax_0))^2+(b(-c-ax_0-by_0))^2}}{a^2+b^2} \\
& = \frac{\sqrt{(a^2+b^2)((-c-by_0-ax_0))^2}}{a^2+b^2} \\
& = \frac{\sqrt{(a^2+b^2)}}{a^2+b^2}\sqrt{(-c-by_0-ax_0))^2} \\
& = \frac{|ax_0+by_0+c|}{\sqrt{a^2+b^2}}.
\end{align*}
\begin{tehtavasivu}

\subsubsection*{Opi perusteet}

\subsubsection*{Hallitse kokonaisuus}

\subsubsection*{Sekalaisia tehtäviä}

TÄHÄN TEHTÄVIÄ SIJOITTAMISTA ODOTTAMAAN

\end{tehtavasivu}
	% kaava pisteen etäisyydelle suorasta

\chapter{Toisen asteen käyrät}
	\section{Ympyrä}

\laatikko{
KIRJOITA TÄHÄN LUKUUN

\begin{itemize}
\item ympyrän määritelmä ja siitä seuraava yhtälö,
origokeskinen ensin
\item muodon $x^2 + y^2 +ax +by +c=0$ täydentäminen neliöksi
ja ympyrän keskipisteen ja säteen selvittäminen siitä
\end{itemize}

KIITOS!}

\textbf{Ympyrä} on klassisesti kaikkien kiinteästä pisteestä, \textbf{keskipisteestä}, vakioetäisyydellä olevien pisteiden joukko. Tätä vakioetäisyyttä kutsutaan ympyrän \textbf{säteeksi}. Jos ympyrän keskipiste on $(x_{0},y_{0})$, ja säde $r$, voidaan ympyrän pisteille muodostaa Pythagoraan lauseen avulla yhtälö.

Määritelmän nojalla mielivaltainen tason piste $(x,y)$ on ympyrällä, jos ja vain jos pisteiden $(x,y)$ ja $(x_{0},y_{0})$ etäisyys on $r$. Pythagoraan lauseella siis

\[
\sqrt{(x-x_{0})^{2}+(y-y_{0})^{2}} = r.
\]

Kun nyt vielä huomioidaan, että $r \geq 0$, voidaan yhtälö yhtä hyvin kirjottaa muodossa

\[
(x-x_{0})^{2}+(y-y_{0})^{2} = r^{2}.
\]
Tällaista lauseketta kutsutaan yleensä ns. \textbf{ympyrän yhtälöksi}. Tapauksessa $r=0$ yhtälön toteuttaa vain pisteestä $(x_{0},y_{0})$ etäisyydellä 0 olevat pisteet, eli piste itse, joten se kuvaa itse keskipistettä. Myös tämä voidaan ajatella surkastuneena, nollasäteisenä ympyränä.

Jos erityisesti $(x_{0},y_{0})= (0,0)$, eli ympyrän keskipiste on origo, saa yhtälö muodon

\[
x^{2}+y^{2} = r^{2}.
\]

\begin{esimerkki}
Ympyrän $\Gamma_{1}$ \footnote{sdf} keskipiste on $(3,-4)$ ja säde $\sqrt{2}$, ja ympyrän $\Gamma_{2}$ keskipiste origo ja säde 1. Määritä ympyröiden yhtälöt. Hahmottele ympyrät koordinaatistoon.

\begin{esimratk}
Edellisen mukaan ympyrän $\Gamma_{1}$ yhtälö on
\[
(x-3)^{2}+(y-(-4))^{2} = (\sqrt{2})^{2}
\]
eli sievennettynä
\[
(x-3)^{2}+(y+4)^{2} = 2.
\]
$\Gamma_{2}$:n yhtälö saadaan vastaavasti:
\[
x^{2}+y^{2} = 1.
\]
Edelliselle yksikkösäteiselle origokeskiselle ympyrälle on vakiintunut nimitys \textbf{yksikköympyrä}.
\end{esimratk}
\end{esimerkki}





\begin{tehtavasivu}

\paragraph*{Opi perusteet}

\paragraph*{Hallitse kokonaisuus}

\paragraph*{Sekalaisia tehtäviä}

TÄHÄN TEHTÄVIÄ SIJOITTAMISTA ODOTTAMAAN

\begin{tehtava}
Ympyrän kespiste on $(0,0)$ ja säde $5$. Muodosta ympyrän yhtälö.
\begin{vastaus}
$x^2+y^2=5$
\end{vastaus}
\end{tehtava}

\begin{tehtava}
Märitä keskipiste ja säde.
\begin{alakohdat}
  \alakohta{$(x-3)^2+(y+7)^2=12$}
	\alakohta{$x^2+y^2=49$}
\end{alakohdat}
\begin{vastaus}
\begin{alakohdat}
	\alakohta{keskpiste $(3,-7)$, säde $2\sqrt{3}$}
	\alakohta{keskipiste $(0,0)$, säde $7$}
\end{alakohdat}
\end{vastaus}
\end{tehtava}

\begin{tehtava}
Märitä keskipiste ja säde.
\begin{alakohdat}
	\alakohta{$x^2+y^2-10x+16y+72=0$}
	\alakohta{$x^2+y^2+8x-22y+129=0$}
\end{alakohdat}
\begin{vastaus}
\begin{alakohdat}
	\alakohta{keskpiste $(5,-8)$, säde $\sqrt{17}$}
	\alakohta{keskipiste $(-4,11)$, säde $2\sqrt{2}$}
\end{alakohdat}
\end{vastaus}
\end{tehtava}

\begin{tehtava}
Määritä ympyrän $(x+10)^2+y^2=2$ keskipiste ja säde ja ratkaise ympyrän yhtälöstä $y$. 
\begin{vastaus}
keskipiste $(-10,0)$, säde $\sqrt{2}$, $y=\pm\sqrt{2-(x+10)^2}$ 
\end{vastaus}
\end{tehtava}

\begin{tehtava}
Ympyrän keskipiste on origo ja säde $3$. Onko piste 
\begin{alakohdat}
	\alakohta{$(10,-2)$}
	\alakohta{$(-3,0)$}
	\alakohta{$(2,\sqrt{5})$ ympyrän kehällä?}
\end{alakohdat}
\begin{vastaus}
\begin{alakohdat}
	\alakohta{ei!}
	\alakohta{joo!}
	\alakohta{joo!}
\end{alakohdat}
\end{vastaus}
\end{tehtava}

\begin{tehtava}
Määritä $k$ niin, että lauseke $(x-3)^2+(y+3)^2=k$ on
\begin{alakohdat}
	\alakohta{ympyrä}
	\alakohta{$\sqrt{7}$-säteinen ympyrä}
	\alakohta{origon kautta kulkeva ympyrä?}
\end{alakohdat}
\begin{vastaus}
\begin{alakohdat}
	\alakohta{$k>0$}
	\alakohta{$k=7$}
	\alakohta{$k=18$}
\end{alakohdat}
\end{vastaus}
\end{tehtava}

\begin{tehtava}
Tutki, mitä yhtälöiden kuvaajat esittävät.
\begin{alakohdat}
	\alakohta{$x^2+y^2-6x+4y+4=0$}
	\alakohta{$x^2+y^2+14x-6y+10=0$}
\end{alakohdat}
\begin{vastaus}
\begin{alakohdat}
	\alakohta{ympyrä}
	\alakohta{piste}
\end{alakohdat}
\end{vastaus}
\end{tehtava}

\begin{tehtava}
Määritä ympyrän keskipiste ja säde.
\begin{alakohdat}
	\alakohta{$(x+t)^2+(y+u)^2=k, k>0$}
	\alakohta{$(x+2)^2+(y-7)^2=-8$}
\end{alakohdat}
\begin{vastaus}
\begin{alakohdat}
	\alakohta{keskipiste $(-t,-u)$, säde  $\sqrt{k}$}
	\alakohta{ei ole ympyrä}
\end{alakohdat}
\end{vastaus}
\end{tehtava}

\begin{tehtava}
Ympyrä sivuaa $y$-akselia pisteessä $(0,-1)$ ja kulkee pisteen $(3,2)$ kautta. Mikä on ympyrän yhtälö?
\begin{vastaus}
$(x-3)^2+(y+1)^2=9$
\end{vastaus}
\end{tehtava}

\begin{tehtava}
Ympyrä kulkee pisteiden $(1,6), (-2,5)$ ja $(5,4)$ kautta. Mikä on ympyrän yhtälö?
\begin{vastaus}
$(x-1)^2+(y-1)^2=16$
\end{vastaus}
\end{tehtava}

\begin{tehtava}
Jana, jonka pituus on $t$ liikkuu koordinaatistossa siten, että sen toinen pää on $x$-akselilla ja toinen $y$-akselilla. Mitä käyrää pitkin liikkuu janan keskipiste?
\begin{vastaus}
$x^2+y^2=\frac{1}{4}t^2$
\end{vastaus}
\end{tehtava}

\end{tehtavasivu}
	% ympyrän yhtälö määritelmästä
	% ensin origokeskeinen
	% keskipiste ja säde muissa tapauksissa neliöksi täydentämällä
	\section{Ympyrä ja suora}

	% suoran ja ympyrän leikkauspisteet
	% tangentit
	\section{Paraabeli}

\laatikko{
KIRJOITA TÄHÄN LUKUUN

\begin{itemize}
\item käyrän $y = ax^2+by+c$ kuvaaja on paraabeli
%%%terminologia? kuvaaja - funktion kuvaaja - käyrä ovatko samoja eivät?
\item mainitaan geometrinen määritelmä
\item paraabelin yhtälön huippumuoto $y-y_0=a(x-x_0)^2$
\end{itemize}

KIITOS!}

\laatikko{
\termi{paraabeli}{Paraabeli} on tason niiden pisteiden joukko, joiden etäisyys kiinteästä pisteestä, \termi{polttopiste}{polttopisteestä} on sama kuin etäisyys kiinteästä suorasta, \termi{johtosuora}{johtosuorasta}.
}

%%%%%%MAA2, luku 3.1 Toisen asteen polynomifunktio
Kussilla 2 mainittiin, että toisen asteen polynomifunktion kuvaaja on paraabeli. Nämä kuvaajat olivat muotoa $y=ax^2+bx+c$ olevia käyriä, joissa $a$ määräsi paraabelin aukeamissuunnan. Jos $a<0$ paraabeli aukeaa alaspäin ja jos $a>0$ ylöspäin.

\begin{kuva}
    kuvaaja.pohja(-1.5, 3.5, -0.5, 2.5, korkeus = 4, nimiX = "$x$", nimiY = "$y$", ruudukko = True)
    kuvaaja.piirra("0.5*x**2-x+0.25", a = -1.5, b = 3.5, nimi = "$y= 0,5x^2-x+0,25$", kohta = (3.2,2.1), suunta = 135)
\end{kuva}

\begin{kuva}
    kuvaaja.pohja(-1.5, 3.5, -0.5, 2.5, korkeus = 4, nimiX = "$x$", nimiY = "$y$", ruudukko = True)
    kuvaaja.piirra("-0.5*x**2+x+1.75", a = -1.5, b = 3.5, nimi = "$y= -0,5x^2+x+1,75$", kohta = (3.2,-0.5), suunta = 135)
\end{kuva}

\begin{esimerkki}
Määritä pistejoukon yhtälö, jolla on seuraava ominaisuus: Jokainen pistejoukon piste on yhtä etäällä pisteestä $(0, 3)$ ja suorasta $y=-3$
\begin{esimratk}
Pisteen $P=(x, y)$ etäisyys annetusta pisteestä on
\[
\sqrt{(x-0)^2+(y-3)^2}=\sqrt{x^2+(y-3)^2}
\]
Pisteen $P$ etäisyys annetusta suorasta on pisteen ja suoran $y$-koordinaattien erotuksen itseisarvo
\[
|y-(-3)| = |y+3| 
\]
Merkitään nämä etäisyydet yhtäsuuriksi ja ratkaistaan saatu yhtälö $y$:n suhteen.
\begin{align*}
|y+3| & = \sqrt{x^2+(y-3)^2} &&\ppalkki \text{neliöönkorotus, kumpikin puoli $>0$}\\
(y+3)^2  &= x^2+(y-3)^2 \\
y^2+6y+9 &=  x^2+y^2-6y+9\\
12y &= x^2 &&\ppalkki : 12\\
y &= \frac{1}{12}x^2
\end{align*}

Pistejoukon yhtälö on $y=\frac{1}{12}x^2$.

\end{esimratk}
\end{esimerkki}

%%% FIX ME ONKO HUIPPUMUOTOINEN HYVÄ TERMI? %%%%%%%%%%%
\subsection{Paraabelin huippumuotoinen yhtälö}

Kaikki muotoa $y=ax^2+bx+c$ olevat paraabelit voidaan ilmoittaa paraabelin huipun $(x_0, y_0)$ avulla seuraavasti.

\[
y-y_0 = a(x-x_0)^2
\]

%%%%% FIX ME Mikä on tämän alaluvun suhde MAA1-kirjan liitteenä olevaan alalukuun "Toisen asteen polynomin kuvaaja"

\begin{esimerkki}
Mitkä ovat paraabelin $y=3x^2-12x+13$ huipun koordinaatit?
\begin{esimratk}
Paraabelin huipun koordinaatit näkisi suoraan huippumuotoiseksi muutetusta yhtälöstä. Helpompaa lienee kuitenkin muuttaa huippumuoitoinen yhtälö $y$:n suhteen ratkaistuksi ja merkitä yhtälöiden kertoimet samoiksi, jolloin saadaan selville $x_0$ ja $y_0$.

\begin{align*}
y-y_0 &= a(x-x_0)^2 \\
y       &= a(x^2-2x_0x+x_0^2)+y_0\\
y       &= ax^2-2ax_0x+(ax_0^2+y_0) &&\ppalkki a=3\\
y       &= 3x^2-6x_0x+(3x_0^2+y_0)
\end{align*}

Merkitään $x$:n kertoimet ja vakiotermit samoiksi.

\begin{align*}
&\begin{cases}
-6x_0=-12 \\
3x_0^2+y_0 =13
\end{cases}\\
&\begin{cases}
x_0=2 \\
y_0 =1
\end{cases}
\end{align*}

Paraabelin huippu on pisteessä $(2, 1)$.

\end{esimratk}
\end{esimerkki}

\begin{tehtavasivu}

\subsubsection*{Opi perusteet}

\subsubsection*{Hallitse kokonaisuus}

\subsubsection*{Sekalaisia tehtäviä}

TÄHÄN TEHTÄVIÄ SIJOITTAMISTA ODOTTAMAAN

\begin{tehtava}
Millaisella käyrällä ovat ympyröiden keskipisteet, kun ympyrät kulkevat pisteen $(0, 0)$ kautta ja sivuavat suoraa $x=4$?
\begin{vastaus}
%keskipiste (x, y)
% x< 4 
% kulkee origon kautta, joten (0-x)^2+(0-y)^2=r^2
% etäisyys suorasta |x-4| = r
$x=-\frac{1}{8}y^2+2$
\end{vastaus}
\end{tehtava}

\begin{tehtava}
Mikä on sen käyrän yhtälö, jonka kukin piste on yhtä etäällä suorasta $y=0$ ja pisteestä $(-3, 1)$.
\begin{vastaus}
% http://www.wolframalpha.com/input/?i=+sqrt%28%28-3-x%29%5E2+%2B+%281-y%29%5E2%29%3D+abs%280-y%29
$y = \frac{1}{2}x^2+3 x+5$
\end{vastaus}
\end{tehtava}

\begin{tehtava}
%%% ONKO JOHTOSUORA NIIN OLEELLINEN KÄSITE, ETTÄ KÄYTETÄÄN TEHTÄVISSÄ?
%% tehtävänhän voi kirjoittaa ilman tuota termiä
Mikä on sen paraabelin yhtälö, jonka polttopiste on $(2, 3)$ ja johtosuora $y=1$?
\begin{vastaus}
% http://www.wolframalpha.com/input/?i=+sqrt%28%282-x%29%5E2+%2B+%283-y%29%5E2%29%3D+abs%281-y%29
$y = \frac{1}{4}x^2-x+3$
\end{vastaus}
\end{tehtava}

\begin{tehtava}
Mitkä ovat suoran $x+y=6$ ja paraabelin $y=4x^2-3x$ leikkauspisteet?
\begin{vastaus}
% http://www.wolframalpha.com/input/?i=y%3D4x%5E2-3x%2C+x%2By%3D6
$x = -1$, $ y = 7$ tai $x = \frac{3}{2}$, $y = \frac{9}{2}$
\end{vastaus}
\end{tehtava}

%%%%%TÄMÄ JA SEURAAVA PITÄISI SIIRTÄÄ VASEMMALLE/OIKEALLE AUKEAVIEN PARAABELIEN JÄLKEEN?
\begin{tehtava}
Millaisella käyrällä ovat ympyröiden keskipisteet, kun ympyrät kulkevat pisteen $(0, 0)$ kautta ja sivuavat suoraa $x=4$?
\begin{vastaus}
%keskipiste (x, y)
% x< 4 
% kulkee origon kautta, joten (0-x)^2+(0-y)^2=r^2
% etäisyys suorasta |x-4| = r
$x=-\frac{1}{8}y^2+2$
\end{vastaus}
\end{tehtava}

\begin{tehtava}
Kuinka monta leikkauspistettä voi olla paraabelilla ja
\begin{enumerate}[a)]
\item suoralla,
\item ympyrällä,
\item toisella paraabelilla?
\end{enumerate}
\begin{vastaus}
\begin{enumerate}[a)]
\item 0--2
\item 0--4
\item 0--4
\end{enumerate}
\end{vastaus}
\end{tehtava}

\begin{tehtava}
Määritä ne paraabelin $y=x^2-1$ pisteet, jotka ovat yhtä kaukana pisteistä $(4, 4)$ ja $(4, 2)$?
\begin{vastaus}
%suoran y=3 ja paraabelin leikkauspisteet
$x=-2$, $y=3$ ja $x=2$, $y=3$
\end{vastaus}
\end{tehtava}

\begin{tehtava}
Määritä ne paraabelin $y=x^2-1$ pisteet, jotka ovat yhtä kaukana pisteistä $(4, 4)$ ja $(3, 3)$?
\begin{vastaus}
%pisteiden keskipisteen (3,5; 3,5) kautta kulkeva suora y=-x+7
%suoran  ja paraabelin leikkauspistee
% http://www.wolframalpha.com/input/?i=y%3Dx%5E2-1%2C+y%3D-x%2B7
$x = -\frac{1+\sqrt{33}}{2}$,   $y = \frac{15+\sqrt{33}}{2}$ tai $x = -\frac{\sqrt{33}-1}{2}$,   $y = \frac{15-\sqrt{33}}{2}$
\end{vastaus}
\end{tehtava}

\begin{tehtava}
Ratkaise paraabelien $y=5(x-2)^2$  ja $y=-x^2+2x+4$ leikkauspisteet?
\begin{vastaus}
%tulee toisen asteen yhtälö ratkaistavaksi
% http://www.wolframalpha.com/input/?i=y%3D5%28x-2%29%5E2%2C+y%3D-x%5E2%2B2x%2B4
$(1, 5)$ ja $(\frac{8}{3}, \frac{20}{9})$
\end{vastaus}
\end{tehtava}

\begin{tehtava}
%tehtävä helpottuu, koska vakiotermin saa suoraan
Määritä sen ylöspäin aukeavan paraabelin yhtälö, joka kulkee pisteiden $(-5, 6)$, $(0, -4)$ ja $(1, 0)$ kautta.
\begin{vastaus}
% http://www.wolframalpha.com/input/?i=y%3D+x%5E2%2B3x-4+at+x%3D%7B-5%2C+0%2C+1%7D
$y= x^2+3x-4$
\end{vastaus}
\end{tehtava}

\begin{tehtava}
Määritä sen alaspäin aukeavan paraabelin yhtälö, joka kulkee pisteiden $(-5, 6)$, $(0, -4)$ ja $(1, 0)$ kautta.
\begin{vastaus}
% http://www.wolframalpha.com/input/?i=y%3D+x%5E2%2B3x-4+at+x%3D%7B-5%2C+0%2C+1%7D
$y= x^2+3x-4$
\end{vastaus}
\end{tehtava}

\begin{tehtava}
% VAIKEA
Määritä kaksi sellaista paraabelia, että niillä on täsmälleen kolme yhteistä pistettä
\begin{vastaus}
%idea valitaan suora, joka on tangetti kummallekin paraabelille esim. y=x ja kumpikin paraabeli sivuaa suoraa samassa kohtaa
% http://www.wolframalpha.com/input/?i=y%3Dx%5E2-x%2C+x%3D2y%5E2-y
Esimerkiksi $y=x^2-x$ ja  $x=2y^2-y$
\end{vastaus}
\end{tehtava}

\begin{tehtava}
Millä parametrin $a$ arvoilla paraabelin $y=x^2-ax+a$ huippu on $x$-akselilla?
\begin{vastaus}
% esim. neliöksi täydentäminen y=(x-a/2)^2 -a^2/4 +a, josta -a^2/4 +a =0
$a=0$ ja $a=4$
\end{vastaus}
\end{tehtava}

\begin{tehtava}
Määritä vakio $a$ siten, että lausekkeen $2x^2+12x+a$ pienin arvo on 10.
\begin{vastaus}
% esim. neliöksi täydentäminen 2x^2+12x+a= 2((x+3)^2-9+a/2), josta 2(-9+a/2)=10
$a=28$
\end{vastaus}
\end{tehtava}



\end{tehtavasivu}


	% merkitys toisen asteen polynomin kuvaajana
	% geometrisen määritelmän maininta
	\section{Paraabelin sovelluksia}

\laatikko{
KIRJOITA TÄHÄN LUKUUN

\begin{itemize}
\item paraabelin huippu on kohdassa $x=-b/2a$, todistus
\item paraabelin yhtälön huippumuoto $y-y_0=a(x-x_0)^2$
\item paraabelin yhtälön ratkaiseminen kolmen pisteen avulla
\item soveltavia tehtäviä, ne iänikuiset holvikaaret jne.
\end{itemize}

KIITOS!}
	% huipun x-koordinaatti on -b/2a
		% todistus liitteeksi -Ville
		% todistus tähän -Niko
	% paraabelin tangentit
	\section{Vasemmalle ja oikealle aukeavat paraabelit}

\laatikko{
KIRJOITA TÄHÄN LUKUUN

\begin{itemize}
\item muotoa $x=ay^2+by+c$ olevat paraabelit aukeavat oikealle tai vasemmalla
\end{itemize}

KIITOS!}

\begin{tehtavasivu}

\subsubsection*{Opi perusteet}

\subsubsection*{Hallitse kokonaisuus}

\subsubsection*{Sekalaisia tehtäviä}

TÄHÄN TEHTÄVIÄ SIJOITTAMISTA ODOTTAMAAN

\end{tehtavasivu}
	% paraabeli x = ay^2  +by + c
	\section{$\star$ Yleinen toisen asteen tasokäyrä}

\laatikko{
KIRJOITA TÄHÄN LUKUUN

\begin{itemize}
\item Tässä luvussa tarkastellaan lyhyesti yleisesti toisen asteen tasokäyriä, joista ympyrä ja paraabeli on jo käsitelty edellä
\item (kahden muuttujan) toisen asteen yhtälön määritelmä
\item joku tuttu esimerkki, vaikka paraabeli
\item esimerkkeinä yksi piste, kaksi suoraa, tyhjä
\item maininta siitä, että voi tulla myös ellipsi tai hyperbeli,
joista sitten joskus kirjoitetaan liite
\end{itemize}

KIITOS!}

\begin{tehtavasivu}

\subsubsection*{Opi perusteet}

\subsubsection*{Hallitse kokonaisuus}

\subsubsection*{Sekalaisia tehtäviä}

TÄHÄN TEHTÄVIÄ SIJOITTAMISTA ODOTTAMAAN

\end{tehtavasivu}
	% esim. piste, kaksi suoraa, tyhjä
	\section{Sekalaista}

\laatikko{
KIRJOITA TÄHÄN LUKUUN

\begin{itemize}
\item kaikkea jännää kurssiin liittyen !
\end{itemize}

KIITOS!}

\begin{tehtavasivu}

\subsubsection*{Opi perusteet}

\subsubsection*{Hallitse kokonaisuus}

\subsubsection*{Sekalaisia tehtäviä}

TÄHÄN TEHTÄVIÄ SIJOITTAMISTA ODOTTAMAAN

\end{tehtavasivu}
	% esimerkkejä ja tehtäviä (erikoisia, vaikeita, yms.)

\chapter{Ulkoasukokeiluja}
	 
\section{Ulkoasukokeiluja}

%\definecolor{Sampo3}{cmyk}{0.2,0.27,0.0,0.0}
\definecolor{ParisGreen}{RGB}{80,200,120}
%\definecolor{PersianGreen}{RGB}{0,166,147}
%\definecolor{darkKhaki}{RGB}{189,183,107}
%\definecolor{cornflowerBlue}{RGB}{154,206,235}
%\definecolor{thulianPink}{RGB}{222,111,161}
%\definecolor{olivine}{RGB}{154,185,115}
%\definecolor{battleshipGray}{RGB}{132,132,130}

\pgfdeclarehorizontalshading{laatikkotaustaC}{100bp}{color(0bp)=(ParisGreen); color(50bp)=(ParisGreen); color(100bp)=(ParisGreen!50)}
%\pgfdeclarehorizontalshading{laatikkotausta}{100bp}{color(0bp)=(white); color(50bp)=(Sampo3); color(100bp)=(Sampo3!50)}
\pgfdeclarehorizontalshading{laatikkotaustaE}{195mm}{color(0mm)=(ParisGreen); color(97mm)=(ParisGreen); color(195mm)=(Sampo3)}
\pgfdeclarehorizontalshading{laatikkotaustaF}{100bp}{color(0bp)=(white); color(25bp)=(white); color(75bp)=(Sampo3); color(100bp)=(Sampo3)}
\colorlet{mycolor}{green}
\pgfdeclarehorizontalshading[mycolor]{myshadingB}{1cm}{rgb(0cm)=(1,0,0); color(1cm)=(white); color(2cm)=(Sampo3)}

\mdfdefinestyle{laatikkotyyliQ}{
  usetwoside=false,
  leftmargin=-46mm,
  rightmargin=-22mm,
  innerleftmargin=0mm,
  innerrightmargin=0mm,
  innertopmargin=0mm,
  innerbottommargin=0mm,
}

\mdfdefinestyle{laatikkotyyliB}{
%  userdefinedwidth=166mm % FIXME: controversial
  usetwoside=true,
  innermargin=15mm,
  outermargin=0mm,
  innerleftmargin=36mm,
  innerrightmargin=12mm,
  middlelinewidth=0pt,
  apptotikzsetting={\tikzset{mdfbackground/.append style = {shading = laatikkotausta}}},
}

\mdfdefinestyle{laatikkotyyliC}{
  userdefinedwidth=127mm,
  leftmargin=0pt,
  innermargin=0pt,
  innerleftmargin=0,
  rightmargin=0pt,
  innerrightmargin=0pt,
  middlelinewidth=0pt,
  apptotikzsetting={\tikzset{mdfbackground/.append style = {shading = laatikkotaustaC}}},
  innertopmargin=0mm,
  innerbottommargin=0mm,
  needspace=3\baselineskip,
}

\mdfdefinestyle{laatikkotyyliW}{
  usetwoside=false,
  leftmargin=-46mm,
  rightmargin=-22mm,
  innerleftmargin=46mm,
  innerrightmargin=22mm,
  innertopmargin=8pt,
  innerbottommargin=8pt,
  hidealllines=true,
  apptotikzsetting={\tikzset{mdfbackground/.append style = {shading = laatikkotaustaF}}},
  needspace=3\baselineskip,
}

\newcommand{\laatikkoB}[1]{%
%\begin{\widepar}
\ifthispageodd
{
  \begin{mdframed}[style=laatikkotyyliB]
    #1
  \end{mdframed}
}
{
  \begin{mdframed}[style=laatikkotyyliC]
    #1
  \end{mdframed}
}
%\end{widepar}
}


\newcommand{\laatikkoE}[1]{%
%\begin{\widepar}
  \begin{mdframed}[style=laatikkotyyliW]
	#1
  \end{mdframed}
%\end{widepar}
}

\laatikkoE{
\textbf{Itseisarvon ominaisuuksia}\\ \\
	\begin{tabular}{l l}
		$|a|\geq0$ & Itseisarvo on aina ei-negatiivinen \\
		$|a|=|-a|$ & Luvun ja sen vastaluvun itseisarvot ovat yhtäsuuret \\
		$|a|^2=a^2$ & Luvun itseisarvon neliö on yhtäsuuri kuin luvun neliö \\
		$|\frac{a}{b}|=\frac{|a|}{|b|}$ & Osamaarän itseisarvo on itseisarvojen osamäärä

	\end{tabular}
}

\newpage

\section{Ulkoasukokeiluja 2}

\laatikkoE{
\textbf{Itseisarvon ominaisuuksia}\\ \\
	\begin{tabular}{l l}
		$|a|\geq0$ & Itseisarvo on aina ei-negatiivinen \\
		$|a|=|-a|$ & Luvun ja sen vastaluvun itseisarvot ovat yhtäsuuret \\
		$|a|^2=a^2$ & Luvun itseisarvon neliö on yhtäsuuri kuin luvun neliö \\
		$|\frac{a}{b}|=\frac{|a|}{|b|}$ & Osamaarän itseisarvo on itseisarvojen osamäärä

	\end{tabular}
}

\newpage

