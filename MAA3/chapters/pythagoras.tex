\section*{Pythagoraan lause}

Pythagoraan lause on eräs vanhimmista matemaattisista lauseista. Sille tunnetaan monia eri
todistuksia, joista kaksi yksinkertaista esitellään alla.

\laatikko{
	\termi{Pythagoraan lause}{Pythagoraan lause}. Olkoot suorakulmaisen kolmion kateettien
	pituudet $a$ ja $b$, ja hypotenuusan pituus $c$. Tällöin pätee
	\[
	a^2 + b^2 = c^2
	\] 
}

\textbf{Pythagoraan lauseen todistus 1}. Olkoot $A$, $B$ ja $C$ suorakulmaisen kolmion
kärjet niin, että kulma $\angle ACB$ on suora. Olkoon piste $D$ sellainen janan $AB$ piste,
jolle $\angle CDA$ on suora.

(KUVA)

Tällöin kolmioilla $ABC$ ja $ACD$ on yhteinen kulma $\angle A$ ja molemmissa on suora kulma, joten ne ovat yhdenmuotoiset (kk). Samoin kolmioilla $ABC$ ja $CBD$ on yhteinen kulma
$\angle B$ ja molemmissa on suora kulma, joten nekin ovat yhdenmuotoiset (kk)

Yhdenmuotoisuuksista saadaan, että
\[
\frac{AB}{AC} = \frac{AC}{AD}
\]
ja
\[
\frac{AB}{CB} = \frac{CB}{DB}.
\]

Ristiinkertomalla saadaan, että
\[
AB \cdot AD = AC^2
\]
ja
\[
AB \cdot DB = CB^2.
\]
Laskemalla nämä saadut yhtälöt yhteen saadaan

\begin{align*}
AC^2 + CB^2  = AB \cdot AD + AB \cdot DB \\
AC^2 + CB^2  = AB(AD + DB) \\
AC^2 + CB^2  = AB^2.
\end{align*}
Tämä on täsmälleen Pythagoraan lause. $\square $

Lauseen voi todistaa myös toisella tavalla.

\textbf{Pythagoraan lauseen todistus 2}. Olkoot $a$, $b$ ja $c$ suorakulmaisen kolmion
sivujen pituudet, joista $c$ on hypotenuusan pituus. Tarkastellaan neliötä, jonka sivun
pituus on $a+b$, ja jaetaan se osiin kahdella eri tavalla alla olevien kuvien mukaisesti.

(KUVAT)

Ensimmäisessä kuvassa neliön osat ovat neljä yhtenevää suorakulmaista kolmiota, joiden
sivujen pituudet ovat $a$, $b$ ja $c$, ja kaksi neliötä, joiden sivujen pituudet ovat $a$ ja
$b$.

Toisessa kuvassa neliön osat ovat neljä yhtenevää suorakulmaista kolmiota, joiden
sivujen pituudet ovat $a$, $b$ ja $c$, ja neliö, jonka sivun pituus on $c$.
Koska yhteenlaskettujen pinta-alojen on oltava samat, on oltava $a^2 + b^2 = c^2$.
Pythagoraan lause on todistettu. $\square $

