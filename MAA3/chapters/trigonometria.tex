
\section*{Trigonometria}

Tarkastellaan kahta suorakulmaista kolmiota, joilla on yhtä suuri terävä kulma $\alpha$.
Silloin kolmiot ovat keskenään yhdenmuotoisia, sillä niillä on kaksi yhtä suurta vastinkulmaa.
Yhdenmuotoisuuden nojalla sivujen keskinäiset suhteet ovat siis täsmälleen samoja kaikilla
suorakulmaisilla kolmioilla, joilla on samansuuruinen terävä kulma. Näille suhteet on
nimetty seuraavasti.

\laatikko{
\termi{sini}{Sini}: $\sin \alpha = \frac{\textrm{kulman vastaisen kateetin pituus}}
{\textrm{hypotenuusan pituus}}$ \\
\termi{kosini}{Kosini}: $\cos \alpha = \frac{\textrm{kulman viereisen kateetin pituus}}
{\textrm{hypotenuusan pituus}}$ \\
\termi{tangentti}{Tangentti}: $\tan \alpha = \frac{\textrm{kulman vastaisen kateetin pituus}}
{\textrm{kulman viereisen kateetin pituus}}$
}

...yksikköympyrä, tylpän kulman siniä ja kosinia tarvitaan...