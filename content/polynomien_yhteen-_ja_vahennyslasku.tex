%%%%%%%%%%%%%%%%%%%%%%%%%%%%%%%%%%%%%%%%%%%%%%%%%%%%%%%%%%%%%%%%%%%%%%%%%%%%%%%%%%%%%%%%%%%%%%%%%%%%%%%%%%%%
% HUOM! Tämän luvun koko sisältö on siirretty tiedostoon "polynomien_yhteen_ja_vahennyslasku_sisalto.tex".
%%%%%%%%%%%%%%%%%%%%%%%%%%%%%%%%%%%%%%%%%%%%%%%%%%%%%%%%%%%%%%%%%%%%%%%%%%%%%%%%%%%%%%%%%%%%%%%%%%%%%%%%%%%%




%\chapter{Polynomien yhteen- ja vähennyslasku}
%
%Polynomeja voidaan laskea yhteen summaamalla samanasteiset termit. Esimerkiksi polynomien $5x^2-x+5$ ja $3x^2-1$ summa on
%   \begin{align*}
%        (\textcolor{red}{5x^2}-x\textcolor{blue}{{}+5})+(\textcolor{red}{3x^2}\textcolor{blue}{-1}) &=\textcolor{red}{(5+3)x^2}-x\textcolor{blue}{{}+(5-1)}=\textcolor{red}{8x^2}-x\textcolor{blue}{{}+4}.
%    \end{align*}
%Samalla tavalla polynomeja voidaan vähentää toisistaan. Polynomien
%$14x^3+69$ ja $3x^3+2x^2+x$ erotus on
%    \begin{align*}
%        (\textcolor{green}{14x^3} + 69) - (\textcolor{green}{3x^3} \textcolor{blue}{{}+ 2x^2} \textcolor{red}{{}+x})
%        &= \textcolor{green}{14x^3} + 69 \textcolor{green}{{}-3x^3} - 
%            \textcolor{blue}{2x^2} \textcolor{red}{{}-x} \\
%        &= \textcolor{green}{(14-3)x^3} \textcolor{blue}{{}-2x^2} \textcolor{red}{{}-x} + 69 \\
%        &= \textcolor{green}{11x^3} \textcolor{blue}{{}-2x^2} \textcolor{red}{{}-x} + 69
%    \end{align*}
%    
%Samanasteisten termien yhteen- ja vähennyslasku perustuu siihen, että vakiokerroin voidaan
%ottaa yhteiseksi tekijäksi:
%\[
%ax^n+bx^n=(a+b)x^n.
%\]
%    
%\begin{esimerkki}
%Lasketaan polynomit
%$x+1$ ja $3x^2-2x+5$ yhteen:
%   \begin{align*}
%        (x+1)+(3x^2-2x+5) =3x^2-x+6.
%    \end{align*}
%Lasketaan polynomit $-4x^4+3x^3-x$ ja $-3x^3+5x^2+2x$ yhteen:
%   \begin{align*}
%        (-4x^4+3x^3-x)+(-3x^3+5x^2+2x) =-4x^4+5x^2+x.
%    \end{align*}
%Vähennetään polynomit $x+1$ ja $-4x^4+3x^3-x$ toisistaan:
%   \begin{align*}
%        (x+1)-(-4x^4+3x^3-x) =x+1+4x^4-3x^3+x=4x^4-3x^3+2x+1.
%    \end{align*}
%\end{esimerkki}
%
%
%%EN AIVAN YMMÄRTÄNYT TÄTÄ
%%Huomaa, että sulkeet voidaan poistaa ja termit yhdistää silloin, kun ne ovat
%%samaa astetta, ihan samaan tapaan kuin sulkeet voidaan poistaa lukujen
%%yhteenlaskussa silloin, kun ne eivät ole tulon tekijöinä.

%\section{Harjoitustehtäviä}
%
%\begin{tehtava}
%    Sievennä
%    \begin{enumerate}
%        \item $(2x + 3) + x $
%        \item $(3x - 1) + (-x + 1)$
%        \item $(5x + 10) + (6x - 6) - (x + 3)$
%    \end{enumerate}
%    \begin{vastaus}
%        \begin{enumerate}
%            \item $3x + 3$
%            \item $2x$
%            \item $10x - 1$
%        \end{enumerate}
%    \end{vastaus}
%\end{tehtava}
%
%\begin{tehtava}
%    Sievennä
%    \begin{enumerate}
%        \item $(x^2 - 2x + 1) + (-x^2 + x) $
%        \item $(3y^3 + 2y^2  + y) - (-y^2 + y)$
%        \item $(z^{10} - z^6 + z^2 + 1) + (z^{10} + 2z^8 - 3z^6) - (z^4 - 3)$
%    \end{enumerate}
%    \begin{vastaus}
%        \begin{enumerate}
%            \item $-x + 1$
%            \item $3y^3 + 2y^2$
%            \item $2z^{10} + 2z^8 - 4z^6 - z^4 + z^2 + 4$
%        \end{enumerate}
%    \end{vastaus}
%\end{tehtava}
