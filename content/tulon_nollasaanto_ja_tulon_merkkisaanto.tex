\chapter{Tulon nollasääntö ja tulon merkkisääntö}

\section{Tulon nollasääntö}

Tulon nollasääntönä tunnetaan seuraava tulos:

\laatikko{
    \emph{Tulon nollasääntö}\\
    Tulo on 0, jos ainakin yksi tulon tekijöistä on 0.
    Jos yksikin tulon tekijöistä on 0, tulo on 0.
}

\section{Tulon merkkisääntö}

Tulon merkkisäännöksi kutsutaan seuraavaa tulosta:

\laatikko{
    \emph{Tulon merkkisääntö kahdelle tulontekijälle}
    Jos tulon tekijät ovat samanmerkkisiä, tulo on positiivinen.
    Jos tulon tekijät ovat erimerkkisiä, tulo on negatiivinen.
}

Tärkeä sovellus merkkisäännölle on, että $x^2 \geq 0$.

\section{Harjoitustehtäviä}
