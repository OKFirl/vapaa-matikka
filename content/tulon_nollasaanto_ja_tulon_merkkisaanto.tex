\chapter{Tulon nollasääntö ja tulon merkkisääntö}

\section*{Tulon merkkisääntö}

Pitkän matematiikan 1. kurssilla on esitetty seuraava sääntö kahden luvun tulolle:

\laatikko{
    \emph{Tulon merkkisääntö kahdelle tulontekijälle}
    \begin{itemize}
        \item Jos tulon tekijät ovat samanmerkkisiä, tulo on positiivinen.
        \begin{itemize}
	        \item Kahden positiivisen luvun tulo on positiivinen.
	        \item Kahden negatiivisen luvun tulo on positiivinen.
        \end{itemize}
        \item Jos tulon tekijät ovat erimerkkisiä, tulo on negatiivinen.
        \begin{itemize}
	        \item Positiivisen ja negatiivisen luvun tulo on negatiivinen.
        \end{itemize}
    \end{itemize}
}

Useasti käytetty merkkisäännöstä saatava tulos on $x^2 \geq 0$.

Tulon merkkisääntö yleistyy mille tahansa määrälle tulontekijöitä.

\laatikko{
    \emph{Tulon merkkisääntö}
    \begin{itemize}
        \item Jos tulossa on parillinen $(0, 2, 4, \ldots)$ määrä negatiivisia tekijöitä, tulo on positiivinen.
        \begin{itemize}
	        \item Erityisesti, jos tulossa on vain positiivisia tekijöitä, tulo on positiivinen.
        \end{itemize}
        \item Jos tulossa on pariton $(1, 3, 5, \ldots)$ määrä negatiivisia tekijöitä, tulo on negatiivinen.
    \end{itemize}
}

Hyödynnämme merkkisääntöä myöhemmin tehdessämme epäyhtälöistä merkkikaavioita.

\section*{Tulon nollasääntö}

Tulon merkkisäännöstä seuraa, että positiivisten ja negatiivisten lukujen tulo on aina positiivista tai negatiivista, ei koskaan nolla.

Jos siis tulo on $0$, tulon tekijöistä ainakin yhden täytyy olla $0$. Toisaalta $0\cdot x = 0$ kaikilla reaaliluvuilla $x$.
(Todistus tälle on liitteessä \ref{tod:tulonolla}.)

Nämä tiedot yhdistämällä saadaan tulon nollasääntö:

\laatikko{
    \emph{Tulon nollasääntö}
    \begin{itemize}
   	    \item Jos jokin tulon tekijöistä on $0$, tulo on $0$.
   	    \item Jos tulo on $0$, ainakin yksi tulon tekijöistä on $0$.
    \end{itemize}
}

Tulon nollasääntöä on kätevää soveltaa monissa tilanteissa.

\begin{esimerkki} Ratkaistaan yhtälö $(x+5) \cdot x =0 $.
    \begin{align*}
        (x+5)\cdot x &=0 \quad \ppalkki \text{ tulon nollasääntö} \\
        x +5= 0 \text{ tai } x &=0 \\
        x= -5 \text{ tai } x &=0.
    \end{align*}
    Ratkaisuja oli siis kaksi, $x= -5$ tai $x= 0$.
\end{esimerkki}

%\begin{esimerkki}
%	\[2(x+5)=0\]
%	Nyt tulon nollasäännön perusteella tiedetään, että $2=0$ tai $x+5=0$.
%	Koska selvästi $2\neq 0$, jää ainoaksi ratkaisuksi $x+5=0$ eli $x=-5$.
%\end{esimerkki}

\begin{esimerkki} Ratkaistaan $y$ yhtälöstä
    \[(x^5+5x+5)\cdot 0\cdot \sqrt{x^3-1} =y\]
    Koska vasemmalla puolella yksi tulon tekijöistä on $0$, tiedämme, että tulo on $0$. Siis $y=0$.
\end{esimerkki}

\begin{esimerkki} Ratkaise yhtälö
    \[xyz=0.\]
Tulon nollasäännön perusteella $x=0$, $y=0$ tai $z=0$. Nollia voi siis
olla 1--3 kappaletta.
\end{esimerkki}

Tulon nollasääntö on yksi tärkeimmistä syistä siihen, miksi polynomien tekijöihinjako on hyödyllistä.

Jos vaikkapa haluamme ratkaista yhtälön $2x^3-14x^2+32x-24=0$ ja satumme tietämään, että $2x^3-14x^2+32x-24=2(x-3)(x-2)^2$,
voimme helposti päätellä, että polynomi saa arvon $0$ jos ja vain jos $x-3=0$ tai $x-2=0$. Yhtälön ainoat ratkaisut ovat siis $x=3$ ja $x=2$.

\begin{esimerkki}
Ratkaistaan yhtälö $6x^3-36x^2+54x=0$ tulon nollasäännön avulla.

\begin{align*}
6x^3-36x^2+54x &= 0 \\
6(x^3-6x^2+9x) &= 0 \ \ \ \ &\ppalkki \text{otetaan } 6 \text{ yhteiseksi tekijäksi} \\
6x(x^2-6x+9) &= 0 &\ppalkki \text{otetaan } x \text{ yhteiseksi tekijäksi} \\
6x(x^2-2\cdot 3\cdot x+3^2)  &= 0 \\
6x(x-3)^2 &= 0 & \\
6x(x-3)(x-3) &= 0 &\ppalkki \text{ tulon nollasääntö}\\
6x=0 \textrm{\quad tai}& \quad x-3=0 \\
x=0 \textrm{\quad tai}& \quad x=3 \\
\end{align*}

Vastaus: $x=0$ tai $x=3$.
\end{esimerkki}

Myöhemmin tässä kirjassa esitetään polynomien jakolause, joka antaa vielä syvällisemmän yhteyden polynomien nollakohtien ja tekijöiden välille.

\Harjoitustehtavat

\begin{tehtava}
    Ratkaise seuraavat yhtälöt tulon nollasääntöä hyödyntäen.
    \begin{enumerate}
        \item $x^2(3+x)=0$
        \item $0=x^3(x-5)$
        \item $(x-4)(x^2-4)=0$
    \end{enumerate}
    \begin{vastaus}
        \begin{enumerate}
            \item $x=0$ tai $x=-3$
            \item $x=0$ tai $x=5$
            \item $x=-2$, $x=2$ tai $x=4$
        \end{enumerate}
    \end{vastaus}
\end{tehtava}

\begin{tehtava}
    Ratkaise seuraavat yhtälöt tulon nollasääntöä hyödyntäen.
    \begin{enumerate}
        \item $(x^2-1)(x-7)=0$
        \item $(x^2-9)(x^2-16)=0$
        \item $(x-4)=x(x-4)$
    \end{enumerate}
    \begin{vastaus}
        \begin{enumerate}
            \item $x=-1$, $x=1$ tai $x=7$
            \item $x=-4$, $x=-3$, $x=3$ tai $x=4$
            \item $x=1$ tai $x=4$
        \end{enumerate}
    \end{vastaus}
\end{tehtava}

\begin{tehtava}
    Ratkaise seuraavat yhtälöt tulon nollasääntöä hyödyntäen.
    \begin{enumerate}
        \item $(\smiley{}+1)\cdot (t+1)=0$
        \item $x(x-5)=0$
        \item $(2w+2)^2=0$
    \end{enumerate}
    \begin{vastaus}
        \begin{enumerate}
            \item $\smiley{}=-1$ tai $t=-1$,\qquad  Symboli $\smiley{}$ esittää jotain lukua, sillä muutoin laskutoimitukset eivät olisi mielekkäitä. Tehtävässä ei myöskään ole selvää, minkä muuttujan suhteen yhtälö pitäisi ratkaista. Siksi on ratkaistu molempien muuttujien suhteen.
            \item $x=0$ tai $x=5$
            \item $w=-1$
        \end{enumerate}
    \end{vastaus}
\end{tehtava}

\begin{tehtava}
    Sievennä seuraava lauseke: $(a-x)\cdot(b-x)\cdot(c-x)\cdot...\cdot(\mathring{a}-x)\cdot(\ddot{a}-x)\cdot(\ddot{o}-x)$.
    \begin{vastaus}
        Tulossa esiintyy tekijänä $(x-x)=0$. Niinpä tulon nollasäännön mukaan
        \begin{align*}
            &(a-x)\cdot(b-x)\cdot(c-x)\cdot...\cdot(x-x)\cdot(y-x)\cdot(z-x)\cdot(\mathring{a}-x)\cdot(\ddot{a}-x)\cdot(\ddot{o}-x) \\
            =&(a-x)\cdot(b-x)\cdot(c-x)\cdot...\cdot 0\cdot(y-x)\cdot(z-x)\cdot(\mathring{a}-x)\cdot(\ddot{a}-x)\cdot(\ddot{o}-x) \\
            =&0
        \end{align*}
    \end{vastaus}
\end{tehtava}
