\chapter{Tulon nollasääntö ja tulon merkkisääntö}

\section{Tulon merkkisääntö}

Pitkän matematiikan 1. kurssilta muistetaan seuraava sääntö kahden luvun tulolle:

\laatikko{
    \emph{Tulon merkkisääntö kahdelle tulontekijälle}
    \begin{itemize}
		\item Jos tulon tekijät ovat samanmerkkisiä, tulo on positiivinen.
		\item Jos tulon tekijät ovat erimerkkisiä, tulo on negatiivinen.    
    \end{itemize}
}

\begin{esimerkki}{\ }
\begin{itemize}
  \item $2\cdot 3 = 6$ (Kahden positiivisen luvun tulo on aina positiivinen.)
  \item $(-2)\cdot (-3) = 6$ (Kahden negatiivisen luvun tulo on aina positiivinen.)
  \item $(-2)\cdot 3 = 2\cdot (-3) = -6$ (Positiivisen ja negatiivisen luvun tulo on aina negatiivinen.)
\end{itemize}
\end{esimerkki}

Useasti käytetty merkkisäännöistä saatava tulos on $x^2 \geq 0$.

\section{Tulon nollasääntö}

Tulon merkkisäännöstä seuraa, että positiivisten ja negatiivisten lukujen tulo on aina positiivista tai negatiivista, ei koskaan nolla.

Jos siis tulo on 0, tulon tekijöistä ainakin yhden täytyy olla 0. Toisaalta
$0\cdot x = 0$ kaikilla reaaliluvuilla $x$. \todo{Lisää todistus liiteeksi.}
Nämä tiedot yhdistämällä saadaan tulon nollasääntö:

%Merkkisäännöstä on helppoa huomata, että positiivisten tai negatiivisten
%lukujen tulot ovat aina positiivisia tai negatiivisia. Näin ollen $0$ ei siis koskaan
%voi syntyä tulokseksi positiivisten tai negatiivisten lukujen kertolaskusta. Jos siis kahden %luvun
%kertolaskusta saadaan tulokseksi $0$, ei jää mitään muuta vaihtoehtoa kuin että vähintään %toinen tekijä on $0$.

\laatikko{
    \emph{Tulon nollasääntö}
    \begin{itemize}
   		\item Jos jokin tulon tekijöistä on 0, tulo on 0.    
   		\item Jos tulo on 0, ainakin yksi tulon tekijöistä on 0.
    \end{itemize}
}

Seuraavassa on muutama esimerkki tulon nollasäännön soveltamisesta.

\begin{esimerkki} Ratkaistaan yhtälö $(x+5) \cdot x =0 $.
	\begin{align*}
	(x+5)\cdot x &=0 \quad || \text{ tulon nollasääntö} \\
	x +5= 0 \text{ tai } x &=0 \\
	x= -5 \text{ tai } x &=0.
	\end{align*}
	Ratkaisuja oli siis kaksi.
\end{esimerkki}

%\begin{esimerkki}
%	\[2(x+5)=0\]
%	Nyt tulon nollasäännön perusteella tiedetään, että $2=0$ tai $x+5=0$.
%	Koska selvästi $2\neq 0$, jää ainoaksi ratkaisuksi $x+5=0$ eli $x=-5$.
%\end{esimerkki}

\begin{esimerkki} Ratkaistaan $y$ yhtälöstä
	\[x^{1+a^2}\cdot 0\cdot\left(x+5+\sqrt[3]{\frac{x}{1+2^a}}\right)=y\]
	Koska toinen tulon tekijöistä on $0$, tiedätään, että myös niiden tulo on $0$.
	Siis $y=0$.
\end{esimerkki}

\todo{Esitetäänkö tulon nollasääntö ainoastaan kahdelle tulon tekijälle vai usemmalle kuten nyt. Ehdotan nykyisen säännön jättämistä. Seuraava esimerkki vaikuttaa siltä kuin tiedettäisiin ainoastaan sääntö 2 tekijälle. Annetaanko sen kuitenkin olla noin?}
\begin{esimerkki}
	\[xyzw=0\]
	Nyt tulon nollasäännö perusteella tiedetään, että $x=0$ tai $yzw=0$. Jos $yzw=0$, niin tulon nollasäännön mukaan $y=0$ tai $zw=0$. Samoin jos $zw=0$, niin $z=0$ tai $w=0$. Näin voidaan päätellä, että vähintään yhden tekijöistä $x$, $y$, $z$ tai $w$ on oltava $0$.
\end{esimerkki}


Tulon nollasääntö on yksi tärkeimmistä syistä siihen, miksi polynomien tekijöihinjako on hyödyllistä.

Jos vaikkapa haluamme ratkaista yhtälön $2x^3-14x^2+32x-24=0$ ja satumme tietämään, että $2x^3-14x^2+32x-24=2(x-3)(x-2)^2$,
voimme helposti päätellä, että polynomi saa arvon $0$ jos ja vain jos $x-3=0$ tai $x-2=0$. Yhtälön ainoat ratkaisut ovat siis $x=3$ ja $x=2$.

\begin{esimerkki}
Ratkaistaan yhtälö $6x^3-24x^2+24x=0$ tulon nollasäännön avulla.

\begin{align*}
6x^3-24x^2+24x &= 0 \\
6(x^3-6x^2+9x) &= 0 \ \ \ \ &\emph{otetaan $5$ yhteiseksi tekijäksi} \\
6x(x^2-6x+9) &= 0 &\emph{otetaan $x$ yhteiseksi tekijäksi} \\
6x(x^2-2\cdot 3\cdot x+3^2)  &= 0 \\
6x(x-3)^2 &= 0 \\
6x(x-3)(x-3) &= 0 \\
x=0 \textrm{\quad tai}& \quad x-3=0 \\
x=0 \textrm{\quad tai}& \quad x=3
\end{align*}

Vastaus: $x=0$ tai $x=3$.
\end{esimerkki}

\section{Harjoitustehtäviä}

\begin{tehtava}
    Ratkaise seuraavat yhtälöt tulon nollasääntöä hyödyntäen.
    \begin{enumerate}
        \item $x^2(3+x)=0$
        \item $0=x^3(x-5)$
        \item $(x-4)(x^2-4)=0$
    \end{enumerate}
    \begin{vastaus}
        \begin{enumerate}
            \item $x=0$ tai $x=-3$
            \item $x=0$ tai $x=5$
            \item $x=-2$, $x=2$ tai $x=4$
        \end{enumerate}
    \end{vastaus}
\end{tehtava}


\begin{tehtava}
    Ratkaise seuraavat yhtälöt tulon nollasääntöä hyödyntäen.
    \begin{enumerate}
        \item $(\smiley{}+1)\cdot (t+1)=0$
        \item $x(x-5)=0$
        \item $(2w+2)^2=0$
    \end{enumerate}
    \begin{vastaus}
        \begin{enumerate}
            \item $\smiley{}=-1$ tai $t=-1$,\qquad  Symboli $\smiley{}$ esittää jotain lukua, sillä muutoin laskutoimitukset eivät olisi mielekkäitä. Tehtävässä ei myöskään ole selvää, minkä muuttujan suhteen yhtälö pitäisi ratkaista. Siksi on ratkaistu molempien muuttujien suhteen.
            \item $x=0$ tai $x=5$
            \item $w=-1$
        \end{enumerate}
    \end{vastaus}
\end{tehtava}

\begin{tehtava}
	Sievennä seuraava lauseke: $(a-x)\cdot(b-x)\cdot(c-x)\cdot...\cdot(\mathring{a}-x)\cdot(\ddot{a}-x)\cdot(\ddot{o}-x)$.
    \begin{vastaus}
		Tulossa esiintyy tekijänä $(x-x)=0$. Niinpä tulon nollasäännön mukaan
		\begin{align*}
 			&(a-x)\cdot(b-x)\cdot(c-x)\cdot...\cdot(x-x)\cdot(y-x)\cdot(z-x)\cdot(\mathring{a}-x)\cdot(\ddot{a}-x)\cdot(\ddot{o}-x) \\
 			=&(a-x)\cdot(b-x)\cdot(c-x)\cdot...\cdot 0\cdot(y-x)\cdot(z-x)\cdot(\mathring{a}-x)\cdot(\ddot{a}-x)\cdot(\ddot{o}-x) \\
 			=&0
		\end{align*}
   \end{vastaus}
\end{tehtava}
