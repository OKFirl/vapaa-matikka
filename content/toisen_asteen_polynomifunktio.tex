\chapter{Toisen asteen polynomin kuvaaja}

Tässä pienessä luvussa perustellaan vihdoin, miksi kaikkien toisen asteen funktioiden
kuvaajat näyttävät samalta. Lisäksi tarkastellaan, mitkä tekijät vaikuttavat kuvaajien
muotoon.

\underline{Funktio $P(x)=x^2$}

Aloitetaan funktiosta $P(x)=x^2$. Mitä tiedämme siitä piirtämättä kuvaajaa?
\begin{itemize}
\item Funktion pienin arvo on $P(0) = 0$, sillä tulon merkkisäännön perusteella $x^2$ ei voi saada negatiivisia arvoja.
\item Kun $x > 0$, lauseke $x^2$  on sitä
suurempi, mitä suurempi $x$ on.
\item Koska $(-x)^2 = x^2$, kuvaaja on symmetrinen $y$-akselin suhteen. 
\end{itemize}  

Näiden tietojen avulla voimme päätellä, että funktion kuvaaja koostuu kahdesta
identtisestä haarasta, jotka kohtaavat, kun $x=0$. Mitä kauempana $x$ on nollasta,
sitä suurempia ovat funktion arvot. Tämän kaiken voi päätellä jo ennen kuvaajan
piirtämistä.

Tässä kuvaaja nyt on:
\begin{center}
\begin{kuvaajapohja}{2}{-2}{2}{-1}{4}
  \kuvaaja{x**2}{$f(x)=x^2$}{blue}
\end{kuvaajapohja}
\end{center}
Kuvaajan muoto on geometriselta nimeltään \emph{paraabeli}. Paraabeleja esiintyy monessa yhteydessä: esimerksi polttopeilin pinta kaareutuu paraabelin muotoisena. Samoin ilmaan heitetyn kappaleen
rata on likimain (ylösalainen) paraabeli, kun ilmanvastus on pieni.

\underline{Funktio $P(x)=ax^2, \quad a\neq 0$}

Polynomien $P(x)=ax^2$ arvot ovat ovat lausekkeeseen $x^2$ nähden $a$-kertaisia. Paraabelin symmetrisyys ja muut keskeiset ominaisuuden siis säilyvät. 

\begin{itemize}
\item Kun $a > 0$, myös $ax^2\geq 0$, joten pienin arvo on yhä $0$.\\
Paraabeli on \termi{ylöspäin aukeava}.
\item Kun $a < 0$, tulon merkkisäännön nojalla $ax^2 \leq 0$. \\
 Nyt $P(0)=0$ onkin suurin arvo, ja funktion arvot ovat sitä pienempiä,
mitä enemmän $x$ poikkeaa nollasta. \\
Paraabeli on \termi{alaspäin aukeava}.
\item Mitä enemmän $a$ poikkeaa nollasta, sitä nopeammin funktion arvot
muuttuvat $x$:n muuttuessa, ja sitä kapeampi paraabeli on.
\end{itemize}

\begin{center}
$a>0$, paraabeli aukeaa ylöpäin:\\
\begin{tabular}{cc}
$f(x)=\frac{1}{2}x^2$& $f(x)=2x^2$ \\ 
\begin{kuvaajapohja}{1}{-2}{2}{-1}{4}
  \kuvaaja{0.5*x**2}{}{blue}
\end{kuvaajapohja} &
\begin{kuvaajapohja}{1}{-2}{2}{-1}{4}
  \kuvaaja{2*x**2}{}{blue}
\end{kuvaajapohja}
\end{tabular}

$a<0$, paraabeli aukeaa alaspäin:\\
\begin{tabular}{cc}
$f(x)=-\frac{1}{2}x^2$ & $f(x)=-2x^2$ \\
\begin{kuvaajapohja}{1}{-2}{2}{-4}{1}
  \kuvaaja{-0.5*x**2}{}{blue}
\end{kuvaajapohja} &
\begin{kuvaajapohja}{1}{-2}{2}{-4}{1}
  \kuvaaja{-2*x**2}{}{blue}
\end{kuvaajapohja}
\end{tabular}
\end{center}

\underline{Funktio $P(x)=ax^2+c$}

Lisäämällä vakiotermin $c$ saadaan $P(x)=ax^2+c$. Vakion lisääminen nostaa tai laskee funktion kuvaajaa (riippuen siitä, onko $c > 0$ tai $c<0$), joten kuvaajan muoto ei muutu.

\underline{Funktio $P(x)=ax^2+bx+c$}

Miksi sitten täydellisen toisen asteen polynomin $P(x)=ax^2+bx+c$ kuvaaja on myös paraabeli? Muokataan lauseketta, aloitetaan ottamalla yhteinen tekijä:
\begin{align*}
P(x)=ax^2+bx+c &= a\left(x^2 +\frac{b}{a}x\right) + c  \quad &  
\text{ lavennetaan } \frac{b}{a} \text{ kahdella} \\
&= a\left(x^2 +2\cdot x \cdot \frac{b}{2a} \quad\right) + c  &  
\text{ täydennetään neliöksi lisäämällä} \left( \frac{b}{2a} \right)^2 \\
&= a \Bigg( \underbrace{x^2 +2\cdot x \cdot \frac{b}{2a}+\left(\frac{b}{2a} \right)^2}_{\left( x+\frac{b}{2a} \right)^2}
- \left(\frac{b}{2a}\right)^2 \Bigg)  + c \\
&= a \left( \left( x + \frac{b}{2a} \right )^2-\frac{b^2}{4a^2} \right) + c &
\text{ kerrotaan sulut auki } \\
&= a \underbrace{\left(  x + \frac{b}{2a} \right)^2}_{\text{neliö}}-
\underbrace{\frac{b^2}{4a} + c}_{\text{vakio}}
\end{align*}

Tästä neliöksi täydennetyksi muodosta nähdään, että $P(x)$ on muotoa
$a\cdot$neliö + vakio. Kuvaaja on siis samanlainen kuin tapauksessa
$ax^2+c$, huippu on vain siirtynyt.
Koska neliö $\geq 0$, saadaan

\begin{itemize}
\item Kun $a>0$, kyseinen vakio on polynomin pienin arvo ja kuvaaja on
ylöspäin aukeava paraabeli.
\item Kun $a<0$, kyseinen vakio on suurin arvo ja kuvaaja on alaspäin
aukeava paraabeli.
\end{itemize}

Paraabelin \termi{huippu} (eli kuvaajan piste, jossa suurin tai pienin arvo
saavutetaan) on aina kohdassa
$x=-\frac{b}{2a}$, koska silloin neliö on nolla.

%Toisen asteen polynomifunktio on muotoa $f(x)=ax^2+bx+c$, jossa $a,b,c \in \mathbb{R}$ ja $a \neq 0$. Toisen asteen polynomifunktion kuvaaja on paraabeli. Toisen asteen polynomifunktioita käytetään matemaattisessa mallinnuksessa talouden, tieteen ja tekniikan aloilla. Esimerkiksi heittoliikkeessä olevan kappaleen lentorata on aina paraabelin muotoinen. \\
%\textbf{Esimerkki 1.}
%a) Piirrä funktion $f(x)=x^2-2$ kuvaaja. \\
%b) Ratkaise funktion $f$ nollakohdat. \\ \\
%
%\begin{kuvaajapohja}{1}{-2}{2}{-3}{1}
%  \kuvaaja{x**2-2}{$f(x)=x^2-2$}{blue}
%\end{kuvaajapohja}
%
%Funtkion kuvaaja on ylöspäin aukeava paraabeli, joka leikkaa x-akselin kohdissa joissa $f(x)=0$. \\
%Graafisesti funktion nollakohdat saadaan ratkaistua kuvaajasta. Kuvaajasta nähdään, että funktion nollakohdat ovat $x_1 \approx 1,4$ ja $x_2 \approx -1,4$. \\
%Algebrallisesti saadaan ratkaistua, että funktion nollakohdat ovat
%\begin{align*}
%f(x)&=0 \\
%x^2-2&=0 \\
%x^2&=2 \\
%x&= \pm \sqrt[]{2}
%\end{align*}
%Funktion $f$ kuvaajasta huomataan, että se on symmetrinen y-akselin suhteen.
%
%\textbf{Esimerkki 2.} \\
%Piirrä funktioiden $f(x)=x^2+1$, $g(x)=2x^2$ ja $h(x)=\frac{1}{3}x^2$ kuvaajat. \\ \\
%
%\begin{kuvaajapohja}{1}{-2}{2}{-1}{3}
%  \kuvaaja{x**2+1}{$f(x)=x^2+1$}{blue}
%\end{kuvaajapohja}
%
%
%\begin{kuvaajapohja}{1}{-2}{2}{-1}{3}
%  \kuvaaja{2*x**2}{$g(x)=2x^2$}{blue}
%\end{kuvaajapohja}
%
%\begin{kuvaajapohja}{1}{-2}{2}{-1}{3}
%  \kuvaaja{(1 / 3.0)*(x**2)}{$h(x)=\frac{1}{3}x^2$}{blue}
%\end{kuvaajapohja}
%
%Mitä tapahtuu funktion kuvaajan muodolle, kun termin $ax^2$ kerroin $a$ muuttuu? \\ \\
%
%\textbf{Esimerkki 3.} \\
%Piirrä funktioiden $f(x)=-x^2+x$, $g(x)=-x^2+2x+1$ ja $h(x)=-x^2+\frac{1}{2}x-1$ kuvaajat. \\
%\missingfigure \\
%Mitä tapahtuu funktion kuvaajan muodolle, kun termin $bx$ kerroin $b$ muuttuu? \\ \\
%
%\textbf{Esimerkki 4.} \\
%Piirrä funktioiden $f(x)=x^2$, $g(x)=x^2-2$ ja $h(x)=x^2+\frac{3}{2}$ kuvaajat. \\ \\
%Mitä tapahtuu funktion kuvaajan muodolle, kun vakiotermi $c$ muuttuu? \\ \\

Koottuna:

\laatikko{Toisen asteen polynomifunktion $f(x)=ax^2+bx+c$ kuvaaja on
\begin{itemize}
\item ylöspäin aukeava paraabeli, kun $a>0$
\item alaspäin aukeava paraabeli, kun $a<0$
\item sitä kapeampi, mitä suurempi $|a|$ on.
\end{itemize}
}

%\textbf{Esimerkki 5.} \\
%Ratkaise funktion $f(x)=4x^2-13x+8$ nollakohdat.
%\begin{align*}
%f(x)&=0 \\
%4x^2-13x+8&=0 \\
%x&=\frac{-(-13) \pm \sqrt[]{(-13)^2-4 \cdot 4 \cdot 8}}{2 \cdot 4} \\
%x&=\frac{13 \pm \sqrt[]{169-128}}{8} \\
%x&=\frac{13 \pm \sqrt[]{41}}{8} \\
%x&=\frac{13 \pm \sqrt[]{41}}{8}
%\end{align*}

\section{Harjoitustehtäviä}

\begin{tehtava}
  Aukeavatko seuraavat paraabelit ylös- vai alaspäin?
  \begin{enumerate}[a)]
    \item $4x^2 + 100x - 3$
    \item $-x^2 + 1337$
    \item $5x^2 - 7x + 5$
    \item $-6(-3x^2 + 5)$
    \item $-13x(9 - 17x)$ 
    \item $100(1-x^2)$
  \end{enumerate}

  \begin{vastaus}
    \begin{enumerate}[a)]
      \item Ylös
      \item Alas
      \item Ylös
      \item Ylös
      \item Ylös
      \item Alas
    \end{enumerate}
  \end{vastaus}
\end{tehtava}

\begin{tehtava}
  \begin{enumerate}[a)]
    \item Ratkaise funktion $2x^2 - 5x - 3$ nollakohdat
    \item Millä arvoilla edellisen kohdan funktio $2x^2 - 5x - 3$ saa positiivisia arvoja?
    \item Onko em. funktiolla globaali raja-arvo (minimi tai maksimi), ja jos on, missä kohtaa funktio saa tämän arvon? Mikä on funktion arvo silloin?
  \end{enumerate}

  \begin{vastaus}
    \begin{enumerate}[a)]
      \item $x = 1.2$ tai $x = -0.2$
      \item $x = \frac{12}{10} = 1.2$ tai $x = -\frac{2}{10} = -0.2$
      \item Koska neliötermin kerroin a on positiivinen (2), funktiolla on globaali minimi (mutta ei ylärajaa). Symmetrian vuoksi minimi on nollakohtien puolivälissä kohdassa 0.5, jossa funktio saa siis pienimmän arvonsa -5.
    \end{enumerate}
  \end{vastaus}
\end{tehtava}

\begin{tehtava}
  Tutki, millä muuttujan x arvoilla seuraavat funktiot saavat positiivisia arvoja.
  \begin{enumerate}[a)]
    \item $x^2 - 4$
    \item $-x^2 - 2x + 3$
    \item $x^2 + 2x + 5$
    \item $-x^2 - 1$    
  \end{enumerate}

  \begin{vastaus}
    \begin{enumerate}[a)]
      \item $x \leq -2$ tai $x \geq 2$
      \item $-3 \geq x \leq 1$
      \item Kaikilla x:n arvoilla.
      \item Ei millään x:n arvoilla.
    \end{enumerate}
  \end{vastaus}
\end{tehtava}