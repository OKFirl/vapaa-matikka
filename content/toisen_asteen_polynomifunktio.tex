\chapter{Paraabelin muoto}

Ville työstää tätä nyt.

Tässä pienessä luvussa perustellaan vihdoin, miksi kaikkien toisen asteen funktioiden
kuvaajat näyttävät samalta. Lisäksi tarkastellaan, mitkä tekijät vaikuttavat kuvaajien
muotoon.

Aloitetaan funktiosta $P(x)=x^2$. Mitä tiedämme siitä piirtämättä kuvaajaa?
\begin{itemize}
\item Funktion pienin arvo on $P(0) = 0$, sillä tulon merkkisäännön perusteella $x^2$ ei voi saada negatiivisia arvoja.
\item Kun $x > 0$, lauseke $x^2$  on sitä
suurempi, mitä suurempi $x$ on.
\item Koska $(-x)^2 = x^2$, kuvaaja on symmetrinen $y$-akselin suhteen. 
\end{itemize}  

Näiden tietojen avulla voimme päätellä, että funktion kuvaaja koostuu kahdesta
identtisestä haarasta, jotka kohtaavat, kun $x=0$. Mitä kauempana $x$ on nollasta,
sitä suurempia ovat funktion arvot. Tämän kaiken voi päätellä jo ennen kuvaajan
piirtämistä.

Tässä kuvaaja nyt on:
\begin{center}
\begin{kuvaajapohja}{2}{-2}{2}{-1}{4}
  \kuvaaja{x**2}{$f(x)=x^2$}{blue}
\end{kuvaajapohja}
\end{center}
Kuvaajan muoto on geometriselta nimeltään \emph{paraabeli}. Paraabeleja esiintyy monessa yhteydessä: esimerksi polttopeilin pinta kaarautuu paraabelin muotoisena. Samoin ilmaan heitetyn kappaleen
rata on likimain (ylösalainen) paraabeli, kun ilmanvastus on pieni.

Polynomien $P(x)=ax^2$ arvot ovat ovat lausekkeeseen $x^2$ nähden $a$-kertaisia. Kun
$a\neq 0$ 

%Toisen asteen polynomifunktio on muotoa $f(x)=ax^2+bx+c$, jossa $a,b,c \in \mathbb{R}$ ja $a \neq 0$. Toisen asteen polynomifunktion kuvaaja on paraabeli. Toisen asteen polynomifunktioita käytetään matemaattisessa mallinnuksessa talouden, tieteen ja tekniikan aloilla. Esimerkiksi heittoliikkeessä olevan kappaleen lentorata on aina paraabelin muotoinen. \\
%\textbf{Esimerkki 1.}
%a) Piirrä funktion $f(x)=x^2-2$ kuvaaja. \\
%b) Ratkaise funktion $f$ nollakohdat. \\ \\
%
%\begin{kuvaajapohja}{1}{-2}{2}{-3}{1}
%  \kuvaaja{x**2-2}{$f(x)=x^2-2$}{blue}
%\end{kuvaajapohja}
%
%Funtkion kuvaaja on ylöspäin aukeava paraabeli, joka leikkaa x-akselin kohdissa joissa $f(x)=0$. \\
%Graafisesti funktion nollakohdat saadaan ratkaistua kuvaajasta. Kuvaajasta nähdään, että funktion nollakohdat ovat $x_1 \approx 1,4$ ja $x_2 \approx -1,4$. \\
%Algebrallisesti saadaan ratkaistua, että funktion nollakohdat ovat
%\begin{align*}
%f(x)&=0 \\
%x^2-2&=0 \\
%x^2&=2 \\
%x&= \pm \sqrt[]{2}
%\end{align*}
%Funktion $f$ kuvaajasta huomataan, että se on symmetrinen y-akselin suhteen.
%
%\textbf{Esimerkki 2.} \\
%Piirrä funktioiden $f(x)=x^2+1$, $g(x)=2x^2$ ja $h(x)=\frac{1}{3}x^2$ kuvaajat. \\ \\
%
%\begin{kuvaajapohja}{1}{-2}{2}{-1}{3}
%  \kuvaaja{x**2+1}{$f(x)=x^2+1$}{blue}
%\end{kuvaajapohja}
%
%
%\begin{kuvaajapohja}{1}{-2}{2}{-1}{3}
%  \kuvaaja{2*x**2}{$g(x)=2x^2$}{blue}
%\end{kuvaajapohja}
%
%\begin{kuvaajapohja}{1}{-2}{2}{-1}{3}
%  \kuvaaja{(1 / 3.0)*(x**2)}{$h(x)=\frac{1}{3}x^2$}{blue}
%\end{kuvaajapohja}
%
%Mitä tapahtuu funktion kuvaajan muodolle, kun termin $ax^2$ kerroin $a$ muuttuu? \\ \\
%
%\textbf{Esimerkki 3.} \\
%Piirrä funktioiden $f(x)=-x^2+x$, $g(x)=-x^2+2x+1$ ja $h(x)=-x^2+\frac{1}{2}x-1$ kuvaajat. \\
%\missingfigure \\
%Mitä tapahtuu funktion kuvaajan muodolle, kun termin $bx$ kerroin $b$ muuttuu? \\ \\
%
%\textbf{Esimerkki 4.} \\
%Piirrä funktioiden $f(x)=x^2$, $g(x)=x^2-2$ ja $h(x)=x^2+\frac{3}{2}$ kuvaajat. \\ \\
%Mitä tapahtuu funktion kuvaajan muodolle, kun vakiotermi $c$ muuttuu? \\ \\

\laatikko{Toisen asteen polynomifunktion $f(x)=ax^2+bx+c$ kuvaaja on
\begin{enumerate}
\item{Ylöspäin aukeava paraabeli, jos $a>0$.}
\item{Alaspäin aukeava paraabeli, jos $a<0$.}
\end{enumerate}
Toisen asteen polynomifunktio leikkaa y-akselin pisteessä $(0,c)$. \\
Toisen asteen polynomifunktio on sitä kapeampi, mitä suurempi on $|a|$.
}

%\textbf{Esimerkki 5.} \\
%Ratkaise funktion $f(x)=4x^2-13x+8$ nollakohdat.
%\begin{align*}
%f(x)&=0 \\
%4x^2-13x+8&=0 \\
%x&=\frac{-(-13) \pm \sqrt[]{(-13)^2-4 \cdot 4 \cdot 8}}{2 \cdot 4} \\
%x&=\frac{13 \pm \sqrt[]{169-128}}{8} \\
%x&=\frac{13 \pm \sqrt[]{41}}{8} \\
%x&=\frac{13 \pm \sqrt[]{41}}{8}
%\end{align*}

\section{Harjoitustehtäviä}

\begin{tehtava}
  Aukeavatko seuraavat paraabelit ylös- vai alaspäin?
  \begin{enumerate}[a)]
    \item $4x^2 + 100x - 3$
    \item $-x^2 + 1337$
    \item $-6(-3x^2 + 5)$
    \item $-13x(9 - 17x)$ 
  \end{enumerate}

  \begin{vastaus}
    \begin{enumerate}[a)]
      \item Ylös
      \item Alas
      \item Ylös
      \item Ylös
    \end{enumerate}
  \end{vastaus}
\end{tehtava}

\begin{tehtava}
  \begin{enumerate}[a)]
    \item Ratkaise funktion $2x^2 - 5x - 3$ nollakohdat
    \item Millä arvoilla edellisen kohdan funktio $2x^2 - 5x - 3$ saa positiivisia arvoja?
    \item Onko em. funktiolla globaali raja-arvo (minimi tai maksimi), ja jos on, missä kohtaa funktio saa tämän arvon? Mikä on funktion arvo silloin?
  \end{enumerate}

  \begin{vastaus}
    \begin{enumerate}[a)]
      \item $x = 1.2$ tai $x = -0.2$
      \item $x = \frac{12}{10} = 1.2$ tai $x = -\frac{2}{10} = -0.2$
      \item Koska neliötermin kerroin a on positiivinen (2), funktiolla on globaali minimi (mutta ei ylärajaa). Symmetrian vuoksi minimi on nollakohtien puolivälissä kohdassa 0.5, jossa funktio saa siis pienimmän arvonsa -5.
    \end{enumerate}
  \end{vastaus}
\end{tehtava}

\begin{tehtava}
  Tutki, millä muuttujan x arvoilla seuraavat funktiot saavat positiivisia arvoja.
  \begin{enumerate}[a)]
    \item $x^2 - 4$
    \item $-x^2 - 2x + 3$
    \item $x^2 + 2x + 5$
    \item $-x^2 - 1$    
  \end{enumerate}

  \begin{vastaus}
    \begin{enumerate}[a)]
      \item $x \leq -2$ tai $x \geq 2$
      \item $-3 \geq x \leq 1$
      \item Kaikilla x:n arvoilla.
      \item Ei millään x:n arvoilla.
    \end{enumerate}
  \end{vastaus}
\end{tehtava}