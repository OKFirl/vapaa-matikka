\chapter{Toisen asteen polynomifunktio}
Toisen asteen polynomifunktio on muotoa $f(x)=ax^2+bx+c$, jossa $a,b,c \in \mathbb{R}$ ja $a \neq 0$. Toisen asteen polynomifunktion kuvaaja on paraabeli. Toisen asteen polynomifunktioita käytetään matemaattisessa mallinnuksessa talouden, tieteen ja tekniikan aloilla. Esimerkiksi heittoliikkeessä olevan kappaleen lentorata on aina paraabelin muotoinen. \\
\textbf{Esimerkki 1.}
a) Piirrä funktion $f(x)=x^2-2$ kuvaaja. \\ 
b) Ratkaise funktion $f(x)$ nollakohdat. \\ \\
%funktion kuvaaja$
Funtkion kuvaaja on ylöspäin aukeava paraabeli, joka leikkaa x-akselin kohdissa joissa $f(x)=0$. \\
Graafisesti funktion nollakohdat saadaan ratkaistua kuvaajasta. Kuvaajasta nähdään, että funktion nollakohdat ovat $x_1 \approx 1,4$ ja $x_2 \approx -1,4$. \\
Algebrallisesti saadaan ratkaistua, että funktion nollakohdat ovat 
\begin{align*}
f(x)=0 \\
x^2-2=0 \\
x^2=2 \\
x= \pm \sqrt[]{2}
\end{align*}
Funktion $f$ kuvaajasta huomataan, että se on symmetrinen y-akselin suhteen. 

\textbf{Esimerkki 2.} \\
Piirrä funktioiden $f(x)=x^2+1$, $g(x)=2x^2$ ja $h(x)=\frac{1}{3}x^2$ kuvaajat. \\ \\

Mitä tapahtuu funktion kuvaajan muodolle, kun termin $ax^2$ kerroin $a$ muuttuu? \\ \\

\textbf{Esimerkki 3.} \\
Piirrä funktioiden $f(x)=-x^2+x$, $g(x)=-x^2+2x+1$ ja $h(x)=-x^2+\frac{1}{2}x-1$ kuvaajat. \\ \\
Mitä tapahtuu funktion kuvaajan muodolle, kun termin $bx$ kerroin $b$ muuttuu? \\ \\

\textbf{Esimerkki 4.} \\
Piirrä funktioiden $f(x)=x^2$, $g(x)=x^2-2$ ja $h(x)=x^2+\frac{3}{2}$ kuvaajat. \\ \\
Mitä tapahtuu funktion kuvaajan muodolle, kun vakiotermi $c$ muuttuu? \\ \\

\laatikko{Toisen asteen polynomifunktion $f(x)=ax^2+bx+c$ kuvaaja on
\begin{enumerate}
\item{Ylöspäin aukeava paraabeli, jos $a>0$.}
\item{Alaspäin aukeava paraabeli, jos $a<0$.}
\end{enumerate}
Toisen asteen polynomifunktio leikkaa y-akselin pisteessä $(0,c)$. \\
Toisen asteen polynomifunktio on sitä kapeampi, mitä suurempi on $|a|$. 
}
%kuvaajat 
\section{Harjoitustehtäviä}
