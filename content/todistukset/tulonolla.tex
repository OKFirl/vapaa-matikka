\section{Tulon nollasääntö}
\label{tod:tulonolla}

\begin{todistus}
Annettuna joukko nollasta poikkeavia lukuja $x_0, x_1, x_2 ... , x_n$ asetetaan
\begin{align*}
    x_0 \cdot x_1 \cdot x_2 \cdot ... \cdot x_n &= 0 & \ppalkki & : (x_0 \cdot x_1 \cdot x_2 \cdot ... \cdot x_{n-1}) \\
    & & & \text{koska kaikille $x_i$ pätee $x_i \neq 0$} \\
    \\
    \frac{x_0 \cdot x_1 \cdot x_2 \cdot ... \cdot x_n}{x_0 \cdot x_1 \cdot x_2 \cdot ... \cdot x_{n-1}} &=
    \frac{0}{x_0 \cdot x_1 \cdot x_2 \cdot ... \cdot x_{n-1}} & \ppalkki & \frac{0}{x_i} = 0\ \text{kaikilla $x_i,\ x_i \neq 0$} \\
    \\
    x_n &= 0 & &
\end{align*}

mikä on ristiriidassa todistuksessa asetetun vaatimuksen kanssa (kaikkien
lukujen piti olla nollasta poikkeavia). Näin alkuperäinen väite on tosi.

\end{todistus}