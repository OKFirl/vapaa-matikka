\chapter{Toisen asteen yhtälö}
\begin{esimerkki}
Milloin funktio $f(x)=x^2+2x+1$ leikkaa x-akselin? \\ \\
\textbf{Ratkaisu:} \\
Funktion kuvaaja leikkaa x-akselin, kun $f(x)=0$. Tätä muuttujan $x$ arvoa kutsutaan funktion $f$ \textbf{nollakohdaksi}. Piirretään funktion $f$ kuvaaja ja etsitään ne kohdat, joissa funktio leikkaa x-akselin. %insert kuvaaja.pic
%funktio käsite on tuttu, joten itse esittelisin tämän asian näin -Lauri

\begin{kuvaajapohja}{1.5}{-2.5}{0.5}{-0.5}{3}
\kuvaaja{x**2+2*x+1}{$f(x)=x^2+2x+1$}{red}
\end{kuvaajapohja}

\end{esimerkki}

Kuvaajasta huomataan, että funktion $f$ nollakohta on $x \approx -1$. Tätä toisen asteen yhtälön ratkaisumenetelmää kutsutaan graafiseksi ratkaisuksi.
Määritettäessä toisen asteen polynomifunktion nollakohtia päädytään \textbf{toisen asteen yhtälöön}, joka on aina saatettavissa yleiseen muotoon
\begin{align*}
ax^2+bx+c=0, \ \ \ a, \ b, \ c  \in \mathbb{R}, a \neq 0.
\end{align*}
Toisen asteen polynomifunktion kuvaaja on aina \textbf{paraabeli}. \\

\laatikko{Toisen asteen yhtälöllä on ratkaisuja joko 0, 1 tai 2 kappaletta.}

Ohessa kuvat eri tapauksista:

\begin{tabular}{c c}

\begin{tabular}{c}
	2 ratkaisua, a positiivinen\\
	\begin{lukusuora}{-1}{1}{4}
	\lukusuoraparaabeli{-0.5}{0.5}{-1}
	\end{lukusuora}
\end{tabular}

&

\begin{tabular}{c}
	2 ratkaisua, a negatiivinen\\
	\begin{lukusuora}{-1}{1}{4}
	\lukusuoraparaabeli{-0.5}{0.5}{1}
	\end{lukusuora}
\end{tabular}

\\ \qquad & \qquad \\

\begin{tabular}{c}
	1 ratkaisu, a positiivinen\\
	\begin{lukusuora}{-2}{2}{4}
	\lukusuorakuvaaja{x**2}
	\end{lukusuora}
\end{tabular}

&

\begin{tabular}{c}
	1 ratkaisu, a negatiivinen\\
	\begin{lukusuora}{-2}{2}{4}
	\lukusuorakuvaaja{-x**2}
	\end{lukusuora}
\end{tabular}

\\ \qquad & \qquad \\

\begin{tabular}{c}
	ei ratkaisuja, a positiivinen\\
	\begin{lukusuora}{-2}{2}{4}
	\lukusuorakuvaaja{x**2+0.3}
	\end{lukusuora}
\end{tabular}

&

\begin{tabular}{c}
	ei ratkaisuja, a negatiivinen\\
	\begin{lukusuora}{-2}{2}{4}
	\lukusuorakuvaaja{-x**2-0.3}
	\end{lukusuora}
\end{tabular}

\\ \qquad & \qquad \\

\end{tabular}

Toisen asteen yhtälö voidaan ratkaista graafisesti tai algebrallisesti. Graafinen ratkaisu on aina likimääräinen eli arvio oikeasta ratkaisusta. \\ \\ Edellä tutustuimme toisen asteen yhtälön graafiseen ratkaisemiseen. Seuraavaksi tutustumme toisen asteen yhtälön algebralliseen ratkaisemiseen.
\section{Vaillinaiset yhtälöt}
Jos toisen asteen yhtälössä $ax^2+bx+c$ missä $a\neq 0$ pätee joko $b = 0$ tai
$c = 0$ on kyseessä niin sanottu \termi{vaillinainen toisen asteen yhtälö}, joka on
siis muotoa
\[ax^2+c=0\]
tai
\[ax^2+bx=0.\]
\subsection*{Toisen asteen yhtälö $ax^2+c=0$}
Muotoa $ax^2+c = 0$ oleva toisen asteen yhtälö saadaan helposti ratkaistua neliöjuuren avulla.

\begin{esimerkki}
Ratkaistaan yhtälö $5x^2-45=0$:
\begin{align*}
5x^2-45 &= 0 &&\ppalkki : 5 \\
x^2-9 &= 0 &&\ppalkki + 9 \\
x^2 &= 9 &&\text{ratkaistaan käyttäen neliöjuurta ($9 \geq 0$)} \\
x &= \pm \sqrt{9} = \pm 3.
\end{align*}
\end{esimerkki}

\begin{esimerkki}
Ratkaistaan yhtälö $13x^2-42=-3$:
\begin{align*}
13x^2-42 &= -3 &&\ppalkki + 42 \\
13x^2 &= 39 &&\ppalkki : 13 \\
x^2 &= 3 &&\text{ratkaistaan käyttäen neliöjuurta ($3 \geq 0$)} \\
x &= \pm \sqrt{3}.
\end{align*}
\end{esimerkki}

\begin{esimerkki}
Ratkaistaan yhtälö $x^2+4=3$:
\begin{align*}
x^2+4 &= 3 &&\ppalkki - 4 \\
x^2 &= -1
\end{align*}
Koska $x^2 \geq 0$ kaikilla $x$, yhtälöllä ei ole ratkaisua.
\end{esimerkki}

\subsection*{Toisen asteen yhtälö $ax^2+bx=0$}
Jos yhtälössä $ax^2+bx+c=0$ kerroin $c=0$, niin toisen asteen yhtälö saa muodon
\begin{align*}
ax^2+bx+0&=0 \\
ax^2+bx=0
\end{align*}
Toisen asteen yhtälöä, joka on muotoa $ax^2+bx=0$ kutsutaan vaillinaiseksi toisen asteen yhtälöksi. Tämän muotoinen yhtälö saadaan ratkaistua tulon nollasäännön avulla. \\
\begin{align*}
ax^2+bx&=0 \ \ \ \ \ &&\ppalkki\text{otetaan yhteinen tekijä} \\
x(ax+b)&=0 \ \ \ \ \ &&\ppalkki\text{tulon nollasääntö} \\
x&=0 \text{ tai } ax+b=0 \\
x&=0 \text{ tai } ax=-b \\
x&=0 \text{ tai } x=-\frac{b}{a}
\end{align*}
\textbf{Esimerkki 3.}
Ratkaise yhtälö $x^2-11x=0$
\begin{align*}
x^2-11x&=0 \ \ \ \ \  &&\ppalkki\text{otetaan yhteinen tekijä} \\
x(x-11)&=0 \ \ \ \ \ &&\ppalkki\text{tulon nollasääntö} \\
x&=0 \text{ tai } x-11=0 \\
x&=0 \text{ tai } x=11
\end{align*}
\textbf{Esimerkki 4.}
Ratkaise yhtälö $55x^2+8x=0$
\begin{align*}
55x^2+8x&=0 \ \ \ \ \ &&\ppalkki\text{otetaan yhteinen tekijä} \\
x(55x+8)&=0 \ \ \ \ \ &&\ppalkki\text{tulon nollasääntö} \\
x&=0 \text{ tai } 55x+8=0 \\
x&=0 \text{ tai } 55x=-8 \\
x&=0 \text{ tai } x=-\frac{8}{55}
\end{align*}
\section{Neliöksi täydentäminen}
Toisen asteen yhtälöä $ax^2+bx+c=0$, jossa $a,b,c \neq 0$ kutsutaan
täydelliseksi toisen asteen yhtälöksi. Tällaiset yhtälöt voidaan palauttaa
vaillinaisiksi toisen asteen yhtälöiksi neliöön täydentämiseksi kutsutulla
tekniikalla. Seuraavassa luvussa todistamme neliöön täydentämisen avulla
yleisen ratkaisukaavan toisen asteen yhtälöille.

\begin{esimerkki}
Haluaisimme ratkaista yhtälön $x^2+4x-16 = 0$. Jotta se saataisiin muotoon
jossa on neliö ja vakiotermi eli $(Ax+B)^2-C=0$, täytyy molempien yhtälöiden
vasempien puolten polynomien olla samat, eli laskemalla muistikaavan avulla
$(Ax+B)^2-C=0$ saadaan
\[A^2x^2+2ABx+B^2-x=x^2+4x-16.\]
Haluamme siis valita kertoimet $A$, $B$ ja $C$ siten, että
\[\left\{\begin{array}{l}
A^2=1 \\ 2AB=4 \\ B^2-C=-16.
\end{array}\right.\]
Valitaan $A$:ksi ensimmäisen yhtälön eräs ratkaisu $A = 1$. Nyt toinen yhtälö
on muotoa $2B=4$ eli $B = 2$. Siis viimeisen yhtälön mukaan voidaan valita
$C = 16+B^2 = 20$.

Näiden valintojen avulla voimme nyt ratkaista yhtälön:
\begin{align*}
x^2+4x-16 &= 0 &&\ppalkki + 20 \text{ ($C = 20$ oikealle puolelle)} \\
x^2+4x+4 &= 20 &&\text{vasen puoli $ = (Ax+B)^2 = (x+2)^2$} \\
(x+2)^2 &= 20 &&\text{ratkaistaan vaillinainen toisen asteen yhtälö} \\
x+2 &= \pm \sqrt{20} &&\ppalkki - 2 \\
x &= -2 \pm \sqrt{20} &&\text{sievennetään vastaus} \\
x &= -2 \pm 2\sqrt{5}.
\end{align*}
\end{esimerkki}

\begin{esimerkki}
Ratkaistaan yhtälö $4x^2-4x-5=0$. Jotta saisimme polynomin $4x^2-4x-5$ muotoon
$(Ax+B)^2-C$, täytyy olla $A^2 = 4$. Valitaan toinen ratkaisu $A = 2$. Täytyy
myös olla $2AB = -4$ eli $B=\frac{-4}{2A}=-1$. Jäljelle jää $B^2 - C = -5$ eli
$C = B^2+5 = 6$.

Nyt voimme ratkaista yhtälön.
\begin{align*}
4x^2-4x-5 &= 0 &&\ppalkki + 6 \text{ ($C = 6$ oikealle puolelle)} \\
4x^2-4x+1 &= 6 &&\text{vasen puoli $ = (Ax+B)^2 = (2x-1)^2$} \\
(2x-1)^2 &= 6 &&\text{ratkaistaan vaillinainen toisen asteen yhtälö} \\
2x-1 &= \pm \sqrt{6} &&\ppalkki + 1 \\
2x &= 1 \pm \sqrt{6} &&\ppalkki : 2 \\
x &= \frac{1\pm \sqrt{6}}{2}
\end{align*}
\end{esimerkki}

% \textbf{Esimerkki 5.} \\
% Ratkaise yhtälö $x^2+4x-16=0$. \\
% \textbf{Ratkaisu}
% \begin{align*}
% x^2+4x-16&=0 \ \ \ \ \ &&\ppalkki +20 \\
% x^2+4x+4&=20 \ \ \ \ \ &&\ppalkki a^2+2ab+b^2=(a+b)^2 \\
% (x+2)^2&=20 \ \ \ \ \ &&\ppalkki \sqrt[]{} \\
% x+2 &= \pm \sqrt[]{20} \\
% x&=-2 \pm \sqrt[]{20} \\
% x&=-2 \pm 2 \sqrt[]{5} \\
% \end{align*}
% \textbf{Esimerkki 6.} \\
% Ratkaise yhtälö $16x^2-64x+2=0$. \\
% \textbf{Ratkaisu}
% \begin{align*}
% 16x^2-16x+2&=0 \ \ \ \ \ &&\ppalkki +2 \\
% 16x^2-16x+4&=2 \ \ \ \ \ &&\ppalkki a^2-2ab+b^2=(a-b)^2 \\
% (4x+2)^2&=2 \ \ \ \ \ &&\ppalkki \sqrt[]{} \\
% 4x+2&=\pm \sqrt[]{2} \\
% 4x&=-2 \pm \sqrt[]{2} \\
% x&=-\frac{1}{2} \pm \frac{\sqrt[]{2}}{4} \\
% x&=-\frac{1}{2} \pm \frac{\sqrt[]{2}}{ 2\sqrt[]{2}\sqrt[]{2}} \\
% x&=-\frac{1}{2} \pm \frac{1}{2 \sqrt[]{2}} \\
% \end{align*}
%
% Ratkaisutapaa, jossa toisen asteen yhtälö täydennetään lisäämällä tai vähentämällä termejä binomin neliöksi, kutsutaan neliöksi täydentämiseksi.
%
% Toisen asteen yhtälö voidaan aina ratkaista neliöön täydentämällä. Yleensä toisen asteen yhtälöt kuitenkin ratkaistaan käyttämällä suoraa kaavaa.
%
% Seuraavassa kappaleessa johdamme toisen asteen yhtälön ratkaisukaavan neliöksi täydentämistä käyttäen.

\section{Harjoitustehtäviä}

\begin{tehtava}
    Ratkaise seuraavat yhtälöt.
    \begin{enumerate}[a)]
        \item $x^2 - 100 = 0$
        \item $x^2 + 100 = 0$
        \item $x^2 - 10 = 0$
        \item $x^2 + 10 = 0$
        \item $-x^2 - 25 = 0$
        \item $-x^2 + 25 = 0$
        \item $2x^2 - 98 = 0$
        \item $2x^2 + 98 = 0$
    \end{enumerate}
    \begin{vastaus}
        \begin{enumerate}[a)]
            \item $x=\pm10$
            \item Ei ratkaisuja.
            \item $x=\pm\sqrt{10}$
            \item Ei ratkaisuja.
            \item Ei ratkaisuja.
            \item $x=\pm5$
            \item $x=\pm7$
            \item Ei ratkaisuja.
        \end{enumerate}
    \end{vastaus}
\end{tehtava}

\begin{tehtava}
    Ratkaise seuraavat yhtälöt.
    \begin{enumerate}[a)]
        \item $x^2 - 72x = 0$
        \item $x^2 + 72x = 0$
        \item $x^2 - 56x = 0$
        \item $x^2 + 56x = 0$
        \item $-x^2 - 13x = 0$
        \item $-x^2 + 13x = 0$
        \item $2x^2 - 43x = 0$
        \item $2x^2 + 43x = 0$
    \end{enumerate}
    \begin{vastaus}
        \begin{enumerate}[a)]
            \item $x=0$ tai $x=72$
            \item $x=-72$ tai $x=0$
            \item $x=0$ tai $x=56$
            \item $x=-56$ tai $x=0$
            \item $x=-13$ tai $x=0$
            \item $x=0$ tai $x=13$
            \item $x=0$ tai $x=21,5$
            \item $x=-21,5$ tai $x=0$
        \end{enumerate}
    \end{vastaus}
\end{tehtava}

\begin{tehtava}
    Ratkaise seuraavat yhtälöt.
    \begin{enumerate}[a)]
        \item $x^2 - 36 = 0$
        \item $x^2 - 85x = 0$
        \item $x^2 + 11x = -6x$
        \item $x^2 + 10x = -4x^2$
    \end{enumerate}
    \begin{vastaus}
        \begin{enumerate}[a)]
            \item $x=\pm6$
            \item $x=0$ tai $x=85$
            \item $x=0$ tai $x=-17$
            \item $x=0$ tai $x=-2$
        \end{enumerate}
    \end{vastaus}
\end{tehtava}


\begin{tehtava}
    Ratkaise seuraavat yhtälöt.
    \begin{enumerate}[a)]
        \item $x^2 - 9 = 0$
        \item $2x^2 + 8 = 0$
        \item $-x^2 + 11 = -5$
        \item $3 - x^2 = -1 + 3x^2$
    \end{enumerate}
    \begin{vastaus}
        \begin{enumerate}[a)]
            \item $x=\pm3$
            \item Ei ratkaisuja.
            \item $x=\pm4$
            \item $x=\pm1$
        \end{enumerate}
    \end{vastaus}
\end{tehtava}

\begin{tehtava}
    Ratkaise seuraavat yhtälöt.
    \begin{enumerate}[a)]
        \item $x^2 - 3x = 0$
        \item $10x + 2x^2 = 0$
        \item $-3x^2 + 8x = -2x$
        \item $2x^2 - x^3 = 0$
    \end{enumerate}
    \begin{vastaus}
        \begin{enumerate}[a)]
            \item $x=0$ tai $x=3$
            \item $x=0$ tai $x=-5$
            \item $x=0$ tai $x=8$
            \item $x=0$ tai $x=2$
        \end{enumerate}
    \end{vastaus}
\end{tehtava}

\begin{tehtava}
    Ratkaise seuraavat yhtälöt.
    \begin{enumerate}[a)]
        \item $x^2+3x+2=0$
        \item $2x^2+5x-12=0$
        \item $3x^2-7x-20=0$
        \item $x^2+3x-5=0$
        \item $x^2+5x-24=0$
    \end{enumerate}
    \begin{vastaus}
        \begin{enumerate}[a)]
            \item $x=-2$ tai $x=-1$
            \item $x=3/2$ tai $x=-4$
            \item $x=4$ tai $x=-5/3$
            \item $x=\frac{3\pm\sqrt{29}}{2}$
            \item $x=3$ tai $x=-8$
        \end{enumerate}
    \end{vastaus}
\end{tehtava}

\begin{tehtava}
    Ratkaise seuraavat yhtälöt.
    \begin{enumerate}[a)]
        \item $x^2+3x-5=4x+8$
        \item $8x^2-5x+1=-36$
        \item $-3x^2-4x+2=-5x^2+3$
        \item $-3x^2+4x+13=-5x^2+10x+9$
    \end{enumerate}
    \begin{vastaus}
        \begin{enumerate}[a)]
            \item $\frac{1\pm\sqrt{53}}{2}$
            \item Ei ratkaisua reaalilukujen joukossa.
            \item $1\pm\frac{\sqrt{6}}{2}$
            \item $x=1$ tai $x=2$
        \end{enumerate}
    \end{vastaus}
\end{tehtava}

\begin{tehtava}
    Elokuvassa \emph{Dredd} pudotetaan ihmisiä kuolemaan noin $1$ kilometrin korkeudesta. Ennen pudotusta heille annetaan huumausainetta, joka hidastaa aikakäsityksen $1$ prosenttiin normaalista. Vapaassa pudotuksessa pudottu matka ajanhetkellä $t$ on $\frac{1}{2} gt^2$, jossa $g$ on putoamiskiihtyvyytenä tunnettu vakio, jolle voimme tässä hyvin käyttää arviota $g \approx 10\frac{\text{m}}{\text{s}^2}$.
    \begin{enumerate}[a)]
    \item Olettaen, että huumausaineen vaikutus kestää koko putoamisen ajan, kuinka pitkältä aika pudotuksesta kuolemaan \textbf{uhrista} tuntuu? (Oleta annetut arvot tarkoiksi ja muodosta relevantti toisen asteen yhtälö.)
    \item Mikä menee fataalisti pieleen, jos a)-kohdan laskee suoraan kuvatulla tavalla?
    \end{enumerate}
    \begin{vastaus}
        \begin{enumerate}[a)]
            \item Vastaukseksi saadaan $1414 \, \text{s} = 23 \, \text{min} \, 34 \, \text{s}$. Käytännössä hyvä vastaustarkkuus voisi olla esimerkiksi $25 \, \text{min}$.
            \item Tehtävä ei huomioi ilmanvastusta. Ihminen saavuttaa korkeimmillaan rajanopeuden $v_{raja} \approx 55\frac{\text{m}}{\text{s}}$. Tehtävän mallissa putoavan ihmisen nopeus nousee $v_{max} \approx 141\frac{\text{m}}{\text{s}}$ asti. Todellisuudessa putoaminen siis kestää vieläkin kauemmin.
        \end{enumerate}
    \end{vastaus}
\end{tehtava}
