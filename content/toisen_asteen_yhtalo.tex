\chapter{Toisen asteen yhtälö}
\textbf{Esimerkki 1.} \\
Milloin funktio $f(x)=x^2+2x+1$ leikkaa x-akselin? \\ 
\textbf{Ratkaisu} \\
Funktio leikkaa x-akselin, kun $f(x)=0$. Tätä muuttujan $x$ arvoa kutsutaan funktion $f$ \textbf{nollakohdaksi}. Piirretään funktion $f$ kuvaaja ja etsitään ne kohdat, joissa funktio leikkaa x-akselin. %insert kuvaaja.pic
%funktio käsite on tuttu, joten itse esittelisin tämän asian näin -Lauri
\\ \\
Kuvaajasta huomataan, että funktion $f$ nollakohta on $x \approx -1$. Tätä toisen asteen yhtälön ratkaisumenetelmää kutsutaan graafiseksi ratkaisuksi.
Määritettäessä toisen asteen polynomifunktion nollakohtia päädytään \textbf{toisen asteen yhtälöön}, joka on aina muotoa
\begin{align*}
ax^2+bx+c=0, \ \ \ a, \ b, \ c  \in \mathbb{R}, a \neq 0.
\end{align*}
Toisen asteen polynomifunktion kuvaaja on aina \textbf{paraabeli}. \\
%kuvaajat paraabeleista 2 nk. 1 nk. 0 nk. ylös ja alas molemmista.

\laatikko{Toisen asteen yhtälöllä on ratkaisuja joko 0, 1 tai 2.} 

Toisen asteen yhtälö voidaan ratkaista graafisesti tai algebrallisesti. Graafinen ratkaisu on aina likimääräinen eli arvio oikeasta ratkaisusta. \\ \\ Edellä tutustuimme toisen asteen yhtälön graafiseen ratkaisemiseen. Seuraavaksi tutustumme toisen asteen yhtälön algebralliseen ratkaisemiseen.
\section{Vaillinaiset yhtälöt}
Toisen asteen yhtälö on aina muotoa $ax^2+bx+c=0$, jossa $a, \ b, \ c \in \mathbb{R}, \ a \neq 0$. \\ 
Jos yhtälössä $ax^2+bx+c=0$ kerroin $b=0$, niin toisen asteen yhtälö saa muodon
\begin{align*}
ax^2+bx+c=0 \\
ax^2+0 \cdot x+c=0 \\
ax^2+c=0 
\end{align*}
\laatikko{Vaillinainen toisen asteen yhtälö $ax^2+c=0$} \\
Toisen asteen yhtälöä, joka on muotoa $ax^2+c=0$ kutsutaan vaillinaiseksi toisen asteen yhtälöksi. Tämän muotoinen toisen asteen yhtälö saadaan helposti ratkaistua.\\ 
\textbf{Esimerkki 1} \\
Ratkaise yhtälö $5x^2-45=0$ \\
\textbf{Ratkaisu} 
\begin{align*}
5x^2-45&=0 \\
5x^2&=45  \ \ \ \ \ &&||:5 \\
x^2&=9 &&|| \sqrt[•]{•} \\
x&= \pm 3   
\end{align*}
\textbf{Esimerkki 2} \\ 
Ratkaise yhtälö $13x^2-42=-3$ \\
\textbf{Ratkaisu} \\
\begin{align*}
13x^2-42&=-3 \\
13x^2&=39 \ \ \ \ \ &&||:13 \\
x^2&=3 \ \ \ \ \ &&|| \sqrt[•]{•} \\
x&=\pm \sqrt[]{3}
\end{align*}
Jos yhtälössä $ax^2+bx+c=0$ kerroin $c=0$, niin toisen asteen yhtälö saa muodon
\begin{align*}
ax^2+bx+0&=0 \\
ax^2+bx=0
\end{align*}
\laatikko{Vaillinainen toisen asteen yhtälö $ax^2+bx=0$} \\
Toisen asteen yhtälöä, joka on muotoa $ax^2+bx=0$ kutsutaan vaillinaiseksi toisen asteen yhtälöksi. Tämän muotoinen yhtälö saadaan ratkaistua tulon nollasäännön avulla. \\
\begin{align*}
ax^2+bx&=0 \ \ \ \ \ &&||\text{otetaan yhteinen tekijä} \\
x(ax+b)&=0 \ \ \ \ \ &&||\text{tulon nollasääntö} \\
x&=0 \text{ tai } ax+b=0 \\
x&=0 \text{ tai } ax=-b \\
x&=0 \text{ tai } x=-\frac{b}{a}
\end{align*}
\textbf{Esimerkki 3.}
Ratkaise yhtälö $x^2-11x=0$
\begin{align*}
x^2-11x&=0 \ \ \ \ \  &&||\text{otetaan yhteinen tekijä} \\
x(x-11)&=0 \ \ \ \ \ &&||\text{tulon nollasääntö} \\
x&=0 \text{ tai } x-11=0 \\ 
x&=0 \text{ tai } x=11
\end{align*} 
\textbf{Esimerkki 4.}
Ratkaise yhtälö $55x^2+8x=0$
\begin{align*}
55x^2+8x&=0 \ \ \ \ \ &&||\text{otetaan yhteinen tekijä} \\
x(55x+8)&=0 \ \ \ \ \ &&||\text{tulon nollasääntö} \\
x&=0 \text{ tai } 55x+8=0 \\
x&=0 \text{ tai } 55x=-8 \\
x&=0 \text{ tai } x=-\frac{8}{55}
\end{align*}
\section{Neliöksi täydentäminen}
Toisen asteen yhtälöä $ax^2+bx+c=0$, jossa $a,b,c \neq 0$ kutsutaan täydelliseksi toisen asteen yhtälöksi. \\ 
\textbf{Esimerkki 5.} \\
Ratkaise yhtälö $x^2+4x-16=0$. \\
\textbf{Ratkaisu} 
\begin{align*}
x^2+4x-16&=0 \ \ \ \ \ &&||+20 \\
x^2+4x+4&=20 \ \ \ \ \ &&||a^2+2ab+b^2=(a+b)^2 \\
(x+2)^2&=20 \ \ \ \ \ &&||\sqrt[]{} \\
x+2 &= \pm \sqrt[]{20} \\
x&=-2 \pm \sqrt[]{20} \\
x&=-2 \pm 2 \sqrt[]{5} \\
\end{align*}
\textbf{Esimerkki 6.} \\
Ratkaise yhtälö $16x^2-64x+2=0$. \\
\textbf{Ratkaisu} 
\begin{align*}
16x^2-16x+2&=0 \ \ \ \ \ &&||+2 \\
16x^2-16x+4&=2 \ \ \ \ \ &&||a^2-2ab+b^2=(a-b)^2 \\
(4x+2)^2&=2 \ \ \ \ \ &&|| \sqrt[]{} \\
4x+2&=\pm \sqrt[]{2} \\
4x&=-2 \pm \sqrt[]{2} \\
x&=-\frac{1}{2} \pm \frac{\sqrt[]{2}}{4} \\
x&=-\frac{1}{2} \pm \frac{\sqrt[]{2}}{ 2\sqrt[]{2}\sqrt[]{2}} \\
x&=-\frac{1}{2} \pm \frac{1}{2 \sqrt[]{2}} \\
\end{align*}
Ratkaisutapaa, jossa toisen asteen yhtälö täydennetään lisäämällä tai vähentämällä termejä binomin neliöksi, kutsutaan neliöksi täydentämiseksi. \\
\laatikko{Toisen asteen yhtälö voidaan aina ratkaista neliöön täydentämällä.} \\
Seuraavassa kappaleessa johdamme toisen asteen yhtälön ratkaisukaavan tämän menetelmän avulla.
\section{Harjoitustehtäviä}

\begin{tehtava}
    Ratkaise seuraavat yhtälöt.
    \begin{enumerate}
        \item $x^2 - 9 = 0$
        \item $2x^2 + 8 = 0$
        \item $-x^2 + 11 = -5$
        \item $3 - x^2 = -1 + 3x^2$
    \end{enumerate}
    \begin{vastaus}
        \begin{enumerate}
            \item $x=\pm3$
            \item Ei ratkaisuja.
            \item $x=\pm4$
            \item $x=\pm1$
        \end{enumerate}
    \end{vastaus}
\end{tehtava}

\begin{tehtava}
    Ratkaise seuraavat yhtälöt.
    \begin{enumerate}
        \item $x^2 - 3x = 0$
        \item $10x + 2x^2 = 0$
        \item $-3x^2 + 8x = -2x$
        \item $2x^2 - x^3 = 0$
    \end{enumerate}
    \begin{vastaus}
        \begin{enumerate}
            \item $x=0$ tai $x=3$
            \item $x=0$ tai $x=-5$
            \item $x=0$ tai $x=8$
            \item $x=0$ tai $x=2$
        \end{enumerate}
    \end{vastaus}
\end{tehtava}

\begin{tehtava}
    Ratkaise seuraavat yhtälöt.
    \begin{enumerate}
        \item $x^2+3x+2=0$
        \item $2x^2+5x-12=0$
        \item $3x^2-7x-20=0$
        \item $x^2+3x-5=0$
        \item $x^2+5x-24=0$
    \end{enumerate}
    \begin{vastaus}
        \begin{enumerate}
            \item $x=-2$ tai $x=-1$
            \item $x=3/2$ tai $x=-4$
            \item $x=4$ tai $x=-5/3$
            \item $x=\frac{3\pm\sqrt{29}}{2}$
            \item $x=3$ tai $x=-8$
        \end{enumerate}
    \end{vastaus}
\end{tehtava}

\begin{tehtava}
    Ratkaise seuraavat yhtälöt.
    \begin{enumerate}
        \item $x^2+3x-5=4x+8$
        \item $8x^2-5x+1=-36$
        \item $-3x^2-4x+2=-5x^2+3$
        \item $-3x^2+4x+13=5x^2+10x+9$
    \end{enumerate}
    \begin{vastaus}
        \begin{enumerate}
            \item $\frac{1\pm\sqrt{53}}{2}$
            \item Ei ratkaisua reaalilukujen joukossa.
            \item $1\pm\frac{\sqrt{6}}{2}$
            \item $x=1$ tai $x=2$
        \end{enumerate}
    \end{vastaus}
\end{tehtava}
