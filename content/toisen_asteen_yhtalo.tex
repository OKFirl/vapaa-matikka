\chapter{Toisen asteen yhtälö}
\textbf{Esimerkki} \\
Milloin funktio $f(x)=x^2+2x+1$ leikkaa x-akselin? \\ 
\textbf{Ratkaisu} \\
Funktio leikkaa x-akselin, kun $f(x)=0$. Tätä muuttujan $x$ arvoa kutsutaan funktion $f$ \textbf{nollakohdaksi}. Piirretään funktion $f$ kuvaaja ja etsitään ne kohdat, joissa funktio leikkaa x-akselin. %insert kuvaaja.pic
\\ \\
Kuvaajasta huomataan, että funktion $f$ nollakohta on $x \approx -1$. Tätä toisen asteen yhtälön ratkaisumenetelmää kutsutaan graafiseksi ratkaisuksi.
Määritettäessä toisen asteen polynomifunktion nollakohtia päädytään \textbf{toisen asteen yhtälöön}, joka on aina muotoa
\begin{align*}
ax^2+bx+c=0, \ \ \ a, \ b, \ c  \in \mathbb{R}, a \neq 0.
\end{align*}
Toisen asteen polynomifunktion kuvaaja on aina \textbf{paraabeli}. \\
%kuvaajat paraabeleista 2 nk. 1 nk. 0 nk. ylös ja alas molemmista.

Toisen asteen yhtälöllä on ratkaisuja joko 0, 1 tai 2.
Toisen asteen yhtälö voidaan ratkaista graafisesti tai algebrallisesti. Graafinen ratkaisu on aina likimääräinen eli arvio oikeasta ratkaisusta. Edellä tutustuimme toisen asteen yhtälön graafiseen ratkaisemiseen. Seuraavaksi tutustumme toisen asteen yhtälön algebralliseen ratkaisemiseen.
\section{Vaillinaiset yhtälöt}
\textbf{Vaillinainen toisen asteen yhtälö} \\ 
Toisen asteen yhtälö on aina muotoa $ax^2+bx+c=0$, jossa $a, \ b, \ c \in \mathbb{R}, \ a \neq 0$. Jos $b=0$ tai $c=0$, niin kyseessä on vaillinainen toisen asteen yhtälö. \\  
\begin{enumerate}
\item{Jos $b=0$} \\
Jos yhtälössä $ax^2+bx+c=0$ kerroin $b=0$, niin toisen asteen yhtälö saa muodon
\begin{align*}
ax^2+bx+c=0 \\
ax^2+0 \cdot x+c=0 \\
ax^2+c=0 
\end{align*}
Toisen asteen yhtälöä, joka on muotoa $ax^2+c=0$ kutsutaan vaillinaiseksi toisen asteen yhtälöksi. Tämän muotoinen toisen asteen yhtälö saadaan helposti ratkaistua.\\ 
\textbf{Esimerkki 1} \\
Ratkaise yhtälö $5x^2-45=0$ \\
\textbf{Ratkaisu} 
\begin{align*}
5x^2-45&=0 \\
5x^2&=45  \ \ \ \ \ &&||:5 \\
x^2&=9 &&|| \sqrt[•]{•} \\
x&= \pm 3   
\end{align*}
\textbf{Esimerkki 2} \\ 
Ratkaise yhtälö $13x^2-42=-3$ \\
\textbf{Ratkaisu} \\
\begin{align*}
13x^2-42&=-3 \\
13x^2&=39 \ \ \ \ \ &&||:13 \\
x^2&=3 \ \ \ \ \ &&|| \sqrt[•]{•} \\
x&=\pm \sqrt[]{3}
\end{align*}
\item{Jos $c=0$} \\
Jos yhtälössä $ax^2+bx+c=0$ kerroin $c=0$, niin toisen asteen yhtälö saa muodon
\begin{align*}
ax^2+bx+0&=0 \\
ax^2+bx=0
\end{align*}
Toisen asteen yhtälöä, joka on muotoa $ax^2+bx=0$ kutsutaan vaillinaiseksi toisen asteen yhtälöksi. Tämän muotoinen yhtälö saadaan ratkaistua tulon nollasäännön avulla. \\
\laatikko{Tulo on nolla, jos ja vain jos jokin tulon tekijöistä on nolla.} \\
\begin{align*}
ax^2+bx&=0 \ \ \ \ \ &&||\text{otetaan yhteinen tekijä} \\
x(ax+b)&=0 \ \ \ \ \ &&||\text{tulon nollasääntö} \\
x&=0 \text{ tai } ax+b=0 \\
x&=0 \text{ tai } ax=-b \\
x&=0 \text{ tai } x=-\frac{b}{a}
\end{align*}
\textbf{Esimerkki 3.}
Ratkaise yhtälö $x^2-11x=0$
\begin{align*}
x^2-11x&=0 \ \ \ \ \  &&||\text{otetaan yhteinen tekijä} \\
x(x-11)&=0 \ \ \ \ \ &&||\text{tulon nollasääntö} \\
x&=0 \text{ tai } x-11=0 \\ 
x&=0 \text{ tai } x=11
\end{align*} 
\textbf{Esimerkki 4.}
Ratkaise yhtälö $55x^2+8x=0$
\begin{align*}
55x^2+8x&=0 \ \ \ \ \ &&||\text{otetaan yhteinen tekijä} \\
x(55x+8)&=0 \ \ \ \ \ &&||\text{tulon nollasääntö} \\
x&=0 \text{ tai } 55x+8=0 \\
x&=0 \text{ tai } 55x=-8 \\
x&=0 \text{ tai } x=-\frac{8}{55}
\end{align*}
\end{enumerate}
\section{Neliöksi täydentäminen}

\section{Harjoitustehtäviä}
