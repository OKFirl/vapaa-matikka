% Tämä luku sisältää aiemmin luvussa 2 (Polynomien yhteen- ja vähennyslasku) olleen sisällön.
% Päätimme siirtää sisällön aliluvuksi lukuun 1 (Polynomit), jotta päästään käsittelemään samassa luvussa polynomien sieventämistä perusmuotoon.
% Sisältö on helposti palautettavissa \input-komennolla kumpaan lukuun halutaan.
% T: Jokke ja Johanna
%Lisäsin qr-linkin. T:Pekka

Polynomeja voidaan laskea yhteen summaamalla samanasteiset termit. Tätä varten on kätevää ensin ryhmitellä polynomin samanasteiset termit vierekkäin.

\begin{esimerkki}
Laske polynomien $5x^2-x+5$ ja $3x^2-1$ summa.
   \begin{align*}
        (\textcolor{blue}{5x^2} \textcolor{red}{{}-x} + 5) + (\textcolor{blue}{+3x^2} -1) 
        &=\textcolor{blue}{5x^2} \textcolor{red}{{}-x} + 5  \textcolor{blue}{{}+3x^2} -1 \\
        &=\textcolor{blue}{5x^2+3x^2} \textcolor{red}{{}-x} +5-1\\
        &=\textcolor{blue}{(5+3)x^2} \textcolor{red}{{}-x}+(5-1)\\
        &=\textcolor{blue}{8x^2} \textcolor{red}{{}-x}+4.
    \end{align*}
\end{esimerkki}

Samalla tavalla polynomeja voidaan vähentää toisistaan.

\begin{esimerkki}
    Laske polynomien $14x^3+69$ ja $3x^3+2x^2+x$ erotus.
    \begin{align*}
        (\textcolor{green}{14x^3} + 69) - (\textcolor{green}{3x^3} \textcolor{blue}{{}+ 2x^2} \textcolor{red}{{}+x})
        &= \textcolor{green}{14x^3} + 69 \textcolor{green}{{}-3x^3} - 
            \textcolor{blue}{2x^2} \textcolor{red}{{}-x} \\
        &= \textcolor{green}{14x^3{}-3x^3} \textcolor{blue}{{}-2x^2} \textcolor{red}{{}-x} + 69 \\
        &= \textcolor{green}{(14{}-3)x^3} \textcolor{blue}{{}-2x^2} \textcolor{red}{{}-x} + 69 \\
        &= \textcolor{green}{11x^3} \textcolor{blue}{{}-2x^2} \textcolor{red}{{}-x} + 69
    \end{align*}
\end{esimerkki}
    
Samanasteisten termien yhteen- ja vähennyslasku perustuu siihen, että reaalilukujen osittelulain (ks. Vapaa matikka 1) nojalla vakiokerroin voidaan
ottaa yhteiseksi tekijäksi:
\[
ax^n+bx^n=(a+b)x^n.
\]

\qrlinkki{http://opetus.tv/maa/maa2/polynomien-yhteen-ja-vahennyslasku/}
{Opetus.tv: \emph{polynomien yhteen- ja vähennyslasku} (7:36)}
   
\begin{esimerkki}
Olkoon polynomit $P(x)=x+1$ ja $Q(x)=3x^2-2x+5$. Määritä summa $P(x)+Q(x)$.
   \begin{align*}
        P(x)+Q(x)&=(x+1)+(3x^2-2x+5) = x+1+3x^2-2x+5 \\
                 &= 3x^2+x-2x+1+5 =3x^2-x+6.
    \end{align*}
\end{esimerkki}

Kun laskurutiinia eli varmuutta on tarpeeksi, voi yllä olevasta esimerkistä jättää yksi tai kaksi välivaihetta pois.

% \begin{esimerkki}
% Määritä polynomien $R(x)=-4x^4+3x^3-x$ ja $S(x)=-3x^3+5x^2+2x$ summa.
%    \begin{align*}
%         R(x)+S(x)=(-4x^4+3x^3-x)+(-3x^3+5x^2+2x) =-4x^4+5x^2+x.
%     \end{align*}
% \end{esimerkki}

\begin{esimerkki}
    Laske polynomien $P(x)$ ja $R(x)$ erotus, kun $P(x)=x+1$ ja $R(x)=-4x^4+3x^3-x$.
   \begin{align*}
        P(x)-R(x) & =(x+1)-(-4x^4+3x^3-x) =x+1+4x^4-3x^3+x \\
        & =4x^4-3x^3+x+x+1 = 4x^4-3x^3+2x+1.
    \end{align*}
\end{esimerkki}

Yhteenlasketut ja vähennetyt polynomit sievennetään yleensä perusmuotoon, jossa on vain yksi termi kutakin astetta kohti. Tämä tehdään esimerkiksi silloin, kun selvitetään polynomin aste.

\begin{esimerkki} Mikä on polynomin $P(x)=(x^2+2x+1)-(x^2+2)$ aste?

\[
P(x)=x^2+2x+1-x^2-2=2x-1.
\]
Ensisilmäyksellä polynomin asteen voisi ajatella olevan 2, mutta tässä tapauksessa toisen asteen termit häviävät sievennettäessä ja asteluku onkin 1.
\end{esimerkki}
