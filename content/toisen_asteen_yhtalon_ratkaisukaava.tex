\chapter{Toisen asteen yhtälön ratkaisukaava}
Edellisessä kappaleessa opittiin, että toisen asteen yhtälö voidaan aina ratkaista täydentämällä se neliöksi. Tätä menetelmää ei kuitenkaan yleensä käytetä, sillä menetelmän voi ilmaista valmiina kaavana. Seuraavaksi johdetaan tämä toisen asteen yhtälön ratkaisukaava. \\ \\

Lähdetään liikkeelle täydellisestä toisen asteen yhtälöstä $ax^2+bx+c=0$.
\begin{align*}
ax^2+bx+c&=0 &\emph{kerrotaan molemmat puolet vakiolla} \ 4a \\
4a \cdot ax^2+4a \cdot bx + 4a \cdot c&=0 \\
4a^2x^2+4abx+4ac&=0 &\emph{vähennetään puolittain termi} \ 4ac  \\
4a^2x^2+4abx&=-4ac
\end{align*}
Täydennetään vasen puoli binomin neliöksi.
\begin{align*}
4a^2x^2+4abx&=-4ac &\emph{lisätään puolittain termi} \ b^2 \\
4a^2x^2+4abx+b^2&=b^2-4ac &\emph{nyt} \ (2ax+b)^2=4a^2x^2+4abx+b^2 \\
(2ax+b)^2&=b^2-4ac &\emph{otetaan puolittain neliöjuuri, jos} \ b^2 \geq 4ac \\
2ax+b&= \pm \sqrt[]{b^2-4ac} &\emph{vähennetään puolittain termi} \ b \\
2ax&=-b \pm \sqrt[]{b^2-4ac} &\emph{jaetaan puolittain termillä} \ 2a \\
x&= \frac{-b \pm \sqrt[]{b^2-4ac}}{2a}
\end{align*}
Saimme johdettua toisen asteen yhtälön ratkaisukaavan. Ratkaisussa jouduimme olettamaan, että $b^2 \geq 4ac$, sillä emme pysty ottamaan negatiivisista luvuista neliöjuuria.\\
\laatikko{\textbf{Toisen asteen yhtälön ratkaisukaava} \\
Yhtälön
$ax^2+bx+c=0$, missä  $a \neq 0$ ja   $b^2 \geq 4ac$,
ratkaisu on \\
\[ x=\frac{-b \pm \sqrt[]{b^2-4ac}}{2a} \]
}
\textbf{Esimerkki 1.}  \\
Ratkaise yhtälö $x^2-8x+16=0$.
\begin{align*}
\underbrace{1}_{=a}x^2 +\underbrace{(-8)}_{=b}x+\underbrace{16}_{=c}=0
\end{align*}
Sijoitetaan vakioiden $a=1$, $b=-8$ ja $c=16$ arvot toisen asteen yhtälön ratkaisukaavaan.
\begin{align*}
x&=\frac{-(-8)\pm \sqrt[]{(-8)^2-4\cdot 1 \cdot 16}}{2 \cdot 1} \\
x&=\frac{8 \pm \sqrt{64- 64}}{2} \\
x&=\frac{8 \pm 0}{2} \\
x&=4
\end{align*}
\textbf{Esimerkki 2.} \\
Ratkaise yhtälö $15x^2+24x+10=0$.
\begin{align*}
\underbrace{15}_{=a}x^2+\underbrace{24}_{=b}x+\underbrace{10}_{=c}=0
\end{align*}
Sijoitetaan vakioiden $a=15$, $b=24$ ja $c=10$ arvot toisen asteen yhtälön ratkaisukaavaan.
\begin{align*}
x&=\frac{-24 \pm \sqrt[]{24^2-4 \cdot 15 \cdot 10}}{2 \cdot 15} \\
x&=\frac{-24 \pm \sqrt[]{576-600}}{30} \\
x&=\frac{-24 \pm \sqrt[]{-24}}{30}
\end{align*}
Koska juurrettava on negatiivinen $-24<0$, niin yhtälöllä ei ole ratkaisuja. \\
\textbf{Esimerkki 3.} \\
Ratkaise yhtälö $x^2+2x-3=0$.
\begin{align*}
\underbrace{1}_{=a} \cdot x^2+\underbrace{2}_{=b}x\underbrace{-3}_{=c}=0
\end{align*}
Sijoitetaan vakioiden $a=1$, $b=2$ ja $c=-3$ arvot toisen asteen yhtälön ratkaisukaavaan.
\begin{align*}
x&=\frac{-2 \pm \sqrt[]{2^2-4 \cdot 1 \cdot (-3)}}{2 \cdot 1} \\
x&=\frac{-2 \pm \sqrt[]{4+12}}{2} \\
x&=\frac{-2 \pm \sqrt[]{16}}{2} \\
x&=\frac{-2 \pm 4}{2} \\
x&=-1 \pm 2 \\
x&=1 \text{ tai } x=-3 \\
\end{align*}
%Yleinen toisen asteen yhtälö on muotoa $ax^2+bx+c=0$.
%Kerrotaan yhtälön molemmat puolet vakiolla $4a$: $4a^2x^2+4abx+4ac=0$.
%Siirretään termi $4ac$ toiselle puolelle: $4a^2x^2+4abx=-4ac$.
%Pyritään täydentämään vasen puoli neliöksi.
%Lisätään puolittain termi $b^2$: $4a^2x^2+4abx+b^2=b^2-4ac$.
%Havaitaan vasemmalla puolella neliö: $(2ax+b)^2=b^2-4ac$.
%Otetaan puolittain neliöjuuri: $2ax+b=\pm\sqrt{b^2-4ac}$.
%Vähennetään puolittain termi $b$: $2ax=-b\pm\sqrt{b^2-4ac}$.
%Jaetaan puolittain vakiolla $2a$: $x=\frac{-b\pm\sqrt{b^2-4ac}}{2a}$.

\section{Diskriminantti}

\begin{esimerkki}
    Ratkaise toisen asteen yhtälö $3x^2-5x+10=0$.
    \begin{align*}
        \underbrace{3}_{=a}x^2\underbrace{-5}_{=b}x+\underbrace{10}_{=c}=0
    \end{align*}
    Sijoitetaan vakiot $a=3$, $b=-5$ ja $c=10$ toisen asteen yhtälön ratkaisukaavaan $x=\frac{-b \pm \sqrt[]{b^2-4ac}}{2a}$.
    \begin{align*}
        x=\frac{-(-5) \pm \sqrt[]{(-5)^2-4\cdot 3 \cdot 10}}{2 \cdot 3} \\
        x=-\frac{5 \pm \sqrt[]{25-120}}{6}
    \end{align*}
    Koska juurrettava on negatiivinen
    \begin{align*}
        b^2-4ac=(-5)^2-4 \cdot 3 \cdot 10=25-120=-95<0
    \end{align*}
    niin toisen asteen yhtälöllä ei ole ratkaisuja.
\end{esimerkki}

%Marginaaliin tai kuvaksi 2. asteen yhtälön ratkaisukaava (tai tekstin sekaan)

Toisen asteen yhtälön ratkaisukaavassa esiintyy neliöjuuri. Tämän neliöjuuren sisällä oleva lauseke $b^2-4ac$ määrää, kuinka monta ratkaisua yhtälöllä on. Joskus riittää pelkkä tieto ratkaisujen olemassaolosta tai lukumäärästä. Tällaisissa tapauksissa ei tarvitse ratkaista yhtälöä, vaan pelkkä edellä mainitun lausekkeen tarkastelu riittää. Tästä lausekkeesta käytetään nimeä \emph{diskriminantti} ja sitä merkitään kirjaimella $D$.

\laatikko{Toisen asteen yhtälön $ax^2+bx+c=0$ ratkaisujen lukumäärän näkee diskriminantin, $D=b^2-4ac$ avulla seuraavasti.
\begin{itemize}
\item
Jos $D<0$, yhtälöllä ei ole ratkaisuja.
\item
Jos $D=0$, yhtälöllä on tasan yksi ratkaisu.
\item
Jos $D>0$, yhtälöllä on kaksi ratkaisua.
\end{itemize}
}

D < 0, ei ratkaisuja
\begin{kuvaajapohja}{1}{-1}{3}{-1}{3}
  \kuvaaja{2*x**2-2*x+1}{$f(x)=2x^2-2x+1$}{blue}
\end{kuvaajapohja}

D = 0, 1 ratkaisu
\begin{kuvaajapohja}{1}{-1}{3}{-1}{3}
  \kuvaaja{x**2-2*x+1}{$f(x)=x^2-2x+1$}{blue}
\end{kuvaajapohja}

D > 0, 2 ratkaisua
\begin{kuvaajapohja}{1}{-1}{3}{-2}{2}
  \kuvaaja{2*x**2-4*x+1}{$f(x)=2x^2-4x+1$}{blue}
\end{kuvaajapohja}

\begin{esimerkki}
Selvitä onko yhtälöllä $x^2+x+2=0$ ratkaisuja.

Tutkitaan diskriminanttia.
\[D=1^2-4\cdot 1 \cdot 2 = 1-8 = -7\]
Koska $D<0$, yhtälöllä ei ole ratkaisuja.

Jos yhtälön ratkaisua yrittäisi ratkaisukaavan avulla, tulisi neliöjuuren alle negatiivinen luku.
\end{esimerkki}

\begin{esimerkki}
Millä $a$:n arvolla yhtälöllä $9x^2+ax+1$ on tasan yksi ratkaisu?

Jotta ratkaisuja olisi tasan yksi, on diskriminantin oltava 0.
\begin{align*}
D &=0\\
a^2-4\cdot 9\cdot 1 &= 0\\
a^2-36&=0\\
a^2&=36\\
a=\pm6
\end{align*}
Yhtälöllä on täsmälleen yksi ratkaisu, jos $a=-6$ tai $a=6$.
\end{esimerkki}


\section{Harjoitustehtäviä}

\begin{tehtava}
    Ratkaise
    \begin{enumerate}[a)]
        \item $x^2 - 2x - 3 = 0$
        \item $-x^2 - 6x - 5 = 0$
        \item $9x^2 - 12x + 4 = 0$
        \item $x + 2x^2 - 6= 0$
        \item $1 + x + 3x^2= 0$
    \end{enumerate}
    \begin{vastaus}
        \begin{enumerate}[a)]
            \item $x = 3 tai x = -1$
            \item $x = -5 tai x = -1$
            \item $x = \dfrac{2}{3}$
            \item $x = -1 + \sqrt{2} tai x = -1 - \sqrt{2}$
            \item Ei ratkaisuja.
        \end{enumerate}
    \end{vastaus}
\end{tehtava}

\begin{tehtava}
    Ratkaise
    \begin{enumerate}[a)]
        \item $x^2 + 2x = -4$
        \item $4x^2 = 12x - 8$
        \item $3x^2 - 13x + 50 = -2x^2 + 17x + 5$
    \end{enumerate}
    \begin{vastaus}
        \begin{enumerate}[a)]
            \item $x = -2$
            \item $x = 1$ tai $x = 2$
            \item $x = 3$
        \end{enumerate}
    \end{vastaus}
\end{tehtava}

\begin{tehtava}
    Tasaisesti kiihtyvässä liikkeessä on voimassa kaavat $v = v_0 + at$ ja $s = v_0t + \dfrac{1}{2}at^2$, missä $v$ on loppunopeus, $v_0$ alkunopeus, $a$ kiihtyvyys, $t$ aika ja $s$ siirtymä.
		\begin{enumerate}[a)]
            \item Auton nopeus on 72~km/h. Auto pysäytetään jarruttamalla tasaisesti. Se pysähtyy 10 sekunnissa. Laske jarrutusmatka.
            \item Kivi heitetään suoraan alas 50 metriä syvään rotkoon nopeudella 3,0~m/s. Kuinka monen sekunnin kuluttua se kohtaa rotkon pohjan?
        \end{enumerate}
    \begin{vastaus}
        \begin{enumerate}[a)]
            \item 100 metriä
            \item Noin 2,9 sekuntia
        \end{enumerate}
    \end{vastaus}
\end{tehtava}

\begin{tehtava}
    Kahden luvun summa on 8 ja tulo 15. Määritä luvut.
    \begin{vastaus}
		3 ja 5
    \end{vastaus}
\end{tehtava}

\begin{tehtava}
    Suorakulmaisen muotoisen alueen piiri on 34~m ja pinta-ala 60~m$^2$. Selvitä alueen mitat.
    \begin{vastaus}
		12 m ja 5 m
    \end{vastaus}
\end{tehtava}

\begin{tehtava}
    Kultaisessa leikkauksessa jana on jaettu siten, että pidemmän osan suhde lyhyempään on sama kuin koko janan suhde pidempään osaan. Tällaista suhdetta (merkitään yleensä kreikkalaisella aakkosella fii $\varphi$) on taiteessa kautta aikojen pidetty ''jumalallisena suhteena''.
		\begin{enumerate}[a)]
            \item Laske kultaiseen suhteen $\varphi$ tarkka arvo ja likiarvo.
            \item Napa jakaa ihmisvartalon pituussuunnassa kultaisen leikkauksen suhteessa. Millä korkeudella napa on 170~cm pitkällä ihmisellä?
        \end{enumerate}
    \begin{vastaus}
        \begin{enumerate}[a)]
            \item $ \varphi = \dfrac{\sqrt{5}-1}{2} \approx 0,618$
            \item Noin 105,1 cm
        \end{enumerate}
    \end{vastaus}
\end{tehtava}
\begin{tehtava}
(K93/T5) Ratkaise yhtälö
        $\frac{2x+a^2-3a}{x-1}=a$ vakion $a$ kaikilla reaaliarvoilla.
\begin{vastaus}
        \begin{enumerate}
         \item{$x=a$, jos $a \neq 2$ ja $a \neq 1$}
         \item{$x\neq 1$, jos $a=2$}
         \item{ei ratkaisua, jos $a=1$}
        \end{enumerate}
    \end{vastaus}
\end{tehtava}
\begin{tehtava}
(K94/T2a) Polynomin $P(x)=ax^3-31x^2+1$ eräs nollakohta on $x=1$. Määritä $a$ ja ratkaise tämän jälkeen $P(x)=0$.
\begin{vastaus}
      $a=30$ yhtälön ratkaisut ovat $1$, $\frac{1}{5}$ ja $-\frac{1}{6}$.
    \end{vastaus}
\end{tehtava}
\begin{tehtava}
(K96/T2b) Yhtälössä $x^2-2ax+2a-1=0$ korvataan luku $a$ luvulla $a+1$. Miten muuttuvat yhtälön juuret?
\begin{vastaus}
     Toinen kasvaa kahdella ja toinen ei muutu.
    \end{vastaus}
\end{tehtava}

\begin{tehtava}
	Kuinka monta ratkaisua yhtälöllä $5x^2+4x-10$ on?
	\begin{vastaus}
		$D=4^2-4\cdot 5 \cdot (-10)>0$, joten kaksi.
	\end{vastaus}
\end{tehtava}

\begin{tehtava}
	Kuinka monta reaalijuurta yhtälöllä $9x^2+12x-4$ on?
	\begin{vastaus}
		$D=12^2-4 \cdot 9 \cdot (-4) >0$, joten kaksi.
	\end{vastaus}
\end{tehtava}

\begin{tehtava}
	Millä vakion $k$ arvoilla yhtälöllä $-x^2-x-k$ ei ole ratkaisua?
	\begin{vastaus}
		Pitää olla $D=(-1)^2-4 \cdot (-1) \cdot (-k)<0$. Siis $k>\frac{1}{4}$.
	\end{vastaus}
\end{tehtava}

\begin{tehtava}
	Millä vakion $t$ arvoilla yhtälöllä $6x^2+2tx+2t=0$ on kaksoisjuuri?
	\begin{vastaus}
		Jos kaksoisjuuri, niin pitää päteä $D=4t^2-48t>0$. Joka toteutuu, kun $-12 < t < 0$.
	\end{vastaus}
\end{tehtava}

\begin{tehtava}
	Osoita, että diskriminantti on $0$ jos ja vain jos yhtälö voidaan esittää muodossa $(gx+h)^2=0$, missä $g$ ja $h$ ovat reaalilukuja.
	\begin{vastaus}
		Suunta "$\Rightarrow$": $(gx+h)^2=0 \Leftrightarrow g^2x^2+2ghx+h^2=0 \Rightarrow D=(2gh)^2-4g^2h^2=4g^2h^2-4g^2h^2=0$ \\
		Suunta "$\Leftarrow$": $D=0 \Leftrightarrow b^2-4ac=0 \Leftrightarrow b^2=4ac \Leftrightarrow c=\frac{b^2}{4a} \Rightarrow ax^2+bx+\frac{b^2}{4a}=0 \Leftrightarrow 4a^2x^2+4abx+b^2=0 \Leftrightarrow (2ax+b)^2=0$
	\end{vastaus}
\end{tehtava}
