\chapter{Toisen asteen yhtälön ratkaisukaava}
Edellisessä kappaleessa opittiin, että toisen asteen yhtälö voidaan aina ratkaista täydentämällä se neliöksi. Tätä menetelmää ei kuitenkaan yleensä käytetä, sillä menetelmän voi ilmaista valmiina kaavana. Seuraavaksi johdetaan toisen asteen yhtälön ratkaisukaava. \\ \\

Lähdetään liikkeelle täydellisestä toisen asteen yhtälöstä $ax^2+bx+c=0$.
\begin{align*}
ax^2+bx+c&=0 \ \ \ \ \ &&|| \text{kerrotaan molemmat puolet vakiolla }4a \\
4a \cdot ax^2+4a \cdot bx + 4a \cdot c&=0 \\
4a^2x^2+4abx+4ac&=0 \ \ \ \ \ &&|| \text{vähennetään puolittain termi }4ac  \\
4a^2x^2+4abx&=-4ac
\end{align*}
Täydennetään vasen puoli binomin neliöksi
\begin{align*}
4a^2x^2+4abx&=-4ac \ \ \ \ \ &&|| \text{lisätään puolittain termi } b^2 \\
4a^2x^2+4abx+b^2&=b^2-4ac \ \ \ \ \ &&||(2ax+b)^2=4a^2x^2+4abx+b^2 \\
(2ax+b)^2&=b^2-4ac  \ \ \ \ \ &&||\text{Otetaan puolittain neliöjuuri } \\
2ax+b&= \pm \sqrt[]{b^2-4ac} \ \ \ \ \ &&||\text{Vähennetään puolittain termi } b \\
2ax&=-b \pm \sqrt[]{b^2-4ac} \ \ \ \ \ &&||\text{Jaetaan puolittain termillä } 2a \\
x&= \frac{-b \pm \sqrt[]{b^2-4ac}}{2a} 
\end{align*}
Saimme johdettua toisen asteen yhtälön ratkaisukaavan. \\ 
\laatikko{\textbf{Toisen asteen yhtälön ratkaisukaava}
\begin{align*}
ax^2+bx+c=0, \ a \neq 0 \\
x=\frac{-b \pm \sqrt[]{b^2-4ac}}{2a}
\end{align*}
}
\textbf{Esimerkki 1.}  \\
Ratkaise yhtälö $4x^2-8x+4=0$.
\begin{align*}
\underbrace{4}_{=a}x^2 \underbrace{-8}_{=b}x+\underbrace{4}_{=c}=0
\end{align*}
Sijoitetaan vakioiden $a=4$, $b=-8$ ja $c=4$ arvot toisen asteen yhtälön ratkaisukaavaan.
\begin{align*}
x&=\frac{-(-8)\pm \sqrt[]{(-8)^2-4\cdot 4 \cdot 4}}{2 \cdot 4} \\
x&=\frac{8 \pm \sqrt[]{64- 64}}{8} \\
x&=\frac{8 \pm 0}{8} \\
x&=1
\end{align*}
\textbf{Esimerkki 2.} \\
Ratkaise yhtälö $15x^2+24x+10=0$.
\begin{align*}
\underbrace{15}_{=a}x^2+\underbrace{24}_{=b}x+\underbrace{10}_{=c}=0 
\end{align*}
Sijoitetaan vakioiden $a=15$, $b=24$ ja $c=10$ arvot toisen asteen yhtälön ratkaisukaavaan.
\begin{align*}
x&=\frac{-24 \pm \sqrt[]{24^2-4 \cdot 15 \cdot 10}}{2 \cdot 15} \\
x&=\frac{-24 \pm \sqrt[]{576-600}}{30} \\
x&=\frac{-24 \pm \sqrt[]{-24}}{30}
\end{align*}
Koska juurrettava on negatiivinen $-24<0$, niin yhtälöllä ei ole ratkaisuja. \\ 
\textbf{Esimerkki 3.} \\
Ratkaise yhtälö $x^2+2x-3=0$. 
\begin{align*}
\underbrace{1}_{=a} \cdot x^2+\underbrace{2}_{=b}x\underbrace{-3}_{=c}=0 
\end{align*}
Sijoitetaan vakioiden $a=1$, $b=2$ ja $c=-3$ arvot toisen asteen yhtälön ratkaisukaavaan.
\begin{align*}
x&=\frac{-2 \pm \sqrt[]{2^2-4 \cdot 1 \cdot (-3)}}{2 \cdot 1} \\
x&=\frac{-2 \pm \sqrt[]{4+12}}{2} \\
x&=\frac{-2 \pm \sqrt[]{16}}{2} \\
x&=\frac{-2 \pm 4}{2} \\
x&=-1 \pm 2
x&=1 \text{ tai } x=-3
\end{align*}
%Yleinen toisen asteen yhtälö on muotoa $ax^2+bx+c=0$.
%Kerrotaan yhtälön molemmat puolet vakiolla $4a$: $4a^2x^2+4abx+4ac=0$.
%Siirretään termi $4ac$ toiselle puolelle: $4a^2x^2+4abx=-4ac$.
%Pyritään täydentämään vasen puoli neliöksi.
%Lisätään puolittain termi $b^2$: $4a^2x^2+4abx+b^2=b^2-4ac$.
%Havaitaan vasemmalla puolella neliö: $(2ax+b)^2=b^2-4ac$.
%Otetaan puolittain neliöjuuri: $2ax+b=\pm\sqrt{b^2-4ac}$.
%Vähennetään puolittain termi $b$: $2ax=-b\pm\sqrt{b^2-4ac}$.
%Jaetaan puolittain vakiolla $2a$: $x=\frac{-b\pm\sqrt{b^2-4ac}}{2a}$.


\section{Harjoitustehtäviä}

\begin{tehtava}
    Ratkaise
    \begin{enumerate}
        \item $x^2 - 2x - 3 = 0$
        \item $-x^2 - 6x - 5 = 0$
        \item $9x^2 - 12x + 4 = 0$
        \item $x + 2x^2 - 6= 0$
        \item $1 + x + 3x^2= 0$
    \end{enumerate}
    \begin{vastaus}
        \begin{enumerate}
            \item $x = 3 tai x = -1$
            \item $x = -5 tai x = -1$
            \item $x = \dfrac{2}{3}$
            \item $x = -1 + \sqrt{2} tai x = -1 - \sqrt{2}$
            \item Ei ratkaisuja.
        \end{enumerate}
    \end{vastaus}
\end{tehtava}

\begin{tehtava}
    Ratkaise
    \begin{enumerate}
        \item $x^2 + 2x = -4$
        \item $4x^2 = 12x - 8$
        \item $3x^2 - 13x + 50 = -2x^2 + 17x + 5$
    \end{enumerate}
    \begin{vastaus}
        \begin{enumerate}
            \item $x = -2$
            \item $x = 1 tai x = 2$
            \item $x = 3$
        \end{enumerate}
    \end{vastaus}
\end{tehtava}

\begin{tehtava}
    Tasaisesti kiihtyvässä liikkeessä on voimassa kaavat $v = v_0 + at$ ja $s = v_0t + \dfrac{1}{2}at^2$, missä $v$ on loppunopeus, $v_0$ alkunopeus, $a$ kiihtyvyys, $t$ aika ja $s$ siirtymä. 
		\begin{enumerate}
            \item Auton nopeus on 72 km/h. Auto pysäytetään jarruttamalla tasaisesti. Se pysähtyy 10 sekunnissa. Laske jarrutusmatka.
            \item Kivi heitetään suoraan alas 50 metriä syvään rotkoon 3,0 m/s- Kuinka monen sekunnin kuluttua se kohtaa rotkon pohjan?
        \end{enumerate}
    \begin{vastaus}
        \begin{enumerate}
            \item 100 metriä
            \item Noin 2.9 sekuntia
        \end{enumerate}
    \end{vastaus}
\end{tehtava}

\begin{tehtava}
    Kahden luvun summa on 8 ja tulo 15. Määritä luvut. 
    \begin{vastaus}
		3 ja 5
    \end{vastaus}
\end{tehtava}

\begin{tehtava}
    Suorakulmaisen muotoisen alueen piiri on 34 m ja pinta-ala 60$m^2$. Selvitä alueen mitat. 
    \begin{vastaus}
		12m ja 5m
    \end{vastaus}
\end{tehtava}

\begin{tehtava}
    Kultaisessa leikkauksessa jana on jaettu siten, että pidemmän osan suhde lyhyempään on sama kuin koko janan suhde pitempään osaan. Tällaista suhdetta (merkitään yleensä kreikkalaisella aakkosella phi $\phi$) on taiteessa kautta aikojen pidetty "jumalallisena suhteena".
		\begin{enumerate}
            \item Laske kultaiseen suhteen $\phi$ tarkka arvo ja likiarvo.
            \item Napa jakaa ihmisvartalon pituussuunnassa kultaisen leikkauksen suhteessa. Millä korkeudella napa on 170 cm pitkällä ihmisellä?
        \end{enumerate}
    \begin{vastaus}
        \begin{enumerate}
            \item $ \phi = \dfrac{sqrt{5}-1}{2} \approx 0,618$
            \item Noin 105,1 cm
        \end{enumerate}
    \end{vastaus}
\end{tehtava}
