\section{Käsitteitä}

\subsection*{Polynomilausekkeet}

\termi[Polynomit]{polynomi} ovat matematiikassa ryhmä hyvin yleisiä ja yksinkertaisia yhden muuttujan laskulausekkeita.
%Usein hyödyllinen lauseketyyppi on sellainen joka koostuu muuttujan $x$ ja vakioiden yhteen- ja kertolaskuista.
%Tällaisia lausekkeita kutsutaan muuttujan $x$ \termi[polynomeiksi]{polynomi}.
Esimerkiksi seuraavat lausekkeet ovat muuttujan $x$ polynomeja:
\begin{itemize}
\item $5x^2+x-7$
\item $x^4-6x^3+2x^2-x$
\item $-24x^{100}$
\end{itemize}
Polynomit tunnistaa siitä, että niissä on käytetty muuttujan potenssien lisäksi vain yhteen-, vähennys- ja kertolaskuja. Esimerkiksi seuraavat muuttujan $x$ lausekkeet \emph{eivät ole} polynomeja:
\begin{itemize}
\item $2/x$
\item $\sqrt{x}+2$
\end{itemize}
Yleisin muuttujakirjain on $x$, mutta myös muita kirjaimia voidaan käyttää
muuttujina. Esimerkiksi $2y^5+y^4-1$ on muuttujan $y$ polynomi ja
$-A^3+2A^2-1$ on muuttujan $A$ polynomi.

Polynomilausekkeet ovat summia, joiden yhteenlaskettavia kutsutaan \termi[termeiksi]{termi}. (Termit, joita edeltää miinusmerkki, voidaan käsittää negatiivisten lukujen yhteenlaskuina.)

Esimerkiksi polynomin $5x^2+x-7$ termit ovat $5x^2$, $x$ ja $-7$. Jokainen termi on muuttujan potenssi kerrottuna jollakin vakiolla, joka voi olla myös  negatiivinen. Termi voi olla myös \termi[vakiotermi]{vakiotermi}, joka ei sisällä muuttujaa. Esimerkiksi $-7$ on vakiotermi.

\qrlinkki{http://opetus.tv/maa/maa2/polynomien-peruskasitteet/}
{Opetusvideo \emph{polynomien peruskäsitteet} (9:00)}

Yhden ja kahden termin polynomeista käytetään erityisnimityksiä \termi[monomi]{monomi} ja \termi[binomi]{binomi}.

Termin muuttujan eksponenttia kutsutaan termin \termi[asteeksi]{termi!aste}.
Vakiotermin aste on nolla.
%Erikoistapaus termistä on nollannen asteen termi, joka on vain vakio eli \termi{vakiotermi}.
Polynomin \termi[aste]{polynomi!aste} (käytetään myös nimitystä \termi[asteluku]{polynomi!asteluku}) on suurin sen termien asteista.

Koska polynomilauseke koostuu potenssien summista ja yhteenlasku on vaihdannainen, termien paikkaa polynomissa voidaan vapaasti vaihtaa. Siksi esimerkiksi polynomi $\frac{1}{3}y^3+y^2-3$ voidaan kirjoittaa myös järjestyksessä $y^2-3+\frac{1}{3}y^3$. Yleensä polynomien termit kirjoitetaan niiden asteen perusteella laskevaan järjestykseen niin, että korkeimman asteen termi kirjoitetaan ensin.

\begin{esimerkki}
Polynomilausekkeen $x^4-2x^3+5$ termit ovat $x^4$, $-2x^3$ ja $5$.
Ensimmäinen termi $x^4$ on neljännen asteen termi. Termin $-2x^3$ aste on puolestaan kolme ja vakiotermin $5$ aste on nolla.
Koko polynomin aste on neljä, koska se on termien asteista suurin.
\end{esimerkki}

\qrlinkki{http://opetus.tv/maa/maa2/polynomin-tasmallinen-maaritelma/}
{Opetusvideo \emph{polynomin täsmällinen määritelmä} (6:10)}

%Huomaa, että polynomilauseke pitää sieventää ennen asteen määrittämistä.
%Määritetään vaikkapa polynomin $x^5-4x^2+6x^3+3-x^5-x$ aste.
%Summataan ensin termejä niin, että
%polynomissa on vain yksi termi jokaista astetta:
%\[x^5-4x^2+6x^3+3-x^5-x = 6x^3-4x^2-x+3.\]
%Nyt nähdään, että polynomin aste on 3.

%\laatikko{
%Polynomi $P$, joka on astetta $n$, voidaan aina sieventää muotoon
%\[P(x) = a_n x^n + a_{n-1} x^{n-1} + \ldots + a_1 x + a_0,\]
%missä $a_0, a_1, \ldots, a_n$ ovat vakioita ja $a_n \neq 0$.
%}

Polynomi määrittää funktion, jota kutsutaan \termi[polynomifunktioksi]{polynomifunktio}. Esimerkiksi polynomi $2x^2+2x-1$ määrittää funktion $P(x)=2x^2+2x-1$. Tämän funktion arvoja voidaan laskea sijoittamalla lukuja muuttujan $x$ paikalle:
\begin{align*}
P(2) & = 5\cdot 2^2-3\cdot 2+2 = 20 - 6 + 2 = 16 \\
P(-1) & = 5(-1)^2-3(-1)+2 = 5 + 3 + 2 = 10 \\
P(-3) & = 5(-3)^2-3(-3)+2 = 45 + 9 + 2 = 56.
\end{align*}

\qrlinkki{http://opetus.tv/maa/maa2/polynomiesimerkkeja/}
{Opetusvideot \emph{polynomiesimerkkejä} (7:43 ja 6:59)}

Toisinaan polynomeja ja polynomifunktioita käsitellään kuin ne olisivat sama asia.
Voidaan esimerkiksi sanoa ''polynomi $P(x)=2x+1$'', vaikka yleensä $P(x)$-merkintää käytetään funktioista.

%Samalla polynomifunktiolla voi olla useampia esityksiä polynomilausekkeena.
%Esimerkiksi jos $P(x) = x^2-x^2+x+1$ ja $Q(x) = x+1$, $P$ ja $Q$ ovat samat
%funktiot koska $x^2 - x^2 = 0$ kaikilla reaaliluvuilla x. Siis saman
%polynomifunktion eri esityksissä lausekkeina suurimman termin aste voi olla
%eri, esimerkiksi $P$:ssä se on 2 ja $Q$:ssa 1. Useimmiten kiinnostavin on
%kuitenkin pienin mahdollinen aste, jolloin polynomifunktion $P$ aste olisi
%myös 1.

\subsection*{Polynomien yhteen- ja vähennyslasku}

% Tämä luku sisältää aiemmin luvussa 2 (Polynomien yhteen- ja vähennyslasku) olleen sisällön.
% Päätimme siirtää sisällön aliluvuksi lukuun 1 (Polynomit), jotta päästään käsittelemään samassa luvussa polynomien sieventämistä perusmuotoon.
% Sisältö on helposti palautettavissa \input-komennolla kumpaan lukuun halutaan.
% T: Jokke ja Johanna
%Lisäsin qr-linkin. T:Pekka

Polynomeja voidaan laskea yhteen summaamalla samanasteiset termit. Tätä varten on kätevää ensin ryhmitellä polynomin samanasteiset termit vierekkäin.

\begin{esimerkki}
Laske polynomien $5x^2-x+5$ ja $3x^2-1$ summa.
   \begin{align*}
        (\textcolor{blue}{5x^2} \textcolor{red}{{}-x} + 5) + (\textcolor{blue}{+3x^2} -1) 
        &=\textcolor{blue}{5x^2} \textcolor{red}{{}-x} + 5  \textcolor{blue}{{}+3x^2} -1 \\
        &=\textcolor{blue}{5x^2+3x^2} \textcolor{red}{{}-x} +5-1\\
        &=\textcolor{blue}{(5+3)x^2} \textcolor{red}{{}-x}+(5-1)\\
        &=\textcolor{blue}{8x^2} \textcolor{red}{{}-x}+4.
    \end{align*}
\end{esimerkki}

Samalla tavalla polynomeja voidaan vähentää toisistaan.

\begin{esimerkki}
    Laske polynomien $14x^3+69$ ja $3x^3+2x^2+x$ erotus.
    \begin{align*}
        (\textcolor{green}{14x^3} + 69) - (\textcolor{green}{3x^3} \textcolor{blue}{{}+ 2x^2} \textcolor{red}{{}+x})
        &= \textcolor{green}{14x^3} + 69 \textcolor{green}{{}-3x^3} - 
            \textcolor{blue}{2x^2} \textcolor{red}{{}-x} \\
        &= \textcolor{green}{14x^3{}-3x^3} \textcolor{blue}{{}-2x^2} \textcolor{red}{{}-x} + 69 \\
        &= \textcolor{green}{(14{}-3)x^3} \textcolor{blue}{{}-2x^2} \textcolor{red}{{}-x} + 69 \\
        &= \textcolor{green}{11x^3} \textcolor{blue}{{}-2x^2} \textcolor{red}{{}-x} + 69
    \end{align*}
\end{esimerkki}
    
Samanasteisten termien yhteen- ja vähennyslasku perustuu siihen, että reaalilukujen osittelulain (ks. Vapaa matikka 1) nojalla vakiokerroin voidaan
ottaa yhteiseksi tekijäksi:
\[
ax^n+bx^n=(a+b)x^n.
\]

\qrlinkki{http://opetus.tv/maa/maa2/polynomien-yhteen-ja-vahennyslasku/}
{Opetusvideo \emph{polynomien yhteen- ja vähennyslasku} (7:36)}
\quad\\
   
\begin{esimerkki}
Olkoon polynomit $P(x)=x+1$ ja $Q(x)=3x^2-2x+5$. Määritä summa $P(x)+Q(x)$.
   \begin{align*}
        P(x)+Q(x)&=(x+1)+(3x^2-2x+5) = x+1+3x^2-2x+5 \\
                 &= 3x^2+x-2x+1+5 =3x^2-x+6.
    \end{align*}
\end{esimerkki}

Kun laskurutiinia eli varmuutta on tarpeeksi, voi yllä olevasta esimerkistä jättää yksi tai kaksi välivaihetta pois.

% \begin{esimerkki}
% Määritä polynomien $R(x)=-4x^4+3x^3-x$ ja $S(x)=-3x^3+5x^2+2x$ summa.
%    \begin{align*}
%         R(x)+S(x)=(-4x^4+3x^3-x)+(-3x^3+5x^2+2x) =-4x^4+5x^2+x.
%     \end{align*}
% \end{esimerkki}

\begin{esimerkki}
    Laske polynomien $P(x)$ ja $R(x)$ erotus, kun $P(x)=x+1$ ja $R(x)=-4x^4+3x^3-x$.
   \begin{align*}
        P(x)-R(x) & =(x+1)-(-4x^4+3x^3-x) =x+1+4x^4-3x^3+x \\
        & =4x^4-3x^3+x+x+1 = 4x^4-3x^3+2x+1.
    \end{align*}
\end{esimerkki}

Yhteenlasketut ja vähennetyt polynomit täytyy yleensä sieventää perusmuotoon, jossa on vain yksi termi kutakin astetta kohti. Tämä on tehtävä esimerkiksi silloin, kun halutaan tarkistaa polynomin aste.

\begin{esimerkki} Mikä on polynomin $P(x)=(x^2+2x+1)-(x^2+2)$ aste?

\[
P(x)=x^2+2x+1-x^2-2=2x-1.
\]
Ensisilmäyksellä polynomin asteen voisi luulla olevan 2, mutta tässä tapauksessa toisen asteen termit häviävät sievennettäessä ja asteluku onkin 1.
\end{esimerkki}


\Harjoitustehtavat

\paragraph*{Opi perusteet}

\begin{tehtava}
    Täydennä taulukko.
        
    \begin{tabular}{|c|c|c|c|c|}
                                                                                           \hline
             & termien   & korkeimman asteen & polynomin &            \\
polynomi     & lukumäärä & termin kerroin    & asteluku  & vakiotermi \\ \hline
$-2x^2+6x$   &        2  &         $-2$      &       2   &    0       \\ \hline 
$7x^3-x-15$  &           &                   &           &            \\ \hline 
             &        2  &          $-9$     &       2   &    5       \\ \hline 
             &        3  &          $-1$     &       5   &    $-17$   \\ \hline 
             &        4  &                   &       3   &            \\ \hline 
             &        1  &          -5       &       99  &            \\ \hline                           
    \end{tabular}
    \begin{vastaus}
    Värilliset kohdat voivat olla jotain muutakin.
    
    \begin{tabular}{|c|c|c|c|c|}
                                                                                           \hline
             &                   & korkeimman asteen &                     &            \\
polynomi     & termien lukumäärä & termin kerroin    & polynomin asteluku  & vakiotermi \\ \hline
$-2x^2+6x$   &        2          &         $-2$      &       2             &    0       \\ \hline 
$7x^3-x-15$  &        3          &           7       &       3             &    $-15$   \\ \hline 
$-9x^2+5$    &        2          &          $-9$     &       2             &    5       \\ \hline 
$-x^5\textcolor{blue}{+4x}-17$%
             &        3          &          $-1$     &       5             &    $-17$   \\ \hline 
$\textcolor{blue}{8}x^3\textcolor{blue}{-x^2+4x}-17$%
             &        4          &\textcolor{blue}{8}  &       3             &\textcolor{blue}{7}\\ \hline 
$-5x^{99}$   &        1          &          $-5$     &       99            &         0      \\ \hline                           
     \end{tabular}
     \end{vastaus}
\end{tehtava}



\begin{tehtava}
    Mitkä seuraavista ovat polynomeja?
    \begin{enumerate}[a)]
        \item $\frac{1}{x}$
       %\item $x^3+4x$
        \item $5x-125$
       %\item $2^x$
        \item $\sqrt{x}+1$
        \item $3x^4+6x^2+9$
        \item $\sqrt{2}x-x$
        \item $4^x+5x+6$
    \end{enumerate}
    \begin{vastaus}
        \begin{enumerate}[a)]
            \item Ei ole.
           %\item On.
            \item On.
           %\item Ei ole.
            \item Ei ole.
            \item On.
            \item On.
            \item Ei ole.
        \end{enumerate}
    \end{vastaus}
\end{tehtava}

\begin{tehtava}
    Olkoot $P(x)=x^2+5$ ja $Q(x)=x^3-1$. Laske
    \begin{enumerate}[a)]
        \item $P(2)$
        \item $Q(1)$
        \item $P(-7)$
        \item $Q(-4)$.
    \end{enumerate}
    \begin{vastaus}
        \begin{enumerate}[a)]
            \item $9$ % 2^2 + 5 = 4 + 5 
            \item $0$ % 1^3 - 1
            \item $54$ % (-7)^2 + 5 = 49 + 5 
            \item $-65$ % (-4)^3 - 1 = -64 -1
        \end{enumerate}
    \end{vastaus}
\end{tehtava}

%\begin{tehtava}
%    Olkoot $P(x)=x^2+3x+4$ ja $Q(x)=x^3-10x+1$. Laske:
%    \begin{enumerate}[a)]
%        \item $P(-1)$
%        \item $Q(-2)$
%        \item $P(3)$
%        \item $Q(0)$
%    \end{enumerate}
%    \begin{vastaus}
%        \begin{enumerate}[a)]
%            \item $2$
%            \item $13$
%            \item $22$
%            \item $1$
%        \end{enumerate}
%    \end{vastaus}
%\end{tehtava}

\begin{tehtava}
    Olkoot $P(x)=x^2+3x+4$ ja $Q(x)=x^3-10x+1$. Sievennä
    \begin{enumerate}[a)]
        \item $P(x)+Q(x)$
        \item $P(x)-Q(x)$
        \item $Q(x)-P(x)$
        \item $2P(3)+Q(2)$.
    \end{enumerate}
    \begin{vastaus}
        \begin{enumerate}[a)]
            \item $x^3+x^2-7x+5$ % x^2+3x+4 + x^3-10x+1
            \item $-x^3+x^2+13x+3$ % x^2+3x+4 -(x^3-10x+1) = x^2+3x+4 -x^3+10x-1
            \item $x^3-x^2-13x-3$ % 
            \item $33$ % 2*(3^2+3*3+4) +  2^3-10*2+1 = 2*(9+9+4)+8-20+1 =44-11 =33
        \end{enumerate}
    \end{vastaus}
\end{tehtava}

%\begin{tehtava}
%	Mikä on/mitkä ovat polynomin $P(x) = x^5-3x^3+2x-1$
%	\begin{enumerate}[a)]
%		\item aste
%		\item termit
%		\item kolmannen asteen termi
%		\item vakiotermi
%	\end{enumerate}
%
%	\begin{vastaus}
%		\begin{enumerate}[a)]
%			\item $5$
%			\item $x^5$, $-3x^3$, $2x$, $-1$
%			\item $-3x^3$
%			\item $-1$
%		\end{enumerate}
%	\end{vastaus}
%\end{tehtava}

%\begin{tehtava}
%	Mitkä on seuraavien polynomien asteet?
%	\begin{enumerate}[a)]
%		\item $x^2 + 3x - 5$
%		\item $100 + x$
%		\item $3x^3 + 90x^8 + 2x$
%		\item $12x^1 + 34x^2 + 56x^3 + 78x^5 + 90x^5$
%	\end{enumerate}
%
%	\begin{vastaus}
%		\begin{enumerate}[a)]
%			\item $2$
%			\item $1$
%			\item $8$
%			\item $5$
%		\end{enumerate}
%	\end{vastaus}
%\end{tehtava}

% \begin{tehtava}
%     Sievennä
%     \begin{enumerate}
%         \item $(2x + 3) + x $
%         \item $(3x - 1) + (-x + 1)$
%         \item $(5x + 10) + (6x - 6) - (x + 3)$
%     \end{enumerate}
%     \begin{vastaus}
%         \begin{enumerate}
%             \item $3x + 3$
%             \item $2x$
%             \item $10x - 1$
%         \end{enumerate}
%     \end{vastaus}
% \end{tehtava}

\paragraph*{Hallitse kokonaisuus}

\begin{tehtava}
    Sievennä
    \begin{enumerate}
        \item $(x^2 - 2x + 1) + (-x^2 + x) $
        \item $(3y^3 + 2y^2  + y) - (-y^2 + y)$
        \item $(z^{10} - z^6 + z^2 + 1) + (z^{10} + 2z^8 - 3z^6) - (z^4 - 3)$.
    \end{enumerate}
    \begin{vastaus}
        \begin{enumerate}
            \item $-x + 1$
            \item $3y^3 + 3y^2$
            \item $2z^{10} + 2z^8 - 4z^6 - z^4 + z^2 + 4$
        \end{enumerate}
    \end{vastaus}
\end{tehtava}

\begin{tehtava}
	Mitkä ovat seuraavien polynomifunktioiden asteet, ts. sievennettyjen muotojen asteet?
	\begin{enumerate}[a)]
		\item $x+5-x$
		\item $x^2+x-2x^2$
		\item $4x^5+x^2-4-x^2$
		\item $x^3+2x^5-x-3x^5+3x^2+5+x^5$
		\item $x^4-2x^3+x-1+x^3-x^4+x^3$
	\end{enumerate}

	\begin{vastaus}
		\begin{enumerate}[a)]
			\item $0$
			\item $2$
			\item $5$
			\item $3$
			\item $1$
		\end{enumerate}
	\end{vastaus}
\end{tehtava}

% Hämäävä tehtävä
% \begin{tehtava}
% 	Sievennä.
% 	\begin{enumerate}[a)]
% 		\item $2 + x$
% 		\item $3x - 1 + 2x^2$
% 		\item $x^3 - \dfrac{2}{3}x^5 + 2x^4 $
% 		\item $1 - 15x + \pi x^{10} + 2$
% 		\item $-2x + 3 + 5x - 5x^2 - 5$
% 	\end{enumerate}
%
% 	\begin{vastaus}
% 		\begin{enumerate}[a)]
% 			\item $x + 2$
% 			\item $2x^2 + 3x - 1$
% 			\item $-\dfrac{2}{3}x^5 + 2x^4 + x^3 $
% 			\item $\pi x^10 - 15x + 3$
% 			\item $-5x^2 + 3x - 2$
% 		\end{enumerate}
% 	\end{vastaus}
% \end{tehtava}

\begin{tehtava}
	Mitkä seuraavista polynomilausekkeista esittävät samaa polynomifunktiota kuin
	$x^3+2x+1$?
	\begin{enumerate}[a)]
		\item $2x+x^3+1$
		\item $x^2+2x+1$
		\item $x+2x^3+1 - (x^3+x)$
		\item $15+x^4+2x+x^3-x^4-14$
	\end{enumerate}
	\begin{vastaus}
		a) ja d)
	\end{vastaus}
\end{tehtava}
