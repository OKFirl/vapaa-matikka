\chapter{Polynomi}
Usein hyödyllinen lauseketyyppi on sellainen joka koostuu muuttujan $x$ ja
vakioiden yhteen- ja kertolaskuista. Tällaisia lausekkeita kutsutaan muuttujan
$x$ \termi{polynomeiksi}{polynomi}. Esimerkiksi seuraavat lausekkeet ovat muuttujan $x$ polynomeja:
\begin{itemize}
\item $24x + 5 - 6x$
\item $5x^2+7$
\item $x^4-6x^3+2x^2-x$
\end{itemize}

Yleisin muuttujakirjain on $x$, mutta myös muita kirjaimia voidaan käyttää
muuttujina. Esimerkiksi $2y^5+y^4-1$ on muuttujan $y$ polynomi ja
$-\epsilon^3+2\epsilon^2-1$ on muuttujan $\epsilon$ polynomi.

Polynomi voidaan kirjoittaa aina summana jonka yhteenlaskettavat eli
\termi{termit}{termi} ovat vakion ja muuttujan potenssin tuloja. Termin muuttujan
eksponenttia kutsutaan termin \termi{asteeksi}{termi!aste}. Erikoistapaus termistä on
nollannen asteen termi, joka on vain vakio eli \termi{vakiotermi}. Polynomin
aste on suurin sen termien asteista.

Yhden termin polynomia kutsutaan \termi{monomiksi}{monomi} ja kahden termin polynomia
\termi{binomiksi}{binomi}.

\begin{esimerkki}
Polynomilausekkeen $2x^4-x^3-7$ termit ovat $2x^4$, $-x^3$ ja $-7$. \\
$2x^4$ on neljännen asteen termi, $-x^3$ on kolmannen asteen termi ja $-7$ on nollannen
asteen termi eli vakiotermi. Polynomin aste on siis 4.

\end{esimerkki}

Kuten muutkin yhden muuttujan lausekkeet, myös polynomit määrittelevät
funktion. Polynomien määrittelemiä funktioita kutsutaan
\termi{polynomifunktioiksi}{polynomifunktio} tai pelkästään polynomeiksi. Polynomifunktioita
merkitään usein kirjaimilla $P$, $Q$ tai $R$. Siis samoin kuin muillakin
funktioilla, merkintä $P(a)$ tarkoittaa polynomin $P$ arvoa kun muuttujan
paikalle sijoitetaan $a$.
\begin{esimerkki}
Jos polynomi $P$ on $P(x) = 5x^2-3x+2$,
\begin{align*}
&P(-3) = 5(-3)^2-3(-3)+2 = 45 + 9 + 2 = 56 \\
&P(-1) = 5(-1)^2-3(-1)+2 = 5 + 3 + 2 = 10 \\
&P(2) = 5\cdot 2^2-3\cdot 2+2 = 20 - 6 + 2 = 16.
\end{align*}
\end{esimerkki}

\laatikko{
Polynomi $P$, joka on astetta $n$, voidaan aina sieventää muotoon
\[P(x) = a_n x^n + a_{n-1} x^{n-1} + \ldots + a_1 x + a_0,\]
missä $a_0, a_1, \ldots, a_n$ ovat vakioita ja $a_n \neq 0$.
}

\section{Harjoitustehtäviä}

\begin{tehtava}
	Mikä on/mitkä ovat polynomin $P(x) = x^5-3x^3+2x-1$
	\begin{enumerate}[a)]
		\item aste
		\item termit
		\item kolmannen asteen termi
		\item vakiotermi
	\end{enumerate}

	\begin{vastaus}
		\begin{enumerate}[a)]
			\item $5$
			\item $x^5$, $-3x^3$, $2x$, $-1$
			\item $-3x^3$
			\item $-1$
		\end{enumerate}
	\end{vastaus}
\end{tehtava}

\begin{tehtava}
	Mikä on seuraavien polynomien aste?
	\begin{enumerate}[a)]
		\item $x^2 + 3x - 5$
		\item $100 + x$
		\item $3x^3 + 90x^8 + 2x$
		\item $12x^1 + 34x^2 + 56x^3 + 78x^5 + 90x^5$
	\end{enumerate}

	\begin{vastaus}
		\begin{enumerate}[a)]
			\item $2$
			\item $1$
			\item $8$
			\item $5$
		\end{enumerate}
	\end{vastaus}
\end{tehtava}

% Hämäävä tehtävä
% \begin{tehtava}
% 	Sievennä.
% 	\begin{enumerate}[a)]
% 		\item $2 + x$
% 		\item $3x - 1 + 2x^2$
% 		\item $x^3 - \dfrac{2}{3}x^5 + 2x^4 $
% 		\item $1 - 15x + \pi x^{10} + 2$
% 		\item $-2x + 3 + 5x - 5x^2 - 5$
% 	\end{enumerate}
%
% 	\begin{vastaus}
% 		\begin{enumerate}[a)]
% 			\item $x + 2$
% 			\item $2x^2 + 3x - 1$
% 			\item $-\dfrac{2}{3}x^5 + 2x^4 + x^3 $
% 			\item $\pi x^10 - 15x + 3$
% 			\item $-5x^2 + 3x - 2$
% 		\end{enumerate}
% 	\end{vastaus}
% \end{tehtava}

\begin{tehtava}
	Mitkä seuraavista polynomilausekkeista esittävät samaa polynomia kuin
	$x^3+2x+1$?
	\begin{enumerate}[a)]
		\item $2x+x^3+1$
		\item $x^2+2x+1$
		\item $x+2x^3+1$
		\item $15+x^4+2x+x^3-x^4-14$
	\end{enumerate}
	\begin{vastaus}
		a) ja d)
	\end{vastaus}
\end{tehtava}
