\chapter{Polynomi}
Usein hyödyllinen lauseketyyppi on sellainen joka koostuu muuttujan $x$ ja
vakioiden yhteen- ja kertolaskuista. Tällaisia lausekkeita kutsutaan muuttujan
$x$ polynomeiksi. Esimerkiksi seuraavat lausekkeet ovat polynomeja:
\begin{itemize}
\item $24x + 5 - 6x$
\item $5x^2+7$
\item $x^4-6x^3+2x^2-x$
\end{itemize}

\laatikko{
Polynomien käsitteitä
\begin{itemize}
\item \emph{Termi} on vakion ja muuttujan potenssin tulo.
\item Termin \emph{aste} on sen muuttujan eksponentti.
\item Termin aste voi myös olla 0. Tällaista termiä kutsutaan \emph{vakiotermiksi}.
\item \emph{Polynomi} on termien summa.
\item Polynomin \emph{asteluku} on sen termien suurin asteluku.
\item Yksitermistä polynomia kutsutaan myös \emph{monomiksi}.
\item Kaksitermistä polynomia kutsutaan myös \emph{binomiksi}.
\todo[inline]{Onko pakko olla samanmuotoisia termejä tai trinomeja jne?}
\end{itemize}
}

\todo{
	Miten selitetään että polynomit $x+3x$ ja $4x$ ovat samoja vaikka niissä on
	eri määrä termejä ja apua? Halutaanko määritellä polynomit suoraa tietyssä
	esitystavassa?
}

Polynomeja merkitään usein kirjaimilla $P$, $Q$, \ldots. Merkintä $P(a)$
tarkoittaa polynomin arvoa kun muuttujan paikalle sijoitetaan $a$. Esimerkiksi
jos polynomi $P$ on muotoa $P(x) = 5x^2-3x+2$,
\begin{align*}
&P(-3) = 5(-3)^2-3(-3)+2 = 45 + 9 + 2 = 56 \\
&P(-1) = 5(-1)^2-3(-1)+2 = 5 + 3 + 2 = 10 \\
&P(2) = 5\cdot 2^2-3\cdot 2+2 = 20 + 6 + 2 = 28.
\end{align*}

\laatikko{
Polynomi $P$, joka on astetta $n$, voidaan aina sieventää muotoon
\[P(x) = a_n x^n + a_{n-1} x^{n-1} + \ldots + a_1 x + a_0,\]
missä $a_0, a_1, \ldots, a_n$ ovat vakioita ja $a_n \neq 0$.
}

\section{Harjoitustehtäviä}
