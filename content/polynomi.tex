\chapter{Polynomi}
Usein hyödyllinen lauseketyyppi on sellainen joka koostuu muuttujan $x$ ja
vakioiden yhteen- ja kertolaskuista. Tällaisia lausekkeita kutsutaan muuttujan
$x$ polynomeiksi. Esimerkiksi seuraavat lausekkeet ovat polynomeja:
\begin{itemize}
\item $24x + 5 - 6x$
\item $5x^2+7$
\item $x^4-6x^3+2x^2-x$
\end{itemize}

\laatikko{
Polynomien käsitteitä
\begin{itemize}
\item \emph{Termi} on vakion ja muuttujan potenssin tulo.
\item Termin \emph{aste} on sen muuttujan eksponentti.
\item Termin aste voi myös olla 0. Tällaista termiä kutsutaan \emph{vakiotermiksi}.
\item \emph{Polynomi} on termien summa.
\item Polynomin \emph{asteluku} on sen termien suurin asteluku.
\item Yksitermistä polynomia kutsutaan myös \emph{monomiksi}.
\item Kaksitermistä polynomia kutsutaan myös \emph{binomiksi}.
\todo[inline]{Onko pakko olla samanmuotoisia termejä jne?}
\end{itemize}
}

\section{Harjoitustehtäviä}
