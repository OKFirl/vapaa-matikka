\chapter{Polynomien kertolasku}

\subsection*{Polynomin kertominen monomilla}

Polynomeja voi kertoa keskenään aivan kuten reaalilukuja. Yksinkertaisin tapaus on polynomin kertominen monomilla.

\subsubsection*{Esimerkkejä}
\begin{itemize}
    \item $2x(3x^2+5x+7) = 6x^3+10x^2+14x$
    \item $5x(3y^2+4y) = 15xy^2+20xy$
    \item $x(y+z) = xy+xz$
\end{itemize}

\subsection*{Kahden binomin tulo}

Toinen yksinkertainen tapaus on kahden binomin tulo. Tällöin kerrotaan termeittäin ja summataan.

\subsubsection*{Esimerkkejä}
\begin{itemize}
    \item $(x+y)(z+w) = xz+xw+yz+yw$
    \item $(ax+by)(x+y) = ax^2+axy+bxy+by^2 = ax^2+(a+b)xy+by^2$
    \item $(x-5)(x-7) = x^2-5x-5x+35 = x^2-10x+35$
\end{itemize}

\subsection*{Yleinen kertolasku}
Kahden polynomin tulo saadaan osittelulailla muotoon jossa kerrotaan polynomia
monomilla.
\begin{esimerkki}
\begin{align*}
&\hspace{0.5cm}(x^4-3x^3+3)(x^3-2x^2+1) \\
&= x^4 (x^3-2x^2+1) - 3x^3 (x^3-2x^2+1) + 3 (x^3-2x^2+1) \\
&= x^4\cdot x^3 + x^4\cdot (-2x^2)+x^4\cdot 1-3x^3\cdot x^3-3x^3(-2x^2)-3x^3\cdot1+3x^3+3(-2x^2)+3\cdot 1 \\
&= x^7-2x^6+x^4-3x^6+6x^5-3x^3+3x^3-6x^2+3 \\
&= x^7-5x^6+6x^5+x^4-6x^2+3
\end{align*}
\end{esimerkki}

Siis kahden polynomin tulo saadaan laskemalla yhteen kaikki monomit, jotka
saadaan kertomalla termi ensimmäisestä ja toinen termi toisesta polynomista.

\missingfigure{Kuva jossa visualisoidaan kivasti termiparien valintaa.}

\subsection*{Polynomin tekijät}

\section{Muistikaavat}

Eräitä polynomien kertolaskuja tarvitaan niin usein, että niitä kutsutaan muistikaavoiksi.

\laatikko{
    Muistikaavat
    \begin{itemize}
        \item $(a+b)^2 = a^2+2ab+b^2$
        \item $(a-b)^2 = a^2-2ab+b^2$
        \item $(a+b)(a-b) = a^2-b^2$
    \end{itemize}
}

\subsection*{Muistikaavojen todistukset}

Todistetaan muistikaavat.

\subsubsection*{Summan neliö}

\begin{align*}
(a+b)^2 &= (a+b)(a+b) &\emph{neliön määritelmä} \\
&= a(a+b)+b(a+b) &\emph{osittelulaki} \\
&= a^2+ab+ba+b^2 &\emph{osittelulaki} \\
&= a^2+ab+ab+b^2 &\emph{reaaliluvuille $ab=ba$} \\
&= a^2+2ab+b^2
\end{align*}

\subsubsection*{Erotuksen neliö}

\begin{align*}
(a-b)^2 &= (a-b)(a-b) &\emph{neliön määritelmä} \\
&= a(a-b)-b(a-b) &\emph{osittelulaki} \\
&= a^2-ab-ba+b^2 &\emph{osittelulaki} \\
&= a^2-ab-ab+b^2 &\emph{reaaliluvuille $ba=ab$} \\
&= a^2-2ab+b^2
\end{align*}

\subsubsection*{Summan ja erotuksen tulo}

\begin{align*}
(a+b)(a-b) &= a(a-b)+b(a-b) &\emph{osittelulaki} \\
&= a^2-ab+ba-b^2 &\emph{osittelulaki} \\
&= a^2-ab+ab-b^2 &\emph{reaaliluvuille $ba=ab$} \\
&= a^2-b^2
\end{align*}

\laatikko{
    Esimerkkejä
    \begin{itemize}
        \item $(3x+2y)^2 = (3x+2y)(3x+2y) = 9x^2+2\cdot 3x\cdot 2y+4y^2 = 9x^2+12xy+4y^2$
        \item $995^2 = (1000-5)^2 = 1000^2-2\cdot 1000\cdot 5+5^2 = 1000000-10000+25 = 990025$
        \item $104\cdot 96 = (100+4)(100-4) = 100^2 - 4^2 = 10000 - 16 = 9984$
    \end{itemize}
    }

\subsection{Tekijöihinjako}

Matematiikan 1. kurssissa on puhuttu lukujen jakamisesta tekijöihin.
Esimerkiksi luvun $12$ {\bf tekijät} ovat $1$, $2$, $3$, $4$, $6$ ja $12$. Nämä ovat sellaisia
lukuja, joista saadaan $12$ kertomalla ne jollain kokonaisluvulla. Sanotaan myös, että luvun $12$
{\bf alkutekijät} ovat $2$, $2$ ja $3$, koska luku $12$ voidaan
ilmaista niiden tulona ($2\cdot 2\cdot 3 = 2^2\cdot 3 = 12$), mutta näitä tekijöitä ei
voi enää jakaa pienempiin osatekijöihin. Kokonaisluvun tekijät ovat aina kokonaislukuja.

Vastavasti voidaan puhua polynomin jakamisesta {\bf tekijöihin} tai {\bf alkutekijöihin}. Polynomin tekijät
ovat aina polynomeja. Olemme jo oppineet kertomaan polynomeja keskenään,
joten voimme helposti laskea, että $(x-3)(2x^2-8x+8)=2x^3-14x^2+32x-24$.
Nyt voidaan sanoa, että $x-3$ ja $2x^2-8x+8$ ovat polynomin $2x^3-14x^2+32x-24$ tekijöitä.
Mutta ovatko ne alkutekijöitä, vai voidaanko ne edelleen jakaa pienempiin tekijöihin?
Itse asiassa laskemalla voidaan todeta että $2x^2-8x+8$ saadaan tulokseksi kertolaskusta $2(x-2)(x-2)=2(x-2)^2$.
Voimme siis ilmoitaa polynomin tekijöidensä avulla: $2x^3-14x^2+32x-24=2(x-3)(x-2)^2$.

Mutta miksi haluaisimme jakaa polynomeja tekijöihin?
Yksi tärkeimmistä hyödyistä liittyy tulon nollasääntöön. Tiedämme nimittäin, että tulo on nolla jos ja vain jos
jokin tulon tekijöistä on nolla. Jos osaamme jakaa polynomin tekijöihin, pystymme näin helposti päättelemään polynomin nollakohdat.

Jos vaikkapa haluamme ratkaista yhtälön $2x^3-14x^2+32x-24=0$ ja satumme tietämään, että $2x^3-14x^2+32x-24=2(x-3)(x-2)^2$,
voimme helposti päätellä, että polynomi saa arvon $0$ jos ja vain jos $x-3=0$ tai $x-2=0$. Yhtälön ainoat ratkaisut ovat siis $x=3$ ja $x=2$.

% Jokainen reaalikertoiminen polynomi voidaan jakaa tekijöihin, jotka ovat korkeintaan toista astetta.

Polynomin jakaminen tekijöihin on siis varsin hyödyllistä. Tekijöihin jakaminen ei aina ole helppoa,
mutta tässä luvussa käsitellään tapoja, joilla tekijöihin jakaminen monissa tilanteissa onnistuu.


\section{Harjoitustehtäviä}

\begin{tehtava}
    Sievennä
    \begin{enumerate}[a)]
        \item $x(x^2 + 1)$
        \item $(x - 5)3x$
        \item $(-2x)(4x - 1)3$
        \item $2x(x + y)$
        \item $(3x^5 + 7)y$
        \item $(-x^3)(10x - 2)$
        \item $5(-2x + 1)(-9x) $
    \end{enumerate}
    \begin{vastaus}
        \begin{enumerate}[a)]
            \item $x^3 + x$
            \item $3x^2 - 15x$
            \item $-24x^2 + 6x$
            \item $2x^2 + xy$
            \item $3x^5y + 7y$
            \item $-10x^4 + 2x^3$
            \item $90x^2 - 45x$
        \end{enumerate}
    \end{vastaus}
\end{tehtava}

\begin{tehtava}
    Sievennä
    \begin{enumerate}[a)]
        \item $(x+2)^2$
        \item $(x-3)^2$
        \item $(x-1)(x+1)$
        \item $(5-x)^2$
        \item $(2x + 8)^2$
        \item $(9 - 7x)(9 + 7x)$
    \end{enumerate}
    \begin{vastaus}
        \begin{enumerate}[a)]
            \item $x^2 + 4x + 4$
            \item $x^2 - 6x + 9$
            \item $x^2 - 1$
            \item $x^2 - 10x + 25$
            \item $4x^2 + 32x + 64$
            \item $-49x^2 + 81$
        \end{enumerate}
    \end{vastaus}
\end{tehtava}

\begin{tehtava}
    Sievennä
    \begin{enumerate}[a)]
        \item $(t+v)^2+(t-v)^2$
        \item $(t+v)^2-(t-v)^2$
    \end{enumerate}
    \begin{vastaus}
        \begin{enumerate}[a)]
            \item $(t+v)^2+(t-v)^2 = t^2+2tv+v^2+t^2-2tv+v^2 = 2t^2+2v^2$
            \item $(t+v)^2-(t-v)^2 = t^2+2tv+v^2-t^2+2tv-v^2 = 4tv$
        \end{enumerate}
    \end{vastaus}
\end{tehtava}

\begin{tehtava}
    Sievennä (vinkki: käytä summakaavoja)
    \begin{enumerate}[a)]
        \item $63^2+37^2$
        \item $101^2+99^2$
    \end{enumerate}
    \begin{vastaus}
        \begin{enumerate}[a)]
            \item $63^2+37^2 = (50+13)^2+(50-13)^2 = 2\cdot 50^2 + 2\cdot 13^2 = 2\cdot 2500 +2\cdot 169 = 5000 + 338 = 5338$
            \item $101^2+99^2 = (100+1)^2+(100-1)^2 = 2\cdot 100^2 + 2\cdot 1^2 = 2\cdot 10000 + 2\cdot 1 = 20000 + 2 = 20002$
        \end{enumerate}
    \end{vastaus}
\end{tehtava}

\begin{tehtava}
    Sievennä
    \begin{enumerate}[a)]
        \item $35^2-25^2$
        \item $170^2-50^2$
    \end{enumerate}
    \begin{vastaus}
        \begin{enumerate}[a)]
            \item $35^2-25^2 = (30+5)^2-(30-5)^2 = 4\cdot 30\cdot 5 = 600$
            \item $170^2-30^2 = (100+70)^2+(100-70)^2 = 4\cdot 100\cdot 70 = 28000$
        \end{enumerate}
    \end{vastaus}
\end{tehtava}

\begin{tehtava}
    Sievennä
    \begin{enumerate}[a)]
        \item $(x+1)(x+3)$
        \item $(x+2)(x-1)$
        \item $(2x+5)(x+7)$
        \item $(x-1)(x+4)x$
    \end{enumerate}
    \begin{vastaus}
        \begin{enumerate}[a)]
            \item $x^2 + 4x + 3$
            \item $x^2 + x - 2$
            \item $2x^2 + 19x + 35$
            \item $x^3 + 3x^2 - 4x$
        \end{enumerate}
    \end{vastaus}
\end{tehtava}

\begin{tehtava}
    Johda ''muistikaavat'' potenssien
    \begin{enumerate}[a)]
            \item $(a+b)^3$
            \item $(a+b)^4$
            \item $(a-b)^3$
        \end{enumerate}
        aukikertomiseksi.
    \begin{vastaus}
        \begin{enumerate}[a)]
            \item $(a+b)^3 = a^3 + 3a^2b + 3ab^2 + b^3$
            \item $(a+b)^4 = a^4 + 4a^3b + 6a^2b^2 + 4ab^3 + b^4$
            \item $(a-b)^3 = a^3 - 3a^2b + 3ab^2 - b^3$
        \end{enumerate}
    \end{vastaus}
\end{tehtava}

\begin{tehtava}
	Laske polynomien kertolaskut
	\begin{enumerate}[a)]
		\item $(x-3)(2x^3-3x+4)$
		\item $(x^2+1)(x^3-2x-4)$
		\item $(x-1)(x^4+x^3+x^2+x+1)$
		\item $(\frac x5-\frac23)(x^2+x+1)$
	\end{enumerate}
	\begin{vastaus}
		\begin{enumerate}[a)]
			\item $2x^4-6x^3-3x^2+13x-12$
			\item $x^5-x^3-4x^2-2x-4$
			\item $x^5-1$
			\item $\frac15x^3-\frac{7}{15}x^2-\frac{7}{15}x-\frac23$
		\end{enumerate}
	\end{vastaus}
\end{tehtava}
