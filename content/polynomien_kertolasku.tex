\chapter{Polynomien kertolasku}

\subsection*{Polynomin kertominen monomilla}

Polynomeja voi kertoa keskenään aivan kuten reaalilukuja. Yksinkertaisin tapaus on polynomin kertominen monomilla.

\subsubsection*{Esimerkkejä}
\begin{itemize}
    \item $2x(3x^2+5x+7) = 6x^3+10x^2+14x$
    \item $5x(3y^2+4y) = 15xy^2+20xy$
    \item $x(y+z) = xy+xz$
\end{itemize}

\subsection*{Kahden binomin tulo}

Toinen yksinkertainen tapaus on kahden binomin tulo. Tällöin kerrotaan termeittäin ja summataan.

\subsection*{Yleinen kertolasku}
Kahden polynomin tulo saadaan osittelulailla muotoon jossa kerrotaan polynomia
monomilla.
\begin{esimerkki}
\begin{align*}
&\hspace{0.5cm}(x^4-3x^3+3)(x^3-2x^2+1) \\
&= x^4 (x^3-2x^2+1) - 3x^3 (x^3-2x^2+1) + 3 (x^3-2x^2+1) \\
&= x^4\cdot x^3 + x^4\cdot (-2x^2)+x^4\cdot 1-3x^3\cdot x^3-3x^3(-2x^2)-3x^3\cdot1+3x^3+3(-2x^2)+3\cdot 1
\end{align*}
\end{esimerkki}


\subsubsection*{Esimerkkejä}
\begin{itemize}
    \item $(x+y)(z+w) = xz+xw+yz+yw$
    \item $(ax+by)(x+y) = ax^2+axy+bxy+by^2 = ax^2+(a+b)xy+by^2$
    \item $(x-5)(x-7) = x^2-5x-5x+35 = x^2-10x+35$
\end{itemize}

\subsection*{Polynomin tekijät}

\section{Muistikaavat}

Eräitä polynomien kertolaskuja tarvitaan niin usein, että niitä kutsutaan muistikaavoiksi.

\laatikko{
    Muistikaavat
    \begin{itemize}
        \item $(x+y)^2 = (x+y)(x+y) = x^2+2xy+y^2$
        \item $(x-y)^2 = (x-y)(x-y) = x^2-2xy+y^2$
        \item $(x+y)(x-y) = x^2-y^2$
    \end{itemize}
    }

\laatikko{
    Esimerkkejä
    \begin{itemize}
        \item $(3a+2b)^2 = (3a+2b)(3a+2b) = 9a^2+2\cdot 3a\cdot 2b+4b^2 = 9a^2+12ab+4b^2$
        \item $995^2 = (1000-5)^2 = (1000-5)(1000-5) = 1000^2-2\cdot 1000\cdot 5+5^2 = 1000000-10000+25 = 990025$
        \item $104\cdot 96 = (100+4)(100-4) = 100^2 - 4^2 = 10000 - 16 = 9984$
    \end{itemize}
    }

\section{Harjoitustehtäviä}

\begin{tehtava}
    Sievennä
    \begin{enumerate}
        \item $(t+v)^2+(t-v)^2$
        \item $(t+v)^2-(t-v)^2$
    \end{enumerate}
    \begin{vastaus}
        \begin{enumerate}
            \item $(t+v)^2+(t-v)^2 = t^2+2tv+v^2+t^2-2tv+v^2 = 2t^2+2v^2$
            \item $(t+v)^2-(t-v)^2 = t^2+2tv+v^2-t^2+2tv-v^2 = 4tv$
        \end{enumerate}
    \end{vastaus}
\end{tehtava}

\begin{tehtava}
    Sievennä
    \begin{enumerate}
        \item $63^2+37^2$
        \item $101^2+99^2$
    \end{enumerate}
    \begin{vastaus}
        \begin{enumerate}
            \item $63^2+37^2 = (50+13)^2+(50-13)^2 = 2\cdot 50^2 + 2\cdot 13^2 = 2\cdot 2500 +2\cdot 169 = 5000 + 338 = 5338$
            \item $101^2+99^2 = (100+1)^2+(100-1)^2 = 2\cdot 100^2 + 2\cdot 1^2 = 2\cdot 10000 + 2\cdot 1 = 20000 + 2 = 20002$
        \end{enumerate}
    \end{vastaus}
\end{tehtava}

\begin{tehtava}
    Sievennä
    \begin{enumerate}
        \item $35^2-25^2$
        \item $170^2-50^2$
    \end{enumerate}
    \begin{vastaus}
        \begin{enumerate}
            \item $35^2-25^2 = (30+5)^2-(30-5)^2 = 4\cdot 30\cdot 5 = 600$
            \item $170^2-30^2 = (100+70)^2+(100-70)^2 = 4\cdot 100\cdot 70 = 28000$
        \end{enumerate}
    \end{vastaus}
\end{tehtava}
