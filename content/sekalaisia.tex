\chapter{Sekalaisia tehtäviä}

Ryhmittele tehtävät luvuittain!
Sijoita nämä tehtävät sopivaan paikkaan, jos ovat hyviä!


\subsubsection*{Polynomifunktion kuvaaja}

\begin{tehtava}
  Aukeavatko seuraavat paraabelit ylös- vai alaspäin?
  \begin{enumerate}[a)]
    \item $4x^2 + 100x - 3$
    \item $-x^2 + 1337$
    \item $5x^2 - 7x + 5$
    \item $-6(-3x^2 + 5)$
    \item $-13x(9 - 17x)$
    \item $100(1-x^2)$
  \end{enumerate}

  \begin{vastaus}
    \begin{enumerate}[a)]
      \item Ylös
      \item Alas
      \item Ylös
      \item Ylös
      \item Ylös
      \item Alas
    \end{enumerate}
  \end{vastaus}
\end{tehtava}

\begin{tehtava}
  \begin{enumerate}[a)]
    \item Ratkaise funktion $2x^2 - 5x - 3$ nollakohdat
    \item Millä arvoilla edellisen kohdan funktio $2x^2 - 5x - 3$ saa positiivisia arvoja?
    \item Onko em. funktiolla globaali raja-arvo (minimi tai maksimi), ja jos on, missä kohtaa funktio saa tämän arvon? Mikä on funktion arvo silloin?
  \end{enumerate}

  \begin{vastaus}
    \begin{enumerate}[a)]
      \item $x = 1.2$ tai $x = -0.2$
      \item $x = \frac{12}{10} = 1.2$ tai $x = -\frac{2}{10} = -0.2$
      \item Koska neliötermin kerroin a on positiivinen (2), funktiolla on globaali minimi (mutta ei ylärajaa). Symmetrian vuoksi minimi on nollakohtien puolivälissä kohdassa 0.5, jossa funktio saa siis pienimmän arvonsa -5.
    \end{enumerate}
  \end{vastaus}
\end{tehtava}

\begin{tehtava}
  Tutki, millä muuttujan x arvoilla seuraavat funktiot saavat positiivisia arvoja.
  \begin{enumerate}[a)]
    \item $x^2 - 4$
    \item $-x^2 - 2x + 3$
    \item $x^2 + 2x + 5$
    \item $-x^2 - 1$
  \end{enumerate}

  \begin{vastaus}
    \begin{enumerate}[a)]
      \item $x \leq -2$ tai $x \geq 2$
      \item $-3 \geq x \leq 1$
      \item Kaikilla x:n arvoilla.
      \item Ei millään x:n arvoilla.
    \end{enumerate}
  \end{vastaus}
\end{tehtava}