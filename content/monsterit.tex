\chapter{Tavoittele valaistumista}

\Opensolutionfile{ans}[ans2]

Tässä on joitakin tehtäviä, jotka on arvioitu hauskoiksi ja hyödyllisiksi kaikkein osaavimmille opiskelijoille. Niiden parissa vietetty aika ei mene hukkaan, vaikkei tehtävä ratkeaisikaan.

\begin{tehtava}
    \begin{enumerate}[a)]
        \item Osoita, että jos polynomi $P$ jaetaan polynomilla $x-1$, niin jakojäännös on polynomin $P$ kertoimien summa.
        \item Päättele tästä, että kokonaisluku $n$ on jaollinen yhdeksällä vain jos sen kymmenjärjestelmäesityksen numeroiden summa on jaollinen yhdeksällä.
    \end{enumerate}
\end{tehtava}

\begin{tehtava} %Vaikea!
    Etsi kaikki positiiviset kokonaisluvut $x$ ja $y$, joille pätee $9x^2-y^2=17$.
    \begin{vastaus}
    Opastus: jaa yhtälön vasen puoli tekijöihin muistikaavalla. 
    Ainoa ratkaisu on $x = 3$, $y=8$.    
    \end{vastaus}
\end{tehtava}

\begin{tehtava} % Sikavaikea (tällä tasolla)
%täydentyy kahdeksi neliöksi, joiden summa on 0
    Ratkaise $x$ ja $y$ yhtälöstä $y^2+2xy+x^4-3x^2+4=0$.
    \begin{vastaus}
        $x=\sqrt{2}, y=-\sqrt{2}$ tai $x=-\sqrt{2}, y=\sqrt{2}$
    \end{vastaus}
\end{tehtava}

\begin{tehtava} % Kaunis
    Ratkaise $x$ ja $y$ yhtälöstä $2x^4+2y^4=4xy-1$.
    \begin{vastaus}
        $x=\frac{\sqrt{2}}{2}, y=\frac{\sqrt{2}}{2}$ tai $x=-\frac{\sqrt{2}}{2}, y=-\frac{\sqrt{2}}{2}$
    \end{vastaus}
\end{tehtava}

\Closesolutionfile{ans}[ans2]

\subsection*{Vastaukset}

\begin{Vastaus}{89}
    Opastus: jaa yhtälön vasen puoli tekijöihin muistikaavalla.
    Ainoa ratkaisu on $x = 3$, $y=8$.
    
\end{Vastaus}
\begin{Vastaus}{90}
        $x=\sqrt{2}, y=-\sqrt{2}$ tai $x=-\sqrt{2}, y=\sqrt{2}$
    
\end{Vastaus}
\begin{Vastaus}{91}
        $x=\frac{\sqrt{2}}{2}, y=\frac{\sqrt{2}}{2}$ tai $x=-\frac{\sqrt{2}}{2}, y=-\frac{\sqrt{2}}{2}$
    
\end{Vastaus}
