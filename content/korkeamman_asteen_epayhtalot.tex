\chapter{Korkeamman asteen epäyhtälöt}

Ville työstää tätä nyt (la klo 14.34 alkaen)!

Korkeamman asteen epäyhtälö, kuten epäyhtälö 
$$x^3 -6 \leq x^2 $$
ratkaistaan siirtämällä kaikki termit epäyhtälön toiselle puolelle ja tutkimalla
syntyvän polynomin merkkiä:
\begin{align*}
x^3 -6x &\leq x^2 \quad || \ \ -x^2 \\
\underbrace{x^3-x^2 -6}_{P(x)} &\leq 0.
\end{align*}
Polynomin merkin selvittämiseksi rakaistaan sen nollakohdat:
\begin{align*}
x^3 - x^2-6x &= 0 \quad || \ \ \textrm{yhteinen tekijä} \\
x(x^2 -x -6) &= 0 \quad || \ \ \textrm{tulon nollasääntö} \\
x = 0 \textrm{ tai } x^2 -x -6 &= 0 \quad|| \ \ \textrm{ratkaisukaava} \\
x &=\frac{-(-1) \pm \sqrt{(-1)^2-4\cdot 1 \cdot (-6)}}{2\cdot 1} \\
x &= -2 \textrm{ tai } x = 3.
\end{align*}
Polynomin $P$ nollakohdat ovat siis $0$, $2$ ja $3$. Tästä voidaan jatkaa kahdella eri tavalla.

\textbf{Tapa 1:} Tekijöihin jako

Jaetaan polynomi tekijöihin nollakohtien avulla: 
$$P(x) = x^3 - x^2-6x = x(x^2-x-6) = x(x+2)(x-3).$$
Tutkitaan kunkin tulon tekijän merkkiä:\\
\begin{tabular}{ll} 
$x+2>0$ & kun $x > -2$\\
$x-3>0$ & kun $x > 3$\\
$x>0$ & kun $x > 0$.
\end{tabular}
Kootaan tulokset merkkikaavioon:

\missingfigure{Tähän merkkikaavio, jossa x-akselilla -2, 0 ja 3,
rivejä neljä: termit x, (x+2), (x-3) ja tulo x(x+2)(x-3) }

Merkkikaaviosta voidaan lukea vastaus: $x^3-x^2 -6 \leq 0$, kun
$x\leq -2$ tai $0\leq x \leq 3$.

\textbf{Tapa 2:} Testipisteet

\section{Harjoitustehtäviä}
