\chapter{Korkeamman asteen epäyhtälöt}

Ville työstää tätä nyt (la klo 14.34 alkaen)!

Korkeamman asteen epäyhtälö, kuten epäyhtälö 
$$x^3 -6 \leq x^2 $$
ratkaistaan siirtämällä kaikki termit epäyhtälön toiselle puolelle ja tutkimalla
syntyvän polynomin merkkiä:
\begin{align*}
x^3 -6 &\leq x^2 \quad || \ \ -x^2 \\
\underbrace{x^3-x^2 -6}_{P(x)} &\leq 0.
\end{align*}
Polynomin merkin selvittämiseksi rakaistaan sen nollakohdat:
\begin{align*}
x^3 - x^2-6 &= 0 \quad || \ \ \textrm{yhteinen tekijä} \\
x(x^2 -x -6) &= 0 \quad || \ \ \textrm{tulon nollasääntö} \\
x = 0 \textrm{ tai } x^2 -x -6 &= 0 \quad|| \ \ \textrm{ratkaisukaava} \\
x &=\frac{-(-1) \pm \sqrt{(-1)^2-4\cdot 1 \cdot (-6)}}{2\cdot 1} \\
x &= 2 \textrm{ tai } x = 3.
\end{align*}

\section{Harjoitustehtäviä}
