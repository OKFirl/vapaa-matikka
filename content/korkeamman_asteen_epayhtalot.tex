\chapter{Korkeamman asteen epäyhtälöt}

Korkeamman asteen epäyhtälö, kuten epäyhtälö 
$$x^3 -6 \leq x^2 $$
ratkaistaan siirtämällä kaikki termit epäyhtälön toiselle puolelle ja tutkimalla
syntyvän polynomin merkkiä:
\begin{align*}
x^3 -6x &\leq x^2 \quad || \ \ -x^2 \\
\underbrace{x^3-x^2 -6}_{P(x)} &\leq 0.
\end{align*}
Polynomin $P(x)$ merkin selvittämiseksi rakaistaan sen nollakohdat:
\begin{align*}
x^3 - x^2-6x &= 0 \quad || \ \ \text{yhteinen tekijä} \\
x(x^2 -x -6) &= 0 \quad || \ \ \text{tulon nollasääntö} \\
x = 0 \textrm{ tai } x^2 -x -6 &= 0 \quad|| \ \ \text{ratkaisukaava} \\
x &=\frac{-(-1) \pm \sqrt{(-1)^2-4\cdot 1 \cdot (-6)}}{2\cdot 1} \\
x &= -2 \textrm{ tai } x = 3.
\end{align*}
Polynomin $P$ nollakohdat ovat siis $0$, $2$ ja $3$. Tästä voidaan jatkaa kahdella eri tavalla.

\textbf{Tapa 1:} Tekijöihin jako

Jaetaan polynomi tekijöihin nollakohtien avulla: 
$$P(x) = x^3 - x^2-6x = x(x^2-x-6) = x(x+2)(x-3).$$
Tutkitaan kunkin tulon tekijän merkkiä:\\
\quad \begin{tabular}{ll} 
$x+2>0$ & kun $x > -2$\\
$x-3>0$ & kun $x > 3$\\
$x>0$ & kun $x > 0$.
\end{tabular}

Kootaan tulokset merkkikaavioon:

\missingfigure{Tähän merkkikaavio, jossa x-akselilla -2, 0 ja 3,
rivejä neljä: termit x, (x+2), (x-3) ja tulo x(x+2)(x-3) }

Merkkikaaviosta voidaan lukea vastaus: $x^3-x^2 -6 \leq 0$, kun
$x\leq -2$ tai $0\leq x \leq 3$.

\textbf{Tapa 2:} Testipisteet\\
Polynomit ovat niin sanottuja jatkuvia funktioita. (Asiasta lisää kurssilla 7, nyt
riittää, että niiden kuvaajat muodostavat yhtenäisen viivan.) Jos siis polynomilla on positiivinen ja negatiivinen arvo, niiden välissä täytyy olla nollakohta. Polynomin ei voi vaihtaa merkkiä ilman, että sen kuvaaja kulkee nollan kautta.

Polynomin arvon merkki ei siis muutu kahden vierekkäisen nollakohdan välissä. Merkin saa selville ottamalla kultakin välitä testipisteen.

Esimerkissä nollakohdat olivat $-2$, $0$ ja $3$. Valitaan niiden välistä ja
ympäriltä testipisteiksi vaikkapa $x=-3$, $x=-1$, $x=1$, $x=4$. Tarkistetaan funktion merkki:

\begin{tabular}{c|l|r|c}
Väli & Testipiste & $f(x)=x^3-x^2-6x$ & Funktion merkki \\
\hline
\ \ \ \ \  $x < -2$ & $x = -3$ & $(-3)^3 -(-3)^2 - 6(-3) = -18$ & $-$ \\
$-2 <x < 0$ & $x = -1$ & $(-1)^3 -(-1)^2 - 6(-1) =4$ & $+$ \\
$0 <x < 3$ & $x = 1$ & $1^3 -1^2 - 6\cdot 1 =  -6$ & $-$ \\
$3 <x $ \ \ \ \ \ & $x = 4$ & $4^3 -4^2 - 6\cdot 4 = 24$ & $+$ 
\end{tabular}

Vastaukseksi saadaan taas $x\leq -2$ tai $0\leq x \leq 3$.


\section{Harjoitustehtäviä}
