\chapter{Polynomifunktion kuvaaja}
Polynomifunktioita on monesti hyödyllistä
havainnollistaa koordinaatistoon piirrettynä kuvaajana.
%Tavallisesti $xy$-koordinaatistossa pystyakselin arvot eli korkeus vastaavat
%funktion arvoja,
%joten polynomin $P$ kuvaaja on siis sama asia kuin käyrän $y = P(x)$.

Nämä kuvaajat voisi piirtää eri koordinaatistoihin.

\begin{kuvaajapohja}{1.5}{-2}{2}{-3}{3}
\kuvaaja{-x-1}{$P(x) = -x-1$}{red}
\kuvaaja{x**2+x}{$Q(x) = x^2+x$}{blue}
\kuvaaja{x**3-3*x-1}{$R(x) = x^3-3x-1$}{green}
\end{kuvaajapohja}


\section*{Kuvaajan piirtäminen}

Funktion kuvaajien piirtäminen käsin onnistuu yleensä helpoiten laskemalla
funktion arvoja muutamilla muuttujan arvoilla, merkitsemällä pisteet
$xy$-koordinaatistoon ja hahmottamalla pisteiden kautta kulkeva kuvaaja.

\begin{esimerkki}
Hahmotellaan polynomifunktion $f(x) = \dfrac{1}{2}x^2 - x - 2$ kuvaaja.
Lasketaan ensin joitakin funktion $f(x) = \dfrac{1}{2}x^2 - x - 2$ arvoja ja piirretään niitä vastaavat pisteet
koorinaatistoon. Lopuksi hahmotellan kuvaaja, joka kulkee pisteiden kautta.

\begin{tabular}{c c c}
	\begin{tabular}{|c|r @{,} l|}
	\hline $x$ & \multicolumn{2}{c|}{$f(x)$} \\
	\hline
	-3 & 5&5 \\
	-2 & 2&0 \\
	-1 & -0&5 \\
	0 & -2&0 \\
	1 & -2&5 \\
	2 & -2&0 \\
	3 & -0&5 \\
	\hline
	\end{tabular}
	&
	\vcent{\begin{kuvaajapohja}{0.6}{-4}{4}{-3}{6}
	\kuvaajapiste{-3}{5.5}
	\kuvaajapiste{-2}{2}
	\kuvaajapiste{-1}{-0.5}
	\kuvaajapiste{0}{-2}
	\kuvaajapiste{1}{-2.5}
	\kuvaajapiste{2}{-2}
	\kuvaajapiste{3}{-0.5}
	\end{kuvaajapohja}}
	&
	\vcent{\begin{kuvaajapohja}{0.6}{-4}{4}{-3}{6}
	\kuvaajapiste{-3}{5.5}
	\kuvaajapiste{-2}{2}
	\kuvaajapiste{-1}{-0.5}
	\kuvaajapiste{0}{-2}
	\kuvaajapiste{1}{-2.5}
	\kuvaajapiste{2}{-2}
	\kuvaajapiste{3}{-0.5}
	\kuvaaja{0.5*x**2-x-2}{$f(x) = \dfrac{1}{2}x^2 - x - 2$}{red}
	\end{kuvaajapohja}}
\end{tabular}

\end{esimerkki}

\section*{Kuvaajan tulkintaa}

%Ensimmäisen asteen polynomin kuvaaja on luonnollisesti aina suora. 
%miten niin luonnollisesti?

Kuvaajan avulla voidaan tehdä johtopäätöksiä funktion ominaisuuksista.
Esimeriksi funktion arvoja voidaan lukea kuvaajasta.

\begin{esimerkki}
Alla on esitetty erään polynomifunktion $P(x)$ kuvaaja. Kuvaajan perusteella näyttäisi siltä, että $P(1)=(-1)$,
mutta tarkkaa arvoa kuvaajasta ei voi päätellä. Hyvin varmoja voidaan kuitenkin olla siitä, että arvo on negatiivinen.

Tähän tarvitaan funktion $x^4-2x^3-x^2+9/10$ kuvaaja.
\end{esimerkki}

Funktion nollakohta on sellainen muuttujan arvo, jolla funktio saa arvon nolla. Esimerkiksi funktiolla $P(x)=x^2-1$
on nollakohdat $x=-1$ ja $x=1$, sillä $P(1)=0$ ja $P(-1)=0$. Polynomien tapauksessa nollakohtia on tapana kutsua
\termi[juuriksi]{juuri, polynomin}.

Nollakohdat näkyvät funktion kuvaajassa. Niiden kohdalla kuvaaja leikkaa $x$-akselin.

\begin{esimerkki}
Tähän kuva vaikkapa funktiosta $P(x)=x^2-1$.
\end{esimerkki}

%Nollakohta tarkoittaa sitä annetun polynomin muuttujan arvoa, jolla koko
%polynomi saa arvon nolla. Kuvaajasta sen voi helposti lukea niinä kohtina,
%joissa kuvaaja leikkaa muuttujan koordinaattiakselin. Funktion $P(x)$ ja
%$xy$-koordinaatiston tapauksessa funktion nollakohdat ovat täsmälleen ne
%$x$-koordinaatit, joilla funktion kuvaaja leikkaa $x$-akselin.

%\section{Taylorin sarja}
%Eräs mielenkiintoinen ja hyvin tunnettu potenssisarja on Taylorin sarja.
%Se on päättymätön potenssisarja, jolla voidaan approksimoida muiden funktioiden
%arvoja.
%
%Yleisesti Taylorin sarjalla saadaan (rajatta derivoituvan) funktion $f$ arvo
%pisteessä $x_0$:
%
%\begin{align*}
%	f(x_0) = \sum\limits_{n=0}^\infty a_n(x-x_0)^n
%\end{align*}
%
%missä
%
%\begin{align*}
%a_n = \frac{f^n(x_0)}{n!}
%\end{align*}
%
%Koska sarja on äärettömän pitkä, sarjan arvoja edelleen arvioidaan Taylorin
%polynomilla, joka on muotoa
%
%\begin{align*}
%	P_k(x) = \sum\limits_{n=0}^k a_k(x-x_0)^k
%\end{align*}
%
%Polynomin avulla voidaan laskea esimerkiksi likiarvo funktiolle
%$(1-x)^{-1} = \frac{1}{1-x}$ pisteen a ympäristössä, kun $a \neg 1$:
%
%\begin{align*}
%	\frac{1}{1-x} \approx \frac{1}{1-a} + \frac{x-a}{(1-a)^2} +
%\frac{(x-a)^2}{(1-a)^3} + \frac{(x-a)^3}{(1-a)^4} ...
%\end{align*}
%
%\missingfigure{Funktion $(x-1)^-1$ kuvaaja}
%\missingfigure{Funktion $\frac{1}{1-a} + \frac{x-a}{(1-a)^2} +
%\frac{(x-a)^2}{(1-a)^3} +$ kuvaaja}

\Harjoitustehtavat

\begin{tehtava}
    Piirrä polynomien kuvaajat.
    \begin{enumerate}[a)]
        \item $5$
        \item $2x-3$
        \item $x^2+x-2$
        \item $x^3-x+3$
    \end{enumerate}   
    \begin{vastaus}
    	\item \begin{kuvaajapohja}{0.4}{-4}{4}{-1}{7}
				\kuvaaja{5}{}{red}
			  \end{kuvaajapohja}
    	\item \begin{kuvaajapohja}{0.4}{-4}{4}{-5}{3}
				\kuvaaja{2*x-3}{}{red}
			  \end{kuvaajapohja}
		\item \begin{kuvaajapohja}{0.4}{-4}{4}{-3}{5}
				\kuvaaja{x**2+x-2}{}{red}
			  \end{kuvaajapohja}
		\item \begin{kuvaajapohja}{0.4}{-4}{4}{-3}{5}
				\kuvaaja{x**3-x+2}{}{red}
			  \end{kuvaajapohja}
    \end{vastaus}
\end{tehtava}

\begin{tehtava}
    Piirrä polynomien kuvaajat.
    \begin{enumerate}[a)]
        \item $x+4$
        \item $2x-9$
        \item $5x+2$
        \item $6x+1$
    \end{enumerate}
    \begin{vastaus}
        \item \begin{kuvaajapohja}{0.4}{-4}{4}{-1}{7}
				\kuvaaja{x+4}{}{red}
			  \end{kuvaajapohja}
    	\item \begin{kuvaajapohja}{0.4}{-2}{6}{-6}{2}
				\kuvaaja{2*x-9}{}{red}
			  \end{kuvaajapohja}
		\item \begin{kuvaajapohja}{0.4}{-4}{4}{-2}{6}
				\kuvaaja{5*x+2}{}{red}
			  \end{kuvaajapohja}
		\item \begin{kuvaajapohja}{0.4}{-4}{4}{-2}{6}
				\kuvaaja{6*x+1}{}{red}
			  \end{kuvaajapohja}
    \end{vastaus}
\end{tehtava}

\begin{tehtava}
    Piirrä polynomien kuvaajat.
    \begin{enumerate}[a)]
        \item $x^2-1$
        \item $2x^2$
        \item $4x^2+4$
        \item $x^2-6x+3$
    \end{enumerate}
    \begin{vastaus}
        \item \begin{kuvaajapohja}{0.4}{-4}{4}{-2}{6}
				\kuvaaja{x+4}{}{red}
			  \end{kuvaajapohja}
    	\item \begin{kuvaajapohja}{0.4}{-4}{4}{-1}{7}
				\kuvaaja{2*x-9}{}{red}
			  \end{kuvaajapohja}
		\item \begin{kuvaajapohja}{0.4}{-4}{4}{-1}{11}
				\kuvaaja{5*x+2}{}{red}
			  \end{kuvaajapohja}
		\item \begin{kuvaajapohja}{0.4}{-1}{7}{-7}{5}
				\kuvaaja{6*x+1}{}{red}
			  \end{kuvaajapohja}
    \end{vastaus}
\end{tehtava}

\begin{tehtava}
    Piirrä polynomien kuvaajat.
    \begin{enumerate}[a)]
        \item $5x^2$
        \item $3x^2+4$
        \item $2x^2-10$
        \item $x^2-x-1$
    \end{enumerate}
    \begin{vastaus}
        \item \begin{kuvaajapohja}{0.4}{-4}{4}{-1}{7}
				\kuvaaja{5*x**2}{}{red}
			  \end{kuvaajapohja}
    	\item \begin{kuvaajapohja}{0.4}{-4}{4}{-1}{11}
				\kuvaaja{3*x**2+4}{}{red}
			  \end{kuvaajapohja}
		\item \begin{kuvaajapohja}{0.4}{-4}{4}{-11}{1}
				\kuvaaja{2*x**2-10}{}{red}
			  \end{kuvaajapohja}
		\item \begin{kuvaajapohja}{0.4}{-3}{5}{-2}{6}
				\kuvaaja{x**2-x-1}{}{red}
			  \end{kuvaajapohja}
    \end{vastaus}
\end{tehtava}

\begin{tehtava}
	Monia funktioita voidaan esittää likimääräisesti polynomeina (ns.
Taylorin polynomi). Esimerkiksi

	\begin{tabular}{lcll}
	$\frac{1}{1+x^2}$ &$\approx$ & $1-x^2+x^4-x^6+x^8-x^{10}$, & kun
$-1<x<1$ \\
	$\sqrt{1+x}$ & $\approx $ & $ 1+\frac{x}{2}
	-\frac{x^2}{8}+\frac{x^3}{16}-\frac{5x^4}{128}$, & kun $-1<x<1$
	\end{tabular}

	Piirrä alkuperäinen funktio ja polynomi samaan kuvaajaan tietokoneella
tai graafisella laskimella. Kokeile, kuinka polynomin viimeisten termien pois
jättäminen vaikuttaa tarkkuuteen. Mitä havaitset? (Termejä voi laskea lisääkin,
mutta siihen ei puututa tässä.)

	\begin{vastaus}
		Mitä enemmän termejä, sitä parempi vastaavuus.
	\end{vastaus}
\end{tehtava}
