\chapter{Polynomifunktion kuvaaja}
Kuten muitakin funktioita, polynomifunktioita on monesti hyödyllistä
havainnollistaa koordinaatistoon piirrettynä kuvaajana. Polynomin
$P$ kuvaaja on siis sama asia kuin käyrän $y = P(x)$ kuvaaja.

\begin{kuvaajapohja}{1.5}{-2}{2}{-3}{3}
\kuvaaja{-x-1}{$P(x) = -x-1$}{red}
\kuvaaja{x**2+x}{$Q(x) = x^2+x$}{blue}
\kuvaaja{x**3-3*x-1}{$R(x) = x^3-3x-1$}{green}
\end{kuvaajapohja}

\section{Taylorin sarja}
Eräs mielenkiintoinen ja hyvin tunnettu potenssisarja on Taylorin sarja.
Se on päättymätön potenssisarja, jolla voidaan aproksimoida muiden funktioiden arvoja.

Yleisesti Taylorin sarjalla saadaan (rajatta derivoituvan) funktion $f$ arvo pisteessä $x_0$:

\begin{align*}
	f(x_0) = \sum\limits_{n=0}^\infty a_n(x-x_0)^n
\end{align*}

missä

\begin{align*}
a_n = \frac{f^n(x_0)}{n!}
\end{align*}

Koska sarja on äärettömän pitkä, sarjan arvoja edelleen arvioidaan Taylorin polynomilla, joka on muotoa

\begin{align*}
	P_k(x) = \sum\limits_{n=0}^k a_k(x-x_0)^k
\end{align*}

Polynomin avulla voidaan laskea esimerkiksi likiarvo funktiolle
$(1-x)^-1 = \frac{1}{1-x}$ pisteen a ympäristössä:

\begin{align*}
	\frac{1}{1-x} \approx \frac{1}{1-a} + \frac{x-a}{(1-a)^2} + \frac{(x-a)^2}{(1-a)^3} + \frac{(x-a)^3}{(1-a)^4} ...
\end{align*}


\section{Harjoitustehtäviä}
\begin{tehtava}
	Piirrä polynomien kuvaajat.
	\begin{enumerate}
		\item $5$
		\item $2x-3$
		\item $x^2+x-2$
		\item $x^3-x+3$
	\end{enumerate}

	\begin{vastaus}
		\todo[inline]{laitetaanko piirtotehtäviin vastaukset tänne}
	\end{vastaus}
\end{tehtava}