\chapter{Epäyhtälöt}

\section{Epäyhtälömerkit}

Olemme yhtälöiden yhteydessä käyttäneet merkintää $=$ osoittamaan, että kaksi lukua (tai monimutkaisempaa lauseketta) ovat arvoltaan yhtä suuret. Lukujen välillä on olemassa kuitenkin muitakin käytännön kannalta hyödyllisiä riippuvuussuhteita eli \emph{relaatioita}. Seuraavia \emph{epäyhtälömerkkejä} käytetään kuvaamaan yhtäsuuruudesta poikkeavia kahden reaaliluvun välisiä yhteyksiä:

\begin{tabular}{ll}
lukujen välinen yhteys & lausumistapa \\
$a<b$ &  "$a$ on pienempi kuin $b$"\\
$a>b$ & "$a$ on suurempi kuin $b$"\\
$a \leq b$ & "$a$ on pienempi tai yhtä suuri kuin $b$" \\
$a \geq b$ & "$a$ on suurempi tai yhtä suuri kuin $b$" \\
$a\neq b$ & "$a$ ei ole yhtä suuri kuin $b$" \\
\end{tabular}

Luonnollisen kielen ilmaisu "suurempi tai yhtä suuri kuin" voidaan myös ilmaista
"vähintään yhtä suuri kuin" ja vastaavasti "pienempi tai yhtä suuri kuin"
voidaan sanoa "korkeintaan yhtä suuri kuin".

Kaikki yhteydet voi halutessaan esittää myös kielteisessä muodossa:

\begin{tabular}{ll}
$a \nless b$ &  "$a$ ei ole pienempi kuin $b$"\\
$a \ngtr b$ & "$a$ ei ole suurempi kuin $b$"\\
$a \nleq b$ & "$a$ ei ole pienempi tai yhtä suuri kuin $b$" \\
$a \ngeq b$ & "$a$ ei ole suurempi tai yhtä suuri kuin $b$" \\
\end{tabular}

Yleensä \termi[epäyhtälöistä]{epäyhtälö} puhuttaessa viitataan erityisesti lausekkeisiin, joissa käytetään ''pienempi kuin''- ja ''pienempi tai yhtä suuri kuin'' -merkkiä ja niiden vastineita. Kieltomuodot ovat harvinaisia.

\section{Epäyhtälöiden teoriaa}

Epäyhtälöt käyttäytyvät monessa mielessä yhtälön tavoin. Yhtälö tai epäyhtälö voi olla:

\begin{enumerate}
\item Aina tosi (identtisesti tosi). Tällöin yhtälö esittää väitteen, joka pitää aina paikkansa jokaisessa tilanteessa. Esimerkiksi $\pi=\pi$ tai $x-5\leq x$ ovat aina tosia.
\item Joskus tosi (ehdollisesti tosi). Yhtälö tai epäyhtälö pätee vain tietyissä tilanteissa, yleensä vain tietyillä jonkun muuttujan arvoilla. Esimerkiksi $x^2=4$ pitää paikkansa vain, kun muuttuja $x$ saa arvon $2$ tai $-2$. Epäyhtälö $x<0$ pätee vain negatiivisilla $x$:n arvoilla.
\item Aina epätosi (identtisesti epätosi). Tällaiset yhtälöt ja epäyhtälöt esittävät mahdottomia väitteitä, jotka eivät koskaan pidä paikkaansa. Esimerkiksi yhtäsuuruusväite $-1=3$ tai epäyhtälö $a < a-1$ ovat aina epätosia.
\end{enumerate}

Yhtälö säilyy aina totena, kun yhtälön molemmille puolille tehdään sama laskutoimitus -- toisin sanoen molempiin yhtälön puoliin käytetään samaa funktiota. Huomaamme kuitenkin kohta, että epäyhtälöiden käsittelyssä on huomioitava joitakin uusia piirteitä.

Tutkitaan seuraavaksi todeksi tunnetun epäyhtälön $2<3$ avulla, mitä epäyhtälölle on luvallista tehdä, jotta lukujen $2$ ja $3$ keskinäinen järjestys (eli epäyhtälön totuusarvo) säilyy. Testataan ensin peruslaskutoimituksia. Havainnollistetaan lukusuoralla tilannetta, jossa epäyhtälön molemmille puolille lisätään luku 4:

\begin{lukusuora}{-1}{10}{12}
	\lukusuoranuolialas{2}{6}
	\lukusuoranuolialas{3}{7}

	\lukusuorapiste{2}{$2$}
	\lukusuorapiste{3}{$3$}
	\lukusuorapystyviiva{0}{$0$}
\lukusuorauusi
	\lukusuorapiste{6}{$2\!+\!4$}
	\lukusuorapiste{7}{$3\!+\!4$}
	\lukusuorapystyviiva{0}{$0$}
\end{lukusuora}

Huomataan, että positiivisen luvun lisääminen vain siirtää lukuja lukusuoralla oikealle, mutta säilyttää niiden järjestyksen. Epäyhtälöstä $2<3$ siis seuraa $2+4<3+4$ eli $6<7$.

Myös vähennyslasku, joka voidaan käsittää negatiivisen luvun lisäämisenä, säilyttää epäyhtälön totena:

\begin{lukusuora}{-1}{10}{12}
	\lukusuoranuolialas{2}{1}
	\lukusuoranuolialas{3}{2}

	\lukusuorapiste{2}{$2$}
	\lukusuorapiste{3}{$3$}
	\lukusuorapystyviiva{0}{$0$}
\lukusuorauusi
	\lukusuorapiste{1}{$2\!-\!1$}
	\lukusuorapiste{2}{$3\!-\!1$}
	\lukusuorapystyviiva{0}{$0$}
\end{lukusuora}

Tarkastellaan seuraavaksi kerto- ja jakolaskua, joista löydämme erään tärkeän poikkeuksen. Epäyhtälön puolten kertominen tai jakaminen positiivisella luvulla skaalaa lukusuoran lukuja, mutta sekin säilyttää suuruusjärjestyksen:

\begin{lukusuora}{-1}{10}{12}
	\lukusuoranuolialas{2}{6}
	\lukusuoranuolialas{3}{9}

	\lukusuorapiste{2}{$2$}
	\lukusuorapiste{3}{$3$}
	\lukusuorapystyviiva{0}{$0$}
\lukusuorauusi
	\lukusuorapiste{6}{$2\cdot 3$}
	\lukusuorapiste{9}{$3\cdot 3$}
	\lukusuorapystyviiva{0}{$0$}
\end{lukusuora}

\begin{lukusuora}{-1}{10}{12}
	\lukusuoranuolialas{2}{1}
	\lukusuoranuolialas{3}{1.5}

	\lukusuorapiste{2}{$2$}
	\lukusuorapiste{3}{$3$}
	\lukusuorapystyviiva{0}{$0$}
\lukusuorauusi
	\lukusuorapiste{1}{$2 \cdot \frac{1}{2}$}
	\lukusuorapiste{1.5}{$3 \cdot \frac{1}{2}$}
	\lukusuorapystyviiva{0}{$0$}
\end{lukusuora}

Epäyhtälön puolittainen kertominen tai jakaminen negatiivisella luvulla aiheuttaa lukusuoralla peilauksen nollan suhteen:

\begin{lukusuora}{-6}{10}{6}
	\lukusuoranuolialas{2}{-2}
	\lukusuoranuolialas{5}{-5}

	\lukusuorapiste{2}{$2$}
	\lukusuorapiste{5}{$5$}
	\lukusuorapystyviiva{0}{$0$}
\lukusuorauusi
	\lukusuorapiste{-2}{$-1\cdot 2$}
	\lukusuorapiste{-5}{$-1\cdot 5$}
	\lukusuorapystyviiva{0}{$0$}
\end{lukusuora}

Negatiivisella luvulla kertominen (tai jakaminen) siis käänsi suuruusjärjestyksen.
Epäyhtälöstä $2<3$ siis \textit{ei} loogisesti seuraa, että $-2<-3$, vaan $-2>-3$!

Tämän perusteella ja yleistäen, kaikille epäyhtälöille pätee seuraava erittäin tärkeä tulos:

\laatikko{Jos epäyhtälö kerrotaan tai jaetaan puolittain negatiivisella luvulla, epäyhtälömerkki kääntyy.}

Lopuksi mainitaan, että nollalla kertominen ei säilytä epäyhtälöä totena, ellei käytetty epäyhtälömerkki hyväksy myös yhtäsuuruutta (esimerkiksi $\leq$). Tällöinkään epäyhtälön kertominen puolittain nollalla ei ole ratkaisun etsimisen kannalta mielekästä: koska mikä tahansa luku kerrottuna nollalla antaa nollan, olisivat molemmat epäyhtälön puolet nollalla kertomisen jälkeen yhtä suuret alkuperäisestä yhtälöstä riippumatta.

\begin{esimerkki}

Esimerkki epäyhtälön käsittelystä:
\begin{align*}
4&<6  \ \ \ \ \ &&\ppalkki +4 \\
8&<10 &&\ppalkki \cdot 5 \\
40&<50 &&\ppalkki :2 \\
20&<25 &&\ppalkki \cdot (-3) \\
-60&>-75
\end{align*}
\end{esimerkki}

Yllä oleva esimerkki havainnollistaa sitä, että epäyhtälö käyttäytyy aivan kuin
yhtälökin, kun sama laskutoimitus tehdään epäyhtälön kummallekin puolelle.
Viimeisessä vaiheessa kuitenkin huomataan, että negatiivisella luvulla
kerrottaessa epähtälön keskellä oleva merkki pitää kääntää toisin päin, että
epäyhtälö olisi edelleen voimassa. $20<25$, mutta kun molemmat kerrotaan
$-3$:lla, saadaan $-60$ ja $-75$, jolloin epäyhtälö kääntyykin toisin päin.

Yleisemmin pätee: jos epäyhtälön puoliin käytetään \emph{kasvavaa funktiota}, esitetty suuruusjärjestys säilyy. Jos epäyhtälön puoliin käytetään \emph{vähenevää funktiota}, epäyhtälömerkki täytyy kääntää. Tähän yleisempään käsittelyyn ja monimutkaisempien laskutoimitusten kuten potenssiin korottamisen ja juurenotton käyttöön epäyhtälöiden ratkaisemisessa palataan myöhemmissä luvuissa ja myöhemmillä kursseilla.

\section{Reaalilukuvälit}

Epäyhtälöiden ratkaisemissa on oleellista ymmärtää, että yhtälöiden kuten $x^3-2x=0$ ratkaisuna saadaan yleensä äärellinen määrä yksittäisiä, erillisiä lukuja:

$x^3-2x=0$
$x(x^2-2)=0$
$x=0$ tai $x^2=2$
$x=0$ tai $x=\pm \sqrt{2}$

Mainittu yhtälö siis pitää paikkansa vain ja ainoastaan kolmella $x$:n arvolla. Epäyhtälöiden tapauksessa on mahdollista, että epäyhtälön avulla esitetyn ehdollisen väitteen toteuttaa äärettömän monta lukua!

\begin{esimerkki}
Epäyhtälö $t \leq 2$ toteutuu, kun $t=1$, sillä $1<2$. Epäyhtälö toteutuu myös esimerkiksi $t$:n arvoilla $2, 0, -\pi, -71,8, -10^{100}, ...$ – kaikilla reaaliluvuilla, jotka ovat pienempiä tai yhtä suuria kuin 2.
\end{esimerkki}

Esitetään tämä ratkaisujoukko vielä lukusuoralla:

\begin{lukusuora}{-6}{10}{6}
	\lukusuorapystyviiva{0}{$0$}
	\lukusuoravalias{}{2}{}{2}
\end{lukusuora}

Edellisen esimerkkiepäyhtälön kaikkien ratkaisujen joukon voi esittää myös seuraavasti: $t\in ]-\infty, 2]$. Tällaisessa \emph{reaalilukuvälin} merkinnässä ilmoitetaan muuttujalle sen pienin ja suurin mahdollinen arvo. Alhaalta (eli lukusuoralla vasemmalta) epäyhtälön toteuttavia $t$:n arvoja ei rajoita mikään, eli $x$ voi saada mielivaltaisen pieniä arvoja. Tämän vuoksi alarajan merkintänä käytetään miinus ääretöntä.

Merkinnässä hakasulkujen suunta kertoo, voiko muuttuja oikeasti saavuttaa mainitut rajat. Ääretön ei ole luku, jota voisi saavuttaa, ja tämän vuoksi hakasulje on auki ulospäin. Sen sijaan epäyhtälö $t \leq 2$ kyllä sallii sen, että $t$ saa täsmälleen arvon kaksi, jolloin hakasulje osoittaa sisäänpäin.

\begin{esimerkki}
a) Epäyhtälön $x>\pi$ antaman rajoituksen $x$:n arvoille voi esittää myös muodossa $x \in ]\pi, \infty [$. \\
b) Väite $t \geq 3$ on yhtäpitävä väitteen $t \in [3, \infty]$ kanssa.
c) $y<0$ tarkoittaa samaa kuin $y \in ]-\infty,0[$.
\end{esimerkki}

Välejä kutsutaan joko \emph{avoimiksi}, \emph{puoliavoimiksi} tai \emph{suljetuiksi} sen perusteella, saavuttaako muuttuja annetut rajat. Jos muuttujaa on rajattu sekä ala- että yläpuolelta, päädytään käyttämään niin sanottua \emph{kaksoisepäyhtälöä}. Tällöin epäyhtälössä esiintyvät luvut kirjoitetaan usein vasemmalta oikealle kasvavaan järjestykseen.

\begin{esimerkki}

a) Epäyhtälö $2<x<10$ vaatii, että $x$ saa arvoja kahden ja kymmenen väliltä, mutta se ei koskaan saa täsmälleen näitä reuna-arvoja. Kyseessä on avoin väli kahdesta kymmeneen, $]2,10[$. Annetulle epäyhtälölle yhtäpitävä ilmaisu on $x \in ]2,10[$. \\
b) Epäyhtälö $0\leq y \leq 2$ rajaa muuttujan $y$ välille suljetulle välille $[0,2]$. Väli on suljettu, koska $y$ voi myös saada täsmälleen arvot $0$ ja $2$. \\
c) Joskus kirjallisuudessa näkee äärettömyyssymbolin käyttöä myös kaksoisepäyhtälöissä, esimerkiksi $3<x<\infty $, mutta ilmaistaan yleisemmin muodossa $x \in ]3,\infty[$. Kyseessä on avoin väli. \\
d) Epäyhtälöt $-100<k\leq 0$ ja $u\leq 90$ ovat puoliavoimia välejä, koska ne rajaavat muuttujan yhtäsuuruuden avulla vain toiselta puolelta. \\
e) Kaksoisepäyhtälö $\frac{1}{5}\geq x>-\sqrt{3}$ tarkoittaa samaa kuin kaksoisepäyhtälö $-\sqrt{3}<x\leq \frac{1}{5}$. Kaksoisepäyhtälö vaatii, että muuttujalle $x$ pätee erikseen sekä epäyhtälö $-\sqrt{3}<x$ että $x\leq \frac{1}{5}$.
\end{esimerkki}

Seuraavaan taulukkoon on koottu esimerkinomaisesti reaalilukuvälien olennainen käsitteistö ja merkinnät. Lukusuoraesityksessä tyhjä ympyrä merkitsee, että päätepiste ei kuulu kyseiselle välille ja täytetty ympyrä taasen sitä, että väli sisältää annetun rajan.

\begin{tabular}{|c|p{2.5cm}|p{2.5cm}|c|}
\hline
Välin nimitys & epäyhtälö\-merkintä & Joukko-opillinen merkintä & Esitys lukusuoralla \\
\hline
Avoin väli & $-3<x<5$ & $x \in ]-3,5[$ & \begin{lukusuora}{-6}{10}{4}\lukusuorapystyviiva{0}{$0$}\lukusuoravalias{-3}{5}{$-3$}{$5$}\end{lukusuora}\\
\hline
Puoliavoin väli & $-3<x \leq 5$ & $x \in ]-3,5]$ & \begin{lukusuora}{-6}{10}{4}\lukusuorapystyviiva{0}{$0$}\lukusuoravalias{-3}{5}{$-3$}{$5$}\end{lukusuora}\\
\hline
Puoliavoin väli & $-3\leq x < 5$ & $x \in [-3,5[$ & \begin{lukusuora}{-6}{10}{4}\lukusuorapystyviiva{0}{$0$}\lukusuoravalias{-3}{5}{$-3$}{$5$}\end{lukusuora}\\
\hline
Suljettu väli & $-3\leq x \leq 5$ & $x \in [-3,5]$ & \begin{lukusuora}{-6}{10}{4}\lukusuorapystyviiva{0}{$0$}\lukusuoravalias{-3}{5}{$-3$}{$5$}\end{lukusuora}\\
\hline
Puoliavoin väli & $-3\leq x$ & $x \in ]-3,\infty[$ & \begin{lukusuora}{-6}{10}{4}\lukusuorapystyviiva{0}{$0$}\lukusuoravalias{-3}{}{$-3$}{}\end{lukusuora}\\
\hline
Avoin väli & $-3<x$ & $x \in ]-3,\infty[$ & \begin{lukusuora}{-6}{10}{4}\lukusuorapystyviiva{0}{$0$}\lukusuoravalias{-3}{}{$-3$}{}\end{lukusuora}\\
\hline
Puoliavoin väli & $x \leq 5$ & $x \in ]-\infty,5]$ & \begin{lukusuora}{-6}{10}{4}\lukusuorapystyviiva{0}{$0$}\lukusuoravalias{}{5}{}{$5$}\end{lukusuora}\\
\hline
Avoin väli & $x < 5$ & $x \in ]-\infty,5[$ & \begin{lukusuora}{-6}{10}{4}\lukusuorapystyviiva{0}{$0$}\lukusuoravalias{}{5}{}{$5$}\end{lukusuora}\\
\hline
\end{tabular}

\section{Ensimmäisen asteen epäyhtälö}

Harjoittelemme nyt erityisesti 1. asteen epäyhtälöiden ratkaisemista -- toisen asteen ja sitä korkeampien polynomiepäyhtälöiden ratkaisemista käsitellään toisen asteen yhtälön käsittelyn jälkeen.

Samoin kuin yhtälöiden kohdalla, epäyhtälö pyritään muuttamaan niin yksinkertaiseen muotoon kuin mahdollista, jotta yksinkertaisesta tilanteesta nähdään välittömästi, mitkä luvut kuuluvat ratkaisuun ja mitkä eivät. Tuntemattomat pyritään yhdistämään, ja epäyhtälöä muokataan niin, että tuntematon saadaan yksin omalle puolelleen yhtälöä.

%Kokeilemalla tai vähän miettimällä on helppoa huomata, että epäyhtälöiden ratkaisemisessa voi käyttää suurelta osin samanlaisia menetelmiä kuin yhtälön ratkaisemisessa. Helposti huomataan, että myös epäyhtälöön voidaan lisätä puolittain jotakin tai siitä voidaan vähentää puolittain jotakin. Samoin epäyhtälö voidaan kertoa tai jakaa puolittain positiivisella luvulla. Negatiivisella luvulla jakamisessa ja kertomisessa täytyy kuitenkin huomata jotain erityistä.

\begin{esimerkki}
Ratkaistaan epäyhtälö $2x+1<0$.
\begin{align*}
2x+1<0 \ \ \ \ \ &&\ppalkki -1 \\
2x<-1 \ \ \ \ \ &&\ppalkki :2 \\
x<\-\frac{1}{2}
\end{align*}

Epäyhtälön ratkaisu voidaan esittää myös muodossa $x \in ]-\infty, -\frac{1}{2}[$.

Ratkaisua voidaan motivoida myös graafisesti tutkimalla lausekkeeseen $2x+1$ liittyvää kuvaajaa:

\begin{kuvaajapohja}{1}{-2}{2}{-2}{2}
	\kuvaaja{2*x+1}{$f(x)=2x+1$}{red}
\end{kuvaajapohja}

Alkuperäinen epäyhtälö $2x+1<0$ vaatii, että lauseke $2x+1$ saa negatiivisia arvoja. Kuvasta tämä on mahdollista nähdä niinä kaikkina $x$:n arvoina, joilla funktion $2x+1$ kuvaaja laskee vaaka-akselin alapuolelle. Tällöin funktio eli toisaalta lauseke $2x+1$ saa negatiivisia arvoja

\end{esimerkki}

\begin{esimerkki}Selvitetään, millä $w$:n arvoilla pätee $-8w-(8-w)\geq w:2+5$?

\begin{align*}
-8w-(8-w)&\geq w:2+5 \\
-8w-8+w&\geq w:2+5 \\
-7w-8&\geq \frac12 w+5  \ \ \ \ \ &&\ppalkki -\frac12 w \\
-7\frac12 w-8&\geq 5  \ \ \ \ \ &&\ppalkki +8 \\
-7\frac12 w&\geq 13  \ \ \ \ \ &&\ppalkki :(-7\frac12) \\
w&\leq 13:(-7\frac12) \\
w&\leq 13:(-\frac{15}{2}) \\
w&\leq -13\cdot \frac{2}{15} \\
w&\leq -\frac{26}{15} \\
w&\leq -1\frac{11}{15}
\end{align*}

Vastaus: $w\leq -1\frac{11}{15}$
\end{esimerkki}

Ensimmäisen asteen polynomiepäyhtälö ratkaistaan siis aivan kuten vastaava yhtälö, mutta negatiivisella luvulla jaettaessa tai kerrottaessa epäyhtälömerkki kääntyy toisin päin.

\begin{esimerkki}
Ratkaistaan kaksoisepäyhtälö $1\leq q+7<-5q+4$.

Tässä on itse asiassa kaksi epäyhtälöä $1\leq q+7$ ja $q+7<-5q+4$. Haluamme siis löytää ne $q$:n arvot, joilla molemmat epäyhtälöt pätevät.
\begin{align*}
1&\leq q+7 \ \ \ \ \ &&\ppalkki -7 \\
-6&\leq q
\end{align*}
Vastaavasti toiselle yhtälölle:
\begin{align*}
q+7&<-5q+4  \ \ \ \ \ &&\ppalkki +5q \\
6q+7&<4 &&\ppalkki -7 \\
6q&<-3 &&\ppalkki :6 \\
q&< -\frac12 \\
\end{align*}

Nämä yhdistämällä saadaan $-6\leq q$ ja $q< -\frac12$ eli $-6\leq q < -\frac12$ eli $q\in [-6, -\frac12[$.

\begin{tabular}{cc}
\begin{lukusuora}{-8}{2}{6} \lukusuoravalisa{-6}{}{$-6$}{} \lukusuorapystyviiva{0}{$0$} \end{lukusuora} & $-6\leq q$ \\
\begin{lukusuora}{-8}{2}{6} \lukusuoravaliaa{}{-0.5}{}{$-\frac12$} \lukusuorapystyviiva{0}{$0$} \end{lukusuora} & $q< -\frac12$ \\
\begin{lukusuora}{-8}{2}{6} \lukusuoravalisa{-6}{-0.5}{$-6$}{$-\frac12$} \lukusuorapystyviiva{0}{$0$} \end{lukusuora} & $-6\leq q < -\frac12$ \\
\end{tabular}
\end{esimerkki}

\Harjoitustehtavat

\begin{tehtava}
    Esitä joukko-opillisilla merkinnöillä ja lukusuoralla.
    \begin{enumerate}[a)]
        \item $-9<x \leq 7$
        \item $5\leq c$
        \item $5\leq s \leq 7\frac{1}{2}$
        \item $5\geq x>1$
        \item $a<b$
    \end{enumerate}
    \begin{vastaus}
        \begin{enumerate}[a)]
            \item $x \in ]-9,7]$
            \item $c \in [5,\infty]$
            \item $s \in [5,7\frac{1}{2}]$
            \item $x \in ]1,5]$
            \item $a \in ]-\infty,b[$ \quad tai \quad $b \in ]a, \infty[$
        \end{enumerate}
    \end{vastaus}
\end{tehtava}

\begin{tehtava}
    Ratkaise seuraavat yhtälöt tai epäyhtälöt.
    \begin{enumerate}[a)]
        \item $-2r+6=0$
        \item $-2r+6\leq 0$
        \item $5y-2<y+6$
        \item $8(x+2)\geq -5(5-x)+3$
        \item $\frac{x+3}{2}+\frac{-2x+1}{3}>\frac{x-9}{4}$
    \end{enumerate}
    \begin{vastaus}
        \begin{enumerate}[a)]
            \item $r=3$
            \item $r\geq 3$
            \item $y<2$
            \item $x=-12\frac{2}{3}$
            \item $x<9\frac{4}{5}$
        \end{enumerate}
    \end{vastaus}
\end{tehtava}

\begin{tehtava}
    Ratkaise seuraavat epäyhtälöt.
    \begin{enumerate}[a)]
        \item $3x+6<4x$
        \item $3x-6<2x+57$
        \item $5y-2<12$
        \item $3\leq y+9$
        \item $z-5\geq-888$
    \end{enumerate}
    \begin{vastaus}
        \begin{enumerate}[a)]
            \item $x>6$
            \item $x<63$
            \item $y<2,8$
            \item $y\geq -6$
            \item $z\leq 883$
        \end{enumerate}
    \end{vastaus}
\end{tehtava}

\begin{tehtava}
    Ratkaise seuraavat epäyhtälöt.
    \begin{enumerate}[a)]
        \item $3x+6<2x\leq 9-x$
        \item $3x+6<2x\leq 1+3x$
    \end{enumerate}
    \begin{vastaus}
        \begin{enumerate}[a)]
            \item $x<-6$
            \item ei ratkaisua
        \end{enumerate}
    \end{vastaus}
\end{tehtava}


\begin{tehtava}
	Millä $x$:n arvoilla luvut $2x - 5$, $-x$ ja $x + 4$ ovat erisuuria ja $2x - 5$ on luvuista
	\begin{enumerate}[a)]
		\item suurin
		\item toiseksi suurin
		\item pienin?
	\end{enumerate}
	\begin{vastaus}
		\begin{enumerate}[a)]
			\item $x > 9$
			\item $\frac{5}{3} < x < 9$
			\item $x < \frac{5}{3}$
		\end{enumerate}
	\end{vastaus}
\end{tehtava}

\begin{tehtava}
Lukion päättötodistuksessa tietyn aineen keskiarvo määräytyy tavallisesti suoraan kurssien keskiarvosta. Opiskelija haluaa filosofian keskiarvokseen 7 tai paremman. Jos filosofian kursseja on tarjolla kolme ja kahden ensimmäisen kurssin arvosanojen keskiarvoksi on tullut 6, niin minkä arvosanan opiskelijan on vähintään saatava kolmannesta kurssista? Jokainen kurssi arvioidaan kokonaislukuasteikolla 4–-10.
\begin{vastaus}
Muodostettava epäyhtälö on muotoa $\frac{2\cdot 6+x}{3}\geq 7$, josta ratkaisuna saadaan $x\geq9$.
\end{vastaus}
\end{tehtava}

\begin{tehtava}
	Tietyn auton käyttövoimavero on 450 EUR/vuosi, ja keskimääräinen kulutus on 5 litraa dieselöljyä / 100 km. Saman valmistajan vastaava bensiinikäyttöinen auto kuluttaa 8 litraa / 100 km. Diesel maksaa 1,55 EUR/litra, ja bensiini 1,65 EUR/litra. Kun vain annetut tiedot huomioidaan, niin kuinka paljon esimerkin dieselajoneuvolla tulee vähintään ajaa vuodessa, jotta se on edullisempi? Dieselauton mahdollista kalliimpaa ostohintaa ei huomioida.
    \begin{vastaus}
        \item 8257 km
    \end{vastaus}
\end{tehtava}
