\chapter{Epäyhtälöt}

\section{Epäyhtälömerkit}

Olemme yhtälöiden yhteydessä käyttäneet merkintää = osoittamaan, että kaksi lukua tai pidempää lauseketta ovat arvoltaan yhtä suuret. Lukujen välillä on olemassa kuitenkin muitakin riippuvuussuhteita eli \emph{relaatioita}. Seuraavia \emph{epäyhtälömerkkejä} käytetään kuvaamaan yhtäsuuruudesta poikkeavia kahden reaaliluvun välisiä yhteyksiä:


\begin{tabular}{ll}
lukujen välinen yhteys & lausumistapa \\
$a<b$ &  "$a$ on pienempi kuin $b$"\\
$a>b$ & "$a$ on suurempi kuin $b$"\\
$a \leq b$ & "$a$ on pienempi tai yhtä suuri kuin $b$" \\
$a \geq b$ & "$a$ on suurempi tai yhtä suuri kuin $b$" \\
\end{tabular}

Luonnollisen kielen ilmaisu "suurempi tai yhtä suuri kuin" voidaan myös ilmaista
"vähintään yhtä suuri kuin" ja vastaavasti "pienempi tai yhtä suuri kuin"
voidaan sanoa "korkeintaan yhtä suuri kuin".

Tutut yhteydet voidaan aina esittää myös kielteisessä muodossa:

\begin{tabular}{ll}
$a\neq b$ & "$a$ ei ole yhtä suuri kuin $b$" \\
$a \nless b$ &  "$a$ ei ole pienempi kuin $b$"\\
$a \ngtr b$ & "$a$ ei ole suurempi kuin $b$"\\
$a \nleq b$ & "$a$ ei ole pienempi tai yhtä suuri kuin $b$" \\
$a \ngeq b$ & "$a$ ei ole suurempi tai yhtä suuri kuin $b$" \\
\end{tabular}

Yleensä epäyhtälöistä puhuttaessa viitataan erityisesti lausekkeisiin, joissa käytetään pienempi kuin - ja pienempi tai yhtä suuri kuin -merkkiä ja niiden vastineita. Kielteisten muotojen käyttö on harvinaista.

\section{Epäyhtälöiden teoriaa}

Yhtälö voi olla:

\begin{enumerate}
\item Aina tosi (identtisesti tosi). Tällöin yhtälö esittää väitteen, joka pitää aina paikkansa jokaisessa tilanteessa. Esimerkiksi $\pi=\pi$ on aina tosi.
\item Joskus tosi eli ehdollisesti tosi (ehdollisesti tosi). Yhtälö pätee vain tietyissä tilanteissa, yleensä vain tietyillä jonkun muuttujan arvoilla. Esimerkiksi $x^2=4$ pitää paikkansa vain, kun muuttuja $x$ saa arvon $2$ tai $-2$.
\item Aina epätosi (identtisesti epätosi). Tällaiset yhtälöt esittävät mahdottomia väitteitä, jotka eivät koskaan pidä paikkaansa. Esimerkiksi yhtäsuuruusväite $-1=3$ on aina epätosi.
\end{enumerate}

[[tuo edelliseen yhtälölukuun?]]


Yhtälö säilyy totena aina, kun yhtälön molemmille puolille tehdään sama laskutoimitus – toisin sanoen molempiin yhtälön puoliin käytetään samaa funktiota. Huomaamme kuitenkin, että epäyhtälöiden käsittelyssä on huomioitava joitakin uusia piirteitä

Tutkitaan seuraavaksi todeksi tunnetun epäyhtälön $2<3$ avulla, mitä epäyhtälölle on luvallista tehdä, jotta lukujen $2$ ja $3$ keskinäinen järjestys (eli epäyhtälön totuusarvo) säilyy. Testataan ensin peruslaskutoimituksia. Havainnollistetaan lukusuoralla tilannetta, jossa epäyhtälön molemmille puolille lisätään luku 4:

\begin{lukusuora}{-1}{10}{12}
	\lukusuoranuolialas{2}{6}
	\lukusuoranuolialas{3}{7}

	\lukusuorapiste{2}{$2$}
	\lukusuorapiste{3}{$3$}
	\lukusuorapystyviiva{0}{$0$}
\lukusuorauusi
	\lukusuorapiste{6}{$2\!+\!4$}
	\lukusuorapiste{7}{$3\!+\!4$}
	\lukusuorapystyviiva{0}{$0$}
\end{lukusuora}

Huomataan, että positiivisen luvun lisääminen vain siirtää lukuja lukusuoralla oikealle, mutta säilyttää niiden järjestyksen. Epäyhtälöstä $2<3$ siis seuraa $2+4<3+4$ eli $6<7$.

Myös vähennyslasku, joka voidaan käsittää negatiivisen luvun lisäämisenä, säilyttää pitää epäyhtälön totena:

\begin{lukusuora}{-1}{10}{12}
	\lukusuoranuolialas{2}{1}
	\lukusuoranuolialas{3}{2}

	\lukusuorapiste{2}{$2$}
	\lukusuorapiste{3}{$3$}
	\lukusuorapystyviiva{0}{$0$}
\lukusuorauusi
	\lukusuorapiste{1}{$2\!-\!1$}
	\lukusuorapiste{2}{$3\!-\!1$}
	\lukusuorapystyviiva{0}{$0$}
\end{lukusuora}

Tarkastellaan seuraavaksi kerto- ja jakolaskua, joista löydämme erään tärkeän poikkeuksen. Epäyhtälön puolten kertominen tai jakaminen positiivisella luvulla skaalaa lukusuoran lukuja, mutta sekin säilyttää suuruusjärjestyksen:

\begin{lukusuora}{-1}{10}{12}
	\lukusuoranuolialas{2}{6}
	\lukusuoranuolialas{3}{9}

	\lukusuorapiste{2}{$2$}
	\lukusuorapiste{3}{$3$}
	\lukusuorapystyviiva{0}{$0$}
\lukusuorauusi
	\lukusuorapiste{6}{$2\cdot 3$}
	\lukusuorapiste{9}{$3\cdot 3$}
	\lukusuorapystyviiva{0}{$0$}
\end{lukusuora}

\begin{lukusuora}{-1}{10}{12}
	\lukusuoranuolialas{2}{1}
	\lukusuoranuolialas{3}{1.5}

	\lukusuorapiste{2}{$2$}
	\lukusuorapiste{3}{$3$}
	\lukusuorapystyviiva{0}{$0$}
\lukusuorauusi
	\lukusuorapiste{1}{$2 \cdot \frac{1}{2}$}
	\lukusuorapiste{1.5}{$3 \cdot \frac{1}{2}$}
	\lukusuorapystyviiva{0}{$0$}
\end{lukusuora}

Epäyhtälön puolittainen kertominen tai jakaminen negatiivisella luvulla aiheuttaa lukusuoralla peilauksen nollan suhteen:

\begin{lukusuora}{-6}{10}{6}
	\lukusuoranuolialas{2}{-2}
	\lukusuoranuolialas{3}{-3}

	\lukusuorapiste{2}{$2$}
	\lukusuorapiste{3}{$3$}
	\lukusuorapystyviiva{0}{$0$}
\lukusuorauusi
	\lukusuorapiste{-2}{$2\cdot -1$}
	\lukusuorapiste{-3}{$3 \cdot -1$}
	\lukusuorapystyviiva{0}{$0$}
\end{lukusuora}

Negatiivisella luvulla kertominen (tai jakaminen) siis käänsi suuruusjärjestyksen.
Epäyhtälöstä $2<3$ siis \textit{ei} seuraa, että $-2<-3$, vaan $-2>-3$!

Kaikille epäyhtälöille pätee seuraava erittäin tärkeä tulos:
LAATIKKO: Jos epäyhtälö kerrotaan tai jaetaan negatiivisella luvulla, epäyhtlömerkki kääntyy.

Lisäksi mainitaan, että nollalla kertominen ei säilytä epäyhtälöä totena, ellei käytetty epäyhtälömerkki hyväksy myös yhtäsuuruutta (esimerkiksi esimerkiksi $\leq$). Koska mikä tahansa luku kerrottuna nollalla antaa nollan, olisivat molemmat epäyhtälön puolet nolla kertomisen jälkeen yhtä suuret.

\textbf{Esimerkki epäyhtälön käsittelystä}
\begin{align*}
4&<6  \ \ \ \ \ &&\ppalkki +4 \\
8&<10 &&\ppalkki \cdot 5 \\
40&<50 &&\ppalkki :2 \\
20&<25 &&\ppalkki \cdot (-3) \\
-60&>-75
\end{align*}

Yllä oleva esimerkki havainnollistaa sitä, että epäyhtälö säilyy aivan kuin
yhtälökin, kun sama laskutoimitus tehdään epäyhtälön kummallekin puolelle.
Viimeisessä vaiheessa huomataan kuitenkin, että negatiivisella luvulla
kerrottaessa epähtälön keskellä oleva merkki pitää kääntää toisin päin, että
epäyhtälö olisi edelleen voimassa. $20<25$, mutta kun molemmat kerrotaan
$-3$:lla, saadaan $-60$ ja $-75$, jolloin epäyhtälö kääntyykin toisin päin.

Yleisemmin pätee: jos epäyhtälön puoliin käytetään \emph{kasvavaa funktiota}, esitetty suuruusjärjestys säilyy. Jos epäyhtälön puoliin käytetään \emph{vähenevää funktiota}, epäyhtälömerkki täytyy kääntää. Tähän yleisempään käsittelyyn ja monimutkaisempien laskutoimitusten kuten potenssiin korottamisen ja juurenotton käyttöön epäyhtälöiden ratkaisemisessa palataan myöhemmissä luvuissa ja myöhemmillä kursseilla.

\section{Reaalilukuvälit}

Epäyhtälöiden ratkaisemissa on oleellista ymmärtää, että yhtälöiden kuten $x^3-2x=0$ ratkaisuna saadaan yleensä äärellinen määrä yksittäisiä, erillisiä lukuja:

$x^3-2x=0$
$x(x^2-2)=0$
$x=0$ tai $x^2=2$
$x=0$ tai $x=\pm \sqrt{2}$

Mainittu yhtälö siis pitää paikkansa vain ja ainoastaan kolmella $x$:n arvolla. Epäyhtälöiden tapauksessa on mahdollista, että epäyhtälön avulla esitetyn ehdollisen väitteen toteuttaa äärettömän monta lukua!

\textbf{Esimerkki}
Epäyhtälö $t \leq 2$ toteutuu, kun $t=1$, sillä $1<2$. Epäyhtälö toteutuu myös esimerkiksi $t$:n arvoilla $2, 0, -\pi, -71,8, -10^100, ...$ – kaikilla reaaliluvuilla, jotka ovat pienempiä tai yhtä suuria kuin 2.

Esitetään tämä ratkaisujoukko vielä lukusuoralla:

\todo{ratkaisujoukon esittäminen lukusuoralla (musta pallukka päätyyn)}
\begin{lukusuora}{-6}{10}{6}
	\lukusuorapiste{2}{$2$}
	\lukusuorapystyviiva{0}{$0$}
\end{lukusuora}

Edellisen esimerkkiepäyhtälön kaikkien ratkaisujen joukon voi esittää myös seuraavasti: $t \in ]-\inf, 2]$. Tällaisessa \emph{reaalilukuvälin} merkinnässä ilmoitetaan muuttujalle sen pienin ja suurin mahdollinen arvo. Alhaalta (eli lukusuoralla vasemmalta) epäyhtälön toteuttavia $t$:n arvoja ei rajoita mikään, eli $x$ voi saada mielivaltaisen pieniä arvoja. Tämän vuoksi alarajan merkintänä käytetään miinus ääretöntä.

Merkinnässä hakasulkujen suunta kertoo, voiko muuttuja oikeasti saavuttaa mainitut rajat. Ääretön ei ole luku, jota voisi saavuttaa, ja tämän vuoksi hakasulje on auki ulospäin. Sen sijaan epäyhtälö $t \leq 2$ kyllä sallii sen, että $t$ saa täsmälleen arvon kaksi, jolloin hakasulje osoittaa sisäänpäin.

\textbf{Esimerkki}
a) Epäyhtälön $x>\pi$ antaman rajoituksen $x$:n arvoille voi esittää myös muodossa $x \in ]\pi, \inf[$. \\
b) Väite $t \geq 3$ on yhtäpitävä väitteen $t \in [3, \inf]$ kanssa.
c) $y<0$ tarkoittaa samaa kuin $y \in ]-\inf,0[$.

Välejä kutsutaan joko \emph{avoimiksi}, \emph{puoliavoimiksi} tai \emph{suljetuiksi} sen perusteella, saavuttaako muuttuja annetut rajat. Jos muuttujaa on rajattu sekä ala- että yläpuolelta, päädytään käyttämään niin sanottua \emph{kaksoisepäyhtälöä}.

\textbf{Esimerkki}
a) Epäyhtälö $2<x<10$ vaatii, että $x$ saa arvoja kahden ja kymmenen väliltä, mutta se ei koskaan saa täsmälleen näitä reuna-arvoja. Kyseessä on avoin väli kahdesta kymmeneen, $]2,10[$. Annetulle epäyhtälölle yhtäpitävä ilmaisu on $x \in ]2,10[$.
b) Epäyhtälö $0\leq y \leq 2$ rajaa muuttujan $y$ välille suljetulle välille $[0,2]$. Väli on suljettu, koska $y$ voi myös saada täsmälleen arvot $0$ ja $2$.
c) Joskus kirjallisuudessa näkee äärettömyyssymbolin käyttöä myös kaksoisepäyhtälöissä, esimerkiksi $3<x<\inf$, mutta ilmaistaan yleisemmin muodossa $x \in ]3,\inf[$. Kyseessä on avoin väli.
d) Epäyhtälöt $-100<k\leq 0$ ja $u\leq 90$ ovat puoliavoimia välejä, koska ne rajaavat muuttujan yhtäsuuruuden avulla vain toiselta puolelta.

Seuraavaan taulukkoon on koottu reaalilukuvälien olennainen käsitteistö.

\todo{reaalilukuvälitaulukkoon myös lukusuoraesitykset!}
\todo{reaalilukuvälitaulukkoon myös äärettömyysesimerkit}
\begin{tabular}{|c|c|c|}
\hline 
Välin nimitys & epäyhtälömerkintä & joukko-opillinen merkintä \\ 
\hline 
Avoin väli & $-3<x<5$ & $x \in ]-3,5[$ \\ 
\hline 
Puoliavoin väli & $-3<x \leq 5$ & $x \in ]-3,5]$ \\ 
\hline 
Puoliavoin väli & $-3\leq x < 5$ & $x \in [-3,5[$ \\ 
\hline 
Suljettu väli & $-3\leq x \leq 5$ & $x \in [-3,5]$ \\ 
\hline 
\end{tabular} 

\todo{välimerkintöjen muunnostehtäviä}

\section{Ensimmäisen asteen epäyhtälö}

Erityisesti harjoittelemme 1. asteen epäyhtälöiden ratkaisemista ja myöhemmin yleisesti polynomiepäyhtälöiden ratkaisemista.

\textbf{Esimerkki 1}
\begin{align*}
ax^2+bx+c<0 \\
9 \cdot x+15\geq 0 \\
1>0>-1
\end{align*}

Samoin kuin yhtälöiden kohdalla, myös epäyhtälöistä halutaan usein selvittää
tuntemattoman muuttujan arvot. Epäyhtälöllä ratkaisuja on tietysti paljon ja ratkaistaessa
pyritään löytämään ne kaikki. Käytännössä ratkaiseminen tarkoittaa sitä, että epäyhtälö
muutetaan niin yksinkertaiseen muotoon kuin mahdollista siten, että yksinkertaisesta epäyhtälöstä
on helppoa nähdä, mitkä luvut kuuluvat ratkaisuun ja mitkä eivät.

Kokeilemalla tai vähän miettimällä on helppoa huomata, että epäyhtälöiden ratkaisemisessa
voi käyttää suurelta osin samanlaisia menetelmiä kuin yhtälön ratkaisemisessa. Helposti huomataan,
että myös epäyhtälöön voidaan lisätä puolittain jotakin tai siitä voidaan vähentää puolittain jotakin.
Samoin epäyhtälö voidaan kertoa tai jakaa puolittain positiivisella luvulla. Negatiivisella luvulla
jakamisessa ja kertomisessa täytyy kuitenkin huomata jotain erityistä.


\textbf{Esimerkki}
Millä $w$:n arvoilla pätee $-8w-(8-w)\geq w:2+5$?

\textbf{Ratkaisu}
\begin{align*}
-8w-(8-w)&\geq w:2+5 \\
-8w-8+w&\geq w:2+5 \\
-7w-8&\geq \frac12 w+5  \ \ \ \ \ &&\ppalkki -\frac12 w \\
-7\frac12 w-8&\geq 5  \ \ \ \ \ &&\ppalkki +8 \\
-7\frac12 w&\geq 13  \ \ \ \ \ &&\ppalkki :(-7\frac12) \\
w&\leq 13:(-7\frac12) \\
w&\leq 13:(-\frac{15}{2}) \\
w&\leq -13\cdot \frac{2}{15} \\
w&\leq -\frac{26}{15} \\
w&\leq -1\frac{11}{15}
\end{align*}

Vastaus: $w\leq -1\frac{11}{15}$

Ensimmäisen asteen polynomiepäyhtälö ratkaistaan siis aivan kuten vastaava yhtälö, mutta negatiivisella luvulla jaettaessa tai kerrottaessa epäyhtälömerkki kääntyy toisin päin.

\textbf{Esimerkki 3}
Ratkaise epäyhtälö $1\leq q+7<-5q+4$.

\textbf{Ratkaisu}
Tässä on itse asiassa kaksi epäyhtälöä $1\leq q+7$ ja $q+7<-5q+4$. Haluamme siis löytää ne $q$:n arvot, joilla molemmat epäyhtälöt pätevät.
\begin{align*}
1&\leq q+7 \ \ \ \ \ &&\ppalkki -7 \\
-6&\leq q
\end{align*}
Vastaavasti toiselle yhtälölle:
\begin{align*}
q+7&<-5q+4  \ \ \ \ \ &&\ppalkki +5q \\
6q+7&<4 &&\ppalkki -7 \\
6q&<-3 &&\ppalkki :6 \\
q&< -\frac12 \\
\end{align*}

Nämä yhdistämällä saadaan $-6\leq q$ ja $q< -\frac12$ eli $-6\leq q < -\frac12$ eli $q\in [-6, -\frac12[$.


\textbf{Ratkaisu}
Luvun itseisarvo on pienempi kuin $5$ jos ja vain jos se on $-5$:n ja $5$:n välillä.
Itseisarvoepäyhtälö voidaan siis jakaa kahdeksi epäyhtälöksi $-5<-5q+4<5$ eli $-5<-5q+4$ ja $-5q+4<5$.

\begin{align*}
-5&<-5q+4 \ \ \ \ \ &&\ppalkki -4 \\
-9&<-5q &&\ppalkki :(-5) \\
\frac95&>q
\end{align*}
Vastaavasti toiselle yhtälölle:
\begin{align*}
-5q+4&<5  \ \ \ \ \ &&\ppalkki -4 \\
-5q&<1 &&\ppalkki :(-5) \\
q&>-\frac15 \\
\end{align*}

(Huomaa missä kohtaa epäyhtälön merkki kääntyi, kun kerrottiin puolittain negatiivisella luvulla.)

Vastaus: $-\frac15<q<\frac95$ eli $q\in ]-\frac15,\frac95[$

\section{Harjoitustehtäviä}

\begin{tehtava}
    Ratkaise seuraavat yhtälöt tai epäyhtälöt.
    \begin{enumerate}[a)]
        \item $-2r+6=0$
        \item $-2r+6\leq 0$
        \item $5y-2<y+6$
        \item $8(x+2)\geq -5(5-x)+3$
        \item $\frac{x+3}{2}+\frac{-2x+1}{3}>\frac{x-9}{4}$
    \end{enumerate}
    \begin{vastaus}
        \begin{enumerate}[a)]
            \item $r=3$
            \item $r\geq 3$
            \item $y<2$
            \item $x=-12\frac{2}{3}$
            \item $x<9\frac{4}{5}$
        \end{enumerate}
    \end{vastaus}
\end{tehtava}

\begin{tehtava}
    Ratkaise seuraavat epäyhtälöt.
    \begin{enumerate}[a)]
        \item $3x+6<4x$
        \item $3x-6<2x+57$
        \item $5y-2<12$
        \item $3\leq y+9$
        \item $z-5\geq-888$
    \end{enumerate}
    \begin{vastaus}
        \begin{enumerate}[a)]
            \item $x>6$
            \item $x<63$
            \item $y<2,8$
            \item $y\geq -6$
            \item $z\leq 883$
        \end{enumerate}
    \end{vastaus}
\end{tehtava}

\begin{tehtava}
    Ratkaise seuraavat epäyhtälöt.
    \begin{enumerate}[a)]
        \item $3x+6<2x\leq 9-x$
        \item $3x+6<2x\leq 1+3x$
    \end{enumerate}
    \begin{vastaus}
        \begin{enumerate}[a)]
            \item $x<-6$
            \item ei ratkaisua
        \end{enumerate}
    \end{vastaus}
\end{tehtava}

\begin{tehtava}
    \item Tietyn auton käyttövoimavero on 450 EUR/vuosi, ja keskimääräinen kulutus on 5 litraa dieselöljyä / 100 km. Saman valmistajan vastaava bensiinikäyttöinen auto kuluttaa 8 litraa / 100 km. Diesel maksaa 1.55 EUR/litra, ja bensiini 1.65 EUR/litra. Kun vain annetut tiedot huomioidaan, niin kuinka paljon esimerkin dieselajoneuvolla tulee vähintään ajaa vuodessa, jotta se on edullisempi? Dieselauton mahdollista kalliimpaa ostohintaa ei huomioida.
    \begin{vastaus}
        \item 8257 km
    \end{vastaus}
\end{tehtava}
