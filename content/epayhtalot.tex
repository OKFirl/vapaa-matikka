\chapter{Epäyhtälöt}

\textbf{Esimerkki 1} \\ 

\begin{align*}
ax^2+bx+c<0 \\
9 \cdot x+15\geq 0 \\
1>0>-1 
\end{align*}

Lauseketta, jossa yhtälön $=$-merkki on korvattu
"$<$"-, "$\leq$"-, "$>$"-, "$\geq$"- tai "$\neq$"-merkillä sanotaan epäyhtälöiksi.

Samoin kuin yhtälöiden kohdalla, myös epäyhtälöistä halutaan usein selvittää
tuntemattoman muuttujan arvot. Epäyhtälöllä ratkaisuja on tietysti paljon ja ratkaistaessa
pyritään löytämään ne kaikki. Käytännössä ratkaiseminen tarkoittaa sitä, että epäyhtälö
muutetaan niin yksinkertaiseen muotoon kuin mahdollista siten, että yksinkertaisesta epäyhtälöstä
on helppoa nähdä, mitkä luvut kuuluvat ratkaisuun ja mitkä eivät.

Kokeilemalla tai vähän miettimällä on helppoa huomata, että epäyhtälöiden ratkaisemisessa
voi käyttää suurelta osin samanlaisia menetelmiä kuin yhtälön ratkaisemisessa. Helposti huomataan,
että myös epäyhtälöön voidaan lisätä puolittain jotakin tai siitä voidaan vähentää puolittain jotakin.
Samoin epäyhtälö voidaan kertoa tai jakaa positiivisela luvulla.

\section{Ensimmäisen asteen epäyhtälö}

\section{Harjoitustehtäviä}
