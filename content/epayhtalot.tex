\chapter{Epäyhtälöt}
(Joonas kirjoittaa aamuyöllä loppuun)

\section{Epäyhtälömerkit}

Olemme yhtälöiden yhteydessä käyttäneet merkintää = osoittamaan, että kaksi lukua tai pidempää lauseketta ovat arvoltaan yhtä suuret. Lukujen välillä on olemassa kuitenkin muitakin riippuvuussuhteita eli \emph{relaatioita}. Seuraavia \emph{epäyhtälömerkkejä} käytetään kuvaamaan yhtäsuuruudesta poikkeavia kahden reaaliluvun välisiä yhteyksiä:


\begin{tabular}{ll}
lukujen välinen yhteys & lausumistapa \\
$a<b$ &  "$a$ on pienempi kuin $b$"\\
$a>b$ & "$a$ on suurempi kuin $b$"\\
$a \leq b$ & "$a$ on pienempi tai yhtä suuri kuin $b$" \\
$a \geq b$ & "$a$ on suurempi tai yhtä suuri kuin $b$" \\
\end{tabular}

Tutut yhteydet voidaan aina esittää myös kielteisessä muodossa:

\begin{tabular}{ll}
$a\neq b$ & "$a$ ei ole yhtä suuri kuin $b$" \\
$a \nless b$ &  "$a$ ei ole on pienempi kuin $b$"\\
$a \ngtr b$ & "$a$ ei ole suurempi kuin $b$"\\
$a \nleq b$ & "$a$ ei ole pienempi tai yhtä suuri kuin $b$" \\
$a \ngeq b$ & "$a$ ei ole suurempi tai yhtä suuri kuin $b$" \\
\end{tabular}

Yleensä epäyhtälöistä puhuttaessa viitataan erityisesti lausekkeisiin, joissa käytetään pienempi kuin - ja pienempi tai yhtä suuri kuin -merkkiä ja niiden vastineita.

\section{Epäyhtälöiden teoriaa}

Yhtälöt voivat olla aina tosia (esimerkiksi $\pi=\pi$), joskus tosia (yhtälö $x^2=4$ pitää paikkansa vain, kun muuttuja $x$ saa arvon $2$ tai $-2$) tai aina epätosia (esimerkiksi $-1=3$). Yhtälön totuusarvo säilyy aina, kun yhtälön molemmille puolille tehdään sama laskutoimitus– toisin sanoen käytetään samaa funktiota.

Myös jokainen epäyhtälö on väite, jolla on totuusarvo. Tutkitaan epäyhtälöä $2<3$, joka tiedetään todeksi. Mitä epäyhtälön puolille voidaan tehdä, jotta lukujen $2$ ja $3$ keskinäinen järjestys säilyy? Testaatan ensin peruslaskutoimituksia:

Lisätään molemmille puolille jokin positiivinen luku, 4

Havainnollistetaan lukusuoralla:
\missingfigure{2 ja 3 lukusuoralla ja 2+4 sekä 3+4 lukusuoralla (nuolet kuvaamaan hyppyjä?}

vähennyslasku

\missingfigure{lukusuoralla lukujen 2 ja 3 siirtyminen vasemmalle}

kertominen positiivisella luvulla

\missingfigure{ja näitä tulee lisää...}

\textbf{Esimerkki 2}
\begin{align*}
4&<6  \ \ \ \ \ &&|| +4 \\
8&<10 &&|| \cdot 5 \\
40&<50 &&|| :2 \\
20&<25 &&|| \cdot (-3) \\
-60&>-75
\end{align*}

Yllä oleva esimerkki havainnollistaa sitä, että epäyhtälö säilyy aivan kuin yhtälökin, kun sama toimitus tehdään
epäyhtälön kummallekin puolelle. Viimeisessä vaiheessa huomataan kuitenkin, että negatiivisella luvulla kerrottaessa
epähtälön keskellä oleva merkki pitää kääntää toisin päin, että epäyhtälö olisi edelleen voimassa. $20<25$, mutta kun molemmat
kerrotaan $-3$:lla, saadaan $-60$ ja $-75$, jolloin epäyhälö kääntyykin toisin päin.

Esimerkki
Toteuttaako luku epäyhtälön...

\section{Reaalilukuvälit}

Kun ratkaistaan epäyhtälöstä tuntematon, se pyritään.... ....

vastauksena ei yksittäisiä lukuja kuten yhtälöiden tapauksissa, vaan epäyhtälön toteuttaa moni luku! Jopa ääretön! ...


\section{Ensimmäisen asteen epäyhtälö}

Erityisesti harjoittelemme 1. asteen epäyhtälöiden ratkaisemista ja myöhemmin yleisesti polynomiepäyhtälöiden ratkaisemista.
\textbf{Esimerkki 1}
\begin{align*}
ax^2+bx+c<0 \\
9 \cdot x+15\geq 0 \\
1>0>-1
\end{align*}

Samoin kuin yhtälöiden kohdalla, myös epäyhtälöistä halutaan usein selvittää
tuntemattoman muuttujan arvot. Epäyhtälöllä ratkaisuja on tietysti paljon ja ratkaistaessa
pyritään löytämään ne kaikki. Käytännössä ratkaiseminen tarkoittaa sitä, että epäyhtälö
muutetaan niin yksinkertaiseen muotoon kuin mahdollista siten, että yksinkertaisesta epäyhtälöstä
on helppoa nähdä, mitkä luvut kuuluvat ratkaisuun ja mitkä eivät.

Kokeilemalla tai vähän miettimällä on helppoa huomata, että epäyhtälöiden ratkaisemisessa
voi käyttää suurelta osin samanlaisia menetelmiä kuin yhtälön ratkaisemisessa. Helposti huomataan,
että myös epäyhtälöön voidaan lisätä puolittain jotakin tai siitä voidaan vähentää puolittain jotakin.
Samoin epäyhtälö voidaan kertoa tai jakaa puolittain positiivisella luvulla. Negatiivisella luvulla
jakamisessa ja kertomisessa täytyy kuitenkin huomata jotain erityistä.


Yleisesti pätee seuraava
	Kasvava/vähenevä funktio

    \begin{kuvaajapohja}{2}{-1}{1}{-1}{1}
      \kuvaaja{2*x}{$f(x) = 2x$}{red}
    \end{kuvaajapohja}

    \begin{kuvaajapohja}{0.5}{-4}{4}{-4}{4}
      \kuvaaja{x+1}{$f(x) = x+1$}{red}
    \end{kuvaajapohja}


Potenssiin korottamisen, juurenottoon ja muiden funktioiden käyttöön epäyhtälöiden ratkaisemisessa palataan myöhemmissä luvuissa ja myöhemmillä kursseilla.

\textbf{Esimerkki 2}
Millä $w$:n arvoilla pätee $-8w-(8-w)\geq w:2+5$?

\textbf{Ratkaisu}
\begin{align*}
-8w-(8-w)&\geq w:2+5 \\
-8w-8+w&\geq w:2+5 \\
-7w-8&\geq \frac12 w+5  \ \ \ \ \ &&|| -\frac12 w \\
-7\frac12 w-8&\geq 5  \ \ \ \ \ &&|| +8 \\
-7\frac12 w&\geq 13  \ \ \ \ \ &&|| :(-7\frac12) \\
w&\leq 13:(-7\frac12) \\
w&\leq 13:(-\frac{15}{2}) \\
w&\leq -13\cdot \frac{2}{15} \\
w&\leq -\frac{26}{15} \\
w&\leq -1\frac{11}{15}
\end{align*}

Vastaus: $w\leq -1\frac{11}{15}$

Epäyhtälö ratkaistaan siis aivan kuten vastaava yhtälö, mutta negatiivisella luvulla
jaettaessa tai kerrottaessa merkki kääntyy toisin päin.

\textbf{Esimerkki 3}
Ratkaise yhtälö $1\leq q+7<-5q+4$.

\textbf{Ratkaisu}
Tässä on itse asiassa kaksi epäyhtälöä $1\leq q+7$ ja $q+7<-5q+4$. Haluamme siis löytää ne $q$:n arvot, joilla molemmat epäyhtälöt pätevät.
\begin{align*}
1&\leq q+7 \ \ \ \ \ &&|| -7 \\
-6&\leq q
\end{align*}
Vastaavasti toiselle yhtälölle:
\begin{align*}
q+7&<-5q+4  \ \ \ \ \ &&|| +5q \\
6q+7&<4 &&|| -7 \\
6q&<-3 &&|| :6 \\
q&< -\frac12 \\
\end{align*}

Nämä yhdistämällä saadaan $-6\leq q$ ja $q< -\frac12$ eli $-6\leq q < -\frac12$ eli $q\in [-6, -\frac12[$.

\textbf{Esimerkki 4}
Ratkaise yhtälö $|-5q+4|<5$.

\textbf{Ratkaisu}
Luvun itseisarvo on pienempi kuin $5$ jos ja vain jos se on $-5$:n ja $5$:n välillä.
Itseisarvoepäyhtälö voidaan siis jakaa kahdeksi epäyhtälöksi $-5<-5q+4<5$ eli $-5<-5q+4$ ja $-5q+4<5$.

\begin{align*}
-5&<-5q+4 \ \ \ \ \ &&|| -4 \\
-9&<-5q &&|| :(-5) \\
\frac95&>q
\end{align*}
Vastaavasti toiselle yhtälölle:
\begin{align*}
-5q+4&<5  \ \ \ \ \ &&|| -4 \\
-5q&<1 &&|| :(-5) \\
q&>-\frac15 \\
\end{align*}

(Huomaa missä kohtaa epäyhtälön merkki kääntyi, kun kerrottiin puolittain negatiivisella luvulla.)

Vastaus: $-\frac15<q<\frac95$ eli $q\in ]-\frac15,\frac95[$

\section{Harjoitustehtäviä}
\begin{tehtava}
    Ratkaise seuraavat yhtälöt tai epäyhtälöt.
    \begin{enumerate}[a)]
        \item $-2r+6=0$
        \item $-2r+6\leq 0$
        \item $5y-2<y+6$
        \item $8(x+2)\geq 5(5-x)\cdot (-1)+3$
        \item $\frac{x+3}{2}+\frac{-2x+1}{3}>\frac{x-9}{3+1}$
    \end{enumerate}
    \begin{vastaus}
        \begin{enumerate}[a)]
            \item $r=3$
            \item $r\geq 3$
            \item $y<2$
            \item $x=-12\frac{2}{3}$
            \item $x<9\frac{4}{5}$
        \end{enumerate}
    \end{vastaus}
\end{tehtava}

\begin{tehtava}
    Ratkaise seuraavat epäyhtälöt.
    \begin{enumerate}[a)]
        \item $3x+6<2x\leq 9-x$
        \item $3x+6<2x\leq 1+3x$
        \item $|5y-2|<12$
        \item $3\leq |x+9|$
        \item $-\smiley{}-5\geq-888$
    \end{enumerate}
    \begin{vastaus}
        \begin{enumerate}[a)]
            \item $x<-6$
            \item ei ratkaisua
            \item $-2<y<2\frac{4}{5}$
            \item $x\geq -6$ tai $x\leq -12$
            \item $\smiley{}\leq 883$
        \end{enumerate}
    \end{vastaus}
\end{tehtava}


\begin{tehtava}
    Ratkaise seuraavat tehtävät.
    \begin{enumerate}[a)]
        \item Tietyn auton käyttövoimavero on 450 EUR/vuosi, ja keskimääräinen kulutus on 5 litraa dieselöljyä / 100 km. Saman valmistajan vastaava bensiinikäyttöinen auto kuluttaa 8 litraa / 100 km. Diesel maksaa 1.55 EUR/litra, ja bensiini 1.65 EUR/litra. Kun vain annetut tiedot huomioidaan, niin kuinka paljon esimerkin dieselajoneuvolla tulee vähintään ajaa vuodessa, jotta se on edullisempi?
    \end{enumerate}

    \begin{vastaus}
        \begin{enumerate}[a)]
            \item 8256 km
        \end{enumerate}
    \end{vastaus}
\end{tehtava}