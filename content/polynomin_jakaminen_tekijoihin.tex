\chapter{Polynomin jakaminen tekijöihin}

Matematiikan 1. kurssissa on puhuttu lukujen jakamisesta tekijöihin.
Esimerkiksi luvun $12$ {\bf tekijät} ovat $1$, $2$, $3$, $4$, $6$ ja $12$. Nämä ovat sellaisia
lukuja, joista saadaan $12$ kertomalla ne jollain kokonaisluvulla. Sanotaan myös, että luvun $12$
{\bf alkutekijät} ovat $2$, $2$ ja $3$, koska luku $12$ voidaan
ilmaista niiden tulona ($2\cdot 2\cdot 3 = 2^2\cdot 3 = 12$), mutta näitä tekijöitä ei
voi enää jakaa pienempiin osatekijöihin. Kokonaisluvun tekijät ovat aina kokonaislukuja.

Vastavasti voidaan puhua polynomin jakamisesta {\bf tekijöihin} tai {\bf alkutekijöihin}. Polynomin tekijät
ovat aina polynomeja. Olemme jo oppineet kertomaan polynomeja keskenään,
joten voimme helposti laskea, että $(x-3)(2x^2-8x+8)=2x^3-14x^2+32x-24$.
Nyt voidaan sanoa, että $x-3$ ja $2x^2-8x+8$ ovat polynomin $2x^3-14x^2+32x-24$ tekijöitä.
Mutta ovatko ne alkutekijöitä, vai voidaanko ne edelleen jakaa pienempiin tekijöihin?
Itse asiassa laskemalla voidaan todeta että $2x^2-8x+8$ saadaan tulokseksi kertolaskusta $2(x-2)(x-2)=2(x-2)^2$.
Voimme siis ilmoitaa polynomin tekijöidensä avulla: $2x^3-14x^2+32x-24=2(x-3)(x-2)^2$.

Mutta miksi haluaisimme jakaa polynomeja tekijöihin?
Yksi tärkeimmistä hyödyistä liittyy tulon nollasääntöön. Tiedämme nimittäin, että tulo on nolla jos ja vain jos
jokin tulon tekijöistä on nolla. Jos osaamme jakaa polynomin tekijöihin, pystymme näin helposti päättelemään polynomin nollakohdat.

Jos vaikkapa haluamme ratkaista yhtälön $2x^3-14x^2+32x-24=0$ ja satumme tietämään, että $2x^3-14x^2+32x-24=2(x-3)(x-2)^2$,
voimme helposti päätellä, että ainoat mahdolliset ratkaisut saadaan yhtälöistä $x-3=0$ ja $x-2=0$

% Jokainen reaalikertoiminen polynomi voidaan jakaa tekijöihin, jotka ovat korkeintaan toista astetta.

Polynomin jakaminen tekijöihin on siis varsin hyödyllistä. Tekijöihin jakaminen ei aina ole helppoa,
mutta tässä luvussa käsitellään tapoja, joilla tekijöihin jakaminen monissa tilanteissa onnistuu.

\subsection*{Polynomien jakolause}

Polynomin $P(x)=(x-2)\cdot(x-3)$ nollakohdat ovat tulon nollasäännön nojalla $x=2$ ja $x=3$. Nollakohdat siis näkee suoraan tekijöihin jaetusta polynomista. Tämä tekijöiden ja nollakohtien välinen yhteys toimii myös toisin päin: nollakohtien avulla voi selvittää tekijät. Jos luku $b$ on polynomin nollakohta, $x-b$ on polynomin tekijä.

%Esimerkiksi polynomin $P(x)=x²-3x+2$ nollakohdat ovat
%\begin{align*}
%x^2-3x+2&=0 \\
%x&=\frac{-(-3) \pm \sqrt[]{(-3)^2-4 \cdot 1 \cdot 2}}{2 \cdot 1} \\
%x&=\frac{3 \pm \sqrt[]{9-8}}{2} \\
%x&=\frac{3 \pm 1}{2} \\
%x&=1 \textrm{ tai } x = 2.
%\end{align*}
%Tämän perusteella voidaan päätellä, että $x-1$ ja $x-2$ ovat polynomin $P(x)$ tekijöitä. Toden %totta: sulut auki kertomalla voidaan tarkistaa, että $(x-1)(x-2)=x^2-3x+2$.

\laatikko{\textbf{Polynomien jakolause} \\
Jos $x=b$ on polynomin nollakohta, $x-b$ on polynomin tekijä.}

\textbf{Todistus.} Jakolauseen todistus perustuu polynomien jakoyhtälöön, josta tarkemmin kurssilla 12. Vaikka lauseke $x-b$ ei olisi polynomin $P(x)$ tekijä, niin lähelle päästään: jos polynomin $Q(x)$ kertoimet valitaan sopivasti, voidaan kirjoittaa
\begin{align*}
P(x)&=(x-b)Q(x)+r, 
\end{align*}
missä $r$ on jokin vakio, niin sanottu jakojäännös. Jos nyt $b$ on polynomin $P$ nollakohta, sijoitetaan edelliseen yhtälöön $x=b$, jolloin
\begin{align*}
P(b)&=(b-b)Q(b)+r \quad || \ \ P(b)=0 \\
0&=0+r,  
\end{align*}
eli $r=0$, joten $x-b$ on polynomin $P(x)$ tekijä. Jos siis $x=b$ on polynomin $P(x)$ nollakohta, $x-b$ on sen tekijä.

\textbf{Esimerkki.} \\
Jaa polynomi $-2x^2-x+1$ tekijöihinsä.\\
Ratkaistaan ensin nollakohdat:
\begin{align*}
-2x^2-x+1&=0 \\
x&=\frac{-(-1) \pm \sqrt[]{(-1)^2-4 \cdot (-2) \cdot 1}}{2 \cdot (-2)} \\
x&=\frac{1 \pm \sqrt[]{1+8}}{-4} \\
x&=\frac{1 \pm 3}{-4} \\
x&=-1 \textrm{ tai } x = \frac{1}{2}.
\end{align*}
Jakolauseen mukaan $(x-\frac{1}{2})$ ja $(x-(-1))$ ovat kyseisen polynomin tekijöitä.
Ne keskenään kertomalla ei kuitenkaan saada oikeaa tulosta:
$$\left(x-\frac{1}{2}\right)(x-(-1))=\left(x-\frac{1}{2}\right)(x+1)=x^2+\frac{1}{2}x-\frac{1}{2}.$$
Puuttuu vielä korkeimman asteen termin kerroin $-2$. Sillä kertomalla saadaan alkuperäinen polynomi:
$-2(x^2+\frac{1}{2}x-\frac{1}{2})=-2x^2-x+1$.

Vastaus: $-2x^2-x+1 = -2(x-\frac{1}{2})(x+1)$.

\subsection*{Toisen asteen polynomin tekijöihin jako}

Polynomien jakolauseen mukaan

\laatikko{Jos toisen asteen polynomin $ax^2+bx+c$ nollakohdat ovat $x_1$ ja $x_2$,
$ax^2+bx+c=a(x-x_1)(x-x_2)$.}
Huomaa, että kerroin $a$ on edellisessä yhtälössä kummallakin puolella sama.
(Muuten korkeimman asteen termit eivät täsmää.)

\textbf{Esimerkki.}
Jaetaan tekijöihin $P(x)=2x^2 + 4x-30$. \\
Ratkaistaan nollakohdat yhtälöstä $$2x^2 + 4x-30=0$$ toisen asteen yhtälön ratkaisukaavalla.
Nollakohdat ovat $x_1=3$ ja $x_2=-5$. Saadaan siis
$$P(x)= 2(x-3)(x-(-5)) = 2(x-3)(x+5).$$

\subsubsection*{Kun nollakohtia on vain yksi}
Jos toisen asteen polynomilla on vain yksi nollakohta, kyseessä on niin sanottu kaksikertainen juuri. Voidaan tulkita, että nollakohdat $x_1$ ja $x_2$ ovat yhtäsuuret.

Esimerkiksi polynomin $P(x)=2x^2-4x+2$ ainoa nollakohta on $x=1$. Polynomi voidaan siis jakaa tekijöihin seuraavasti: \\ $P(x)=2(x-1)(x-1)=2(x-1)^2$. 


\subsection*{Polynomien kuvaajista}

\section{Harjoitustehtäviä}
