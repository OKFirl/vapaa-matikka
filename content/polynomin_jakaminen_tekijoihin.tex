\section{Toisen asteen polynomin jakaminen tekijöihin}

Polynomin $P(x)=(x-2)(x-3)$ nollakohdat ovat tulon nollasäännön nojalla $x=2$ ja $x=3$. Nollakohdat siis näkee suoraan tekijöihin jaetusta polynomista. Tämä tekijöiden ja nollakohtien välinen yhteys toimii myös toisin päin: nollakohtien avulla voi selvittää tekijät. Jos luku $b$ on polynomin nollakohta, $x-b$ on polynomin tekijä.

\begin{esimerkki}
Polynomin $P(x)=x²-3x+2$ nollakohdat ovat
\begin{align*}
x^2-3x+2&=0 \\
x&=\frac{-(-3) \pm \sqrt[]{(-3)^2-4 \cdot 1 \cdot 2}}{2 \cdot 1} \\
x&=\frac{3 \pm \sqrt[]{9-8}}{2} \\
x&=\frac{3 \pm 1}{2} \\
x&=1 \textrm{ tai } x = 2.
\end{align*}
Nollakohdista näemme, että $x-1$ ja $x-2$ ovat polynomin $P(x)$ tekijöitä. Kertomalla sulut auki näemmekin, että $(x-1)(x-2)=x^2-3x+2$.
\end{esimerkki}

\laatikko{\textbf{Polynomien jakolause} \\
Jos $x=b$ on polynomin nollakohta, $x-b$ on polynomin tekijä.}

Lauseen todistus on liitteessä \ref{tod:poljako}.

\begin{esimerkki}
Jaa polynomi $-2x^2-x+1$ tekijöihinsä.\\
Ratkaistaan ensin nollakohdat:
\begin{align*}
-2x^2-x+1&=0 \\
x&=\frac{-(-1) \pm \sqrt[]{(-1)^2-4 \cdot (-2) \cdot 1}}{2 \cdot (-2)} \\
x&=\frac{1 \pm \sqrt[]{1+8}}{-4} \\
x&=\frac{1 \pm 3}{-4} \\
x&=-1 \textrm{ tai } x = \frac{1}{2}.
\end{align*}
Jakolauseen mukaan $(x-\frac{1}{2})$ ja $(x-(-1))$ ovat kyseisen polynomin tekijöitä.
Ne keskenään kertomalla ei kuitenkaan saada oikeaa tulosta:
$$\left(x-\frac{1}{2}\right)(x-(-1))=\left(x-\frac{1}{2}\right)(x+1)=x^2+\frac{1}{2}x-\frac{1}{2}.$$
Puuttuu vielä korkeimman asteen termin kerroin $-2$. Sillä kertomalla saadaan alkuperäinen polynomi:
$-2(x^2+\frac{1}{2}x-\frac{1}{2})=-2x^2-x+1$.

Vastaus: $-2x^2-x+1 = -2(x-\frac{1}{2})(x+1)$.
\end{esimerkki}

\subsubsection*{Toisen asteen polynomin tekijöihin jako}

Polynomien jakolauseen mukaan

\laatikko{Jos toisen asteen polynomin $ax^2+bx+c$ nollakohdat ovat $x_1$ ja $x_2$,
\[ ax^2+bx+c=a(x-x_1)(x-x_2). \]}
Huomaa, että kerroin $a$ on edellisessä yhtälössä kummallakin puolella sama.
(Muuten korkeimman asteen termit eivät täsmää.)

\begin{esimerkki}
Jaetaan tekijöihin $P(x)=2x^2 + 4x-30$. \\
Ratkaistaan nollakohdat yhtälöstä $$2x^2 + 4x-30=0$$ toisen asteen yhtälön ratkaisukaavalla.
Nollakohdat ovat $x_1=3$ ja $x_2=-5$. Saadaan siis
$$P(x)= 2(x-3)(x-(-5)) = 2(x-3)(x+5).$$
\end{esimerkki}

\qrlinkki{http://opetus.tv/maa/maa2/polynomi-tekijoihin-nollakohtien-avulla/}{Opetus.tv: \emph{polynomin jakaminen tekijöihin nollakohtiensa avulla} (6:48)}

Jos toisen asteen polynomilla on vain yksi nollakohta, kyseessä on niin sanottu kaksinkertainen juuri. Voidaan tulkita, että nollakohdat $x_1$ ja $x_2$ ovat yhtäsuuret. Tällöin tekijöiksi saadaan $a(x-x_1)(x-x_1)=a(x-x_1)^2$.

Esimerkiksi polynomin $P(x)=2x^2-4x+2$ ainoa nollakohta on $x=1$. Polynomi voidaan siis jakaa tekijöihin seuraavasti: \\ $P(x)=2(x-1)(x-1)=2(x-1)^2$. \\

Jos toisen asteen polynomilla ei ole nollakohtia, sitä ei voi jakaa ensimmäisen asteen tekijöihin. % (Sillä ensimmäisen asteen tekijällä on aina nollakohta.)

%\subsubsection*{Joitakin yleistyksiä}
%
%Polynomeille voidaan todistaa seuraavat tulokset:
%
%\laatikko{
%\begin{itemize}
%\item Kaikki polynomit voidaan jakaa tekijöihin, jotka ovat korkeintaan toista astetta.
%\item Paritonasteisilla polynomeilla on vähintään yksi nollakohta
%\item $n$. asteen polynomilla on korkeintaan $n$ nollakohtaa
%\end{itemize}}
%
%Todistetaan näistä viimeinen:
%
%\begin{todistus}
%Jos $n$. asteen polynomilla $P$ olisi yli $n$ nollakohtaa, sillä olisi yli $n$ ensimmäisen asteen
%tekijää. Polynomin $P$ aste olisi siis yli $n$, mikä on ristiriita. Nollakohtia on siis
%korkeintaan $n$.
%\end{todistus}
%
%Tämä tulos kertoo paljon siitä, millaisia polynomien kuvaajat voivat olla.
%
%\begin{esimerkki} Kolmannen asteen polynomilla on 1--3 nollakohtaa.
%
%\begin{lukusuora}{-2.5}{3}{3.6}
%\lukusuoraisobbox
%\lukusuorakuvaaja{(x**3-x-1)/2}
%\lukusuorapienipiste{1.32472}{}
%\end{lukusuora}
%\begin{lukusuora}{-2.8}{2.5}{3.6}
%\lukusuoraisobbox
%\lukusuorakuvaaja{(x**3-x+0.3849)/2}
%\lukusuorapienipiste{-1.1547}{}
%\lukusuorapienipiste{0.577028}{}
%\end{lukusuora}
%\begin{lukusuora}{-2}{2}{3.6}
%\lukusuoraisobbox
%\lukusuorakuvaaja{1.4*(x**3-x)}
%\lukusuorapienipiste{-1}{}
%\lukusuorapienipiste{0}{}
%\lukusuorapienipiste{1}{}
%\end{lukusuora}
%
%\end{esimerkki}
%
%
%\begin{esimerkki} Neljännen asteen polynomilla on 0--4 nollakohtaa.
%
%\begin{lukusuora}{-4}{4}{3.6}
%\lukusuorabboxy{-0.5}{1.5}
%\lukusuorakuvaaja{(x**4-5*x**2+12)/14}
%\end{lukusuora}
%\begin{lukusuora}{-4}{4}{3.6}
%\lukusuorabboxy{-0.5}{1.5}
%\lukusuorakuvaaja{(x**4-5*x**2+3*x+11.2)/14}
%\lukusuorapienipiste{-1.71394}{}
%\end{lukusuora}
%\begin{lukusuora}{-4}{4}{3.6}
%\lukusuorabboxy{-0.5}{1.5}
%\lukusuorakuvaaja{(x**4-5*x**2-3)/14}
%\lukusuorapienipiste{2.354}{}
%\lukusuorapienipiste{-2.354}{}
%\end{lukusuora}
%
%\begin{lukusuora}{-4}{4}{3.6}
%\lukusuorabboxy{-0.5}{1.5}
%\lukusuorakuvaaja{(x**4-5*x**2+3*x+1.75842)/14}
%\lukusuorapienipiste{-2.43622}{}
%\lukusuorapienipiste{-0.367327}{}
%\lukusuorapienipiste{1.402}{}
%\end{lukusuora}
%\begin{lukusuora}{-2.8}{2.8}{3.6}
%\lukusuorabboxy{-0.5}{1.5}
%\lukusuorakuvaaja{0.6*(x+1.5)*(x+0.5)*(x-0.5)*(x-1.5)}
%\lukusuorapienipiste{1.5}{}
%\lukusuorapienipiste{0.5}{}
%\lukusuorapienipiste{-1.5}{}
%\lukusuorapienipiste{-0.5}{}
%\end{lukusuora}
%
%\end{esimerkki}

\Harjoitustehtavat

\begin{tehtava}
    Jaa tekijöihin
    \begin{enumerate}[a)]
        \item $-10x^2+5x+5$
        \item $8x^4-12x^2+4x$
    \end{enumerate}
    \begin{vastaus}
        \item $(-5)\cdot(2x^2-x-1)$
        \item $(4x)\cdot(2x^2-3x+1)$
    \end{vastaus}
\end{tehtava}

\begin{tehtava}
    Toisen asteen polynomille $P$ pätee $P(-3)=P(4)=0$ ja $P(1)=12$. Ratkaise $P(x)$.
    \begin{vastaus}
        $P(x)=-(x+3)(x-4)=-x^2+x+12$
    \end{vastaus}
\end{tehtava}

\begin{tehtava}
    Kolmannen asteen polynomille $P$ pätee $P(-1)=P(2)=P(3)=0$ ja $P(1)=-8$. Ratkaise $P(x)$.
    \begin{vastaus}
        $P(x)=-2(x+1)(x-2)(x-3)=-2x^3+8x^2-2x-2$
    \end{vastaus}
\end{tehtava}
