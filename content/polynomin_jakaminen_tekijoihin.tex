\chapter{Polynomin jakaminen tekijöihin}

Matematiikan 1. kurssissa on puhuttu lukujen jakamisesta tekijöihin.
Esimerkiksi luvun $12$ {\bf tekijät} ovat $1$, $2$, $3$, $4$, $6$ ja $12$. Nämä ovat sellaisia
lukuja, joista saadaan $12$ kertomalla ne jollain kokonaisluvulla. Sanotaan myös, että luvun $12$
{\bf alkutekijät} ovat $2$, $2$ ja $3$, koska luku $12$ voidaan
ilmaista niiden tulona ($2\cdot 2\cdot 3 = 2^2\cdot 3 = 12$), mutta näitä tekijöitä ei
voi enää jakaa pienempiin osatekijöihin. Kokonaisluvun tekijät ovat aina kokonaislukuja.

Vastavasti voidaan puhua polynomin jakamisesta {\bf tekijöihin} tai {\bf alkutekijöihin}. Polynomin tekijät
ovat aina polynomeja. Olemme jo oppineet kertomaan polynomeja keskenään,
joten voimme helposti laskea, että $(x-3)(2x^2-8x+8)=2x^3-14x^2+32x-24$.
Nyt voidaan sanoa, että $x-3$ ja $2x^2-8x+8$ ovat polynomin $2x^3-14x^2+32x-24$ tekijöitä.
Mutta ovatko ne alkutekijöitä, vai voidaanko ne edelleen jakaa pienempiin tekijöihin?
Itse asiassa laskemalla voidaan todeta että $2x^2-8x+8$ saadaan tulokseksi kertolaskusta $2(x-2)(x-2)=2(x-2)^2$.
Voimme siis ilmoitaa polynomin tekijöidensä avulla: $2x^3-14x^2+32x-24=2(x-3)(x-2)^2$.

Mutta miksi haluaisimme jakaa polynomeja tekijöihin?
Yksi tärkeimmistä hyödyistä liittyy tulon nollasääntöön. Tiedämme nimittäin, että tulo on nolla jos ja vain jos
jokin tulon tekijöistä on nolla. Jos osaamme jakaa polynomin tekijöihin, pystymme näin helposti päättelemään polynomin nollakohdat.

Jos vaikkapa haluamme ratkaista yhtälön $2x^3-14x^2+32x-24=0$ ja satumme tietämään, että $2x^3-14x^2+32x-24=2(x-3)(x-2)^2$,
voimme helposti päätellä, että ainoat mahdolliset ratkaisut saadaan yhtälöistä $x-3=0$ ja $x-2=0$

% Jokainen reaalikertoiminen polynomi voidaan jakaa tekijöihin, jotka ovat korkeintaan toista astetta.

Polynomin jakaminen tekijöihin on siis varsin hyödyllistä. Tekijöihin jakaminen ei aina ole helppoa,
mutta tässä luvussa käsitellään tapoja, joilla tekijöihin jakaminen monissa tilanteissa onnistuu.

\section{Harjoitustehtäviä}
