\chapter{Polynomien jakolause}

Polynomin $P(x)=(x-2)\cdot(x-3)$ nollakohdat ovat tulon nollasäännön nojalla $x=2$ ja $x=3$. Nollakohdat siis näkee suoraan tekijöihin jaetusta polynomista. Tämä tekijöiden ja nollakohtien välinen yhteys toimii myös toisin päin: nollakohtien avulla voi selvittää tekijät. Jos luku $b$ on polynomin nollakohta, $x-b$ on polynomin tekijä.

%Esimerkiksi polynomin $P(x)=x²-3x+2$ nollakohdat ovat
%\begin{align*}
%x^2-3x+2&=0 \\
%x&=\frac{-(-3) \pm \sqrt[]{(-3)^2-4 \cdot 1 \cdot 2}}{2 \cdot 1} \\
%x&=\frac{3 \pm \sqrt[]{9-8}}{2} \\
%x&=\frac{3 \pm 1}{2} \\
%x&=1 \textrm{ tai } x = 2.
%\end{align*}
%Tämän perusteella voidaan päätellä, että $x-1$ ja $x-2$ ovat polynomin $P(x)$ tekijöitä. Toden %totta: sulut auki kertomalla voidaan tarkistaa, että $(x-1)(x-2)=x^2-3x+2$.

\laatikko{\textbf{Polynomien jakolause} \\
Jos $x=b$ on polynomin nollakohta, $x-b$ on polynomin tekijä.}

Lauseen todistus on liiteessä X.
\todo{Lisää todistus (tässä kohtaa kommenttina) liiteeksi}

%\textbf{Todistus.} Jakolauseen todistus perustuu polynomien jakoyhtälöön, josta tarkemmin kurssilla 12. Vaikka lauseke $x-b$ ei olisi polynomin $P(x)$ tekijä, niin lähelle päästään: jos polynomin $Q(x)$ kertoimet valitaan sopivasti, voidaan kirjoittaa
%\begin{align*}
%P(x)&=(x-b)Q(x)+r,
%\end{align*}
%missä $r$ on jokin vakio, niin sanottu jakojäännös. Jos nyt $b$ on polynomin $P$ nollakohta, sijoitetaan edelliseen yhtälöön $x=b$, jolloin
%\begin{align*}
%P(b)&=(b-b)Q(b)+r \quad || \ \ P(b)=0 \\
%0&=0+r,
%\end{align*}
%eli $r=0$, joten $x-b$ on polynomin $P(x)$ tekijä. Jos siis $x=b$ on polynomin $P(x)$ nollakohta, $x-b$ on sen tekijä. $_\Box$

\textbf{Esimerkki.} \\
Jaa polynomi $-2x^2-x+1$ tekijöihinsä.\\
Ratkaistaan ensin nollakohdat:
\begin{align*}
-2x^2-x+1&=0 \\
x&=\frac{-(-1) \pm \sqrt[]{(-1)^2-4 \cdot (-2) \cdot 1}}{2 \cdot (-2)} \\
x&=\frac{1 \pm \sqrt[]{1+8}}{-4} \\
x&=\frac{1 \pm 3}{-4} \\
x&=-1 \textrm{ tai } x = \frac{1}{2}.
\end{align*}
Jakolauseen mukaan $(x-\frac{1}{2})$ ja $(x-(-1))$ ovat kyseisen polynomin tekijöitä.
Ne keskenään kertomalla ei kuitenkaan saada oikeaa tulosta:
$$\left(x-\frac{1}{2}\right)(x-(-1))=\left(x-\frac{1}{2}\right)(x+1)=x^2+\frac{1}{2}x-\frac{1}{2}.$$
Puuttuu vielä korkeimman asteen termin kerroin $-2$. Sillä kertomalla saadaan alkuperäinen polynomi:
$-2(x^2+\frac{1}{2}x-\frac{1}{2})=-2x^2-x+1$.

Vastaus: $-2x^2-x+1 = -2(x-\frac{1}{2})(x+1)$.

\subsection*{Toisen asteen polynomin tekijöihin jako}

Polynomien jakolauseen mukaan

\laatikko{Jos toisen asteen polynomin $ax^2+bx+c$ nollakohdat ovat $x_1$ ja $x_2$,
\[ ax^2+bx+c=a(x-x_1)(x-x_2). \]}
Huomaa, että kerroin $a$ on edellisessä yhtälössä kummallakin puolella sama.
(Muuten korkeimman asteen termit eivät täsmää.)

\textbf{Esimerkki.}
Jaetaan tekijöihin $P(x)=2x^2 + 4x-30$. \\
Ratkaistaan nollakohdat yhtälöstä $$2x^2 + 4x-30=0$$ toisen asteen yhtälön ratkaisukaavalla.
Nollakohdat ovat $x_1=3$ ja $x_2=-5$. Saadaan siis
$$P(x)= 2(x-3)(x-(-5)) = 2(x-3)(x+5).$$

\subsubsection*{Kun nollakohtia on vain yksi tai ei lainkaan}
Jos toisen asteen polynomilla on vain yksi nollakohta, kyseessä on niin sanottu kaksinkertainen juuri. Voidaan tulkita, että nollakohdat $x_1$ ja $x_2$ ovat yhtäsuuret. Tällöin tekijöiksi saadaan $a(x-x_1)(x-x_1)=a(x-x_1)^2$.

Esimerkiksi polynomin $P(x)=2x^2-4x+2$ ainoa nollakohta on $x=1$. Polynomi voidaan siis jakaa tekijöihin seuraavasti: \\ $P(x)=2(x-1)(x-1)=2(x-1)^2$. \\

Jos toisen asteen polynomilla ei ole nollakohtia, sitä ei voi jakaa ensimmäisen asteen tekijöihin. (Sillä ensimmäisen asteen tekijällä on aina nollakohta.)

\subsection*{Joitakin yleistyksiä}

Polynomeille voidaan todistaa seuraavat tulokset:

\laatikko{
\begin{itemize}
\item Kaikki polynomit voidaan jakaa tekijöihin, jotka ovat korkeintaan toista astetta.
\item Paritonasteisilla polynomeilla on vähintään yksi nollakohta
\item $n$. asteen polynomilla on korkeintaan $n$ nollakohtaa
\end{itemize}}
Todistetaan näistä viimeinen: \\
Jos $n$. asteen polynomilla $P$ olisi yli $n$ nollakohtaa, sillä olisi yli $n$ ensimmäisen asteen
tekijää. Polynomin $P$ aste olisi siis yli $n$, mikä on ristiriita. Nollakohtia on siis
korkeintaan $n$. $_\Box$

Tämä tulos kertoo paljon siitä, millaisia polynomien kuvaajat voivat olla.

\textbf{Esimerkki 1.} Kolmannen asteen polynomilla on 1 - 3 nollakohtaa.
\todo{asemoi esimerkit}

\begin{lukusuora}{-2.5}{3}{3}
\lukusuorakuvaaja{(x**3-x-1)/2}
\lukusuorapienipiste{1.32472}{}
\end{lukusuora}
\begin{lukusuora}{-2.8}{2.5}{3}
\lukusuorakuvaaja{(x**3-x+0.3849)/2}
\lukusuorapienipiste{-1.1547}{}
\lukusuorapienipiste{0.577028}{}
\end{lukusuora}
\begin{lukusuora}{-2}{2}{3}
\lukusuorakuvaaja{1.4*(x**3-x)}
\lukusuorapienipiste{-1}{}
\lukusuorapienipiste{0}{}
\lukusuorapienipiste{1}{}
\end{lukusuora}

\textbf{Esimerkki 2.} Neljännen asteen polynomilla on 0 - 4 nollakohtaa.

\begin{lukusuora}{-4}{4}{3}
\lukusuorakuvaaja{(x**4-5*x**2+12)/14}
\end{lukusuora}
\begin{lukusuora}{-4}{4}{3}
\lukusuorakuvaaja{(x**4-5*x**2+3*x+11.2)/14}
\lukusuorapienipiste{-1.71394}{}
\end{lukusuora}
\begin{lukusuora}{-4}{4}{3}
\lukusuorakuvaaja{(x**4-5*x**2-3)/14}
\lukusuorapienipiste{2.354}{}
\lukusuorapienipiste{-2.354}{}
\end{lukusuora}

\begin{lukusuora}{-4}{4}{3}
\lukusuorakuvaaja{(x**4-5*x**2+3*x+1.75842)/14}
\lukusuorapienipiste{-2.43622}{}
\lukusuorapienipiste{-0.367327}{}
\lukusuorapienipiste{1.402}{}
\end{lukusuora}
\begin{lukusuora}{-2.8}{2.8}{3}
\lukusuorakuvaaja{0.6*(x+1.5)*(x+0.5)*(x-0.5)*(x-1.5)}
\lukusuorapienipiste{1.5}{}
\lukusuorapienipiste{0.5}{}
\lukusuorapienipiste{-1.5}{}
\lukusuorapienipiste{-0.5}{}
\end{lukusuora}

\section{Harjoitustehtäviä}

\begin{tehtava}
  Jaa tekijöihin
  \begin{enumerate}[a)]
    \item $-10x^2+5x+5$
    \item $8x^4-12x^2+4x$
  \end{enumerate}

  \begin{vastaus}
    \item $(-5)\cdot(2x^2-x-1)$
    \item $(4x)\cdot(2x^2-3x+1)$
  \end{vastaus}
\end{tehtava}

\begin{tehtava}
  Kolmanen asteen polynomille $P$ pätee $P(-1)=P(2)=P(3)=0$ ja $P(1)=-8$. Ratkaise $P(x)$.

  \begin{vastaus}
    $P(x)=-2(x+1)(x-2)(x-3)=-2x^3+8x^2-2x-2.$
  \end{vastaus}
\end{tehtava}
