\chapter{Korkeamman asteen yhtälöt}

\section{Sijoituksen avulla ratkeavat yhtälöt}

Malliesimerkki sijoituksen avulla ratkeavasta yhtälöstä on ns. bikvadraattinen yhtälö.
Bikvadraattinen yhtälö on muotoa $ax^4+bx^2+c=0$ ja se ratkeaa sijoituksella $y=x^2$.
Vastaavasti sijoituksella $y=x^n$ voidaan ratkaista kaikki yhtälöt, jotka ovat muotoa $ax^{2n}+bx^n+c=0$.

\begin{esimerkki}{\ }
\begin{itemize}
    \item Ratkaistaan yhtälö $2x^4+14x^2-36=0$.
    \item Tehdään sijoitus $y=x^2$.
    \item Yhtälö saa muodon $2y^2+14y-36=0$.
    \item Sievennetään yhtälöä jakamalla puolittain kahdella: $y^2+7y-18=0$.
    \item Ratkaistaan toisen asteen yhtälön ratkaisukaavalla $y=2$ tai $y=-9$.
    \item Reaaliluvun neliö ei voi olla negatiivinen, joten hylätään jälkimmäinen juuri.
    \item Ratkaistaan ensimmäisestä juuresta $x$: $x^2=2 \Leftrightarrow  x=\pm\sqrt{2}$.
    \item Saadaan vastaukseksi kaksi reaalijuurta yhtälölle: $x=\pm\sqrt{2}$.
\end{itemize}
\end{esimerkki}

\section{Harjoitustehtäviä}
