\section{Korkeamman asteen yhtälöt}

\qrlinkki{http://opetus.tv/maa/maa2/n-asteinen-polynomifunktio/}{Opetus.tv: \emph{N-asteinen polynomifunktio} (10:40)}

\qrlinkki{http://opetus.tv/maa/maa2/n-asteinen-polynomiyhtalo/}{Opetus.tv: \emph{N-asteinen polynomiyhtälö} (8:38, 6:20 ja 15:53)}

Toisen asteen polynomiyhtälön ratkaiseminen on helppoa, sillä voimme aina soveltaa toisen asteen yhtälön ratkaisukaavaa.
Kolmannen ja sitä korkeamman asteen yhtälöiden ratkaiseminen on vaikeampaa.
Kolmannen ja neljännen asteen yhtälöille on olemassa ratkaisukaavat, mutta ne ovat epäkäytännöllisiä.
Vuonna 1823 Niels Abel osoitti, että viidennen ja sitä korkeamman asteen yhtälöille ei ole olemassa ratkaisukaavaa.

Käytännössä korkeamman asteen yhtälöt täytyy ratkaista numeerisesti tietokoneen avulla. Joissain erikoistapauksissa ratkaiseminen onnistuu käsin.

\subsection*{Tekijöihinjako}

Polynomiyhtälöitä voi ratkaista jakamalla polynomin tekijöihin.

%Jos polynomiyhtälössä $P(x) = 0$ polynomi $P(x)$ voidaan jakaa tekijöihin, ratkaisu saadaan etsimällä näiden tekijöiden nollakohdat.
%Esimerkiksi vakiotermittömässä polynomissa voidaan ottaa muuttuja yhteiseksi tekijäksi ja riittää ratkaista yhtä pienemmän asteen %polynomiyhtälö.

\begin{esimerkki}
Ratkaise yhtälö $x^3 - 3x^2 + x = 0$.

\textbf{Ratkaisu:}

Polynomissa $x^3 - 3x^2 + x$ ei ole vakiotermiä. Voidaan siis ottaa muuttuja $x$ yhteiseksi tekijäksi, joten se saadaan muotoon $x(x^2 - 3x + 1)$. Nyt yhtälö ratkeaa tulon nollasäännön avulla:

Nyt yhtälön ratkeaa tulon nollasäännöllä.
\begin{align*}
x^3 - 3&x^2 + x=0 \\
x(x^2 - 3x + 1)=0 \\
x= 0 \quad &\text{tai} \quad x^2 - 3x + 1 = 0 \\
\end{align*}

Käyttämällä toisen asteen yhtälön ratkaisukaavaa saadaan yhtälön $x^2 - 3x + 1 = 0$ ratkaisuksi
\[x = \frac{3\pm \sqrt{5}}{2}.\]

\textbf{Vastaus:}
$x= 0$ tai $x=\dfrac{3\pm \sqrt{5}}{2}$
\end{esimerkki}

Aiemmin esitellyn polynomien jakolauseen mukaan kaikki polynomit voidaan jakaa tekijöihin, jotka ovat korkeintaan toista astetta.
Periaatteessa tällä tavalla voidaan siis ratkaista kaikki polynomiyhtälöt. Tekijöihin jakaminen on kuitenkin yleensä vaikeaa.

\begin{esimerkki}
Ratkaise yhtälö $x^3-17x^2-x+17 = 0$.

\textbf{Ratkaisu:}

Yhtälön vasemmalla puolella olevan polynomin voi jakaa tekijöihin ryhmittelemällä:

\begin{align*}
x^3-17x^2-x+17=x^2(x-17)+(-1)(x-17)=(x^2-1)(x-17)
\end{align*}

Nyt yhtälön ratkeaa tulon nollasäännöllä.
\begin{align*}
x^3-17x^2&-x+17=0 \\
(x^2-1)&(x-17)=0 \\
x - 17 = 0 \quad &\text{tai} \quad x^2-1 = 0 \\
x = 17 \quad &\text{tai} \quad x^2 = 1 \\
x = 17 \quad &\text{tai} \quad x =\pm 1 \\
\end{align*}

\textbf{Vastaus:} $x = 17$ tai $x = \pm 1$
\end{esimerkki}

\subsection*{Sijoitukset}
Joskus yhtälöt ratkeavat ns. sijoituksella eli muuttujanvaihdolla.

%Esimerkiksi muotoa $ax^4+bx^2+c=0$ olevissa yhtälöissä huomataan, että merkitsemällä lauseketta $x^2$ kirjaimella $y$, yhtälö voidaan kirjoittaa muotoon $ay^2+by+c=0$. Uudesta yhtälöstä voidaan ratkaista $y$ toisen asteen yhtälön ratkaisukaavalla, ja sijoituksesta $y = x^2$ voidaan ratkaista $x$.

\begin{esimerkki}
Ratkaise yhtälö $2x^4+14x^2-36=0$.

\textbf{Ratkaisu:}

Yhtälö voidaan kirjoittaa muodossa $2(x^2)^2+14x^2-36=0$. Kun merkitään $y=x^2$, saadaankin muuttujan $y$ yhtälö
\[2y^2+14y-36=0.\]
Tämä on toisen asteen yhtälö, joka osataan ratkaista toisen asteen yhtälön ratkaisukaavalla. Ratkaisut ovat $y = 2$ ja $y = -9$.

Nyt on vielä selvitettävä muuttujan $x$ arvot. Koska $x^2=y$, saadaan yhtälöt $x^2=2$ ja $x^2=-9$. Reaaliluvun neliö ei kuitenkaan voi olla negatiivinen, joten ainoat ratkaisut ovat yhtälön $x^2 = 2$ ratkaisut $x = \pm\sqrt{2}$.

\textbf{Vastaus:} $x=-2$ tai $x=2$
\end{esimerkki}

Muotoa $ax^4+bx^2+c=0$ oleva yhtälö voidaan aina ratkaista sijoituksella $y=x^2$.  Yleisemmin muotoa $ax^{2n}+bx^n+c=0$ olevat yhtälöt voidaan ratkaista sijoituksella $y = x^n$.

\begin{esimerkki}
Ratkaise yhtälö $x^{10}+x^5=2$.

\textbf{Ratkaisu:}

Muutetaan yhtälö muotoon $x^{10}+x^5-2=0$ ja edelleen $(x^5)^2+x^5-2=0$. Tehdään sijoitus $y = x^5$. Nyt yhtälö saa muodon $y^2+y-2 = 0$. Toisen asteen yhtälön ratkaisukaavalla saadaan yhtälön ratkaisuiksi $y = -2$ ja $y = 1$.

Nyt alkuperäisen yhtälön ratkaisut saadaan yhtälöistä $x^5=-2$ ja $x^5=1$. Siten ratkaisut ovat $x = \sqrt[5]{-2}$ ja $x = 1$.

\textbf{Ratkaisu:} $x = \sqrt[5]{-2}$ ja $x = 1$

\end{esimerkki}

\Harjoitustehtavat

\begin{tehtava}
    Ratkaise yhtälöt.
    \begin{enumerate}[a)]
        \item $x^4 - 2x^2 - 24 = 0$
        \item $x^4 - 4x^2 - 5 = 0$
        \item $x^4 - 8x^2 + 15 = 0$
    \end{enumerate}
    \begin{vastaus}
        \begin{enumerate}[a)]
            \item $x = \pm\sqrt{6}$
            \item $x = \pm\sqrt{5}$
            \item $x = \pm\sqrt{3}$ tai $\pm\sqrt{5}$
        \end{enumerate}
    \end{vastaus}
\end{tehtava}

\begin{tehtava}
    Ratkaise yhtälöt.
    \begin{enumerate}[a)]
        \item $x^8 - 1 = 0$
        \item $x^8 - x^4 = 0$
        \item $x^8 - x^4 - 1 = 0$
    \end{enumerate}
    \begin{vastaus}
        \begin{enumerate}[a)]
            \item $x = \pm\sqrt{1}$
            \item $x = 0$ tai $x = \pm\sqrt{1}$
            \item $x = \pm\sqrt[4]{\frac{1+\sqrt{5}}{2}} = \pm\sqrt[4]{\upvarphi}$ ($\upvarphi$ on kultaisena leikkauksena tunnettu vakio)
        \end{enumerate}
    \end{vastaus}
\end{tehtava}

\begin{tehtava}
    Ratkaise yhtälöt.
    \begin{enumerate}[a)]
        \item $x^4 - 16 = 0$
        \item $2x^4 = 8x^2$
        \item $x^6 - 2x^3 = 3$
        \item $x^{100} - 2x^{50} + 1 = 0$
    \end{enumerate}
    \begin{vastaus}
        \begin{enumerate}[a)]
            \item $x = \pm2$
            \item $x = 0$ tai $x=\pm2$
            \item $x = \sqrt[3]{3}$ tai $x= -1$
            \item $x = \pm1$
        \end{enumerate}
    \end{vastaus}
\end{tehtava}

\begin{tehtava}
    Ratkaise yhtälöt.
    \begin{enumerate}[a)]
        \item $x^4 + 7x^3 = 0$
        \item $2x^3 - 16x^2 + 32x = 0$
        \item $x^6 + 6x^5 = -9x^4$
        \item $x^3 - 2x^5 = 0$      
    \end{enumerate}
    \begin{vastaus}
        \begin{enumerate}[a)]
        	\item $x = 0$ tai $x = -7$
        	\item $x = 0$ tai $x = 4$
        	\item $x = 0$ tai $x = -3$
            \item $x = 0$ tai $x = \pm\dfrac{1}{\sqrt{2}}$
        \end{enumerate}
    \end{vastaus}
\end{tehtava}

\begin{tehtava}
    Ratkaise yhtälöt.
    \begin{enumerate}[a)]
        \item $x^5 - 2x^3 + x = 0$
        \item $x^8 + 4x^4 = 5x^6$       
    \end{enumerate}
    \begin{vastaus}
        \begin{enumerate}[a)]
        	\item $x = 0$ tai $x = \pm1$
        	\item $x = 0$ tai $x = \pm1$ tai $x = \pm2$
        \end{enumerate}
    \end{vastaus}
\end{tehtava}

\begin{tehtava}
	Ratkaise yhtälö $x^{627} - 6x^{514} + 5x^{401} = 0$.
	\begin{vastaus}
		$x = 0$, $x = 1$ tai $x = \sqrt[113]{5}$
	\end{vastaus}
\end{tehtava}

\begin{tehtava}
	(K02/T2b) Ratkaise yhtälö $e^{x^3+4x^2+1}=1$. [$e$ on matemaattinen vakio, irrationaaliluku, jonka likiarvo on $2,718$.]
	\begin{vastaus}
	$x=0$ tai $x=-2 + \sqrt[]{3}$ tai $x=-2 - \sqrt[]{3}$
	\end{vastaus}
\end{tehtava}

\begin{tehtava}
	$[ \star ]$ Ratkaise yhtälö $2^x-1=\frac{12}{2^x}$
	\begin{vastaus}
	$x=2$
	\end{vastaus}
\end{tehtava}
