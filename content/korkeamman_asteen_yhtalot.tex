\chapter{Korkeamman asteen yhtälöt}
Toisen asteen polynomiyhtälön ratkaiseminen on helppoa toisen asteen yhtälön ratkaisukaavan avulla. Korkeamman asteen yhtälöiden ratkaiseminen on monimutkaisempaa. Myös kolmannen ja neljännen asteen yhtälöille on olemassa ratkaisukaavat, mutta ne ovat hyvin monimutkaisia. Vuonna 1823 Niels Abel osoitti, että viidennen ja sitä korkeamman asteen yhtälöille ei ole olemassa ratkaisukaavaa.

Siis käytännössä korkeamman asteen yhtälöt täytyy ratkaista numeerisesti tietokoneen avulla. Kuitenkin joissain erikoistapauksissa ratkaiseminen onnistuu käsin.

\section{Tekijöihinjako}
Jos polynomiyhtälön $P(x) = 0$ polynomi $P$ voidaan jakaa tekijöihin, ratkaisu saadaan etsimällä näiden tekijöiden nollakohdat. Esimerkiksi jos polynomista puuttuu vakiotermi, voidaan ottaa muuttuja yhteiseksi tekijäksi ja riittää ratkaista yhtä pienemmän asteen polynomiyhtälö.

\begin{esimerkki}
Ratkaistaan yhtälö $x^3 - 3x^2 + x = 0$. Huomataan, että vasemmalla puolella on yhteisenä tekijänä $x$, eli yhtälö voidaan kirjoittaa muodossa $x(x^2-3x+1) = 0$. Siis yhtälö pätee täsmälleen silloin kun
\[x = 0 \text{ tai } x^2-3x+1 = 0.\]
Siis käyttämällä toisen asteen yhtälön ratkaisukaavaa saadaan, että yhtälön ratkaisu on
\[x = 0 \text{ tai } x = \frac{3\pm \sqrt{5}}{2}.\]
\end{esimerkki}

Koska polynomien jakolauseen mukaan kaikki polynomit voidaan jakaa tekijöihin, jotka ovat korkeintaan toista astetta, periaatteessa tällä tavalla voidaan ratkaista kaikki polynomiyhtälöt. Yleensä kuitenkin tekijöihin jakaminen on vaikeaa.

\section{Sijoituksen avulla ratkeavat yhtälöt}
Joskus yhtälöt ratkeavat ns. sijoituksella eli muuttujanvaihdolla. Esimerkiksi muotoa $ax^4+bx^2+c=0$ olevissa yhtälöissä huomataan, että jos merkitään kirjaimella $y$ lauseketta $x^2$, yhtälö voidaan kirjoittaa muotoon $ay^2+by+c=0$. Nyt $y$ voidaan ratkaista toisen asteen ratkaisukaavalla, ja sijoituksesta $y = x^2$ voidaan ratkaista $x$. 

\begin{esimerkki}
Ratkaistaan yhtälö $2x^4+14x^2-36=0$. Tehdään sijoitus $y = x^2$. Nyt yhtälö saa muodon $2y^2+14y-36 = 0$. Toisen asteen yhtälön ratkaisukaavalla saadaan yhtälön ratkaisuiksi $y = 2$ ja $y = -9$.

Nyt alkuperäisen yhtälön ratkaisut ovat ne $x$:n arvot jotka toteuttavat ainakin toisen yhtälöistä $x^2 = 2$ ja $x^2 = -9$. Reaaliluvun neliö ei ole koskaan negatiivinen, eli ainoat ratkaisut ovat yhtälön $x^2 = 2$ ratkaisut $x = \pm\sqrt{2}$.
\end{esimerkki}

\Harjoitustehtavat

\begin{tehtava}
    Ratkaise yhtälöt.
    \begin{enumerate}[a)]
        \item $x^4 - 2x^2 - 24 = 0$
        \item $x^4 - 4x^2 - 5 = 0$
        \item $x^4 - 8x^2 + 15 = 0$
    \end{enumerate}
    \begin{vastaus}
        \begin{enumerate}[a)]
            \item $x = \pm\sqrt{6}$
            \item $x = \pm\sqrt{5}$
            \item $x = \pm\sqrt{3}$ tai $\pm\sqrt{5}$
        \end{enumerate}
    \end{vastaus}
\end{tehtava}

\begin{tehtava}
    Ratkaise yhtälöt.
    \begin{enumerate}[a)]
        \item $x^8 - 1 = 0$
        \item $x^8 - x^4 = 0$
        \item $x^8 - x^4 - 1 = 0$
    \end{enumerate}
    \begin{vastaus}
        \begin{enumerate}[a)]
            \item $x = \pm\sqrt{1}$
            \item $x = 0$ tai $x = \pm\sqrt{1}$
            \item $x = \pm\sqrt[4]{\frac{1+\sqrt{5}}{2}} = \pm\sqrt[4]{\upvarphi}$ ($\upvarphi$ on kultaisena leikkauksena tunnettu vakio)
        \end{enumerate}
    \end{vastaus}
\end{tehtava}

\begin{tehtava}
    Ratkaise yhtälöt.
    \begin{enumerate}[a)]
        \item $x^4 - 16 = 0$
        \item $2x^4 = 8x^2$
        \item $x^6 - 2x^3 = 3$
        \item $x^{100} - 2x^{50} + 1 = 0$
    \end{enumerate}
    \begin{vastaus}
        \begin{enumerate}[a)]
            \item $x = \pm2$
            \item $x = 0$ tai $x=\pm2$
            \item $x = \sqrt[3]{3}$ tai $x= -1$
            \item $x = \pm1$
        \end{enumerate}
    \end{vastaus}
\end{tehtava}

\begin{tehtava}
    Ratkaise yhtälöt.
    \begin{enumerate}[a)]
        \item $x^4 + 7x^3 = 0$
        \item $2x^3 - 16x^2 + 32x = 0$
        \item $x^6 + 6x^5 = -9x^4$
        \item $x^3 - 2x^5 = 0$      
    \end{enumerate}
    \begin{vastaus}
        \begin{enumerate}[a)]
        	\item $x = 0$ tai $x = -7$
        	\item $x = 0$ tai $x = 4$
        	\item $x = 0$ tai $x = -3$
            \item $x = 0$ tai $x = \pm\dfrac{1}{\sqrt{2}}$
        \end{enumerate}
    \end{vastaus}
\end{tehtava}

\begin{tehtava}
    Ratkaise yhtälöt.
    \begin{enumerate}[a)]
        \item $x^5 - 2x^3 + x = 0$
        \item $x^8 + 4x^4 = 5x^6$       
    \end{enumerate}
    \begin{vastaus}
        \begin{enumerate}[a)]
        	\item $x = 0$ tai $x = \pm1$
        	\item $x = 0$ tai $x = \pm1$ tai $x = \pm2$
        \end{enumerate}
    \end{vastaus}
\end{tehtava}

\begin{tehtava}
	Ratkaise yhtälö $x^{627} - 6x^{514} + 5x^{401} = 0$.
	\begin{vastaus}
		$x = 0$, $x = 1$ tai $x = \sqrt[113]{5}$
	\end{vastaus}
\end{tehtava}

\begin{tehtava}
	(K02/T2b) Ratkaise yhtälö $e^{x^3+4x^2+1}=1$. [$e$ on matemaattinen vakio, irrationaaliluku, jonka likiarvo on $2,718$.]
	\begin{vastaus}
	$x=0$ tai $x=-2 + \sqrt[]{3}$ tai $x=-2 - \sqrt[]{3}$
	\end{vastaus}
\end{tehtava}

\begin{tehtava}
	$[ \star ]$ Ratkaise yhtälö $2^x-1=\frac{12}{2^x}$
	\begin{vastaus}
	$x=2$
	\end{vastaus}
\end{tehtava}
