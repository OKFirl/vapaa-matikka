\section{Korkeamman asteen yhtälöt}
Toisen asteen polynomiyhtälön ratkaiseminen on helppoa, sillä voimme aina soveltaa toisen asteen yhtälön ratkaisukaavaa.
Kolmannen ja sitä korkeamman asteen yhtälöiden ratkaiseminen on vaikeampaa.
Kolmannen ja neljännen asteen yhtälöille on olemassa ratkaisukaavat, mutta ne ovat epäkäytännöllisiä.
Vuonna 1823 Niels Abel osoitti, että viidennen ja sitä korkeamman asteen yhtälöille ei ole olemassa ratkaisukaavaa.

Käytännössä korkeamman asteen yhtälöt täytyy ratkaista numeerisesti tietokoneen avulla. Joissain erikoistapauksissa ratkaiseminen onnistuu käsin.

\subsection*{Tekijöihinjako}
Jos polynomiyhtälön $P(x) = 0$ polynomi $P$ voidaan jakaa tekijöihin, ratkaisu saadaan etsimällä näiden tekijöiden nollakohdat. Esimerkiksi
vakiotermittömässä polynomissa voidaan ottaa muuttuja yhteiseksi tekijäksi ja riittää ratkaista yhtä pienemmän asteen polynomiyhtälö.

\begin{esimerkki}
Ratkaistaan yhtälö $x^3 - 3x^2 + x = 0$. Polynomissa $P(x) = x^3 - 3x^2 + x$ ei ole vakiotermiä. Voimme siis ottaa muuttujan $x$ yhteiseksi tekijäksi ja saamme yhtälölle muodon $x(x^2 - 3x + 1) = 0$. Tulon nollasäännön perusteella yhtälö pätee täsmälleen silloin kun
\[x = 0 \text{ tai } x^2-3x+1 = 0.\]
Käyttämällä toisen asteen yhtälön ratkaisukaavaa saamme yhtälön ratkaisuksi
\[x = 0 \text{ tai } x = \frac{3\pm \sqrt{5}}{2}.\]
\end{esimerkki}

Aiemmin esitellyn polynomien jakolauseen mukaan kaikki polynomit voidaan jakaa tekijöihin, jotka ovat korkeintaan toista astetta.
Periaatteessa voimme siis ratkaista tällä tavalla kaikki polynomiyhtälöt. Tekijöihin jakaminen on kuitenkin yleensä vaikeaa.

\begin{esimerkki}
Ratkaistaan yhtälö $x^3-17x^2-x+17 = 0$. Kolmannen asteen yhtälö on lähtökohtaisesti vaikea ratkaista. Tässä tapauksessa huomataan
kuitenkin, että vasemman puolen polynomilla on tekijänä $x-17$. Näin ollen yhtälö voidaan kirjoittaa muodossa $(x-17)(x^2-1) = 0$.
Siis yhtälö pätee täsmälleen silloin kun
\[x - 17 = 0 \text{ tai } x^2-1 = 0.\]
Ratkaisemalla nämä yhtälöt saamme alkuperäisen yhtälön ratkaisuksi
\[x = 17 \text{ tai } x = \pm 1.\]
\end{esimerkki}

\subsection*{Sijoitukset}
Joskus yhtälöt ratkeavat ns. sijoituksella eli muuttujanvaihdolla. Esimerkiksi muotoa $ax^4+bx^2+c=0$ olevissa yhtälöissä huomataan, että merkitsemällä
lauseketta $x^2$ kirjaimella $y$, yhtälö voidaan kirjoittaa muotoon $ay^2+by+c=0$. Uudesta yhtälöstä voidaan ratkaista $y$ toisen asteen yhtälön
ratkaisukaavalla, ja sijoituksesta $y = x^2$ voidaan ratkaista $x$.

\begin{esimerkki}
Ratkaistaan yhtälö $2x^4+14x^2-36=0$. Sijoituksella $y = x^2$ yhtälö saa muodon $2y^2+14y-36 = 0$. Toisen asteen yhtälön ratkaisukaavalla saadaan yhtälön ratkaisuiksi $y = 2$ ja $y = -9$.

Alkuperäisen yhtälön ratkaisut ovat yhtälön $y = x^2$ ratkaisut muuttujan $x$ suhteen, kun $y = 2$ ja kun $y = -9$. Reaaliluvun neliö ei ole koskaan
negatiivinen, joten ainoat ratkaisut ovat yhtälön $x^2 = 2$ ratkaisut $x = \pm\sqrt{2}$.
\end{esimerkki}

Yleisemmin muotoa $ax^{2n}+bx^n+c=0$ olevat yhtälöt voidaan ratkaista sijoituksella $y = x^n$ ja käyttämällä toisen asteen yhtälön ratkaisukaavaa.

\begin{esimerkki}
Ratkaistaan yhtälö $x^{10}+x^5=2$. Muutetaan yhtälö perusmuotoon $x^{10}+x^5-2=0$. Tehdään sijoitus $y = x^5$. Nyt yhtälö saa muodon $y^2+y-2 = 0$. Toisen asteen yhtälön ratkaisukaavalla saadaan yhtälön ratkaisuiksi $y = -2$ ja $y = 1$.

Nyt alkuperäisen yhtälön ratkaisut ovat yhtälön $y = x^5$ ratkaisut muuttujan $x$ suhteen, kun $y = -2$ ja kun $y = 1$. Siis ratkaisut ovat $x = \sqrt[5]{-2}$ ja $x = 1$.
\end{esimerkki}

\Harjoitustehtavat

\begin{tehtava}
    Ratkaise yhtälöt.
    \begin{enumerate}[a)]
        \item $x^4 - 2x^2 - 24 = 0$
        \item $x^4 - 4x^2 - 5 = 0$
        \item $x^4 - 8x^2 + 15 = 0$
    \end{enumerate}
    \begin{vastaus}
        \begin{enumerate}[a)]
            \item $x = \pm\sqrt{6}$
            \item $x = \pm\sqrt{5}$
            \item $x = \pm\sqrt{3}$ tai $\pm\sqrt{5}$
        \end{enumerate}
    \end{vastaus}
\end{tehtava}

\begin{tehtava}
    Ratkaise yhtälöt.
    \begin{enumerate}[a)]
        \item $x^8 - 1 = 0$
        \item $x^8 - x^4 = 0$
        \item $x^8 - x^4 - 1 = 0$
    \end{enumerate}
    \begin{vastaus}
        \begin{enumerate}[a)]
            \item $x = \pm\sqrt{1}$
            \item $x = 0$ tai $x = \pm\sqrt{1}$
            \item $x = \pm\sqrt[4]{\frac{1+\sqrt{5}}{2}} = \pm\sqrt[4]{\upvarphi}$ ($\upvarphi$ on kultaisena leikkauksena tunnettu vakio)
        \end{enumerate}
    \end{vastaus}
\end{tehtava}

\begin{tehtava}
    Ratkaise yhtälöt.
    \begin{enumerate}[a)]
        \item $x^4 - 16 = 0$
        \item $2x^4 = 8x^2$
        \item $x^6 - 2x^3 = 3$
        \item $x^{100} - 2x^{50} + 1 = 0$
    \end{enumerate}
    \begin{vastaus}
        \begin{enumerate}[a)]
            \item $x = \pm2$
            \item $x = 0$ tai $x=\pm2$
            \item $x = \sqrt[3]{3}$ tai $x= -1$
            \item $x = \pm1$
        \end{enumerate}
    \end{vastaus}
\end{tehtava}

\begin{tehtava}
    Ratkaise yhtälöt.
    \begin{enumerate}[a)]
        \item $x^4 + 7x^3 = 0$
        \item $2x^3 - 16x^2 + 32x = 0$
        \item $x^6 + 6x^5 = -9x^4$
        \item $x^3 - 2x^5 = 0$      
    \end{enumerate}
    \begin{vastaus}
        \begin{enumerate}[a)]
        	\item $x = 0$ tai $x = -7$
        	\item $x = 0$ tai $x = 4$
        	\item $x = 0$ tai $x = -3$
            \item $x = 0$ tai $x = \pm\dfrac{1}{\sqrt{2}}$
        \end{enumerate}
    \end{vastaus}
\end{tehtava}

\begin{tehtava}
    Ratkaise yhtälöt.
    \begin{enumerate}[a)]
        \item $x^5 - 2x^3 + x = 0$
        \item $x^8 + 4x^4 = 5x^6$       
    \end{enumerate}
    \begin{vastaus}
        \begin{enumerate}[a)]
        	\item $x = 0$ tai $x = \pm1$
        	\item $x = 0$ tai $x = \pm1$ tai $x = \pm2$
        \end{enumerate}
    \end{vastaus}
\end{tehtava}

\begin{tehtava}
	Ratkaise yhtälö $x^{627} - 6x^{514} + 5x^{401} = 0$.
	\begin{vastaus}
		$x = 0$, $x = 1$ tai $x = \sqrt[113]{5}$
	\end{vastaus}
\end{tehtava}

\begin{tehtava}
	(K02/T2b) Ratkaise yhtälö $e^{x^3+4x^2+1}=1$. [$e$ on matemaattinen vakio, irrationaaliluku, jonka likiarvo on $2,718$.]
	\begin{vastaus}
	$x=0$ tai $x=-2 + \sqrt[]{3}$ tai $x=-2 - \sqrt[]{3}$
	\end{vastaus}
\end{tehtava}

\begin{tehtava}
	$[ \star ]$ Ratkaise yhtälö $2^x-1=\frac{12}{2^x}$
	\begin{vastaus}
	$x=2$
	\end{vastaus}
\end{tehtava}
