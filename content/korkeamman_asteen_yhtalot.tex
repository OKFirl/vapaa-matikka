\chapter{Korkeamman asteen yhtälöt}

\section{Sijoituksen avulla ratkeavat yhtälöt}

Malliesimerkki sijoituksen avulla ratkeavasta yhtälöstä on ns. bikvadraattinen yhtälö.
Bikvadraattinen yhtälö on muotoa $ax^4+bx^2+c=0$ ja se ratkeaa sijoituksella $y=x^2$.
Vastaavasti sijoituksella $y=x^n$ voidaan ratkaista kaikki yhtälöt, jotka ovat muotoa $ax^{2n}+bx^n+c=0$.

\begin{esimerkki}{\ }
\begin{itemize}
    \item Ratkaistaan yhtälö $2x^4+14x^2-36=0$.
    \item Tehdään sijoitus $y=x^2$.
    \item Yhtälö saa muodon $2y^2+14y-36=0$.
    \item Sievennetään yhtälöä jakamalla puolittain kahdella: $y^2+7y-18=0$.
    \item Ratkaistaan toisen asteen yhtälön ratkaisukaavalla $y=2$ tai $y=-9$.
    \item Reaaliluvun neliö ei voi olla negatiivinen, joten hylätään jälkimmäinen juuri.
    \item Ratkaistaan ensimmäisestä juuresta $x$: $x^2=2 \Leftrightarrow  x=\pm\sqrt{2}$.
    \item Saadaan vastaukseksi kaksi reaalijuurta yhtälölle: $x=\pm\sqrt{2}$.
\end{itemize}
\end{esimerkki}

\section{Muuttuja yhteisenä tekijänä}

Joillakin yhtälöillä saattaa olla muuttuja tai jokin muuttujan potenssi yhtälön tekijänä. Esimerkkinä kolmannen asteen yhtälössä, joka on muotoa $ax^3+bx^2+cx=0$, muuttuja $x$ on yhtälön yksi tekijä. Kyseinen yhtälö voidaan esittää muodossa $x(ax^2+bx+c)=0$, jolloin sen ratkaisut ovat $x=0$ ja yhtälön $ax^2+bx+c=0$ ratkaisut. Vastaavasti muotoa $ax^{n+2}+bx^{n+1}+cx^n=0$ oleva yhtälö voidaan esittää muodossa $x^n(ax^2+bx+c)=0$ ja ratkaista samalla lailla.

\begin{esimerkki}{\ }
\begin{itemize}
    \item Ratkaistaan yhtälö $x^4-8x^3+16x^2=0$.
    \item Yhtälön yhteinen tekijä on $x^2$, joten muotoillaan se irti muista termeistä: $x^2(x^2-8x+16)=0$.
    \item Yksi ratkaisu on $x=0$, loput ratkaisut saadaan ratkaisemalla yhtälö $(x^2-8x+16)=0$.
    \item Ratkaisut ovat $x=0$ ja $x=4$.

\end{itemize}
\end{esimerkki}

\section{Harjoitustehtäviä}

\begin{tehtava}
    Ratkaise yhtälöt.
    \begin{enumerate}
        \item $x^4 - 16 = 0$
        \item $2x^4 = 8x^2$
        \item $x^6 - 2x^3 = -3$
        \item $x^{100} - 2x^{50} + 1 = 0$
    \end{enumerate}
    \begin{vastaus}
        \begin{enumerate}
            \item $x = \pm3$
            \item $x= 0 tai x=\pm2$
            \item $x = \sqrt[3]{3} tai x= -1$
            \item $x = \pm1$
        \end{enumerate}
    \end{vastaus}
\end{tehtava}

\begin{tehtava}
    Ratkaise yhtälöt.
    \begin{enumerate}
        \item $x^4 + 7x^3 = 0$
        \item $2x^3 - 16x^2 + 32x = 0$
        \item $x^6 + 6x^5 = -9x^4$
        \item $x^3 - 2x^5 = 0$      
    \end{enumerate}
    \begin{vastaus}
        \begin{enumerate}
        	\item $x = 0$ tai $x = -7$
        	\item $x = 0$ tai $x = 4$
        	\item $x = 0$ tai $x = 3$
            \item $x = 0$ tai $x = \pm\dfrac{1}{\sqrt{2}}$
        \end{enumerate}
    \end{vastaus}
\end{tehtava}

\begin{tehtava}
    Ratkaise yhtälöt.
    \begin{enumerate}
        \item $x^5 + 2x^3 + x = 0$
        \item $x^8 - 4x^4 = 5x^6$       
    \end{enumerate}
    \begin{vastaus}
        \begin{enumerate}
        	\item $x = 0$ tai $x = -1$
        	\item $x = 0$ tai $x = \pm1$ tai $x = \pm2$
        \end{enumerate}
    \end{vastaus}
\end{tehtava}