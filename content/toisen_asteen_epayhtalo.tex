\chapter{Toisen asteen epäyhtälö}
\begin{esimerkki}
\missingfigure{johdantoesimerkki esim. lämpötilatarkastelu tjsp}
\begin{itemize}
\item{Milloin lämpötila on suurempi kuin nolla?}
\item{Milloin lämpötila on pienempi kuin nolla?}
\item{Milloin lämpötilan merkki voi vaihtua?} 
\end{itemize}
\end{esimerkki}
\section{Toisen asteen epäyhtälön ratkaiseminen kuvaajan avulla.}
\begin{esimerkki}
Milloin toisen asteen polynomifunktion $f(x)=x^2-5$ arvot ovat suurempia kuin nolla?  \\
\textbf{Ratkaisu} \\
Funktion $f$ arvot ovat suurempia kuin nolla niillä muuttujan $x$ arvoilla, joilla pätee $x^2-5>0$. \\

\begin{itemize}
\item{Funktion arvo on positiivinen, kun funktion kuvaaja on x-akselin yläpuolella.}
\item{Funktion arvo on negatiivinen, kun funktion kuvaaja on x-akselin alapuolella.}
\end{itemize}
\begin{lukusuora}{-2.5}{2.5}{5}
\lukusuoraparaabeli{-1}{1}{-1}
\lukusuoraalanimi{0}{$-$}
\lukusuoranimi{-1.8}{$+$}
\lukusuoranimi{1.8}{$+$}
\end{lukusuora}

Ratkaistaan funktion $f$ nollakohdat, koska niissä kohdissa funktion arvojen merkki voi vaihtua.
\begin{align*}
f(x)&=0 \\
x^2-5&=0 \\
x^2&=5 \\
x&=\pm \sqrt[]{5}
\end{align*}
Funktion $f$ kuvaaja on ylöspäin aukeava paraabeli, koska kerroin $a>0$.
Funktio $f$ leikkaa x-akselin kohdissa $x=-\sqrt[]{5}$ ja $x=\sqrt[]{5}$.

\begin{lukusuora}{-2.5}{2.5}{5}
\lukusuoraparaabeli{-1}{1}{-1}
\lukusuoraalanimi{0}{$-$}
\lukusuoranimi{-1.8}{$+$}
\lukusuoranimi{1.8}{$+$}
\lukusuorapienipiste{-1}{\hspace{4mm}-\!$\sqrt{5}$}
\lukusuorapienipiste{1}{$\sqrt{5}$\hspace{3mm}}
\end{lukusuora}

Funktion arvot ovat positiivisia niillä muuttujan $x$ arvoilla, joilla funktion kuvaaja on $x$-akselin yläpuolella.
Kuvaajasta huomataan, että $f(x)>0$, kun $x<-\sqrt[]{5}$ tai $x>\sqrt[]{5}$. 
\end{esimerkki}

Toisen asteen polynomifunktion $f(x)=ax^2+bx+c$, $x \in \mathbb{R}$ arvot voivat vaihtaa merkkiään vain kyseisen polynomifunktion nollakohdissa. Jos haluamme tietää, milloin toisen asteen polynomifunktion arvot ovat positiivisia tai negatiivisia, niin
\begin{itemize}
\item{1.Ratkaistaan toisen asteen polynomifunktion $f$ nollakohdat eli ratkaistaan ne muuttujan $x$ arvot, joilla $ax^2+bx+c=0$. }
\item{2.Hahmotellaan toisen asteen polynomifunktion kuvaajaparaabelin aukeamissuunta ja merkitään kuvaajaan funktion nollakohdat.}
\item{3.Päätellään hahmotelmasta milloin funktion arvot ovat positiivisia tai negatiivisia.}
\end{itemize} 

\begin{esimerkki}
Milloin funktion $f$, jonka lauseke on $f(x)=5x^2+13x+3$, arvot ovat suurempia tai yhtäsuuria kuin nolla? \\
\textbf{Ratkaisu} \\
1. Ratkaistaan funktion nollakohdat.
\begin{align*}
f(x)&=0 \\
5x^2+13x+3&=0 \ \ \ \ \ \ \ &&||x=\frac{-b\pm \sqrt[]{b^2-4ac}}{2a} \\
x&=\frac{-13 \pm \sqrt[]{13^2-4 \cdot 5 \cdot 3}}{2 \cdot 5} \\
x&=\frac{-13 \pm \sqrt[]{169-60}}{10} \\
x&=\frac{-13 \pm \sqrt[]{109}}{10}  
\end{align*}
2. Hahmotellaan polynomifunktion kuvaaja. \\
Kuvaaja on ylöspäin aukeava paraabeli, koska $a=5>0$. 

\begin{lukusuora}{-2.5}{2.5}{5}
\lukusuoraparaabeli{-1}{1}{-1}
\lukusuoraalanimi{0}{$-$}
\lukusuoranimi{-1.8}{$+$}
\lukusuoranimi{1.8}{$+$}
\lukusuorapienipiste{-1}{}
\lukusuorapienipiste{1}{}
\end{lukusuora}

3. Kuvaajasta voidaan päätellä, että $f(x)>0$, kun $x<\frac{-13 - \sqrt[]{109}}{10}$ tai $x> \frac{-13 + \sqrt[]{109}}{10}$.  
\end{esimerkki} 
\begin{esimerkki} 
Milloin epäyhtälö $x^2+5x-6<0$ toteutuu? \\ 
\textbf{Ratkaisu} \\
%Toisen asteen polynomifunktion merkin tarkastelussa päädytään aina toisen asteen %epäyhtälöön. \\ \\
%\laatikko{Toisen asteen epäyhtälö on muotoa  % SIIS HÄH?
%\begin{itemize}
%\item{$ax^2+bx+c>0$}
%\item{$ax^2+bx+c \leq  0$}
%\item{$ax^2+bx+c < 0$}
%\item{$ax^2+bx+c \geq 0$}
%\end{itemize}}
Toisen asteen epäyhtälö ratkaistaan tutkimalla toisen asteen polynomifunktion merkkiä.  \\
1. Ratkaistaan polynomifunktion $f(x)=x^2+5x-6$ nollakohdat.
\begin{align*}
f(x)&=0 \\
x^2+5x-6&=0 \ \  \ \ \ &&||x=\frac{-b \pm \sqrt[]{b^2-4ac}}{2a} \\ 
x&=\frac{-5 \pm \sqrt[]{(5)^2-4 \cdot 1 \cdot(-6)}}{2 \cdot 1} \\
x&=\frac{-5 \pm \sqrt[]{25+24}}{2} \\
x&=\frac{-5 \pm \sqrt[]{49}}{2} \\
x&=\frac{-5 \pm 7}{2} \\
x&=-6 \text{ tai } x=1
\end{align*}
2. Hahmotellaan polynomifunktion kuvaaja. Se aukeaa ylöspäin, koska toisen
asteen termin $x^2$ kerroin on positiivinen. \\ 

\begin{lukusuora}{-2.5}{2.5}{5}
\lukusuoraparaabeli{-1}{1}{-1}
\lukusuoraalanimi{0}{$-$}
\lukusuoranimi{-1.8}{$+$}
\lukusuoranimi{1.8}{$+$}
\lukusuorapienipiste{-1}{\hspace{3mm}-6}
\lukusuorapienipiste{1}{1\hspace{1mm}}
\end{lukusuora}

3.  Kuvaajasta voidaan päätellä, että $f(x)<0$, kun $-6 < x < 1$.  \\
Siis alkuperäinen epäyhtälö toteutuu, kun $-6 < x <1$.  
\end{esimerkki}
\begin{esimerkki}
Raaka-Arska rakentaa suorakulmion muotoista kaniaitausta lemmikkikanilleen, joka rajoittuu yhdeltä sivulta Arskan taloon. Hänellä on yhteensä 14 metriä kaniverkkoa. Miten hänen kannattaa valita aitauksensa mitat, jotta kaniaitauksen koko on vähintään 12 neliömetriä. \\
\textbf{Ratkaisu}\\
Olkoon aitauksen yhdensuuntaiset sivut pituudeltaan $x$. Olkoon talon suuntaisen sivun pituus $y$. Koska tiedämme, että $x+x+y=14$, niin tästä saadaan ratkaistua talon suuntaisen sivun pituudeksi $y=14 - 2x$. \\ \\
Sivujen pituuksille pitää päteä
\begin{align*}
x&>0 \ \ \ \ \ \ \text{ ja } \ \ \ \ \ y>0 \\
x&>0 \ \ \ \ \ \ \text{ ja } \ \ \ \ \ 14-2x>0 \\ 
x&>0 \ \ \ \ \ \ \text{ ja } \ \ \ \ \ 14>2x \\ 
x&>0 \ \ \ \ \ \ \text{ ja } \ \ \ \ \ 7>x \\
\end{align*} 
Aitauksen pinta-alaa kuvaa funktio $A(x)=x(14-2x)=14x-2x^2=-2x^2-14x$, $x \in ]0,7[.$. \\
Halutaan, että $A(x)>12$, josta saadaan epäyhtälö $-2x^2+14x>12$. \\
1. Muokataan alkuperäistä epäyhtälöä siten, että saadaan kaikki termit vasemmalle puolelle epäyhtälöä.
\begin{align*}
-2x^2+14x&>12 \\
-2x^2+14x-12&>0
\end{align*}
Saimme alkuperäisen epäyhtälön kanssa yhtäpitävän epäyhtälön, jota lähdemme ratkaisemaan. \\
2. Ratkaistaan millä muuttujan $x$ arvolla polynomi $-2x^2+14x-12$ saa arvon nolla.
\begin{align*}
-2x^2+14x-12&=0 \\
x&=\frac{-14 \pm \sqrt[]{14^2-4 \cdot (-2) \cdot (-12)}}{2 \cdot (-2)} \\
x&=\frac{-14 \pm \sqrt[]{196-96}}{-4} \\
x&=\frac{-14 \pm 10}{-4} \\
x&=\frac{-24}{-4} \text{ tai } x=\frac{-4}{-4} \\
x&=6 \text{ tai } x=1
\end{align*}

\begin{lukusuora}{-2.5}{2.5}{5}
\lukusuoraparaabeli{-1}{1}{1}
\lukusuoranimi{0}{$+$}
\lukusuoraalanimi{-1.8}{$-$}
\lukusuoraalanimi{1.8}{$-$}
\lukusuorapienipiste{-1}{1\hspace{1mm}}
\lukusuorapienipiste{1}{\hspace{1mm}6}
\end{lukusuora}

Hahmotellaan paraabelin $y=-2x^2+14x-12$ kuvaaja. Huomataan, että paraabeli kulkee x-akselin yläpuolella, kun $1<x<6$. 
Siis epäyhtälö $-2x^2+14x-12$ toteutuu, kun $1<x<6$. Tämä tarkoittaa, että $A(x)>12$ kun $1<x<6$.

Jos aitauksen yhdensuuntaisten sivujen pituus on väliltä $1<x<6$ ja $y=14-2x$, niin aitauksen pinta-ala on vähintään 12 neliömetriä.
\end{esimerkki}
\section{Toisen asteen epäyhtälön ratkaiseminen tekijöihin jakamisen avulla}
\begin{esimerkki} 
Ratkaise epäyhtälö $x^2+10x>0$. \\
\textbf{Ratkaisu} \\
1. Ratkaistaan millä muuttujan x arvolla toisen asteen polynomi $x^2+10x$ saa arvon nolla. 
\begin{align*}
x^2+10x&=0 \ \ \ \ &&||\text{tulon nollasääntö} \\
x&=0 \text{ tai } x+10=0 \\
x&=0 \text{ tai } x=-10
\end{align*}
Koska kahden positiivisen tai negatiivisen luvun tulo on positiivinen, niin
polynomi $x^2+10x=x(x+10)$ saa positiivisen arvon, kun 
\begin{itemize}
\item{$x>0$ ja $x+10>0$} \\tai \\
\item{$x<0$ ja $x+10<0$} 
\end{itemize}
Koska negatiivisen ja positiivisen luvun tulo on negatiivinen, niin
polynomi $x^2+10x=x(x+10)$ saa negatiivisen arvon, kun
\begin{itemize}
\item{$x>0$ ja $x+10<0$} \\ tai \\
\item{$x<0$ ja $x+10>0$} \\
\end{itemize}
\missingfigure{kaavio, jossa käydään tulon merkki}
Tästä huomataan, että epäyhtälö $x^2+10x>0$ on tosi, kun $x>0$ tai $x<-10$. 
\end{esimerkki} 

Toinen tapa ratkaista toisen asteen epäyhtälö on 
\begin{itemize}
\item{jakaa polynomi tekijöihinsä }
\item{tutkia millä muuttujan $x$ arvoilla kahden lausekkeen tulo on positiivinen ja milloin negatiivinen}
\end{itemize}

\begin{esimerkki}
Ratkaise epäyhtälö $x^2-6x+9 \leq 0$. \\
\textbf{Ratkaisu} \\
Ratkaistaan millä muuttujan $x$ arvolla polynomi saa arvon nolla.
\begin{align*}
x^2-6x+9&=0 \\
x&=\frac{-(-6) \pm \sqrt[]{(-6)^2-4 \cdot 1 \cdot 9}}{2 \cdot 1} \\
x&=-3
\end{align*}
Yhtälöllä $x^2-6x+9=0$ on kaksoisjuuri $x=-3$. 
Polynomi saadaan siis muotoon $x^2-6x+9=(x-3)^2$. \\
\missingfigure{merkki $(x-3)$ ja $(x-3)$ ja tulo} \\
Kaaviosta saadaan pääteltyä, että $x^2-6x+9=(x-3)^2$ on positiivinen kun $x \neq -3$. Siis epäyhtälö $x^2-6x+9 \leq 0$ toteutuu, kun $x=-3$ jolloin polynomi saa arvon nolla. 
\end{esimerkki}
\begin{esimerkki}
Ratkaise epäyhtälö $2x^2+2x-12<0$. 
\end{esimerkki}
1. Ratkaistaan millä muuttujan $x$ arvolla polynomi $2x^2+2x-12$ saa arvon nolla.
\begin{align*}
2x^2+2x-12&=0 \\
x&=\frac{-2 \pm \sqrt[]{2^2-4 \cdot 2 \cdot (-12)}}{2 \cdot 2} \\
x&=\frac{-2 \pm \sqrt[]{4+96}}{4} \\
x&=\frac{-2 \pm 10}{4} \\
x&=-3 \text{ tai } x = 2
\end{align*}
2. Polynomi saadaan muotoon $2x^2+2x-12=2(x-2)(x+3)$.  \\
3. Piirretään merkkikaavio:
\begin{center}
\begin{merkkikaavio}{2}
\merkkikaavioKohta{$-3$}
\merkkikaavioKohta{$2$}


	\merkkikaavioFunktio{$2(x-2)$}
	\merkkikaavioMerkki{$-$}
	\merkkikaavioMerkki{$-$}
	\merkkikaavioMerkki{$+$}
\merkkikaavioUusirivi
	\merkkikaavioFunktio{$x+3$}
	\merkkikaavioMerkki{$-$}
	\merkkikaavioMerkki{$+$}
	\merkkikaavioMerkki{$+$}
\merkkikaavioUusiriviKaksoisviiva
	\merkkikaavioFunktio{$2(x-2)(x+3)$}
	\merkkikaavioMerkki{$+$}
	\merkkikaavioMerkki{$-$}
	\merkkikaavioMerkki{$+$}
\end{merkkikaavio}
\end{center}

4. Huomataan, että polynomi $2(x-2)(x+3)$ on negatiivinen, kun $-3<x<2$. 
\section{Harjoitustehtäviä}

\begin{tehtava}
    Ratkaise seuraavat epäyhtälöt.
    \begin{enumerate}[(a)]
        \item $-x^2+5x+8>0$
        \item $-2x^2+6x+9>0$
        \item $4x^2-8x-72<0$
        \item $5x^2+10x-73<0$
    \end{enumerate}
    \begin{vastaus}
        \begin{enumerate}[(a)]
            \item $\frac{1}{2} (5+\sqrt{57}) > x > \frac{1}{2} (5-\sqrt{57})$
            \item $\frac{3}{2} (1+\sqrt{3}) > x > \frac{3}{2} (1-\sqrt{3})$
            \item $1+\sqrt{19} > x > 1-\sqrt{19}$
            \item $-1+\frac{1}{5} \sqrt{390} > x > -1-\frac{1}{5} \sqrt{390}$
        \end{enumerate}
    \end{vastaus}
\end{tehtava}

\begin{tehtava}
    Ratkaise seuraavat epäyhtälöt.
    \begin{enumerate}[(a)]
        \item $4x^2+13x\geq 0$
        \item $4x^2-776\geq 0$
        \item $7x^2-8x\leq 0$
        \item $11x^2-12\leq 0$
    \end{enumerate}
    \begin{vastaus}
        \begin{enumerate}[(a)]
            \item $x \geq 0$ tai $x \leq -3,25$
            \item $x \geq \sqrt{194}$ tai $x \leq -\sqrt{194}$
            \item $0,875 \geq x \geq 0$
            \item $2\sqrt{\frac{3}{11}} \geq x \geq -2\sqrt{\frac{3}{11}}$
        \end{enumerate}
    \end{vastaus}
\end{tehtava}

\begin{tehtava}
    Ratkaise epäyhtälö $ax^2+(a+1)x+1 > 0$ vapaan parametrin $a$ funktiona.
    \begin{vastaus}
        $a < 0$: $-\frac{1}{a} > x > -1$ \\ $a = 0$: $x > -1$ \\ $1 > a > 0$: $x > -1$ tai $x < -\frac{1}{a}$ \\ $a = 1$: $x \neq -1$ \\ $a > 1$: $x \in \mathbb{R}$
    \end{vastaus}
\end{tehtava}

\begin{tehtava}
    Ratkaise epäyhtälö $a^2x^2+ax+1 \geq 0$ vapaan parametrin $a$ funktiona.
    \begin{vastaus}
        $a < 0$: $x \geq -\frac{2}{a}$ tai $x \leq \frac{1}{a}$ \\ $a = 0$: $x \in \mathbb{R} $ \\ $a > 0$: $x \geq \frac{1}{a}$ tai $x \leq -\frac{2}{a}$
    \end{vastaus}
\end{tehtava}

% $ax^2+x+5 < 0$
% $x^2+x+a \leq 0$

\begin{tehtava}
(K93/T3b) Autoilijan työmatkan kesto $t$ riippuu liikennevirrasta $m$ kaavan 
        $t=0,01m^2+0,03m+18$ mukaisesti, missä $t$ on ajoaika minuutteina ja $m$ liikenteen mittauspisteen minuutissa ohittavien autojen määrä. Kuinka suuri saa liikennevirta enintään olla, jotta autoilijan työmatka kestäisi enintään puoli tuntia?
\begin{vastaus}
        33 $autoa/minuutti$
    \end{vastaus}
\end{tehtava}
