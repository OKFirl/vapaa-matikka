%% encoding: utf-8
\chapter{Ensimmäisen asteen yhtälö}

Yhtälöistä yksinkertaisin on ensimmäisen asteen yhtälö. Käydään se lyhyesti
vielä läpi kertauksen vuoksi.

Ensimmäisen asteen yhtälö on siis yhtälö, jossa vakiotermin \emph{C} lisäksi
oleva muuttuja on kerrottu jollain vakiolla \emph{a}, eikä sitä ole korotettu
mihinkään potenssiin. Yleisesti ensimmäisen asteen yhtälö on siten muotoa

\begin{align*}
    ax + C = 0
\end{align*}

Mikä vain 1. asteen yhtälö on muokattavissa aina
annettuun muotoon laskemalla kaikki vakiotermit yhteen,
ja siirtämällä ne vasemmalle puolelle. Esim. yhtälö

\begin{align*}
    4x + 5 = 2 \Leftrightarrow 4x + 3 = 0
\end{align*}

Ylempänä oleva ns. yleisä muotoa oleva yhtälö voidaan ratkaista helposti siirtämällä
ensiksi vakiotermi yhtälön oikealle puolelle, ja jakamalla sitten molemmat puolet vakiotermillä:

\begin{align*}
    x = \frac{-C}{a}
\end{align*}

Erityisesti kannattaa huomata, että kaikkia 1. asteen yhtälöitä ei toki
kannata saattaa aina yleiseen muotoon. Jos vakiotermit ovat valmiiksi oikealla
puolella, ratkaisu on luonnollisesti helpompaa vain jakamalla molemmat puolet
muuttujan kertoimella.

\section{Harjoitustehtäviä}

\begin{tehtava}
  Ratkaise yhtälö
  \begin{enumerate}[a)]
    \item $x + 5 = 47$
    \item $2x = 64$
    \item $3x - 5 = 16$
  \end{enumerate}

  \begin{vastaus}
    \begin{enumerate}[a)]
      \item $x = 42$
      \item $x = 32$
      \item $x = 7$
    \end{enumerate}
  \end{vastaus}
\end{tehtava}

\begin{tehtava}
  Muodosta yhtälö seuraaville lauseille:
  \begin{enumerate}[a)]
    \item Hilavitkuttimen vuokraus maksaa 42 EUR / tunti. Vuokraan tulee ottaa myös pakollinen laitteistovakuutus, jonka hinta on 25 EUR.
    \item Määrä euroina, kun 1 EUR vastaa 1.26 USD ja halutaan vaihtaa dollareita euroiksi. Valuutanvaihtaja veloittaa lisäksi palvelumaksun 50 senttiä.
  \end{enumerate}

  \begin{vastaus}
    \begin{enumerate}[a)]
      \item $42t + 25$
      \item $\frac{1}{1.26}c + 0.5$
    \end{enumerate}
  \end{vastaus}
\end{tehtava}
