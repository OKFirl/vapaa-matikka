%% encoding: utf-8
\chapter{Kertausta: Ensimmäisen asteen yhtälö}

Yhtälöistä yksinkertaisin on ensimmäisen asteen yhtälö. Käydään se lyhyesti
läpi kertauksen vuoksi.

Ensimmäisen asteen yhtälössä ratkaistavana on vain mahdollisesti vakiolla
kerrottu muuttuja, joka voidaan ratkaista vain käyttämällä korkeintaan
kahta neljästä aritmeettisesta peruslaskutoimitusta. Eli yleensä
yhtälön molempia puolia lisätään tai vähennetään jollain luvulla, jolla
vasemmalle puolelle saadaan jäämään vain muuttuja jollain vakiolla kerrottuna.
Sitten vain jaetaan molemmat puolet muuttujan kertoimella, jolloin saadaan
suoraan ratkaisu.

\todo{tyylikorjausta, termiä juuri ei vielä esitellä}

Yleisesti ensimmäisen asteen yhtälö on siten muotoa

\begin{align*}
    ax + b = 0.
\end{align*}

Mikä vain 1. asteen yhtälö on muokattavissa aina annettuun
muotoon siirtämällä kaiken vasemmalle puolelle ja
soveltamalla polynomien yhteen- ja vähennyslaskusääntöjä.

\subsection*{Esimerkkejä}

\begin{esimerkki}
Ratkaise yhtälö $3x - 6 = 0$ \\
\textbf{Ratkaisu} \\
  \begin{align*}
    3x - 6 &= 0 & \ppalkki & +6 \\
        3x &= 6 & \ppalkki & :3 \\
         x &= \frac{6}{3} \\
           &= 2
  \end{align*}
\end{esimerkki}

\begin{esimerkki}
Ratkaise yhtälö $4x + 5 = 2$ \\
\textbf{Ratkaisu}
\begin{align*}
    4x + 5 &= 2  & \ppalkki & -5 \\
        4x &= -3 & \ppalkki & :4 \\
         x &= \frac{-3}{4}
 \end{align*}
\end{esimerkki}

Yleisen yhtälön ratkaisu on siten muotoa

\begin{align*}
  x = \frac{-b}{a}.
\end{align*}

Erityisesti kannattaa huomata, että kaikkia 1. asteen yhtälöitä ei toki
kannata saattaa aina yleiseen muotoon. Jos vakiotermit ovat valmiiksi oikealla
puolella, ratkaisu on luonnollisesti helpompaa vain suorittamalla
molemmille puolille jakolasku muuttujan kertoimella.

\Harjoitustehtavat

\subsubsection*{Opi perusteet}

\begin{tehtava}
    Ratkaise yhtälöt.
    \begin{enumerate}[(a)]
        \item $x + 5 = 47$
        \item $2x = 64$
        \item $3x - 5 = 16$
    \end{enumerate}
    \begin{vastaus}
        \begin{enumerate}[(a)]
            \item $x = 42$
            \item $x = 32$
            \item $x = 7$
        \end{enumerate}
    \end{vastaus}
\end{tehtava}

\begin{tehtava}
    Ratkaise yhtälöt.
    \begin{enumerate}[(a)]
        \item $x + 8 = 2x - 1$
        \item $2x + 4 = 60$
        \item $3x - 5 = -x + 11$
    \end{enumerate}
    \begin{vastaus}
        \begin{enumerate}[(a)]
            \item $x = 9$
            \item $x = 28$
            \item $x = 4$
        \end{enumerate}
    \end{vastaus}
\end{tehtava}

%ei ole yhtälö tehtävä, mutta mallinnusharjoituksena ok? 
\begin{tehtava}
    Muodosta tilannetta kuvaavat lausekkeet.
    \begin{enumerate}[(a)]
        \item Kuinka paljon maksaa hilavitkuttimen vuokraus $x$:ksi tunniksi, kun vuokra on 42 \euro /tunti. Vuokrajan tulee ottaa myös pakollinen 25 euron laitteistovakuutus.
        \item Kuinka monta euroa saa $x$:llä dollarilla, kun 1~EUR vastaa 1,23~USD ja halutaan vaihtaa dollareita euroiksi. Valuutanvaihtaja veloittaa lisäksi palvelumaksun 0,50 euroa.
    \end{enumerate}
    \begin{vastaus}
        \begin{enumerate}[(a)]
            \item $42x + 25$
            \item $\frac{1}{1{,}23}x + 0{,}5$
        \end{enumerate}
    \end{vastaus}
\end{tehtava}



\subsubsection*{Hallitse kokonaisuus}

\begin{tehtava}
    Ratkaise yhtälöt.
    \begin{enumerate}[(a)]
        \item $3(x+7)=7x$
        \item $2(3x-1)=-7x $
        \item $3-2x-(4-x)=2 $
    \end{enumerate}
    \begin{vastaus}
        \begin{enumerate}[(a)]
            \item $x = \frac{7}{6} =1\frac{1}{6} $
            \item $x = \frac{2}{13}$
            \item $x = -3$
        \end{enumerate}
    \end{vastaus}
\end{tehtava}

\begin{tehtava}
    Ratkaise yhtälöt.
    \begin{enumerate}[(a)]
        \item $-2\cdot\frac{x-5}{3}-\frac{5}{7}(1-x)=5x+3$
        \item $\frac{4x-5}{3}-\frac{3}{2}(x-8)=-\frac{x+5}{6}$
        \item $3(x-3)+x=4x-9$
    \end{enumerate}
    \begin{vastaus}
        \begin{enumerate}[(a)]
            \item $x = -\frac{1}{13}$
            \item ei ratkaisuja
            \item yhtälö on toteutuu kaikilla reaaliluvuilla
        \end{enumerate}
    \end{vastaus}
\end{tehtava}

\begin{tehtava}
 
    \begin{vastaus}
		
    \end{vastaus}
\end{tehtava}

%vaatii Pythagoraan lauseen, jota ei vielä ole käsitelty lukiossa.
\begin{tehtava}
    Tässä tehtävässä pitäisi muistaa peruskoulussa käsitelty Pythagoraan lause.
    Suorakulmaisen kolmion sivujen pituuden kateetittien pituudet ovat $x+1$ ja $4$. Hypotenuusan pituus $x+3$. Mikä $x$ on?
    \begin{vastaus}
		$x=2$
    \end{vastaus}
\end{tehtava}

%hankala
\begin{tehtava}
    Määritä sekunnin tarkkuudella se ajanhetki, kun kellotaulun minuutti- ja tuntiviisarit ovat päällekkäin ensimmäisen kerran klo 12.00 jälkeen.
    \begin{vastaus}
		$13.05.27$
    \end{vastaus}
\end{tehtava}
