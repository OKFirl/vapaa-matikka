%% encoding: utf-8
\chapter{Kertausta: Ensimmäisen asteen yhtälö}

Yhtälöistä yksinkertaisin on ensimmäisen asteen yhtälö. Käydään se lyhyesti
vielä läpi kertauksen vuoksi.

Ensimmäisen asteen yhtälössä ratkaistavana on vain mahdollisesti vakiolla
kerrottu muuttuja, joka voidaan ratkaista vain käyttämällä korkeintaan
kahta neljästä aritmeettisesta peruslaskutoimitusta. Eli yleensä
yhtälön molempia puolia lisätään tai vähennetään jollain luvulla, jolla
vasemmalle puolelle saadaan jäämään vain muuttuja jollain vakiolla kerrottuna.
Sitten vain jaetaan molemmat puolet muuttujan kertoimella, jolloin saadaan
suoraan ratkaisu.

Yleisesti ensimmäisen asteen yhtälö on siten muotoa

\begin{align*}
    ax + b = 0
\end{align*}

Mikä vain 1. asteen yhtälö on muokattavissa aina
annettuun muotoon laskemalla kaikki vakiotermit yhteen,
ja siirtämällä ne vasemmalle puolelle.

\subsection*{Esimerkkejä}

\begin{esimerkki}
Ratkaise epäyhtälö $3x - 6 = 0$ \\
\textbf{Ratkaisu} \\
    $3x - 6 = 0$   \quad || +6 \\
    $3x = 6$       \quad || :3 \\
     $x = \frac{6}{3} = 2$
\end{esimerkki}

\begin{esimerkki}
Ratkaise epäyhtälö $4x + 5 = 2$ \\
\textbf{Ratkaisu} \\
    4x + 5 = 2     \quad || -5 \\
    4x = -3        \quad || :4 \\
     x = $\frac{-3}{4}$
\end{esimerkki}

Yleisen yhtälön ratkaisu on siten muotoa

\begin{align*}
  x = \frac{-b}{a}
\end{align*}

Erityisesti kannattaa huomata, että kaikkia 1. asteen yhtälöitä ei toki
kannata saattaa aina yleiseen muotoon. Jos vakiotermit ovat valmiiksi oikealla
puolella, ratkaisu on luonnollisesti helpompaa vain suorittamalla
molemmille puolille jakolasku muuttujan kertoimella.

\section{Harjoitustehtäviä}

\begin{tehtava}
  Ratkaise yhtälö
  \begin{enumerate}[a)]
    \item $x + 5 = 47$
    \item $2x = 64$
    \item $3x - 5 = 16$
  \end{enumerate}

  \begin{vastaus}
    \begin{enumerate}[a)]
      \item $x = 42$
      \item $x = 32$
      \item $x = 7$
    \end{enumerate}
  \end{vastaus}
\end{tehtava}

\begin{tehtava}
  Muodosta yhtälö seuraaville lauseille:
  \begin{enumerate}[a)]
    \item Hilavitkuttimen vuokraus maksaa 42 EUR / tunti. Vuokraan tulee ottaa myös pakollinen laitteistovakuutus, jonka hinta on 25 EUR.
    \item Määrä euroina, kun 1 EUR vastaa 1.26 USD ja halutaan vaihtaa dollareita euroiksi. Valuutanvaihtaja veloittaa lisäksi palvelumaksun 50 senttiä.
  \end{enumerate}

  \begin{vastaus}
    \begin{enumerate}[a)]
      \item $42t + 25$
      \item $\frac{1}{1.26}u + 0.5$, missä u = summa dollareina
    \end{enumerate}
  \end{vastaus}
\end{tehtava}
