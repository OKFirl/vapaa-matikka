\chapter{Diskriminantti}

\begin{esimerkki}
    Ratkaise toisen asteen yhtälö $3x^2-5x+10=0$.
    \begin{align*}
        \underbrace{3}_{=a}x^2\underbrace{-5}_{=b}x+\underbrace{10}_{=c}=0
    \end{align*}
    Sijoitetaan vakiot $a=3$, $b=-5$ ja $c=10$ toisen asteen yhtälön ratkaisukaavaan $x=\frac{-b \pm \sqrt[]{b^2-4ac}}{2a}$.
    \begin{align*}
        x=\frac{-(-5) \pm \sqrt[]{(-5)^2-4\cdot 3 \cdot 10}}{2 \cdot 3} \\
        x=-\frac{5 \pm \sqrt[]{25-120}}{6}
    \end{align*}
    Koska juurrettava on negatiivinen
    \begin{align*}
        b^2-4ac=(-5)^2-4 \cdot 3 \cdot 10=25-120=-95<0
    \end{align*}
    niin toisen asteen yhtälöllä ei ole ratkaisuja.
\end{esimerkki}

%Marginaaliin tai kuvaksi 2. asteen yhtälön ratkaisukaava (tai tekstin sekaan)

Toisen asteen yhtälön ratkaisukaavassa esiintyy neliöjuuri. Tämän neliöjuuren sisällä oleva lauseke $b^2-4ac$ määrää, kuinka monta ratkaisua yhtälöllä on. Joskus riittää pelkkä tieto ratkaisujen olemassaolosta tai lukumäärästä. Tälläisissä tapauksissa ei tarvitse ratkaista yhtälöä, vaan pelkkä edellä mainitun lausekkeen tarkastelu riittää. Tästä lausekkeesta käytetään nimeä \emph{diskriminantti} ja sitä merkitään kirjaimella $D$.

\laatikko{Toisen asteen yhtälön $ax^2+bx+c=0$ ratkaisujen lukumäärän näkee diskriminantin, $D=b^2-4ac$ avulla seuraavasti.
\begin{itemize}
\item
Jos $D<0$, yhtälöllä ei ole ratkaisuja.
\item
Jos $D=0$, yhtälöllä on tasan yksi ratkaisu.
\item
Jos $D>0$, yhtälöllä on kaksi ratkaisua.
\end{itemize}
}

D < 0, ei ratkaisuja
\begin{kuvaajapohja}{1}{-1}{3}{-1}{3}
  \kuvaaja{2*x**2-2*x+1}{$f(x)=2x^2-2x+1$}{blue}
\end{kuvaajapohja}

D = 0, 1 ratkaisu
\begin{kuvaajapohja}{1}{-1}{3}{-1}{3}
  \kuvaaja{x**2-2*x+1}{$f(x)=x^2-2x+1$}{blue}
\end{kuvaajapohja}

D > 0, 2 ratkaisua
\begin{kuvaajapohja}{1}{-1}{3}{-2}{2}
  \kuvaaja{2*x**2-4*x+1}{$f(x)=2x^2-4x+1$}{blue}
\end{kuvaajapohja}

\todo[noline]{Esimerkeille voisi laittaa oman makron, jos ne ovat aina samalla otsikkotasolla?}
\subsection*{Esimerkkejä}

\begin{esimerkki}
Selvitä onko yhtälöllä $x^2+x+2=0$ ratkaisuja.

Tutkitaan diskriminanttia.
\[D=1^2-4\cdot 1 \cdot 2 = 1-8 = -7\]
Koska $D<0$, yhtälöllä ei ole ratkaisuja.

Jos yhtälön ratkaisua yrittäisi ratkaisukaavan avulla, tulisi neliöjuuren alle negatiivinen luku.
\end{esimerkki}

\begin{esimerkki}
Millä $a$:n arvolla yhtälöllä $9x^2+ax+1$ on tasan yksi ratkaisu?

Jotta ratkaisuja olisi tasan yksi, on diskriminantin oltava 0.
\begin{align*}
D &=0\\
a^2-4\cdot 9\cdot 1 &= 0\\
a^2-36&=0\\
a^2&=36\\
a=\pm6
\end{align*}
Yhtälöllä on täsmälleen yksi ratkaisu, jos $a=-6$ tai $a=6$.
\end{esimerkki}

\section{Harjoitustehtäviä}

\begin{tehtava}
	Kuinka monta ratkaisua yhtälöllä $5x^2+4x-10$ on?
	\begin{vastaus}
		$D=4^2-4\cdot 5 \cdot (-10)>0$, joten kaksi.
	\end{vastaus}
\end{tehtava}

\begin{tehtava}
	Kuinka monta reaalijuurta yhtälöllä $9x^2+12x-4$ on?
	\begin{vastaus}
		$D=12^2-4 \cdot 9 \cdot (-4) >0$, joten kaksi.
	\end{vastaus}
\end{tehtava}

\begin{tehtava}
	Millä vakion $k$ arvoilla yhtälöllä $-x^2-x-k$ ei ole ratkaisua?
	\begin{vastaus}
		Pitää olla $D=(-1)^2-4 \cdot (-1) \cdot (-k)<0$. Siis $k>\frac{1}{4}$.
	\end{vastaus}
\end{tehtava}

\begin{tehtava}
	Millä vakion $t$ arvoilla yhtälöllä $6x^2+2tx+2t=0$ on kaksoisjuuri?
	\begin{vastaus}
		Jos kaksoisjuuri, niin pitää päteä $D=4t^2-48t>0$. Joka toteutuu, kun $-12 < t < 0$.
	\end{vastaus}
\end{tehtava}

\begin{tehtava}
	Osoita, että diskriminantti on $0$ jos ja vain jos yhtälö voidaan esittää muodossa $(gx+h)^2=0$, missä $g$ ja $h$ ovat reaalilukuja.
	\begin{vastaus}
		Suunta "$\Rightarrow$": $(gx+h)^2=0 \Leftrightarrow g^2x^2+2ghx+h^2=0 \Rightarrow D=(2gh)^2-4g^2h^2=4g^2h^2-4g^2h^2=0$ \\
		Suunta "$\Leftarrow$": $D=0 \Leftrightarrow b^2-4ac=0 \Leftrightarrow b^2=4ac \Leftrightarrow c=\frac{b^2}{4a} \Rightarrow ax^2+bx+\frac{b^2}{4a}=0 \Leftrightarrow 4a^2x^2+4abx+b^2=0 \Leftrightarrow (2ax+b)^2=0$
	\end{vastaus}
\end{tehtava}
