\chapter{Diskriminantti}

%
%Marginaaliin tai kuvaksi 2. asteen yhtälön ratkaisukaava (tai tekstin sekaan)
%
Toisen asteen yhtälön ratkaisukaavassa esiintyy neliöjuuri. Tämän neliöjuuren sisällä oleva lauseke $b^2-4ac$ määrää, kuinka monta ratkaisua yhtälöllä on. Joskus riittää pelkkä tieto ratkaisujen olemassaolosta tai lukumäärästä. Tälläisissä tapauksissa siis ei tarvitse ratkaista yhtälöä, vaan pelkkä edellä mainitun lausekkeen tarkastelu riittää. Lausekkeesta käytetään nimeä \emph{diskriminantti} ja sitä merkitään kirjaimella $D$.

\laatikko{Toisen asteen yhtälön $ax^2+bx+c=0$ ratkaisujen lukumäärän näkee diskriminantin, $D=b^2-4ac$ avulla seuraavasti.
\begin{itemize}
\item
Jos $D<0$, ei ole ratkaisuja.
\item
Jos $D=0$, on tasan yksi ratkaisu.
\item
Jos $D>0$, on 2 ratkaisua.
\end{itemize}
}
\missingfigure{kuvaajat eri tapauksista}
\begin{esimerkki}
Selvitä onko yhtälöllä $x^2+x+2=0$ ratkaisuja.

Tutkitaan diskriminanttia.
\[D=1^2-4\cdot 1 \cdot 2 = 1-8 = -7\]
Koska $D<0$, yhtälöllä ei ole ratkaisuja.

Jos yhtälön ratkaisua yrittäisi ratkaisukaavan avulla, tulisi neliöjuuren alle negatiivinen luku.
\end{esimerkki}

\begin{esimerkki}
Millä $a$:n arvolla yhtälöllä $9x^2+ax+1$ on tasan yksi ratkaisu.

Jotta ratkaisuja olisi tasan yksi, on diskriminantin oltava 0.
\begin{align*}
D &=0\\
a^2-4\cdot 9\cdot 1 &= 0\\
a^2-36&=0\\
a^2&=36\\
a=\pm6
\end{align*}
Yhtälöllä on täsmälleen yksi ratkaisu, jos $a=-6$ tai $a=6$.
\end{esimerkki}



\section{Harjoitustehtäviä}
