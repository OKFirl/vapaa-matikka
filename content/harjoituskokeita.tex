% keskeneräistä

\subsection*{Harjoituskoe 1}

\begin{enumerate}
\item Ratkaise epäyhtälöt.\\ a) $1-\dfrac{1-x}{6}<x$\\ b) $(x+1)(x^2-2x-1)\geq0$
\item Ratkaise yhtälöt.\\ a) $4x^2-1=0$\\ b) $x^3=-3x$\\ c) $2y^2=y-8$
\item Millä parametrin $k$:n arvoilla yhtälöllä $kx^2-(k+1)x+1=0$ on kaksi erisuurta reaalijuurta? 

\end{enumerate}

\subsection*{Harjoituskoe 2}

\begin{enumerate}
\item Ratkaise yhtälöt.\\ a) $x^2-5x=0$\\ b) $x^4-1=0$\\ c) $(x-1)(x+4) = x(x-5)$
\item Ratkaise epäyhtälöt.\\ a) $x^2-8\geq0$\\ b) $x^2-8\geq(x-3)^2$\\ c) $x^2-6x+9\leq0$
\item Tiedämme, että $(a+b)^2=12$ ja $(a-b)^2=4$. Ratkaise tulo $ab$.
\item Kolmannen asteen polynomifunktiolle pätee $P(-1)=0$, $P(0)=0$ ja $P(1)=0$. Lisäksi $P(3)=3$. Määritä polynomi $P$.
\item Millä vakion $t$ arvoilla yhtälöllä $tx^2+tx-6=0$ ei ole ratkaisuja?
\end{enumerate}


\subsection*{Harjoituskoe 3}


\subsection*{Harjoituskoe 4}
