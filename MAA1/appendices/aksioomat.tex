\chapter{Syventävää sisältöä: reaalilukujen aksioomat}
\label{aksioomat}
Reaaliluvut ovat kunta, eräs algebrallinen rakenne. Myös esimerkiksi rationaaliluvut ja seuraavassa liitteessä esiteltävät kompleksiluvut muodostavat kunnan. Sen sijaan luonnolliset luvut ja kokonaisluvut eivät ole kuntia.

% Pitäisikö 'kunta-aksioomat' erottaa 'kunta-aksioomista reaalilukujen tapauksessa'

% Vaatii mathtools-paketin. mathtools implementoitu vain tiedostossa kirja.tex. Tämä on aika purkka (-- NVI).

Reaalilukujen aksiomaattinen määritelmä muodostuu kolmesta osasta:

\begin{flalign*}
&\textbf{Kunta-aksioomat} &\\
&\textbf{K1.} \, \forall x, y \in \mathbb{R}: & &x+(y+z) = (x+y)+z & &| \, \text{summan liitäntälaki} &\\
&\textbf{K2.} \, \exists 0 \in \mathbb{R}: & &x+0 = x & &| \, \text{summan neutraalialkio} &\\
&\textbf{K3.} \, \forall x \in \mathbb{R} & &\exists (-x) \in \mathbb{R}: \quad x+(-x)=0 & &| \, \text{vasta-alkio} &\\
&\textbf{K4.} \, \forall x, y \in \mathbb{R}: & &x+y = y+x & &| \, \text{summan vaihdantalaki} &\\
&\textbf{K5.} \, \forall x, y, z \in \mathbb{R}: & &x \cdot (y+z) = x \cdot y + x \cdot z & &| \, \text{osittelulaki} &\\
&\textbf{K6.} \, \forall x, y, z \in \mathbb{R}: & &x \cdot (y \cdot z) = (x \cdot y) \cdot z & &| \, \text{tulon liitäntälaki} &\\
&\textbf{K7.} \, \exists 1 \in \mathbb{R}: & &1 \cdot x = x & &| \, \text{tulon neutraalialkio} &\\
&\textbf{K8.} \, \forall x \in \mathbb{R} \setminus \{0\} & &\exists x^{-1} \in \mathbb{R} \setminus \{0\}: \quad x \cdot x^{-1}=1 & &| \, \text{tulon käänteisalkio} &\\
&\textbf{K9.} \, \forall x, y \in \mathbb{R}: & &x \cdot y = y \cdot x & &| \, \text{tulon vaihdantalaki} \\
&\textbf{Järjestysaksioomat} &\\
&\textbf{J1.} \, \forall x, y \in \mathbb{R}: & &\text{täsmälleen yksi seuraavista:} & \\
& & &(x > y), \, (x = y), \, (x < y) & &\\
&\textbf{J2.} \, \forall x, y, z \in \mathbb{R}: & &(x < y) \land (y < z) \Rightarrow (x < z) & &\\
&\textbf{J3.} \, \forall x, y, z \in \mathbb{R}: & &(x < y) \Leftrightarrow (x + z < y + z) & &\\
&\textbf{J4.} \, \forall x, y \in ]0,\infty[: & &x \cdot y \in ]0,\infty[ & &\\
&\textbf{Täydellisyysaksiooma} &\\
\shortintertext{\textbf{T1.} Jokaisella ylhäältä rajoitetulla epätyhjällä reaalilukujen osajoukolla on pienin yläraja.} 
\end{flalign*}

\begin{tehtava}
Todista aksioomista lähtien:
\begin{enumerate}[(1)]
\item $\forall x \in \mathbb{R}: 0 \cdot x = 0$
\item $\forall x \in \mathbb{R}: -1 \cdot x = -x$
\item $\forall x, y \in ]-\infty,0[: x \cdot y \in ]0,\infty[$
% lisää
\end{enumerate}
\begin{vastaus}
\begin{enumerate}[(1)]
% ???
\end{enumerate}
\end{vastaus}
\end{tehtava}

% Hanatehtävä on nyt tässä:
\begin{tehtava}
Kylpyhuoneessa on kolme hanaa. Hana A täyttää kylpyammeen 60 minuutissa, hana B 30 minuutissa ja hana C 15 minuutissa. Kuinka kauan kylpyammeen täyttymisessä kestää, jos kaikki hanat ovat yhtäaikaa auki?
\begin{vastaus}
$8$ min $34$ s
\end{vastaus}
\end{tehtava}
