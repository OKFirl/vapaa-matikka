\section{Pääsykoetehtäviä}

\subsection*{Arkkitehtivalinta}

\begin{description}
    \item[(2012/1)] Taloyhtiössä on $210$ asukasta. Taloyhtiö perii vesimaksua
        $15$ \euro/henkilö/kk. Kaupunki perii taloyhtiöltä vesimaksua
        kokonaiskulutuksen mukaan $3,17$~\euro $/ \mathrm{m}^3$.
        Taloyhtiön toteutunut kokonaisvedenkulutus on asukasta kohden
        155 litraa vuorokaudessa, mihin sisältyy hukkaan valuvia
        vuotoja yhtiötä kohden $1500~\mathrm{m}^3$ vuodessa. Oletamme,
        että vuodessa on $365$ vuorokautta.
                    
    \begin{alakohdat}
        \alakohta{Kuinka monta litraa vettä taloyhtiössä valuu hukkaan minuutissa?}
        \alakohta{Vuodot tukitaan. Montako prosenttia vesimaksua voitaisiin laskea tai tulee korottaa, jotta vesimaksu tällöin kattaisi taloyhtiön vuotuiset vesikulut?}
    \end{alakohdat}
    
    Anna vastaukset kolmen numeron tarkkuudella.
\end{description}

% Vastaus: a) 2,85~l/min b) laskea 12,9 prosenttia

\begin{description}
    \item[(2010/3)] Arkkitehdit M. Uoto, F. Örm ja S. Hapé ovat suunnitelleet Aalto-yliopiston aulaan kullatusta teräskuutiosta muodostuvan taideteoksen. Kultaus on ohut.
    
    Toteutuksen aikana teoksen sijoituspaikka muuttuu, jolloin teräskuution tilavuutta kasvatetaan 19~\% alkuperäisestä. Loppulaskutuksessa materiaalikustannusten todetaan kasvaneen samaiset 19~\% alkuperäisestä budjetista.                   
    \begin{alakohdat}
        \alakohta{Paljonko kultaukseen käytetyn kullan määrä kasvoi toteutuksen aikana?}
        \alakohta{Teräksen yksikköhinta ei toteutuksen aikana muuttunut. Miten kullan yksikköhinta siis muuttui?}
    \end{alakohdat}
    
    Anna vastaukset prosentteina $0,1$ prosenttiyksikön tarkkuuteen pyöristettynä.
\end{description}

\subsection*{Diplomi-insinöörivalinta}
\begin{description}
	\item[(2012/3)] Asumistukea maksetaan 80~\% vuokran määrästä, siltä osin kuin
        vuokra ei ylitä 252 euroa. Vuokran määrää vähennettynä asumistuella
        kutsutaan omavastuuksi.
        
		\begin{alakohdat}
			\alakohta{Minkä suuruinen vuokra on, kun omavastuu on puolet vuokrasta?}
		\end{alakohdat}
	
	\item[(2009/1)] Kokonaistuotanto jaetaan materian ja palveluiden tuotantoon.
        Verrataan tuotantoa tammikuussa 2008 tammikuuhun 2009. Tänä vuoden pituisen
        tarkastelujakson aikana materiatuotanto kasvoi 2,0~\% ja palvelutuotanto laski 7,0~\%.
	
	   Kuinka suuri oli materiatuotannon osuus kokonaistuotannosta tammikuussa 2009,
	   
    	\begin{alakohdat}
    		\alakohta{kun tammikuussa 2008 materia- ja palvelutuotanto olivat yhtäsuuret?}
    		\alakohta{kun vertailuaikana kokonaistuotanto laski 2,0~\%?}
    	\end{alakohdat}
    	
	   Anna kummatkin vastaukset 0,1~\%-yksikön tarkkuuteen pyöristettynä.

	\item[(2008/2)] Yritys hankkii 5000~kg raaka-ainetta, josta on vettä 5,40~\%
        (painoprosenttia) ja väripigmenttiä 2,60~\%. Ennen käyttöä raaka-aine on
        laimennettava siten, että lisäyksen jälkeen sekoituksesta 6,60~\% on vettä.
	
    	\begin{alakohdat}
    		\alakohta{Miten paljon hankittuun raaka-aineeseen tulee lisätä vettä, jotta haluttu vesipitoisuus saavutetaan?}
    		\alakohta{Miten paljon vettä ja väripigmenttiä tulee lisätä hankittuun raaka-aineeseen, jotta haluttu vesipitoisuus saavutetaan, ja lisäksi väripigmentin suhteellinen osuus massasta säilyy alkuperäisenä 2,60~\%:na?}
    	\end{alakohdat}
    	
    	Anna vastaukset sadan gramman tarkkuudella.

	\item[(2007/1)] Vaaleissa kaikkiaan 39~300 äänestäjästä 45~\% äänestää varmasti
        puoluetta A ja 47~\% puoluetta B. Loput ovat ns. liikkuvia äänestäjiä,
        jotka eivät ole vielä päättäneet kantaansa.
	
    	\begin{alakohdat}
    		\alakohta{Oletetaan, että kaikki äänioikeutetut äänestävät. Kuinka monta liikkuvien äänestäjien ääntä puolueen A täytyy tällöin kerätä saadakseen enemmistön, vähintään puolet annetuista äänistä?}
    		\alakohta{Oletetaan, että täsmälleen kolmasosa liikkuvista äänestäjistä jättää äänestämättä. Kuinka monta prosenttia liikkuvien äänestäjien annetuista äänistä puolueen A täytyy tällöin kerätä saadakseen enemmistön kaikista annetuista äänistä?}
    	\end{alakohdat}	 	
	
\end{description}

\subsection*{Matematiikan ja tilastotieteen valinta}

\begin{description}
	\item[(2012/1a)] Ratkaise yhtälö $\frac{3}{2}x - \frac{2}{3} = \frac{2}{3}x - \frac{1}{4}$.
	\item[(2011/1)] Oletetaan, että polttoaineessa E05 on etanolia 5~\% ja
        bensiiniä 95~\% ja polttoaineessa E10 etanolia 10~\% ja bensiiniä 90~\%.
        Oletetaan myös, että etanolin energiasisältö on $\frac{2}{3}$ puhtaan bensiinin
		energiasisällöstä. Jos tietyllä autolla 100~km kulutus on 10 litraa
        polttoainetta E05, paljonko kulutus on polttoainetta E10? Anna vastaus
        sievennettynä murto- tai sekalukuna. Jos polttoaineen E10 hinta on 1,60~€/l
        ja polttoaineen E05 hinta on 1,65~€/l, kumpaa on edullisempaa käyttää?
	\item[(2008/1)] Matkailuauton nopeus on 80~km/h, mutta kolmasosalla matkasta
        Jyväskylästä Heinolaan se laskee tietöiden takia 40~kilometriin tunnissa.
        Kuinka paljon tietyöt alentavat matkailuauton keskinopeutta välillä Jyväskylä--Heinola?
\end{description}

\subsection*{Tekniikan ja liikenteen alan AMK-valinta}

\begin{description}
	\item[(K2012/3)] Erään tuotteen valmistuskustannuksista raaka-aineiden osuus on
        65~\% ja palkkojen osuus on 35~\%.
        
    	\begin{alakohdat}
    		\alakohta{Jos työntekijät saavat 5~\% palkankorotuksen, niin kuinka monta prosenttia tuotteen valmistuskustannukset kasvavat?}
    		\alakohta{Jos toisaalta valmistuskustannukset halutaan pitää ennallaan palkankorotuksen jälkeen, niin montako prosenttia raaka-ainekustannusten pitää pienentyä?}
    	\end{alakohdat}	 

	\item[(K2012/4)] Esitä luku $\frac{1}{1+\frac{1}{1+\frac{1}{1+1}}}$ yhtenä murtolukuna (siis muodossa $\frac{m}{n}$ [missä $m$ ja $n$ ovat kokonaislukuja, toim. lis.]).
	\item[(S2011/2)] Olkoon $a=132$ ja  $b=112$. Kuinka monta prosenttia 
		\begin{alakohdat}
			\alakohta{luku a on suurempi kuin luku b}
			\alakohta{luku b on pienempi kuin luku a}
			\alakohta{luku b on luvusta a? }
		\end{alakohdat}
	\item[(K2011/1)] Laske lukujen $\frac{1}{3}$ ja $-\frac{7}{3}$
		\begin{alakohdat}
			\alakohta{summan vastaluku}
			\alakohta{summan käänteisluku }
			\alakohta{käänteislukujen summa.}
		\end{alakohdat}
	\item[(K2011/2)] Yksi kilogramma etanolia tuottaa palaessaan energiaa 26,8~MJ
        ja vastaavasti yksi kilogramma puhdasta bensiiniä tuottaa palaessaan energiaa
        42,6~MJ. Kuinka monta prosenttia enemmän energiaa puhdas bensiini tuottaa
        palaessaan kuin 95~E10 -bensiini, joka sisältää 10~\% etanolia? Ilmoita
        vastaus yhden desimaalin tarkkuudella. 
\end{description}

\subsection*{Kauppatieteiden valintakoe}

Kauppatieteiden valintakokeen tehtävät ovat monivalintoja. Kussakin tehtävässä on täsmälleen yksi oikea vastausvaihtoehto. Kokeessa saa käyttää ainoastaan ns. nelilaskinta, jolla voi laskea vain yhteen-, vähennys-, kerto- ja jakolaskuja sekä neliöjuuren. Seuraavia tehtäviä ratkoessasi sinun kannattaa miettiä, miten toimisit, jos käytössäsi olisi vain nelilaskin.

\begin{description}
	\item[(2002/38)] Mikä on lukujen $a=2^{1/2}$, $b=3^{1/3}$ ja $c=5^{1/5}$ suuruusjärjestys?
        
		\begin{alakohdat}
			\alakohta{$a>b>c$}
			\alakohta{$a>c>b$}
			\alakohta{$b>a>c$}
			\alakohta{$c>a>b$}
		\end{alakohdat}

%Oikea vastausvaihtoehto: c

	\item[(2008/43)] TietoEnatorin liikevaihto vuonna $2007$ oli $1\,772{,}4$ MEUR (miljoona Euroa). Mikä
	alla olevista vaihtoehdoista on lähimpänä vuoden $1999$ liikevaihtoa, kun
	liikevaihdon keskimääräinen vuotuinen muutosprosentti kahdella desimaalilla on
	ollut $3{,}96 \%$? Keskimääräinen muutosprosentti on luku, joka ilmaisee kuinka paljon
	liikevaihto olisi vuosittain prosentuaalisesti kasvanut, jos prosentuaalinen kasvu olisi
	ollut vakio.
        
		\begin{alakohdat}
			\alakohta{$1\,299{,}1$ MEUR}
			\alakohta{$1\,346{,}0$ MEUR}
			\alakohta{$1\,704{,}9$ MEUR}
			\alakohta{$1\,249{,}6$ MEUR}
		\end{alakohdat}

%Oikea vastausvaihtoehto: a

	\item[(2004/40)] Piensijoittajan rahavarat $r$ vuoden alussa ovat 1 000 \euro, ja ne kasvavat korkoa vuotuisen korkotekijän $R=1{,}04$ mukaisesti. Vuoden lopussa korko lisätään pääomaan, ja seuraavana vuotena vuotuinen korkotekijä on $R=1{,}10$. Korkotuotto kahdelta vuodelta on (lähimpään kokonaislukuun pyöristettynä)
        
		\begin{alakohdat}
			\alakohta{$145$ \euro.}
			\alakohta{$140$ \euro.}
			\alakohta{$100$ \euro.}
			\alakohta{$144$ \euro.}
		\end{alakohdat}

%Oikea vastausvaihtoehto: d

	\item[(2003/33)] Yhtä tuotetta valmistavan monopoliyrityksen kuukauden tarjontamäärän ollessa $q$ (yks/kuukausi) on tuotteen hinta $p$ (euroa/yks) annettu hintafunktiolla $p = 100-q$. Kun kyseessä on monopoli, yritys määrää markkinahinnan $p$ valitsemalla tuotantomäärän $q$, jolloin hinta määräytyy hintafunktion mukaisesti. Yrityksen tuotantokustannukset $c(q)$ (euroa/kuukausi) tuotantomäärän $q$ funktiona ovat\\
$c(q)=100+80q$, kun $q\leq 15$, ja\\
$c(q)=700+40q$, kun $q\geq 15$.\\
Yrityksen voitto $pq-c(q)$ saavuttaa maksimiarvonsa, kun
        
		\begin{alakohdat}
			\alakohta{$q=10$.}
			\alakohta{$q=15$.}
			\alakohta{$q=30$.}
			\alakohta{$q=60$.}
		\end{alakohdat}

%Oikea vastausvaihtoehto: c

	\item[(2000/33)] Erään tuotteen kysyntämäärä $d$ (yksikköä) riippuu yksikköhinnasta $p$ (mk/yksikkö) funktion $d=5000p^{-1{,}5}$ mukaisesti. Tuotteen yksikkökustannukset ovat vakio $15$ mk/yksikkö. Suurin nettotuotto saadaan tällöin hinnalla
        
		\begin{alakohdat}
			\alakohta{$40$ mk.}
			\alakohta{$45$ mk.}
			\alakohta{$50$ mk.}
			\alakohta{$55$ mk.}
		\end{alakohdat}

%Oikea vastausvaihtoehto: b
	
\end{description}
