% Ympäristö välien kuvaamiseen lukusuoralla.
% Käyttö:
% \begin{lukusuora}{-1}{1}{10}
% \lukusuoravalisa{0.2}{0.8}{x}{y}
% \lukusuoravaliaa{-0.5}{-0.3}{a}{b}
% \end{lukusuora}
% 
% luo lukusuoran joka piirretään 10 pituisena, käyttää sisäisesti
% koordinaatteja -1..1, ja jolla näytetään väli [0.2, 0.8[ ja
% väli ]-0.5, -0.3[ nimillä [x, y[ ja ]a, b[.



% lukusuora-ympäristön parametrit:
% alkux loppux viivanpituus
\newenvironment{lukusuora}[3]
{
	\begin{tikzpicture}
		\draw[arrows=-triangle 45,line width=0.2mm] (0, 0) -- (#3, 0);
		\pgfmathsetmacro{\LSax}{#1}
		\pgfmathsetmacro{\LSdx}{#2-#1}
		\pgfmathsetmacro{\LSpituus}{#3}
}
{
	\end{tikzpicture}
}

% lukusuoravali??-komennon parametrit:
% alkux loppux alkunimi loppunimi

% avoimet ja suljetut välit:
% lukusuoravaliss [a, b]
% lukusuoravalisa [a, b[
% lukusuoravalias ]a, b]
% lukusuoravaliaa ]a, b[

\newcommand{\lukusuoravaliss}[4]{
	\lukusuoravali{#1}{#2}{#3}{#4}{black}{black}
}
\newcommand{\lukusuoravalisa}[4]{
	\lukusuoravali{#1}{#2}{#3}{#4}{black}{white}
}
\newcommand{\lukusuoravalias}[4]{
	\lukusuoravali{#1}{#2}{#3}{#4}{white}{black}
}
\newcommand{\lukusuoravaliaa}[4]{
	\lukusuoravali{#1}{#2}{#3}{#4}{white}{white}
}

\newcommand{\lukusuoravali}[6]{
	\pgfmathsetmacro{\LSxa}{\LSpituus*(#1-\LSax)/\LSdx}
	\pgfmathsetmacro{\LSxb}{\LSpituus*(#2-\LSax)/\LSdx}
	\draw[line width=0.6mm] (\LSxa,0) -- (\LSxb,0);
	\fill (\LSxa,0) circle (0.1);
	\fill (\LSxb,0) circle (0.1);
	\fill[color=#5] (\LSxa,0) circle (0.08);
	\fill[color=#6] (\LSxb,0) circle (0.08);
	% Integraalimerkit ovat riittävän korkeita saamaan bounding boxit saman
	% kokoisiksi.
	\draw (\LSxa,0) node[above] {\phantom{$\int$}#3\phantom{$\int$}};
	\draw (\LSxb,0) node[above] {\phantom{$\int$}#4\phantom{$\int$}};
}
