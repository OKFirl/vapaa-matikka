% muokattava korostus termeille
\newcommand{\termi}[2][] {%
  \ifthenelse{\isempty{#2}}{\korostettuindeksi{#1}{#1}}%
  {\korostettuindeksi{#1}{#2}}%
}

% käytetään kun sovelletaan sama funktio molemmille puolille
% (pystypalkki kaksoisviivalla)
\newcommand{\ppalkki}{\big\|}

% \lowercase poistettu, aiheutti ongelmia makeindexin kanssa
\newcommand{\korostettuindeksi}[2]{\textbf{#1}\index{#2}}

\newenvironment{todistus}[0]
{
  \begin{proof}
}
{
  \end{proof}
}

% Tarvitaan kuvien ja taulukkojen vierekkäin laittamiseen.
\def\vcent#1{\mathsurround0pt$\vcenter{\hbox{#1}}$}


\newcommand{\Harjoitustehtavat}{%
	\newpage
	\section*{Harjoitustehtäviä}%
	%\addcontentsline{toc}{section}{Harjoitustehtäviä}
}

