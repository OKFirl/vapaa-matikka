% marginaalihuomautuksia varten (lähinnä termeille), ei vielä käytössä
% \newcommand{\marginaali}[1]{%
%  \marginpar{\raggedleft\sffamily\small\selectfont{#1}}}

% muokattava korostus termeille
\newcommand{\termi}[2][] {%
 \ifthenelse{\isempty{#2}}{\korostettuindeksi{#1}{#1}}%
 {\korostettuindeksi{#1}{#2}}%
}

% käytetään kun sovelletaan sama funktio molemmille puolille
% (pystypalkki kaksoisviivalla)
\newcommand{\ppalkki}{\big\|}

% \lowercase poistettu, aiheutti ongelmia makeindexin kanssa
\newcommand{\korostettuindeksi}[2]{\textbf{#1}\index{#2}}

\newenvironment{todistus}[0]
{
  \begin{proof}
}
{
  \end{proof}
}

% Tarvitaan kuvien ja taulukkojen vierekkäin laittamiseen.
\def\vcent#1{\mathsurround0pt$\vcenter{\hbox{#1}}$}


\newcommand{\Harjoitustehtavat}{%
	\newpage
	\subsection*{Harjoitustehtäviä}%
	%\addcontentsline{toc}{section}{Harjoitustehtäviä}
}

% Linkki ja QR-koodi. Parametrit: URL ja kuvaus.
\newcommand{\qrlinkki}[2]{
	
	\begin{minipage}[c]{0.7in}
		\begin{pspicture}(0.6in,0.6in)
			\psbarcode{#1}{width=0.6 height=0.6}{qrcode}
		\end{pspicture}
		\vspace{0.1in}
	\end{minipage}
	\begin{minipage}[l]{\textwidth-0.7in}
		#2
		
		{\scriptsize\url{#1}}
	\end{minipage}
	
}
