% Ympäristö kuvaajan plottaamiseen. Käytä jos riittää jotta on helppo päivittää.
% Käyttö:
% \begin{kuvaajapohja}{2}{-3.14}{3.14}{-1.25}{1.25}
% 	\kuvaaja{sin(x)}{hassu kuvaaja}{red}
% 	\kuvaaja{cos(x)}{hassumpi kuvaaja}{blue}
% \end{kuvaajapohja}
% piirtää punaisen sinin ja sinisen kosinin välille -3.14 .. 3.14,
% y-koordinaatti välillä -1.25 .. 1.25, sinin nimenä hassu kuvaaja
% ja kosinin nimenä hassumpi kuvaaja, skaalattuna kertoimella 2.


% kuvaajapohja-ympäristön parametrit: skaala minx maxx miny maxy
\newenvironment{kuvaajapohja}[5]
{
	\begin{tikzpicture}
		\draw[color=black!10!white, step=#1*0.25] (#1*#2, #1*#4) grid (#1*#3, #1*#5);
		\draw[color=black!40!white, step=#1] (#1*#2, #1*#4) grid (#1*#3, #1*#5);
		\draw[arrows=-triangle 45] (#1*#2,0) -- (#1*#3,0);
		\draw[arrows=-triangle 45] (0,#1*#4) -- (0,#1*#5);
		\providecommand\kuvaajascale{#1}
		\providecommand\kuvaajaminx{#2}
		\providecommand\kuvaajamaxx{#3}
		\providecommand\kuvaajaminy{#4}
		\providecommand\kuvaajamaxy{#5}
}
{
	\end{tikzpicture}
}

% kuvaaja-komennon parametrit: funktio(x) nimi väri
\newcommand{\kuvaaja}[3]{
\draw[smooth,color=#3,thick,domain=\kuvaajaminx:\kuvaajamaxx,scale=\kuvaajascale,samples=300] plot function{(#1) < \kuvaajamaxy ? ((#1) > \kuvaajaminy ? (#1) : NaN) : NaN} node[right] {#2};
}
