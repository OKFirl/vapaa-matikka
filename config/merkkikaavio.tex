% merkkikaavio-ympäristön parametrit:
% leveys
\newenvironment{merkkikaavio}[1]
{
	\begin{tikzpicture}
		\pgfmathsetmacro{\MKruutuja}{#1+1}
		\pgfmathsetmacro{\MKruutuw}{1.4}
		\pgfmathsetmacro{\MKruutuh}{0.7}
		\pgfmathsetmacro{\MKkaikkiw}{\MKruutuw*\MKruutuja}
		\pgfmathsetmacro{\MKsuorakorkeus}{0.22}
		
		\draw[arrows=-triangle 45,line width=0.2mm] (0, \MKsuorakorkeus) -- (\MKkaikkiw + 0.4, \MKsuorakorkeus) node[below] {$x$};
		
		\pgfmathsetmacro{\MKvikaviiva}{\MKruutuja-1}
		\foreach \sarake in {0,...,\MKvikaviiva}
		{
			\if\sarake0
			\else
				\draw (\sarake * \MKruutuw, \MKsuorakorkeus - 0.1) -- ++(0, 0.2);
			\fi
			\draw (\sarake * \MKruutuw, 0) -- ++(0, -\MKruutuh);
		}
		
		\pgfmathsetmacro{\MKposx}{0}
		\pgfmathsetmacro{\MKposy}{0}
		\pgfmathsetmacro{\MKkohtapos}{\MKruutuw}
}
{
	\end{tikzpicture}
}

% Lisää kohdan x-akselille.
\newcommand{\merkkikaavioKohta}[1]{
	\draw (\MKkohtapos, 0.26) node[above] {\phantom{$\int$}#1\phantom{$\int$}};
	\pgfmathsetmacro{\MKkohtapos}{\MKkohtapos+\MKruutuw}
}

% Piirtää funktion nimen merkkikaaviossa.
\newcommand{\merkkikaavioFunktio}[1]{
	\draw (0, \MKposy-\MKruutuh/2) ++(-0.1, 0) node[left] {#1};
}

% Aloittaa uuden rivin merkkikaaviossa.
\newcommand{\merkkikaavioUusirivi}[0]{
	\pgfmathsetmacro{\MKposx}{0}
	\pgfmathsetmacro{\MKposy}{\MKposy-\MKruutuh}
	\draw (-0.3, \MKposy) -- (\MKkaikkiw, \MKposy);
	\foreach \sarake in {0,...,\MKvikaviiva}
	{
		\draw (\sarake * \MKruutuw, \MKposy) -- ++(0, -\MKruutuh);
	}
}

% Aloittaa uuden rivin merkkikaaviossa käyttäen kaksoisviivaa.
\newcommand{\merkkikaavioUusiriviKaksoisviiva}[0]{
	\pgfmathsetmacro{\MKposx}{0}
	\pgfmathsetmacro{\MKposy}{\MKposy-\MKruutuh}
	\draw (-0.3, \MKposy) -- (\MKkaikkiw, \MKposy);
	\foreach \sarake in {0,...,\MKvikaviiva}
	{
		\draw (\sarake * \MKruutuw, \MKposy) -- ++(0, -\MKruutuh);
	}
	\pgfmathsetmacro{\MKposy}{\MKposy-0.06}
	\draw (-0.3, \MKposy) -- (\MKkaikkiw, \MKposy);
	\foreach \sarake in {0,...,\MKvikaviiva}
	{
		\draw (\sarake * \MKruutuw, \MKposy) -- ++(0, -\MKruutuh);
	}
}

% Lisää merkin merkkikaavioon.
\newcommand{\merkkikaavioMerkki}[1]{
	\draw (\MKposx+\MKruutuw/2, \MKposy-\MKruutuh/2) node {#1};
	\pgfmathsetmacro{\MKposx}{\MKposx+\MKruutuw}
}
