\section{Polynomien jakolause}
\label{tod:poljako}

Jakolauseen todistus perustuu polynomien jakoyhtälöön, josta tarkemmin kurssilla 12.

\begin{todistus}
Vaikka lauseke $x-b$ ei olisi polynomin $P(x)$ tekijä, niin lähelle päästään: jos polynomin $Q(x)$ kertoimet valitaan sopivasti, voidaan kirjoittaa
\begin{align*}
P(x)&=(x-b)Q(x)+r,
\end{align*}
missä $r$ on jokin vakio, niin sanottu jakojäännös. Jos nyt $b$ on polynomin $P$ nollakohta, sijoitetaan edelliseen yhtälöön $x=b$, jolloin
\begin{align*}
P(b)&=(b-b)Q(b)+r \quad || \ \ P(b)=0 \\
0&=0+r,
\end{align*}
eli $r=0$, joten $x-b$ on polynomin $P(x)$ tekijä. Jos siis $x=b$ on polynomin $P(x)$ nollakohta, $x-b$ on sen tekijä.
\end{todistus}
