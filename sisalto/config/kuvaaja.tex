% Ympäristö kuvaajan plottaamiseen. Käytä jos riittää jotta on helppo päivittää.
% Käyttö:
% \begin{kuvaajapohja}{2}{-3.14}{3.14}{-1.25}{1.25}
% 	\kuvaaja{sin(x)}{hassu kuvaaja}{red}
% 	\kuvaaja{cos(x)}{hassumpi kuvaaja}{blue}
% \end{kuvaajapohja}
% piirtää punaisen sinin ja sinisen kosinin välille -3.14 .. 3.14,
% y-koordinaatti välillä -1.25 .. 1.25, sinin nimenä hassu kuvaaja
% ja kosinin nimenä hassumpi kuvaaja, skaalattuna kertoimella 2.


% kuvaajapohja-ympäristön parametrit: skaala minx maxx miny maxy
% asetuksia voi muuttaa laittamalla hakasulkuihin \kuvaajaAsetus*-komentoja.
\newenvironment{kuvaajapohja}[6][]
{
\begin{tikzpicture}
	\newcommand\kuvaajaNaytaLuvut1
	\newcommand\kuvaajaNaytaRuudukko1
	#1
	
	\pgfmathsetmacro{\kuvaajasmallgrid}{0.25}
	\pgfmathsetmacro{\kuvaajastep}{1}
	\ifthenelse{\lengthtest{#2 pt < 0.8 pt}}{
		\pgfmathsetmacro{\kuvaajasmallgrid}{0.5}
	}{}
	\ifthenelse{\lengthtest{#2 pt < 0.4 pt}}{
		\pgfmathsetmacro{\kuvaajasmallgrid}{1}
		\pgfmathsetmacro{\kuvaajastep}{2}
	}{}
	
	
	\if\kuvaajaNaytaRuudukko1
		\draw[color=black!10!white, step=#2*\kuvaajasmallgrid] (#2*#3, #2*#5) grid (#2*#4, #2*#6);
		\draw[color=black!50!white, step=#2] (#2*#3, #2*#5) grid (#2*#4, #2*#6);
	\fi
	
	\draw[arrows=-triangle 45, thick] (#2*#3,0) -- (#2*#4,0);
	\draw[arrows=-triangle 45, thick] (0,#2*#5) -- (0,#2*#6);
	
	\if\kuvaajaNaytaLuvut1
		\pgfmathsetmacro{\xalaraja}{-#3-0.01}
		\pgfmathsetmacro{\xylaraja}{#4-0.01}
		\pgfmathsetmacro{\yalaraja}{-#5-0.01}
		\pgfmathsetmacro{\yylaraja}{#6-0.01}
		\foreach \sarake in {0,...,\xylaraja}
		{
			\ifthenelse{\lengthtest{\sarake pt > 0 pt}}{
				\pgfmathsetmacro{\kuvaajajakojaannos}{mod(\sarake, \kuvaajastep)}
				\ifthenelse{\lengthtest{\kuvaajajakojaannos pt = 0 pt}}{
					\draw (#2*\sarake + 0.1, 0) node[above] {\footnotesize\sarake};
					\draw[thick] (#2*\sarake, -0.07) -- (#2*\sarake, 0.07);
				}{}
			}{}
		}
		\foreach \sarake in {0,...,\xalaraja}
		{
			\ifthenelse{\lengthtest{\sarake pt > 0 pt}}{
				\pgfmathsetmacro{\kuvaajajakojaannos}{mod(\sarake, \kuvaajastep)}
				\ifthenelse{\lengthtest{\kuvaajajakojaannos pt = 0 pt}}{
					\draw (-#2*\sarake + 0.1, 0) node[above] {\footnotesize-\sarake};
					\draw[thick] (-#2*\sarake, -0.07) -- (-#2*\sarake, 0.07);
				}{}
			}{}
		}
		\foreach \rivi in {0,...,\yylaraja}
		{
			\ifthenelse{\lengthtest{\rivi pt > 0 pt}}{
				\pgfmathsetmacro{\kuvaajajakojaannos}{mod(\rivi, \kuvaajastep)}
				\ifthenelse{\lengthtest{\kuvaajajakojaannos pt = 0 pt}}{
					\draw (0, #2*\rivi) node[right] {\footnotesize\rivi};
					\draw[thick] (-0.07, #2*\rivi) -- (0.07, #2*\rivi);
				}{}
			}{}
		}
		\foreach \rivi in {0,...,\yalaraja}
		{
			\ifthenelse{\lengthtest{\rivi pt > 0 pt}}{
				\pgfmathsetmacro{\kuvaajajakojaannos}{mod(\rivi, \kuvaajastep)}
				\ifthenelse{\lengthtest{\kuvaajajakojaannos pt = 0 pt}}{
					\draw (0, -#2*\rivi) node[right] {\footnotesize-\rivi};
					\draw[thick] (-0.07, -#2*\rivi) -- (0.07, -#2*\rivi);
				}{}
			}{}
		}
	\fi
%		\draw (#2, 0) node[above] {1};
%		\draw (0, #2) node[right] {1};
	\newcommand\kuvaajascale{#2}
	\newcommand\kuvaajaminx{#3}
	\newcommand\kuvaajamaxx{#4}
	\newcommand\kuvaajaminy{#5}
	\newcommand\kuvaajamaxy{#6}
}
{
\end{tikzpicture}
}

% kuvaajapohjaan hakasulkuihin annettava komento, joka poistaa luvut akseleilta.
\newcommand{\kuvaajaAsetusEiLukuja}{
	\renewcommand\kuvaajaNaytaLuvut0
}

% kuvaajapohjaan hakasulkuihin annettava komento, joka poistaa kaikki ruudukot
\newcommand{\kuvaajaAsetusEiRuudukkoa}{
	\renewcommand\kuvaajaNaytaRuudukko0
}

% kuvaaja-komennon parametrit: funktio(x) nimi väri
\newcommand{\kuvaaja}[3]{
	\draw[smooth,color=#3,thick,domain=\kuvaajaminx:\kuvaajamaxx,scale=\kuvaajascale,samples=300] plot function{(#1) < \kuvaajamaxy ? ((#1) > \kuvaajaminy ? (#1) : NaN) : NaN} node[right] {#2};
}

% Piirtää pisteen kuvaajaan. Parametrit:
% x y
\newcommand{\kuvaajapiste}[2]{
	\fill (#1*\kuvaajascale, #2*\kuvaajascale) circle (0.07);
}

% Näytä katkoviivalla funktion kohta ja arvo.
% parametrit:
% kohta arvo kohtanimi arvonimi
\newcommand{\kuvaajakohtaarvo}[4]{
	\kuvaajapiste{#1}{#2}
	\draw[thick, dashed] (#1*\kuvaajascale, 0) -- (#1*\kuvaajascale, #2*\kuvaajascale);
	\draw[thick, dashed] (0, #2*\kuvaajascale) -- (#1*\kuvaajascale, #2*\kuvaajascale);
	\ifthenelse{\lengthtest{#2 pt > 0 pt}}{
		\draw (#1*\kuvaajascale, 0) node[below] {#3};
	}{
		\draw (#1*\kuvaajascale, 0) node[above] {#3};
	}
	\ifthenelse{\lengthtest{#1 pt > 0 pt}}{
		\draw (0, #2*\kuvaajascale) node[left] {#4};
	}{
		\draw (0, #2*\kuvaajascale) node[right] {#4};
	}
}
