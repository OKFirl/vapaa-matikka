\chapter{Harjoituskokeita}

\section*{Harjoituskoe 1}

\begin{description}
	\item Kaikki tehtävät tehdään. Kaikista tehtävistä voi saada 6 p, paitsi tehtävästä 7 vielä ylimääräiset 2 p.
	\item[1.] Laske \\
	(a) $\sqrt{144}$ \qquad
	(b) $\sqrt{196}$ \qquad
	(c) $\sqrt[3]{64}$ \qquad
	(d) $\sqrt[3]{216}$
	\item[2.] Sievennä \\
	(a) $\frac{a^2 b^2}{a}$, $a \neq 0$ \qquad
	(b) $3(a^2+1)-2(a^2-1)$ \\
	(c) $ab(a+2a)$ \qquad
	(d) $(a^3 b^2 c)^2$.
	\item[3.] Ratkaise \\
	(a) $x^2 = 1$
	(b) $x^4 = x^2$
	\item[4.] Mitkä seuraavista luvuista ovat alkulukuja? \\
	(a) $11$ \qquad
	(b) $4$ \qquad
	(c) $29$ \qquad
	(d) $39$
	\item[5.] Olkoon $f(t) = 35 \cdot 2^t$ bakteerien lukumäärä soluviljelmässä ajanhetkellä $t$ (sekuntia). Monenko sekunnin kuluttua bakteereita on yli 1000? Yhden sekunnin tarkkuus ylöspäin pyöristettynä riittää.
	\item[6.] Muuta seuraavat kymmenjärjestelmän luvut binääriluvuiksi. \\
	(a) $100$ \qquad
	(b) $128$
	\item[7.] Kerro jokin \\
	(a) aina tosi yhtälö \\
	(b) joskus tosi yhtälö (ja milloin se on tosi) \\
	(c) aina epätosi yhtälö \\
	(+) mahdollisesti tosi yhtälö (tästä voi saada ylimääräiset 2 p).
	\item[8.] Funktio $f$ määritellään kaavalla $f(x) = x^2 + 2x + 3$. Ilmaise $f(f(x))$ muodossa, jossa ei ole termiä $f(x)$.
	\item Arvosanat: 45 - 50 p: 10, 39 - 44 p: 9, 33 - 38 p: 8, 27 - 32 p: 7, 21 - 26 p: 6, 15 - 20 p: 5, 0 - 14 p: 4.
\end{description}

\section*{Harjoituskoe 2}

% tämä voisi olla esimerkki kokeesta, jossa voi valita
% jokin 8p jokeri/tähtitehtävä mukaan
