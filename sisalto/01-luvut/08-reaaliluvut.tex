\chapter{Reaaliluvut}

Joskus yksinkertaisen geometrisen ongelman ratkaisu voi olla luku, jota ei voida esittää rationaalilukuna. Siksi on tarpeen tutkia lukuja, jotka eivät ole rationaalilukuja. Tällaisia lukuja kutsutaan \emph{irrationaaliluvuiksi}.

Esimerkiksi $\sqrt{2}$ on irrationaaliluku, mikä todistetaan luvun lopussa. Toinen tuttu peruskoulusta tuttu irrationaaliluku on ympyrän kehän pituuden suhde halkaisijaan: $\pi$.

%%%%TÄMÄ ON KUVA LUKUSUORALLA OLEVASTA PIISTÄ JA SQRT 2:STA%%%
\begin{scriptsize}

\end{scriptsize}
\definecolor{ffqqqq}{rgb}{1,0,0}
\begin{tikzpicture}[line cap=round,line join=round,>=triangle 45,x=6.0cm,y=6.0cm]
\draw[->,color=black] (1.2,0) -- (3.3,0);
\foreach \x in {1.2,1.3,1.4,1.5,1.6,1.7,1.8,1.9,2,2.1,2.2,2.3,2.4,2.5,2.6,2.7,2.8,2.9,3,3.1,3.2,3.3}
\draw[shift={(\x,0)},color=black] (0pt,2pt) -- (0pt,-2pt) node[below] {\footnotesize $\x$};
\clip(1.2,-0.1) rectangle (3.3,0.3);
\draw (0.5,2.1) node[anchor=north west] {$\sqrt[]{2}$};
\draw (2.8,2.4) node[anchor=north west] {$\pi$};
\draw [->] (1.2,0) -- (3.3,0);
\draw [->,line width=2pt,color=ffqqqq] (1.414,0.1) -- (1.414,0);
\draw [->,line width=2pt,color=ffqqqq] (pi,0.1) -- (pi,0);
\draw [color=ffqqqq](1.4,0.2) node[anchor=north west] {$\sqrt[]{2}$};
\draw [color=ffqqqq](3.1,0.2) node[anchor=north west] {$\pi$};
\end{tikzpicture}

\laatikko{
Rationaalilukuja ja irrationaalilukuja kutsutaan yhdessä \emph{reaaliluvuiksi}. Reaalilukujen joukkoja merkitään $\mathbb{R}$.
}

Rationaaliluvuilla voidaan approksimoida irrationaalilukuja mielivaltaisen tarkasti. Esimerkiksi $\pi$ voidaan kirjoittaa halutusta tarkkuudesta riippuen seuraaviin muotoihin:
\[
3, \quad 3,1, \quad 3,14, \quad 3,145, \quad 3,1459, \quad 3,14592, \quad 3,145926, \ldots
\]

Rationaaliluvut eivät ns. täytä koko lukusuoraa, mutta niitä on niin tiheästi, että jokaista irrationaaliluku voidaan approksimoida mielivaltaisen tarkasti jollain rationaaliluvulla.

\missingfigure{kuva tähän lukusuorasta, jolla on neliöjuuri 2 ja sen vieressä rationaalilukuja yhä lähempänä ja lähempänä - ne approksimoivat neliöjuuri kakkosta}

Vaikka menisimme kuinka lähelle jotain irrationaalilukua lukusuoralla, välistä löytyisi vielä jokin rationaaliluku.


\laatikko{
Kaikki rationaalilukuja koskevat laskusäännöt pätevät myös reaaliluvuille.
}

Tämän perusteleminen on valitettavasti liian pitkällinen asia lukion ykköskurssille.

\todo{pidä huolta, että aiemmin on oikeasti puhuttu rationaalilukujen desimaalilukuesitysten jaksollisuudesta!! tai sitten perustele tässä}

Siinä missä rationaalilukujen desimaaliesitykset ovat päättyviä tai jaksollisia, ovat irrationaalilukujen desimaaliesitykset päättymättömiä ja
jaksottomia.

Luvun
\[\sqrt{2} \approx 1,414213562373095048801688724209\ldots\]
desimaaliesityksessä on kyllä toistuvia kohtia, esimerkiksi numeropari $88$ esiintyy kahdesti. Siinä ei kuitenkaan ole mitään jaksoa, jonka jälkeen luvut alkaisivat toistua, toisin kuin esimerkiksi luvussa $3,80612312312312\ldots$.

Reaalilukujen ominaisuuksista kerrotaan lisää liitteessä \ref{aksioomat}.


Reaalilukujen myötä kaikki lukiokursseissa esiintyvät lukujoukot on esitelty. Ne on lueteltu seuraavassa:
\begin{center}\begin{tabular}{l|c|l}
Joukko & Symboli & Mitä ne ovat\\
\hline
Luonnolliset luvut & $\mathbb{N}$ &
Luvut 0, 1, 2, 3, $\ldots$ \\
Kokonaisluvut & $\mathbb{Z}$ & Luvut $\ldots$ -2, -1, 0, 1, 2 $\ldots$ \\
Rationaaliluvut & $\mathbb{Q}$ & Luvut, jotka voidaan esittää
murtolukuina \\
Reaaliluvut & $\mathbb{R}$ & Kaikki lukusuoran luvut
\end{tabular} \end{center} 

\todo{Kuva lukujoukoista täytyy korjata: kaikki esimerkkiluvut suppeimman
joukon sisään, johon ne kuuluvat. Ei omaa rinkulaa irrationaaliluvuille.
Teksti irrationaaliluvut saa jäädä mukaan.}

\begin{tikzpicture}[line cap=round,line join=round,>=triangle 45,x=0.5cm,y=0.5cm]
\clip(-7.4,-8.8) rectangle (16.8,8.6);
\draw [rotate around={0.5:(2.2,0)}] (2.2,0) ellipse (1.1cm and 0.9cm);
\draw [rotate around={-0.8:(2.5,0)}] (2.5,0) ellipse (2cm and 1.6cm);
\draw [rotate around={-0.8:(2.5,0)}] (2.5,0) ellipse (2.9cm and 2.6cm);
\draw (2,1.5) node[anchor=north west] {$\mathbb{N}$};
\draw (4.3,2.7) node[anchor=north west] {$\mathbb{Z}$};
\draw (5.8,3.9) node[anchor=north west] {$\mathbb{Q}$};
\draw (7.1,-4.2) node[anchor=north west] {{\scriptsize Irrationaaliluvut}}; %TODO: rotate
\draw (8.4,6.2) node[anchor=north west] {$\mathbb{R}$};
\draw [rotate around={0.5:(4.4,0)}] (4.4,0) ellipse (5cm and 4.2cm);
\draw [rotate around={18.2:(7.9,-5.4)}] (7.9,-5.4) ellipse (2.7cm and 0.6cm);
\draw (0.8,1.6) node[anchor=north west] {$1$};
\draw (1,-0.4) node[anchor=north west] {$5$};
\draw (2.4,-0.2) node[anchor=north west] {$101$};
\draw (4.8,0.7) node[anchor=north west] {$-5$};
\draw (1.2,-1.7) node[anchor=north west] {$0$};
\draw (1.2,3.1) node[anchor=north west] {$-14$};
\draw (4.1,-1) node[anchor=north west] {$75$};
\draw (4.2,-2.4) node[anchor=north west] {$\frac{1}{3}$};
\draw (7.3,1.4) node[anchor=north west] {$\frac{5}{2}$};
\draw (-1.4,0.9) node[anchor=north west] {$-3$};
\draw (0.4,-3.1) node[anchor=north west] {$-4$};
\draw (2.4,4.7) node[anchor=north west] {$3$};
\draw (-1.3,4.1) node[anchor=north west] {$\frac{5}{7}$};
\draw (-2.3,-1) node[anchor=north west] {$4$};
\draw (4.5,-5.6) node[anchor=north west] {$\pi$};
\draw (10.7,-3) node[anchor=north west] {$\sqrt[]{2}$};
\draw (6,-4.7) node[anchor=north west] {$-\frac{\pi}{2}$};
\draw (9.8,1.6) node[anchor=north west] {$\frac{5}{2}$};
\draw (11.1,4) node[anchor=north west] {$\frac{1}{3}$};
\draw (4.4,7.3) node[anchor=north west] {$3$};
\draw (-0.1,-5.3) node[anchor=north west] {$0$};
\draw (-4.9,1.5) node[anchor=north west] {$-5$};
\draw (11.8,-0.9) node[anchor=north west] {$-\frac{\pi}{2}$};
\draw (-0.6,6.8) node[anchor=north west] {$75$};
\draw (-3.7,-3) node[anchor=north west] {$-3$};
\end{tikzpicture}

Lukualueita voidaan laajentaa lisää vielä tästäkin, esimerkiksi \emph{imaginaariluvut} ovat lukuja, jotka eivät ole reaalilukuja, vaan lukuja, joita ei voi edes sijoittaa lukusuoralle. Reaalilukuja ja imaginaarilukuja kutsutaan yhdessä \emph{kompleksiluvuiksi}.

Kompleksilukuja tarvitaan mm. insinöörialoilla yliopistoissa ja ammattikorkeakouluissa. Esimerkiksi vaihtosähköpiirien analyysissä, signaalinkäsittelyssä ja säätötekniikassa käytetään runsaasti kompleksilukuja. Kompleksiluvut ovat tärkeitä myös matematiikan tutkimuksessa itsessään.
