
\chapter{Esipuhe}

%%%%%%%%%%%%%%%%%%%%%%%%%%%%%%%%%%%%%%%%%%%%%%%%%%%%%%%%%%%%%%%%%%%%%%%%%%%%%%%%
%%%%  /usr/share/doc/texlive-fonts-extra-doc/fonts/arev/mathtesty.tex

% mathtesty.tex, by Stephen Hartke 20050522
% based on mathtestx.tex in the mathptmx package
% and symbols.tex by David Carlisle

Matematiikka tarjoaa työkaluja asioiden täsmälliseen jäsentämiseen, päättelyyn ja mallintamiseen. Alasta riippuen käsittelemme matematiikassa erilaisia \textbf{objekteja}: Geometriassa tarkastelemme kaksiulotteisia \textbf{tasokuvioita} ja kolmiulotteisia \textbf{avaruuskappaleita}. Algebra tutkii \textbf{lukualueita} ja niissä määriteltäviä \textbf{laskutoimituksia}. Todennäköisyyslaskenta tarkastelee satunnaisten \textbf{tapahtumien} esiintymistä. Matemaattinen analyysi tutkii \textbf{funktioita} ja niiden ominaisuuksia, esimerkiksi \textbf{jatkuvuutta}, \textbf{derivoituvuutta} ja \textbf{integroituvuutta}. Matemaattista analyysiä käsittelevät pitkässä matematiikassa kurssit 7, 8, 10 ja 13 sekä osin kurssit 9 ja 12. Voidaankin sanoa, että analyysi on keskeisin aihealue lukion pitkässä matematiikassa.\footnote[1]{Tämä Suomessa käytetty lähestymistapa on käytössä monissa länsimaissa. Sen sijaan esimerkiksi Balkanilla geometrialle ja matemaattiselle todistamiselle annetaan edelleen paljon suurempi painoarvo.}

Jokaiseen tarkastelukohteeseen liitetään myös niille ominaisia \textbf{operaatioita}. Tämä kurssi käsittelee lähinnä lukuja ja niiden operaatioita eli laskutoimituksia. Tässä kirjassa esittelemme luvun käsitteen ja yleisimmin käytetyt lukualueet laskutoimituksineen, ja jatkamme niistä \textbf{yhtälöihin} ja funktioihin.

Uusimpien (vuoden 2003) lukion opetussuunnitelman perusteiden mukaan pitkän matematiikan ensimmäisen kurssin tavoitteena on, että opiskelija
\begin{itemize}
\item vahvistaa yhtälön ratkaisemisen ja prosenttilaskennan taitojaan
\item syventää verrannollisuuden, neliöjuuren ja potenssin käsitteiden ymmärtämistään
\item tottuu käyttämään neliöjuuren ja potenssin laskusääntöjä
\item syventää funktiokäsitteen ymmärtämistään tutkimalla potenssi- ja eksponenttifunktioita
\item oppii ratkaisemaan potenssiyhtälöitä.
\end{itemize}

Opetussuunnitelman perusteet määrittelevät kurssin keskeisiksi sisällöiksi
\begin{itemize}
\item potenssifunktion
\item potenssiyhtälön ratkaisemisen
\item juuret ja murtopotenssin
\item eksponenttifunktion.
\end{itemize}

\section*{Kiitämme}
\begin{itemize}
\item Metropolia AMK
\item Tekniikan akateemisten liitto TEK
\item Onnibus
\item Kebab Pizza Service
\item Heikki Hakkarainen
\item Eveliina Heinonen
\item Jokke Häsä
\item Katleena Kortesuo
\item Hannu Köngäs
\item Senja Larsen
\item Jussi Partanen
\item Arto Piironen
\item Antti Ruonala
\item Anni Saarelainen
\item Tiina Salola
\item Kaj Sotala
\item Paula Thitz
\item Juhapekka ''Naula'' Tolvanen
\item Daniel Valtakari
\item Kaikki tärkeät tyypit, joiden nimi unohtui mainita!
\end{itemize}

%%%%%%%%%%%%%%%%%%%%%%%%%%%%%%%%%%%%%%%%%%%%%%%%%%%%%%%%%%%%%%%%%%%%%%%%%%%%%%%%
%%%% /usr/share/doc/texlive-doc-en/fonts/free-math-font-survey/source/textfragment.tex

%%% Local Variables: 
%%% mode: latex
%%% End: 
