\chapter*{Liitteet}
\addcontentsline{toc}{chapter}{Liitteet}

\renewcommand*{\thesection}{\Alph{section}}
\titleformat{\section}[hang]
{\sffamily\fontsize{15}{10}\selectfont\filleft}
{Liite \thesection}
{0.5em}
{\MakeUppercase}
[\vspace{0.5ex}{\titlerule[1pt]}]

\fancypagestyle{plain}{%
 \fancyhf{}
 \fancyfoot[RE]{{\sffamily\fontsize{9}{7} Liitteet}\hspace{1em}\thepage}
 \fancyfoot[LO]{\thepage\hspace{1em}{\sffamily\fontsize{9}{7} Liitteet}}
}
\fancyhf{}
\fancyfoot[RE]{{\sffamily\fontsize{9}{7} Liitteet}\hspace{1em}\thepage}
\fancyfoot[LO]{\thepage\hspace{1em}{\sffamily\fontsize{9}{7} Liitteet}}

\setcounter{section}{0}

\section{Vastaukset}
\input{ans1}

\section{Harjoituskokeita}

% keskeneräistä

\subsection*{Harjoituskoe 1}

\begin{enumerate}
\item Ratkaise epäyhtälöt.\\ a) $1-\dfrac{1-x}{6}<x$\\ b) $(x+1)(x^2-2x-1)\geq0$
\item Ratkaise yhtälöt.\\ a) $4x^2-1=0$\\ b) $x^3=-3x$\\ c) $2y^2=y-8$
\item Millä parametrin $k$:n arvoilla yhtälöllä $kx^2-(k+1)x+1=0$ on kaksi erisuurta reaalijuurta? 

\end{enumerate}

\subsection*{Harjoituskoe 2}

\begin{enumerate}
\item Ratkaise yhtälöt.\\ a) $x^2-5x=0$\\ b) $x^4-1=0$\\ c) $(x-1)(x+4) = x(x-5)$
\item Ratkaise epäyhtälöt.\\ a) $x^2-8\geq0$\\ b) $x^2-8\geq(x-3)^2$\\ c) $x^2-6x+9\leq0$
\item Tiedämme, että $(a+b)^2=12$ ja $(a-b)^2=4$. Ratkaise tulo $ab$.
\item Kolmannen asteen polynomifunktiolle pätee $P(-1)=0$, $P(0)=0$ ja $P(1)=0$. Lisäksi $P(3)=3$. Määritä polynomi $P$.
\item Millä vakion $t$ arvoilla yhtälöllä $tx^2+tx-6=0$ ei ole ratkaisuja?
\end{enumerate}


\subsection*{Harjoituskoe 3}


\subsection*{Harjoituskoe 4}




\section{Todistuksia}

\subsection*{Tulon nollasääntö}
\label{tod:tulonolla}

\begin{todistus}
Annettuna joukko nollasta poikkeavia lukuja $x_0, x_1, x_2 ... , x_n$ asetetaan
\begin{align*}
    x_0 \cdot x_1 \cdot x_2 \cdot ... \cdot x_n &= 0 & \ppalkki & : (x_0 \cdot x_1 \cdot x_2 \cdot ... \cdot x_{n-1}) \\
    & & & \text{koska kaikille $x_i$ pätee $x_i \neq 0$} \\
    \\
    \frac{x_0 \cdot x_1 \cdot x_2 \cdot ... \cdot x_n}{x_0 \cdot x_1 \cdot x_2 \cdot ... \cdot x_{n-1}} &=
    \frac{0}{x_0 \cdot x_1 \cdot x_2 \cdot ... \cdot x_{n-1}} & \ppalkki & \frac{0}{x_i} = 0\ \text{kaikilla $x_i,\ x_i \neq 0$} \\
    \\
    x_n &= 0 & &
\end{align*}

mikä on ristiriidassa todistuksessa asetetun vaatimuksen kanssa (kaikkien
lukujen piti olla nollasta poikkeavia). Näin alkuperäinen väite on tosi.

\end{todistus}

\section*{Toisen asteen polynomin kuvaaja}

Tässä liitteessä perustellaan, miksi kaikkien toisen asteen funktioiden kuvaajat näyttävät samalta. Lisäksi tarkastellaan, mitkä tekijät vaikuttavat kuvaajien muotoon.

\underline{Funktio $P(x)=x^2$}

Aloitetaan funktiosta $P(x)=x^2$. Mitä tiedämme siitä piirtämättä kuvaajaa?
\begin{itemize}
\item Algebrallisen päättelyn avulla voimme todeta, että funktion pienin arvo on $P(0) = 0$, sillä tulon merkkisäännön perusteella lauseke $x^2$ ei voi saada negatiivisia arvoja – ei ole olemassa sellaista reaalilukua $x$, jonka neliö olisi negatiivinen. (Kompleksiluvuilla tällaista rajoitusta ei ole, tästä lisää liitteessä.) Kuvaajan alin kohta sijoittuu siis origoon ja kuvaajan kaikki muut pisteet sen yläpuolelle.
\item Kun $x > 0$, lauseke $x^2$  on sitä
suurempi, mitä suurempi $x$ on. Tästä tiedämme, että nollasta oikealle siirryttäessä funktion kuvaaja nousee.
\item Koska $(-x)^2 = x^2$, kuvaaja on symmetrinen $y$-akselin suhteen.
\end{itemize}

Näiden tietojen avulla voimme päätellä, että funktion kuvaaja koostuu kahdesta
identtisestä haarasta, jotka kohtaavat, kun $x=0$. Mitä kauempana $x$ on nollasta, sitä suurempia ovat funktion arvot. Tämän kaiken voi päätellä jo ennen kuvaajan piirtämistä.

Merkitsemällä koordinaatistoon yhtälön $y=x^2$ toteuttavia pisteitä, muodostuu lopulta funktion kuvaaja:
\begin{center}
\begin{kuvaajapohja}{2}{-2}{2}{-1}{4}
  \kuvaaja{x**2}{$f(x)=x^2$}{blue}
\end{kuvaajapohja}
\end{center}

Kuvaajan muoto on geometriselta nimeltään \emph{paraabeli}. Paraabeleja esiintyy monessa yhteydessä: esimerksi polttopeilin ja radioteleskoopin pinta kaareutuu paraabelin muotoisena. Samoin ilmaan heitetyn kappaleen rata on likimain (ylösalainen) paraabeli, kun ilmanvastus on pieni.

\underline{Funktio $P(x)=ax^2, \quad a\neq 0$}

Polynomien $P(x)=ax^2$ arvot ovat ovat lausekkeeseen $x^2$ nähden $a$-kertaisia. Muuten paraabelin symmetrisyys ja muut keskeiset ominaisuudet säilyvät.

\begin{itemize}
\item Kun $a > 0$, myös $ax^2\geq 0$, joten pienin arvo on yhä $0$.\\
Paraabeli on \termi{ylöspäin aukeava}.
\item Kun $a < 0$, tulon merkkisäännön nojalla $ax^2 \leq 0$. \\
 Nyt $P(0)=0$ onkin suurin arvo, ja funktion arvot ovat sitä pienempiä,
mitä enemmän $x$ poikkeaa nollasta. \\
Paraabeli on \termi{alaspäin aukeava}.
\item Mitä enemmän $a$ poikkeaa nollasta, sitä nopeammin funktion arvot
muuttuvat $x$:n muuttuessa ja sitä kapeampi paraabeli on.
\end{itemize}

\begin{center}
$a>0$, paraabeli aukeaa ylöpäin:\\
\begin{tabular}{cc}
$f(x)=\frac{1}{2}x^2$& $f(x)=2x^2$ \\
\begin{kuvaajapohja}{1}{-2}{2}{-1}{4}
  \kuvaaja{0.5*x**2}{}{blue}
\end{kuvaajapohja} &
\begin{kuvaajapohja}{1}{-2}{2}{-1}{4}
  \kuvaaja{2*x**2}{}{blue}
\end{kuvaajapohja}
\end{tabular}

$a<0$, paraabeli aukeaa alaspäin:\\
\begin{tabular}{cc}
$f(x)=-\frac{1}{2}x^2$ & $f(x)=-2x^2$ \\
\begin{kuvaajapohja}{1}{-2}{2}{-4}{1}
  \kuvaaja{-0.5*x**2}{}{blue}
\end{kuvaajapohja} &
\begin{kuvaajapohja}{1}{-2}{2}{-4}{1}
  \kuvaaja{-2*x**2}{}{blue}
\end{kuvaajapohja}
\end{tabular}
\end{center}

\underline{Funktio $P(x)=ax^2+c$}

Lisäämällä vakiotermi $c$ saadaan $P(x)=ax^2+c$. Vakion lisääminen nostaa tai laskee funktion kuvaajaa (riippuen siitä, onko $c > 0$ tai $c<0$), joten kuvaajan muoto ei muutu.

\underline{Funktio $P(x)=ax^2+bx+c$}

Miksi sitten täydellisen toisen asteen polynomin $P(x)=ax^2+bx+c$ kuvaaja on myös paraabeli? Muokataan lauseketta, aloitetaan ottamalla yhteinen tekijä:
\begin{align*}
P(x) &=ax^2+bx+c \\
&= a\left(x^2 +\frac{b}{a}x\right) + c  \quad &
\text{ lavennetaan } \frac{b}{a} \text{ kahdella} \\
&= a\left(x^2 +2\cdot x \cdot \frac{b}{2a} \quad\right) + c  &
\text{ täydennetään neliöksi lisäämällä} \left( \frac{b}{2a} \right)^2 \\
&= a \Bigg( \underbrace{x^2 +2\cdot x \cdot \frac{b}{2a}+\left(\frac{b}{2a} \right)^2}_{\left( x+\frac{b}{2a} \right)^2}
- \left(\frac{b}{2a}\right)^2 \Bigg)  + c \\
&= a \left( \left( x + \frac{b}{2a} \right )^2-\frac{b^2}{4a^2} \right) + c &
\text{ kerrotaan sulut auki } \\
&= a \underbrace{\left(  x + \frac{b}{2a} \right)^2}_{\text{neliö}}-
\underbrace{\frac{b^2}{4a} + c}_{\text{vakio}}
\end{align*}

Tästä neliöksi täydennetyksi muodosta nähdään, että $P(x)$ on muotoa
$a\cdot$neliö + vakio. Kuvaaja on siis samanlainen kuin tapauksessa
$ax^2+c$, huippu on vain siirtynyt.
Koska neliö $\geq 0$, saadaan

\begin{itemize}
\item Kun $a>0$, kyseinen vakio on polynomin pienin arvo ja kuvaaja on
ylöspäin aukeava paraabeli.
\item Kun $a<0$, kyseinen vakio on suurin arvo ja kuvaaja on alaspäin
aukeava paraabeli.
\end{itemize}

Paraabelin \termi{huippu} (eli kuvaajan piste, jossa suurin tai pienin arvo
saavutetaan) on aina kohdassa
$x=-\frac{b}{2a}$, koska silloin neliö on nolla.

%Toisen asteen polynomifunktio on muotoa $f(x)=ax^2+bx+c$, jossa $a,b,c \in \mathbb{R}$ ja $a \neq 0$. Toisen asteen polynomifunktion kuvaaja on paraabeli. Toisen asteen polynomifunktioita käytetään matemaattisessa mallinnuksessa talouden, tieteen ja tekniikan aloilla. Esimerkiksi heittoliikkeessä olevan kappaleen lentorata on aina paraabelin muotoinen. \\
%\textbf{Esimerkki 1.}
%a) Piirrä funktion $f(x)=x^2-2$ kuvaaja. \\
%b) Ratkaise funktion $f$ nollakohdat. \\ \\
%
%\begin{kuvaajapohja}{1}{-2}{2}{-3}{1}
%  \kuvaaja{x**2-2}{$f(x)=x^2-2$}{blue}
%\end{kuvaajapohja}
%
%Funtkion kuvaaja on ylöspäin aukeava paraabeli, joka leikkaa x-akselin kohdissa joissa $f(x)=0$. \\
%Graafisesti funktion nollakohdat saadaan ratkaistua kuvaajasta. Kuvaajasta nähdään, että funktion nollakohdat ovat $x_1 \approx 1,4$ ja $x_2 \approx -1,4$. \\
%Algebrallisesti saadaan ratkaistua, että funktion nollakohdat ovat
%\begin{align*}
%f(x)&=0 \\
%x^2-2&=0 \\
%x^2&=2 \\
%x&= \pm \sqrt[]{2}
%\end{align*}
%Funktion $f$ kuvaajasta huomataan, että se on symmetrinen y-akselin suhteen.
%
%\textbf{Esimerkki 2.} \\
%Piirrä funktioiden $f(x)=x^2+1$, $g(x)=2x^2$ ja $h(x)=\frac{1}{3}x^2$ kuvaajat. \\ \\
%
%\begin{kuvaajapohja}{1}{-2}{2}{-1}{3}
%  \kuvaaja{x**2+1}{$f(x)=x^2+1$}{blue}
%\end{kuvaajapohja}
%
%
%\begin{kuvaajapohja}{1}{-2}{2}{-1}{3}
%  \kuvaaja{2*x**2}{$g(x)=2x^2$}{blue}
%\end{kuvaajapohja}
%
%\begin{kuvaajapohja}{1}{-2}{2}{-1}{3}
%  \kuvaaja{(1 / 3.0)*(x**2)}{$h(x)=\frac{1}{3}x^2$}{blue}
%\end{kuvaajapohja}
%
%Mitä tapahtuu funktion kuvaajan muodolle, kun termin $ax^2$ kerroin $a$ muuttuu? \\ \\
%
%\textbf{Esimerkki 3.} \\
%Piirrä funktioiden $f(x)=-x^2+x$, $g(x)=-x^2+2x+1$ ja $h(x)=-x^2+\frac{1}{2}x-1$ kuvaajat. \\
%\missingfigure \\
%Mitä tapahtuu funktion kuvaajan muodolle, kun termin $bx$ kerroin $b$ muuttuu? \\ \\
%
%\textbf{Esimerkki 4.} \\
%Piirrä funktioiden $f(x)=x^2$, $g(x)=x^2-2$ ja $h(x)=x^2+\frac{3}{2}$ kuvaajat. \\ \\
%Mitä tapahtuu funktion kuvaajan muodolle, kun vakiotermi $c$ muuttuu? \\ \\

Koottuna:

\laatikko{Toisen asteen polynomifunktion $f(x)=ax^2+bx+c$ kuvaaja on
\begin{itemize}
\item ylöspäin aukeava paraabeli, kun $a>0$
\item alaspäin aukeava paraabeli, kun $a<0$
\item sitä kapeampi, mitä suurempi $|a|$ on.
\end{itemize}
}

\todo{selitys tulisi hoitaa ilman itseisarvoa}

%\textbf{Esimerkki 5.} \\
%Ratkaise funktion $f(x)=4x^2-13x+8$ nollakohdat.
%\begin{align*}
%f(x)&=0 \\
%4x^2-13x+8&=0 \\
%x&=\frac{-(-13) \pm \sqrt[]{(-13)^2-4 \cdot 4 \cdot 8}}{2 \cdot 4} \\
%x&=\frac{13 \pm \sqrt[]{169-128}}{8} \\
%x&=\frac{13 \pm \sqrt[]{41}}{8} \\
%x&=\frac{13 \pm \sqrt[]{41}}{8}
%\end{align*}




%Kuvaajaan transformaatioihin esimerkkiä myös muuttujan x muuttamisesta. Piirrä f(x+1):n kuvaaja jne.
%funktion nollakohtien ja yhtälön juurien yhteys
%paraabelin muodon perustelu: P(x)=ax^2+bx+c=a(x-b/2a)^2+b^2/4a^2, siis pienin/suurin arvo kun x=-b/2a -> ylöspäin ja alaspäin aukeavat paraabelit

\subsection*{Polynomien jakolause}
\label{tod:poljako}

Jakolauseen todistus perustuu polynomien jakoyhtälöön, josta tarkemmin kurssilla 12.

\begin{todistus}
Vaikka lauseke $x-b$ ei olisi polynomin $P(x)$ tekijä, niin lähelle päästään: jos polynomin $Q(x)$ kertoimet valitaan sopivasti, voidaan kirjoittaa
\begin{align*}
P(x)&=(x-b)Q(x)+r,
\end{align*}
missä $r$ on jokin vakio, niin sanottu jakojäännös. Jos nyt $b$ on polynomin $P$ nollakohta, sijoitetaan edelliseen yhtälöön $x=b$, jolloin
\begin{align*}
P(b)&=(b-b)Q(b)+r \quad || \ \ P(b)=0 \\
0&=0+r,
\end{align*}
eli $r=0$, joten $x-b$ on polynomin $P(x)$ tekijä. Jos siis $x=b$ on polynomin $P(x)$ nollakohta, $x-b$ on sen tekijä.
\end{todistus}


\chapter{Tehtäviä ylioppilaskokeista}

\subsection*{Lyhyen oppimäärän tehtäviä}

\begin{itemize}
  \item[] (k2011, 1a) Ratkaise yhtälö $4x+(5x-4) = 12+3x$.
  \item[] (k2011, 1b) Sievennä lauseke $x^2+x-(x^2-x)$.
  \item[] (k2011, 2b) Sievennä lauseke $(\sqrt{x}-1)^2+2\sqrt{x}$.
  \item[] (s2011, 1c) Ratkaise yhtälö $x^2-3(x+3) = 3x-18$.
  \item[] (k2012, 1a) Ratkaise yhtälö $7x+3 = 31$.
  \item[] (s2012, 1a) Ratkaise yhtälö $x^2-2x = 0$.
\end{itemize}

\subsection*{Pitkän oppimäärän tehtäviä}

\begin{itemize}
  \item[] (K2011, 1b) Ratkaise epäyhtälö $x^2-2 \leq x$.
  \item[] (S2011, 1b) Suorakulmaisen kolmion hypotenuusan pituus on 5
   ja toisen kateetin pituus 2. Laske toisen kateetin pituus.
  \item[] (S2011, 3b) Ratkaise epäyhtälö $\frac{2x+1}{x-1} \geq 3$.
  \item[] (S2012, 1a) Ratkaise yhtälö $2(1-3x+3x^2) = 3(1+2x+2x^2)$.
\end{itemize}






\chapter{Sekalaisia tehtäviä}

Rymittele tehtävät luvuittain!
Sijoita nämä tehtävät sopivaan paikkaan, jos ovat hyviä!


\subsubsection*{Polynomifunktion kuvaaja}

\begin{tehtava}
  Aukeavatko seuraavat paraabelit ylös- vai alaspäin?
  \begin{enumerate}[a)]
    \item $4x^2 + 100x - 3$
    \item $-x^2 + 1337$
    \item $5x^2 - 7x + 5$
    \item $-6(-3x^2 + 5)$
    \item $-13x(9 - 17x)$
    \item $100(1-x^2)$
  \end{enumerate}

  \begin{vastaus}
    \begin{enumerate}[a)]
      \item Ylös
      \item Alas
      \item Ylös
      \item Ylös
      \item Ylös
      \item Alas
    \end{enumerate}
  \end{vastaus}
\end{tehtava}

\begin{tehtava}
  \begin{enumerate}[a)]
    \item Ratkaise funktion $2x^2 - 5x - 3$ nollakohdat
    \item Millä arvoilla edellisen kohdan funktio $2x^2 - 5x - 3$ saa positiivisia arvoja?
    \item Onko em. funktiolla globaali raja-arvo (minimi tai maksimi), ja jos on, missä kohtaa funktio saa tämän arvon? Mikä on funktion arvo silloin?
  \end{enumerate}

  \begin{vastaus}
    \begin{enumerate}[a)]
      \item $x = 1.2$ tai $x = -0.2$
      \item $x = \frac{12}{10} = 1.2$ tai $x = -\frac{2}{10} = -0.2$
      \item Koska neliötermin kerroin a on positiivinen (2), funktiolla on globaali minimi (mutta ei ylärajaa). Symmetrian vuoksi minimi on nollakohtien puolivälissä kohdassa 0.5, jossa funktio saa siis pienimmän arvonsa -5.
    \end{enumerate}
  \end{vastaus}
\end{tehtava}

\begin{tehtava}
  Tutki, millä muuttujan x arvoilla seuraavat funktiot saavat positiivisia arvoja.
  \begin{enumerate}[a)]
    \item $x^2 - 4$
    \item $-x^2 - 2x + 3$
    \item $x^2 + 2x + 5$
    \item $-x^2 - 1$
  \end{enumerate}

  \begin{vastaus}
    \begin{enumerate}[a)]
      \item $x \leq -2$ tai $x \geq 2$
      \item $-3 \geq x \leq 1$
      \item Kaikilla x:n arvoilla.
      \item Ei millään x:n arvoilla.
    \end{enumerate}
  \end{vastaus}
\end{tehtava}

\section{Tavoittele valaistumista}

\Opensolutionfile{ans}[ans2]

Tässä on joitakin tehtäviä, jotka on arvioitu hauskoiksi ja hyödyllisiksi kaikkein osaavimmille opiskelijoille. Niiden parissa vietetty aika ei mene hukkaan, vaikkei tehtävä ratkeaisikaan.

\begin{tehtava}
    \begin{enumerate}[a)]
        \item Osoita, että jos polynomi $P$ jaetaan polynomilla $x-1$, niin jakojäännös on polynomin $P$ kertoimien summa.
        \item Päättele tästä, että kokonaisluku $n$ on jaollinen yhdeksällä vain jos sen kymmenjärjestelmäesityksen numeroiden summa on jaollinen yhdeksällä.
    \end{enumerate}
\end{tehtava}

\begin{tehtava} %Ehkä käsitteellisesti vaikea
    $P$ on toisen asteen polynomi, jonka vakiotermi on $1$. Polynomi $Q$ määritellään lausekkeella $Q(x)=P(x+1)-P(x)$ ja siitä tiedetään, että $Q(0)=7$ ja $Q(1)=13$. Määritä polynomin $P$ lauseke.
    \begin{vastaus}
        $P(x) = 3x^2+4x+1$
    \end{vastaus}
\end{tehtava}

\begin{tehtava} %Vaikea!
    Etsi kaikki positiiviset kokonaisluvut $x$ ja $y$, joille pätee $9x^2-y^2=17$.
    \begin{vastaus}
    Opastus: jaa yhtälön vasen puoli tekijöihin muistikaavalla. 
    Ainoa ratkaisu on $x = 3$, $y=8$.
    \end{vastaus}
\end{tehtava}

\begin{tehtava} % Sikavaikea (tällä tasolla)
%täydentyy kahdeksi neliöksi, joiden summa on 0
    Ratkaise $x$ ja $y$ yhtälöstä $y^2+2xy+x^4-3x^2+4=0$.
    \begin{vastaus}
        $x=\sqrt{2}, y=-\sqrt{2}$ tai $x=-\sqrt{2}, y=\sqrt{2}$
    \end{vastaus}
\end{tehtava}

\begin{tehtava} % Kaunis
    Ratkaise $x$ ja $y$ yhtälöstä $2x^4+2y^4=4xy-1$. %lisää tai vähennä kiva termi puolittain
    \begin{vastaus}
        $x=\frac{\sqrt{2}}{2}, y=\frac{\sqrt{2}}{2}$ tai $x=-\frac{\sqrt{2}}{2}, y=-\frac{\sqrt{2}}{2}$
    \end{vastaus}
\end{tehtava}

\Closesolutionfile{ans}[ans2]

\subsubsection*{Vastaukset}

\begin{Vastaus}{89}
    Opastus: jaa yhtälön vasen puoli tekijöihin muistikaavalla.
    Ainoa ratkaisu on $x = 3$, $y=8$.
    
\end{Vastaus}
\begin{Vastaus}{90}
        $x=\sqrt{2}, y=-\sqrt{2}$ tai $x=-\sqrt{2}, y=\sqrt{2}$
    
\end{Vastaus}
\begin{Vastaus}{91}
        $x=\frac{\sqrt{2}}{2}, y=\frac{\sqrt{2}}{2}$ tai $x=-\frac{\sqrt{2}}{2}, y=-\frac{\sqrt{2}}{2}$
    
\end{Vastaus}
