\section{Toisen asteen polynomifunktio}

\qrlinkki{http://opetus.tv/maa/maa2/toisen-asteen-polynomifunktio/}{Opetus.tv: \emph{toisen asteen polynomifunktio} (7:59)}

Toisen asteen polynomifunktio on muotoa
\begin{align*}
ax^2+bx+c,
\end{align*}
missä vakiot $b$ ja $c$ voivat olla mitä tahansa reaalilukuja $(b, \ c \in \mathbb{R})$ ja $a$ voi olla mikä tahansa reaaliluku, paitsi luku nolla $(a \in \mathbb{R}, \ a \neq 0)$.

Toisen asteen polynomifunktion kuvaajaa nimitetään \termi[paraabeliksi]{paraabeli}. Jos toisen asteen polynomifunktiossa $a < 0$ sanomme sen kuvaajaa
alaspäin aukeavaksi paraabeliksi ja vastaavasti, kun $a > 0$, kuvaajaa nimitetään ylöspäin aukeavaksi paraabeliksi.

% FIXME: kuvat alaspäin ja ylöspäin aukeavista paraabeleista
