\section{Polynomifunktion kuvaaja}
Polynomifunktiota voi
havainnollistaa koordinaatistoon piirretyn kuvaajan avulla:

%\begin{kuvaajapohja}{1.5}{-2}{2}{-3}{3}
%\kuvaaja{-x-1}{$P(x) = -x-1$}{red}
%\kuvaaja{x**2+x}{$Q(x) = x^2+x$}{blue}
%\kuvaaja{x**3-3*x-1}{$R(x) = x^3-3x-1$}{green}
%\end{kuvaajapohja}

% \begin{kuvaajapohja}{1.5}{-2}{2}{-3}{3}
% %\kuvaaja{-x-1}{$P(x) = -x-1$}{red}
% \kuvaaja{x**2+x}{$Q(x) = x^2+x$}{black}
% \kuvaaja{x**3-3*x-1}{$R(x) = x^3-3x-1$}{black}
% \end{kuvaajapohja}
% %

\begin{kuva}
kuvaajapohja(-2, 2, -3, 3, korkeus = 9, nimiX = '$x$')
kuvaaja("x**2+x", nimi = "$Q(x) = x^2+x$", kohta = 1, suunta = (1, -1))
kuvaaja("x**3-3*x-1", nimi = "$R(x) = x^3-3x-1$", kohta = 1.8)
\end{kuva}

\subsection*{Kuvaajan piirtäminen}

\qrlinkki{http://opetus.tv/maa/maa2/suoran-piirtaminen/}{Opetus.tv: \emph{suoran piirtäminen} (5:47)}

Funktioiden kuvaajia voi piirtää laskimella tai tietokoneella. Kuvaajaa voi myös hahmotella käsin laskemalla funktion arvoja muutamilla muuttujan arvoilla, merkitsemällä pisteet
$xy$-koordinaatistoon ja piirtämällä lopuksi kuvaajaan pisteiden kautta kulkeva viiva.

% Esimerkin kuvat.
\begin{luoKuva}{esimkuva1}
kuvaajapohja(-4, 4, -3, 6, leveys = 3.5, nimiX = "$x$")
piste((-3, 5.5))
piste((-2, 2))
piste((-1, -0.5))
piste((0, -2))
piste((1, -2.5))
piste((2, -2))
piste((3, -0.5))
\end{luoKuva}
\begin{luoKuva}{esimkuva2}
kuvaajapohja(-4, 4, -3, 6, leveys = 3.5, nimiX = "$x$")
piste((-3, 5.5))
piste((-2, 2))
piste((-1, -0.5))
piste((0, -2))
piste((1, -2.5))
piste((2, -2))
piste((3, -0.5))
kuvaaja("0.5*x**2-x-2")
\end{luoKuva}

\begin{esimerkki}
Hahmotellaan polynomifunktion $f(x) = \dfrac{1}{2}x^2 - x - 2$ kuvaaja.
Lasketaan ensin joitakin funktion $f(x) = \dfrac{1}{2}x^2 - x - 2$ arvoja ja piirretään niitä vastaavat pisteet
koordinaatistoon. Lopuksi hahmotellaan kuvaaja, joka kulkee pisteiden kautta.

\begin{tabular}{c c c}
	\begin{tabular}{|c|r @{,} l|}
	\hline $x$ & \multicolumn{2}{c|}{$f(x)$} \\
	\hline
	-3 & 5&5 \\
	-2 & 2&0 \\
	-1 & -0&5 \\
	0 & -2&0 \\
	1 & -2&5 \\
	2 & -2&0 \\
	3 & -0&5 \\
	\hline
	\end{tabular}
	&
	\vcent{\naytaKuva{esimkuva1}}
	&
	\vcent{\naytaKuva{esimkuva2}}
\end{tabular}
% 
% \begin{tabular}{c c c}
% 	\begin{tabular}{|c|r @{,} l|}
% 	\hline $x$ & \multicolumn{2}{c|}{$f(x)$} \\
% 	\hline
% 	-3 & 5&5 \\
% 	-2 & 2&0 \\
% 	-1 & -0&5 \\
% 	0 & -2&0 \\
% 	1 & -2&5 \\
% 	2 & -2&0 \\
% 	3 & -0&5 \\
% 	\hline
% 	\end{tabular}
% 	&
% 	\vcent{\begin{kuvaajapohja}{0.6}{-4}{4}{-3}{6}
% 	\kuvaajapiste{-3}{5.5}
% 	\kuvaajapiste{-2}{2}
% 	\kuvaajapiste{-1}{-0.5}
% 	\kuvaajapiste{0}{-2}
% 	\kuvaajapiste{1}{-2.5}
% 	\kuvaajapiste{2}{-2}
% 	\kuvaajapiste{3}{-0.5}
% 	\end{kuvaajapohja}}
% 	&
% 	\vcent{\begin{kuvaajapohja}{0.6}{-4}{4}{-3}{6}
% 	\kuvaajapiste{-3}{5.5}
% 	\kuvaajapiste{-2}{2}
% 	\kuvaajapiste{-1}{-0.5}
% 	\kuvaajapiste{0}{-2}
% 	\kuvaajapiste{1}{-2.5}
% 	\kuvaajapiste{2}{-2}
% 	\kuvaajapiste{3}{-0.5}
% 	\kuvaaja{0.5*x**2-x-2}{}{black}
% 	\end{kuvaajapohja}}
% \end{tabular}

\end{esimerkki}

\subsection*{Kuvaajan tulkintaa}

%Ensimmäisen asteen polynomin kuvaaja on luonnollisesti aina suora. 
%miten niin luonnollisesti?

Kuvaajan avulla voidaan tehdä johtopäätöksiä funktion ominaisuuksista.
Esimerkiksi funktion arvoja voidaan lukea kuvaajasta.

\begin{esimerkki}
Seuraavassa on esitetty polynomifunktion $P(x)=-3x^2+2x+4$ kuvaaja.

\begin{kuvaajapohja}[\kuvaajaAsetusEiRuudukkoa]{0.7}{-3}{3}{-5}{5}
\kuvaajapiste{2}{-4}
\kuvaajakohtaarvo{2}{-4}{}{}
\kuvaaja{-3*x**2+2*x+4}{$P(x)$}{black}
\end{kuvaajapohja}

Kuvaajasta voi lukea funktion arvoja tai ainakin niiden likiarvoja. Kuvaajan perusteella näyttää siltä, että $P(2)=-4$. Näin todellakin on, sillä $P(1)=-3\cdot 1^2+2\cdot 1+4=1$.
\end{esimerkki}

\begin{esimerkki}
Kuvaajasta ei välttämättä näe tarkkoja arvoja. Seuraavassa on esitetty erään polynomifunktion $P(x)$ kuvaaja. Kuvaajan perusteella näyttäisi siltä, että $P(1)=1$, mutta tarkkaa arvoa kuvaajasta ei voi päätellä.
 
\begin{kuvaajapohja}[\kuvaajaAsetusEiRuudukkoa]{1}{-2}{2}{-2}{2}
\kuvaaja{19./20*x**2+19./20*x-1}{$P(x)$}{black}
\end{kuvaajapohja}
 
Itse asiassa edellinen kuvaaja kuuluu funktiolle $P(x)=\dfrac{19}{20} x^2-\dfrac{19}{20} x-1$. Nyt tiedetään, että funktion $P(x)$ arvo kohdassa $x=1$ on
$$\dfrac{19}{20}\cdot 1^2-\dfrac{19}{20} \cdot 1-1=\dfrac{9}{10}$$
eikä $1$, kuten kuvaajan perusteella voisi luulla. Kuvaajasta ei siis voi lukea tarkkoja tietoja funktiosta.
\end{esimerkki}

Funktion \termi[nollakohta]{nollakohta} on sellainen muuttujan arvo, jolla funktio saa arvon nolla. Esimerkiksi funktiolla $Q(x)=x^2-1$
on nollakohdat $-1$ ja $1$, sillä $Q(-1)=0$ ja $Q(1)=0$.

Funktion kuvaaja antaa tietoa nollakohdista. Niiden kohdalla kuvaaja leikkaa $x$-akselin.

\begin{esimerkki}
Funktion $P(x) = \dfrac{1}{3}x^2-4x+\dfrac{5}{2}$ kuvaajasta nähdään, että funktiolla on ainakin kaksi nollakohtaa. Toinen niistä on lähellä lukua $1$ ja toinen lukua $11$.

\begin{kuvaajapohja}{0.3}{-5}{15}{-10}{5}
\kuvaajapiste{0.66146}{0}
\kuvaajapiste{11.3385}{0}
\kuvaaja{1./3*x**2-4*x+5./2}{$P(x) = \dfrac{1}{3}x^2-4x+\dfrac{5}{2}$}{black}
\end{kuvaajapohja}
\end{esimerkki}

%Nollakohta tarkoittaa sitä annetun polynomin muuttujan arvoa, jolla koko
%polynomi saa arvon nolla. Kuvaajasta sen voi helposti lukea niinä kohtina,
%joissa kuvaaja leikkaa muuttujan koordinaattiakselin. Funktion $P(x)$ ja
%$xy$-koordinaatiston tapauksessa funktion nollakohdat ovat täsmälleen ne
%$x$-koordinaatit, joilla funktion kuvaaja leikkaa $x$-akselin.

%\subsection{Taylorin sarja}
%Eräs mielenkiintoinen ja hyvin tunnettu potenssisarja on Taylorin sarja.
%Se on päättymätön potenssisarja, jolla voidaan approksimoida muiden funktioiden
%arvoja.
%
%Yleisesti Taylorin sarjalla saadaan (rajatta derivoituvan) funktion $f$ arvo
%pisteessä $x_0$:
%
%\begin{align*}
%	f(x_0) = \sum\limits_{n=0}^\infty a_n(x-x_0)^n
%\end{align*}
%
%missä
%
%\begin{align*}
%a_n = \frac{f^n(x_0)}{n!}
%\end{align*}
%
%Koska sarja on äärettömän pitkä, sarjan arvoja edelleen arvioidaan Taylorin
%polynomilla, joka on muotoa
%
%\begin{align*}
%	P_k(x) = \sum\limits_{n=0}^k a_k(x-x_0)^k
%\end{align*}
%
%Polynomin avulla voidaan laskea esimerkiksi likiarvo funktiolle
%$(1-x)^{-1} = \frac{1}{1-x}$ pisteen a ympäristössä, kun $a \neg 1$:
%
%\begin{align*}
%	\frac{1}{1-x} \approx \frac{1}{1-a} + \frac{x-a}{(1-a)^2} +
%\frac{(x-a)^2}{(1-a)^3} + \frac{(x-a)^3}{(1-a)^4} ...
%\end{align*}
%
%\missingfigure{Funktion $(x-1)^-1$ kuvaaja}
%\missingfigure{Funktion $\frac{1}{1-a} + \frac{x-a}{(1-a)^2} +
%\frac{(x-a)^2}{(1-a)^3} +$ kuvaaja}

\begin{tehtavasivu}

\begin{tehtava}
    Piirrä polynomien kuvaajat.
    \begin{enumerate}[a)]
        \item $5$
        \item $2x-3$
        \item $x^2+x-2$
        \item $x^3-x+3$
    \end{enumerate}   
    \begin{vastaus}
    	\item \begin{kuvaajapohja}{0.4}{-4}{4}{-1}{7}
				\kuvaaja{5}{}{red}
			  \end{kuvaajapohja}
    	\item \begin{kuvaajapohja}{0.4}{-4}{4}{-5}{3}
				\kuvaaja{2*x-3}{}{red}
			  \end{kuvaajapohja}
		\item \begin{kuvaajapohja}{0.4}{-4}{4}{-3}{5}
				\kuvaaja{x**2+x-2}{}{red}
			  \end{kuvaajapohja}
		\item \begin{kuvaajapohja}{0.4}{-4}{4}{-3}{5}
				\kuvaaja{x**3-x+2}{}{red}
			  \end{kuvaajapohja}
    \end{vastaus}
\end{tehtava}

\begin{tehtava}
    Piirrä polynomien kuvaajat.
    \begin{enumerate}[a)]
        \item $x+4$
        \item $2x-9$
        \item $5x+2$
        \item $6x+1$
    \end{enumerate}
    \begin{vastaus}
        \item \begin{kuvaajapohja}{0.4}{-4}{4}{-1}{7}
				\kuvaaja{x+4}{}{red}
			  \end{kuvaajapohja}
    	\item \begin{kuvaajapohja}{0.4}{-2}{6}{-6}{2}
				\kuvaaja{2*x-9}{}{red}
			  \end{kuvaajapohja}
		\item \begin{kuvaajapohja}{0.4}{-4}{4}{-2}{6}
				\kuvaaja{5*x+2}{}{red}
			  \end{kuvaajapohja}
		\item \begin{kuvaajapohja}{0.4}{-4}{4}{-2}{6}
				\kuvaaja{6*x+1}{}{red}
			  \end{kuvaajapohja}
    \end{vastaus}
\end{tehtava}

\begin{tehtava}
    Piirrä polynomien kuvaajat.
    \begin{enumerate}[a)]
        \item $x^2-1$
        \item $2x^2$
        \item $4x^2+4$
        \item $x^2-6x+3$
    \end{enumerate}
    \begin{vastaus}
        \item \begin{kuvaajapohja}{0.4}{-4}{4}{-2}{6}
				\kuvaaja{x+4}{}{red}
			  \end{kuvaajapohja}
    	\item \begin{kuvaajapohja}{0.4}{-4}{4}{-1}{7}
				\kuvaaja{2*x-9}{}{red}
			  \end{kuvaajapohja}
		\item \begin{kuvaajapohja}{0.4}{-4}{4}{-1}{11}
				\kuvaaja{5*x+2}{}{red}
			  \end{kuvaajapohja}
		\item \begin{kuvaajapohja}{0.4}{-1}{7}{-7}{5}
				\kuvaaja{6*x+1}{}{red}
			  \end{kuvaajapohja}
    \end{vastaus}
\end{tehtava}

\begin{tehtava}
    Piirrä polynomien kuvaajat.
    \begin{enumerate}[a)]
        \item $5x^2$
        \item $3x^2+4$
        \item $2x^2-10$
        \item $x^2-x-1$
    \end{enumerate}
    \begin{vastaus}
        \item \begin{kuvaajapohja}{0.4}{-4}{4}{-1}{7}
				\kuvaaja{5*x**2}{}{red}
			  \end{kuvaajapohja}
    	\item \begin{kuvaajapohja}{0.4}{-4}{4}{-1}{11}
				\kuvaaja{3*x**2+4}{}{red}
			  \end{kuvaajapohja}
		\item \begin{kuvaajapohja}{0.4}{-4}{4}{-11}{1}
				\kuvaaja{2*x**2-10}{}{red}
			  \end{kuvaajapohja}
		\item \begin{kuvaajapohja}{0.4}{-3}{5}{-2}{6}
				\kuvaaja{x**2-x-1}{}{red}
			  \end{kuvaajapohja}
    \end{vastaus}
\end{tehtava}

\begin{tehtava}
	Monia funktioita voidaan esittää likimääräisesti polynomeina (ns.
Taylorin polynomi). Esimerkiksi

	\begin{tabular}{lcll}
	$\frac{1}{1+x^2}$ &$\approx$ & $1-x^2+x^4-x^6+x^8-x^{10}$, & kun
$-1<x<1$ \\
	$\sqrt{1+x}$ & $\approx $ & $ 1+\frac{x}{2}
	-\frac{x^2}{8}+\frac{x^3}{16}-\frac{5x^4}{128}$, & kun $-1<x<1$
	\end{tabular}

	Piirrä alkuperäinen funktio ja polynomi samaan kuvaajaan tietokoneella
tai graafisella laskimella. Kokeile, kuinka polynomin viimeisten termien pois
jättäminen vaikuttaa tarkkuuteen. Mitä havaitset? (Termejä voi laskea lisääkin,
mutta siihen ei puututa tässä.)

	\begin{vastaus}
		Mitä enemmän termejä, sitä parempi vastaavuus.
	\end{vastaus}
\end{tehtava}

\end{tehtavasivu}
