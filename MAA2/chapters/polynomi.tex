\section{Käsitteitä}

\qrlinkki{http://opetus.tv/maa/maa2/polynomien-peruskasitteet/}
{Opetus.tv: \emph{polynomien peruskäsitteet} (9:00)}

\qrlinkki{http://opetus.tv/maa/maa2/polynomin-tasmallinen-maaritelma/}
{Opetus.tv: \emph{polynomin täsmällinen määritelmä} (6:10)}

\subsection*{Polynomit}

\termi{polynomi}{Polynomit} ovat matematiikassa tärkeitä lausekkeita.
Polynomissa voi olla yksi tai useampia muuttujia, tai se voi olla muuttujaton vakiopolynomi.
% koostuu muuttujan $x$ ja vakioiden yhteen- ja kertolaskuista
Esimerkiksi seuraavat lausekkeet ovat polynomeja:
\begin{itemize}
\item $4$, ei muuttujia
\item $5x^2+x-7$, muuttujana $x$
\item $x^4-6x^3+2x^2-x$, muuttujana $x$
\item $-3x^{100}$, muuttujana $x$
\item $y$, muuttujana $y$
\item $2y-5$, muuttujana $y$
\item $y^2+1$, muuttujana $y$
\item $xy^2+x^2y$, muuttujina $x$ ja $y$
\end{itemize}
Polynomeissa on muuttujien potenssien (eksponentit luonnollisia lukuja) lisäksi ainoastaan vakiolla kertomista sekä yhteen- ja vähennyslaskua.
Polynomin yleinen muoto on
\[
a_n x^n + a_{n-1} x^{n-1} + \ldots + a_1 x + a_0 \; \textnormal{jollakin} \; n\in\N; \; a_0, a_1, \ldots, a_{n-1}, a_n \; \textnormal{ovat vakioita}.
\] % miten nollapolynomin laita? asteista ei puhuta tässä, joten a_n \neq 0 ei välttämätön
Jokainen lauseke, joka on esitettävissä polynomin yleisessä muodossa, on polynomi.
Esimerkiksi seuraavat muuttujan $x$ lausekkeet \emph{eivät ole} polynomeja:
\begin{itemize}
\item $2/x = 2x^{-1}$
\item $\sqrt{x}+2 = x^{\frac{1}{2}}+2$
\end{itemize}
Yhden muuttujan polynomeissa muuttujana käytetään tyypillisesti kirjainta $x$.
Yleisimmät muut kirjaimet muuttujien merkitsemiseen ovat $y$ ja $z$.

Polynomi on summalauseke, joiden yhteenlaskettavia kutsutaan \termi{termi}{termeiksi}.
Termejä edeltävät miinukset ymmärretään termin osaksi negatiivisina kertoimina.
Esimerkiksi polynomin $-2x^3+5x^2+x-7$ termit ovat $-2x^3$, $5x^2$, $x$ ja $-7$.
Termejä, jotka eivät sisällä muuttujaa, kutsutaan  \termi{vakiotermi}{vakiotermeiksi}.
Vakiotermejä ovat siis esimerkiksi $1$ ja $-7$.

Yhden ja kahden termin polynomeja kutsutaan \termi{monomi}{monomeiksi} ja \termi{binomi}{binomeiksi}, vastaavasti.

Muuttujan eksponenttia kutsutaan termin \termi{aste}{asteeksi} tai \termi{asteluku}{asteluvuksi}.
Vakiotermin aste on nolla. Polynomin aste on suurin sen termien asteista.
On tärkeää huomata, että polynomin asteen saaminen selville voi vaatia polynomin sieventämistä.
Esimerkiksi $2x^2-x^2+1-x^2 = 1$ ei ole toisen, vaan nollannen asteen polynomi.

Koska yhteenlasku on vaihdannainen, polynomin termit voi kirjoittaa missä tahansa järjestyksessä.
Esimerkiksi polynomi $y^4+y^2-1$ voidaan kirjoittaa myös järjestyksessä $-1+y^4+y^2$.
Yleensä polynomien termit kirjoitetaan niiden asteen perusteella laskevaan järjestykseen niin, että korkeimman asteen termi kirjoitetaan ensin.

\begin{esimerkki}
    Mitkä ovat polynomin $x^4-2x^3+5$ termit ja niiden asteet? Mikä on polynomin aste?
    \begin{esimvast}
        Polynomin $x^4-2x^3+5$ termit ovat $x^4$, $-2x^3$ ja $5$ ja niiden asteet ovat
        neljä, kolme ja nolla, vastaavasti. Polynomin aste on sama kuin sama kuin
        korkein termien asteista, eli neljä.
    \end{esimvast}
\end{esimerkki}

\subsection*{Polynomifunktion arvo}

\qrlinkki{http://opetus.tv/maa/maa2/polynomiesimerkkeja/}
{Opetus.tv: \emph{polynomiesimerkkejä} (6:59 ja 7:43)}

Polynomi määrittää funktion, jota kutsutaan \termi{polynomifunktio}{polynomifunktioksi}.
Esimerkiksi polynomi $2x^2+2x-1$ määrittää funktion $P(x)=2x^2+2x-1$.
Tämän funktion arvoja voidaan laskea sijoittamalla lukuja muuttujan $x$ paikalle:
\begin{align*}
    P(2) & = 5\cdot 2^2-3\cdot 2+2 = 20 - 6 + 2 = 16 \\
    P(-1) & = 5(-1)^2-3(-1)+2 = 5 + 3 + 2 = 10 \\
    P(-3) & = 5(-3)^2-3(-3)+2 = 45 + 9 + 2 = 56.
\end{align*}

Polynomeja ja polynomifunktioita käsitellään usein yhtäläisesti;
voidaan esimerkiksi sanoa ''polynomi $P(x)=2x+1$'',
vaikka tarkoitetaan vastaavaa polynomifunktiota.

\subsection*{Polynomien yhteen- ja vähennyslasku}

\qrlinkki{http://opetus.tv/maa/maa2/polynomien-yhteen-ja-vahennyslasku/}
{Opetus.tv: \emph{polynomien yhteen- ja vähennyslasku} (7:36)}

Polynomeja voidaan laskea yhteen summaamalla samanasteiset termit.
On kätevää aloittaa ryhmittelemällä samanasteiset termit vierekkäin.

\begin{esimerkki}
Laske polynomien $5x^2-x+5$ ja $3x^2-1$ summa.
    \begin{esimratk}
        \begin{align*}
            (\textcolor{blue}{5x^2} \textcolor{red}{{}-x} + 5) + (\textcolor{blue}{+3x^2} -1) 
            &=\textcolor{blue}{5x^2} \textcolor{red}{{}-x} + 5  \textcolor{blue}{{}+3x^2} -1 \\
            &=\textcolor{blue}{5x^2+3x^2} \textcolor{red}{{}-x} +5-1\\
            &=\textcolor{blue}{(5+3)x^2} \textcolor{red}{{}-x}+(5-1)\\
            &=\textcolor{blue}{8x^2} \textcolor{red}{{}-x}+4.
        \end{align*}
    \end{esimratk}
    \begin{esimvast}
        Polynomien summa on $8x^2-x+4$.
    \end{esimvast}
\end{esimerkki}

Polynomeja voidaan vastaavalla tavalla vähentää toisistaan.

\begin{esimerkki}
    Laske polynomien $14x^3+69$ ja $3x^3+2x^2+x$ erotus.
    \begin{esimratk}
        \begin{align*}
            (\textcolor{green}{14x^3} + 69) - (\textcolor{green}{3x^3} \textcolor{blue}{{}+ 2x^2} \textcolor{red}{{}+x})
            &= \textcolor{green}{14x^3} + 69 \textcolor{green}{{}-3x^3} - \textcolor{blue}{2x^2} \textcolor{red}{{}-x} \\
            &= \textcolor{green}{14x^3{}-3x^3} \textcolor{blue}{{}-2x^2} \textcolor{red}{{}-x} + 69 \\
            &= \textcolor{green}{(14{}-3)x^3} \textcolor{blue}{{}-2x^2} \textcolor{red}{{}-x} + 69 \\
            &= \textcolor{green}{11x^3} \textcolor{blue}{{}-2x^2} \textcolor{red}{{}-x} + 69
        \end{align*}
    \end{esimratk}
    \begin{esimvast}
        Polynomien erotus on $11x^3-2x^2-x+69$.
    \end{esimvast}
\end{esimerkki}

Samanasteisten termien yhteen- ja vähennyslasku perustuu siihen,
että reaalilukujen osittelulain (ks. Vapaa matikka 1) nojalla vakiokerroin voidaan ottaa yhteiseksi tekijäksi:
\[
ax^n+bx^n=(a+b)x^n.
\]

Polynomifunktioita nimetään tyypillisesti isoilla kirjaimilla, kuten $P$, $Q$ tai $R$.

\begin{esimerkki}
    Olkoot polynomit $P(x)=2x+1$ ja $Q(x)=3x^2-2x+5$. Määritä summa $R(x)=P(x)+Q(x)$.
    \begin{esimratk}
        \begin{align*}
            R(x) = P(x)+Q(x) &= (2x+1)+(3x^2-2x+5) \\
                             &= 2x+1+3x^2-2x+5 \\
                             &= 3x^2+2x-2x+1+5 \\
                             &= 3x^2+6.
        \end{align*}
    \end{esimratk}
    \begin{esimvast}
        $R(x) = 3x^2+6$.
    \end{esimvast}
\end{esimerkki}

Polynomit sievennetään yleensä yleiseen muotoon, jossa on vain yksi termi kutakin astetta kohti.
Tämä tehdään esimerkiksi silloin, kun selvitetään polynomin aste.

\begin{esimerkki}
    Laske polynomien $P$ ja $Q$ erotus $R$, kun $P(x)=-3x^4+x^2+1$ ja $Q(x)=-3x^4+3x^3-x$.
    Mikä on polynomin $R$ aste?
   \begin{esimratk}
        \begin{align*}
            R(x) = P(x)-Q(x) &= (-3x^4+x^2+1)-(-3x^4+3x^3-x) \\
                             &= -3x^4+x^2+1+3x^4-3x^3+x \\
                             &= -3x^4+3x^4-3x^3+x^2+x+1 \\
                             &= -3x^3+x^2+x+1.
        \end{align*}
    \end{esimratk}
    \begin{esimvast}
        $R(x) = -3x^3+x^2+x+1$. Polynomin $R$ aste on kolme.
    \end{esimvast}
\end{esimerkki}

\begin{tehtavasivu}

\paragraph*{Opi perusteet}

\begin{tehtava}
    Täydennä taulukko.
        
    \begin{tabular}{|c|c|c|c|c|}
                                                                         \hline
polynomi     & \begin{sideways}\begin{minipage}{3.5cm}termien\\lukumäärä\end{minipage}\end{sideways}%
& \begin{sideways}\begin{minipage}{3.5cm}korkeimman asteen\\termin kerroin\end{minipage}\end{sideways}%
& \begin{sideways}\begin{minipage}{3.5cm}polynomin\\asteluku\end{minipage}\end{sideways}%
& \begin{sideways}vakiotermi\end{sideways} \\ \hline
$-2x^2+6x$   &        2  &         $-2$      &       2   &    0       \\ \hline 
$7x^3-x-15$  &           &                   &           &            \\ \hline 
             &        2  &          $-9$     &       2   &    5       \\ \hline 
             &        3  &          $-1$     &       5   &    $-17$   \\ \hline 
             &        4  &                   &       3   &            \\ \hline 
             &        1  &          -5       &       99  &            \\ \hline                           
    \end{tabular}

%      \begin{tabular}{|l|c|c|c|c|c|c|}
%                                                                                            \hline
% polynomi     & \begin{sideways}$-2x^2+6x$\end{sideways} & \begin{sideways}$7x^3-x-15$\end{sideways}    &     &          &     &     \\ \hline
% termien      &            &                &     &          &     &     \\ \hline 
% lukumäärä    &        2   &                & 2   &    3     &  4  &  1  \\ \hline 
% korkeimman & & & & & & \\  
% asteen & & & & & & \\  
% termin & & & & & & \\  
% kerroin      &    $-2$    &                &$-9$ &   $-1$   &     &$-5$ \\ \hline 
% polynomin & & & & & & \\  
% asteluku     &        2   &                & 2   &    5     &  3  & 99  \\ \hline 
% vakiotermi   &        0   &                & 5   &    $-17$ &     &     \\ \hline 
%     \end{tabular}
%      \begin{tabular}{|c|c|c|c|c|}
%                                                                                           \hline
%              & termien   & korkeimman asteen & polynomin &            \\
% polynomi     & lukumäärä & termin kerroin    & asteluku  & vakiotermi \\ \hline
% $-2x^2+6x$   &        2  &         $-2$      &       2   &    0       \\ \hline 
% $7x^3-x-15$  &           &                   &           &            \\ \hline 
%              &        2  &          $-9$     &       2   &    5       \\ \hline 
%              &        3  &          $-1$     &       5   &    $-17$   \\ \hline 
%              &        4  &                   &       3   &            \\ \hline 
%              &        1  &          -5       &       99  &            \\ \hline                           
%     \end{tabular}

    
    \begin{vastaus}
    Värilliset kohdat voivat olla jotain muutakin.
    
    \begin{tabular}{|c|c|c|c|c|}
                                                                                           \hline
polynomi     & \begin{sideways}\begin{minipage}{3.5cm}termien\\lukumäärä\end{minipage}\end{sideways}%
& \begin{sideways}\begin{minipage}{3.5cm}korkeimman asteen\\termin kerroin\end{minipage}\end{sideways}%
& \begin{sideways}\begin{minipage}{3.5cm}polynomin\\asteluku\end{minipage}\end{sideways}%
& \begin{sideways}vakiotermi\end{sideways} \\ \hline
$-2x^2+6x$   &        2          &         $-2$      &       2             &    0       \\ \hline 
$7x^3-x-15$  &        3          &           7       &       3             &    $-15$   \\ \hline 
$-9x^2+5$    &        2          &          $-9$     &       2             &    5       \\ \hline 
$-x^5\textcolor{blue}{+4x}-17$%
             &        3          &          $-1$     &       5             &    $-17$   \\ \hline 
$\textcolor{blue}{8}x^3\textcolor{blue}{-x^2+4x}-17$%
             &        4          &\textcolor{blue}{8}  &       3             &\textcolor{blue}{7}\\ \hline 
$-5x^{99}$   &        1          &          $-5$     &       99            &         0      \\ \hline                           
     \end{tabular}
     \end{vastaus}
\end{tehtava}

\begin{tehtava}
    Olkoot $P(x)=x^2+5$ ja $Q(x)=x^3-1$. Laske
    \begin{enumerate}[a)]
        \item polynomin $P(x)$ arvo, kun $x=2$
        \item polynomin $Q(x)$ arvo, kun $x=1$
        \item $P(-7)$
        \item $Q(-4)$.
    \end{enumerate}
    \begin{vastaus}
        \begin{enumerate}[a)]
            \item $9$ % 2^2 + 5 = 4 + 5 
            \item $0$ % 1^3 - 1
            \item $54$ % (-7)^2 + 5 = 49 + 5 
            \item $-65$ % (-4)^3 - 1 = -64 -1
        \end{enumerate}
    \end{vastaus}
\end{tehtava}

\begin{tehtava}
    Olkoot $P(x)=x^2+3x+4$ ja $Q(x)=x^3-10x+1$. Sievennä
    \begin{enumerate}[a)]
        \item $P(x)+Q(x)$
        \item $P(x)-Q(x)$
        \item $Q(x)-P(x)$
        \item $2P(3)+Q(2)$.
    \end{enumerate}
    \begin{vastaus}
        \begin{enumerate}[a)]
            \item $x^3+x^2-7x+5$ % x^2+3x+4 + x^3-10x+1
            \item $-x^3+x^2+13x+3$ % x^2+3x+4 -(x^3-10x+1) = x^2+3x+4 -x^3+10x-1
            \item $x^3-x^2-13x-3$ % 
            \item $33$ % 2*(3^2+3*3+4) +  2^3-10*2+1 = 2*(9+9+4)+8-20+1 =44-11 =33
        \end{enumerate}
    \end{vastaus}
\end{tehtava}



\paragraph*{Hallitse kokonaisuus}

\begin{tehtava}
    Sievennä
    \begin{enumerate}
        \item $(x^2 - 2x + 1) + (-x^2 + x) $
        \item $(3y^3 + 2y^2  + y) - (-y^2 + y)$
        \item $(z^{10} - z^6 + z^2 + 1) + (z^{10} + 2z^8 - 3z^6)$.
    \end{enumerate}
    \begin{vastaus}
        \begin{enumerate}
            \item $-x + 1$
            \item $3y^3 + 3y^2$
            \item $2z^{10} + 2z^8 - 4z^6 + z^2 + 1$
        \end{enumerate}
    \end{vastaus}
\end{tehtava}

\begin{tehtava}
	Mitkä ovat seuraavien polynomifunktioiden asteet, ts. sievennettyjen muotojen asteet?
	\begin{enumerate}[a)]
		\item $x+5-x$
		\item $x^2+x-2x^2$
		\item $4x^5+x^2-4-x^2$
		\item $x^3+2x^5-x-3x^5+3x^2+5+x^5$
		\item $x^4-2x^3+x-1+x^3-x^4+x^3$
	\end{enumerate}

	\begin{vastaus}
		\begin{enumerate}[a)]
			\item $0$
			\item $2$
			\item $5$
			\item $3$
			\item $1$
		\end{enumerate}
	\end{vastaus}
\end{tehtava}

\begin{tehtava}
	Mitkä seuraavista polynomilausekkeista esittävät samaa polynomifunktiota kuin
	$x^3+2x+1$?
	\begin{enumerate}[a)]
		\item $2x+x^3+1$
		\item $x^2+2x+1$
		\item $x+2x^3+1 - (x^3+x)$
		\item $15+x^4+2x+x^3-x^4-14$
	\end{enumerate}
	\begin{vastaus}
		a) ja d)
	\end{vastaus}
\end{tehtava}

\begin{tehtava}
	Sievennä polynomifunktiot ja laske funktioiden arvot muuttujan arvoilla $1$, $-1$ ja $3$.
	\begin{enumerate}[a)]
		\item $P(x)=(x^3-4x+5)+(-x^3+x^2+4x-2)$
		\item $Q(x)=(2x^3+x^2-10x)-(3x^3-4x^2+5x)$
	\end{enumerate}
	
	\begin{vastaus}
		\begin{enumerate}[a)]
			\item $P(x)=x^2+3$, $P(1)=-2$, $P(-1)=-2$ ja $P(3)=5$
			\item $Q(x)=-x^3+5x^2-15x$, $Q(1)=-11$, $Q(-1)=-9$ ja $R(3)=-27$
		\end{enumerate}
	\end{vastaus}
\end{tehtava}

\begin{tehtava}
	Hyödynnä edellisen tehtävän polynomifunktioita ja laske
	\begin{enumerate}[a)]
		 \item polynomien P(x) ja Q(x) summa
		 \item polynomien P(1) ja Q(-2) erotus
	\end{enumerate}
	
	\begin{vastaus}
		\begin{enumerate}[a)]
			\item $-x^3+6x^2-12$
			\item $-54$
	\end{enumerate}
	\end{vastaus}
\end{tehtava}

\paragraph*{Lisää tehtäviä}

\begin{tehtava}
    Mitkä seuraavista ovat polynomeja?
    \begin{enumerate}[a)]
        \item $\frac{1}{x}$
       %\item $x^3+4x$
        \item $5x-125$
       %\item $2^x$
        \item $\sqrt{x}+1$
        \item $3x^4+6x^2+9$
        \item $\sqrt{2}x-x$
        \item $4^x+5x+6$
    \end{enumerate}
    \begin{vastaus}
        \begin{enumerate}[a)]
            \item Ei ole.
           %\item On.
            \item On.
           %\item Ei ole.
            \item Ei ole.
            \item On.
            \item On.
            \item Ei ole.
        \end{enumerate}
    \end{vastaus}
\end{tehtava}

\begin{tehtava}
    Olkoot $P(x)=x^2+3x+4$ ja $Q(x)=x^3-10x+1$. Laske:
    \begin{enumerate}[a)]
        \item $P(-1)$
        \item $Q(-2)$
        \item $P(3)$
        \item $Q(0)$
    \end{enumerate}
    \begin{vastaus}
        \begin{enumerate}[a)]
            \item $2$
            \item $13$
            \item $22$
            \item $1$
        \end{enumerate}
    \end{vastaus}
\end{tehtava}

\begin{tehtava}
	Mikä on/mitkä ovat polynomin $P(x) = x^5-3x^3+2x-1$
	\begin{enumerate}[a)]
		\item aste
		\item termit
		\item kolmannen asteen termi
		\item vakiotermi
	\end{enumerate}

	\begin{vastaus}
		\begin{enumerate}[a)]
			\item $5$
			\item $x^5$, $-3x^3$, $2x$, $-1$
			\item $-3x^3$
			\item $-1$
		\end{enumerate}
	\end{vastaus}
\end{tehtava}

\begin{tehtava}
	Mitkä on seuraavien polynomien asteet?
	\begin{enumerate}[a)]
		\item $x^2 + 3x - 5$
		\item $100 + x$
		\item $3x^3 + 90x^8 + 2x$
		\item $12x^1 + 34x^2 + 56x^3 + 78x^5 + 90x^5$
	\end{enumerate}

	\begin{vastaus}
		\begin{enumerate}[a)]
			\item $2$
			\item $1$
			\item $8$
			\item $5$
		\end{enumerate}
	\end{vastaus}
\end{tehtava}

\begin{tehtava}
     Sievennä
     \begin{enumerate}
         \item $(2x + 3) + x $
         \item $(3x - 1) + (-x + 1)$
         \item $(5x + 10) + (6x - 6) - (x + 3)$
     \end{enumerate}
     \begin{vastaus}
         \begin{enumerate}
             \item $3x + 3$
             \item $2x$
             \item $10x - 1$
         \end{enumerate}
     \end{vastaus}
 \end{tehtava}

\begin{tehtava}
	Sievennä polynomifunktiot ja laske funktioiden arvot muuttujan arvoilla $1$, $-1$ ja $3$.
	\begin{enumerate}[a)]
		\item $R(x)=4(2x-4)+(-x^3+1)$
		\item $S(x)=-(2x^2-x+8)+6(x^5-3x^2+1)-2(3x^5-10x^2)$
	\end{enumerate}
	\begin{vastaus}
		\begin{enumerate}[a)]
			\item $R(x)=-x^3+8x-15$, $R(1)=-8$, $R(-1)=-6$ ja $R(3)=-18$
			\item $S(x)=-2$, $S(1)=-2$, $S(-1)=-2$ ja $S(3)=-2$
		\end{enumerate}
	\end{vastaus}
\end{tehtava}

\end{tehtavasivu}
