%% encoding: utf-8
\section{Kertausta: Ensimmäisen asteen yhtälö}

Yhtälöistä yksinkertaisin on ensimmäisen asteen yhtälö. Käydään se lyhyesti
läpi kertauksen vuoksi.

Ensimmäisen asteen yhtälössä ratkaistavana on vain mahdollisesti vakiolla
kerrottu muuttuja. Yhtälön ratkaisemiseksi riittää käyttää neljää
aritmeettista peruslaskutoimitusta. Aluksi yhtälön molemmille puolille 
lisätään tai vähennetään jokin luku, niin että 
vasemmalle puolelle saadaan jäämään pelkkä vakiolla kerrottu muuttuja.
Sen jälkeen jaetaan yhtälön molemmat puolet muuttujan kertoimella, jolloin
yhtälön ratkaisu jää oikealle puolelle.

% fixme tyylikorjausta, termiä juuri ei vielä esitellä

Yleisesti ensimmäisen asteen yhtälö on muotoa

\begin{align*}
    ax + b = 0.
\end{align*}

Kaikki 1. asteen yhtälöt voidaan muokata tähän yleiseen
muotoon siirtämällä kaikki termit vasemmalle puolelle ja
sieventämällä.

% fixme Esimerkit 2.30-2.31 voisi siirtää ennen yleisen muodon esittelyä, koska niissä ei sitä mihinkään tarvita.

\subsubsection*{Esimerkkejä}

\begin{esimerkki}
Ratkaise yhtälö $4x + 5 = 2 + 2x$.

\textbf{Ratkaisu}
\begin{align*}
    4x + 5 &= 2 + 2x && \ppalkki -2x \\
    2x + 5 &= 2      && \ppalkki -5 \\
        2x &= -3     && \ppalkki :2 \\
         x &= -\frac{3}{2}
 \end{align*}
\end{esimerkki}

\begin{esimerkki}
Ratkaise yhtälö $3x - 6 = 0$.

\textbf{Ratkaisu}
  \begin{align*}
    3x - 6 &= 0 && \ppalkki +6 \\
        3x &= 6 && \ppalkki :3 \\
         x &= \frac{6}{3} \\
           &= 2
  \end{align*}
\end{esimerkki}

Yleisen yhtälön $ax + b = 0$ ratkaisu on siis

\begin{align*}
  x = -\frac{b}{a}.
\end{align*}

Erityisesti kannattaa huomata, että kaikkia 1. asteen yhtälöitä ei tarvitse
saattaa yleiseen muotoon. Esimerkiksi jos vakiotermit ovat valmiiksi
oikealla puolella, yhtälön ratkaisemiseksi riittää luonnollisesti jakaa
molemmat puolet muuttujan kertoimella.

\begin{tehtavasivu}

\paragraph*{Opi perusteet}

\begin{tehtava}
    Ratkaise yhtälöt.
    \begin{enumerate}[(a)]
        \item $x + 5 = 47$
        \item $2x = 64$
        \item $3x - 5 = 16$
    \end{enumerate}
    \begin{vastaus}
        \begin{enumerate}[(a)]
            \item $x = 42$
            \item $x = 32$
            \item $x = 7$
        \end{enumerate}
    \end{vastaus}
\end{tehtava}

\begin{tehtava}
    Ratkaise yhtälöt.
    \begin{enumerate}[(a)]
        \item $x + 8 = 2x - 1$
        \item $2x + 4 = 60$
        \item $3x - 5 = -x + 11$
    \end{enumerate}
    \begin{vastaus}
        \begin{enumerate}[(a)]
            \item $x = 9$
            \item $x = 28$
            \item $x = 4$
        \end{enumerate}
    \end{vastaus}
\end{tehtava}

%ei ole yhtälötehtävä, mutta mallinnusharjoituksena ok? 
\begin{tehtava}
    Muodosta tilannetta kuvaavat lausekkeet.
    \begin{enumerate}[(a)]
        \item Kuinka paljon maksaa hilavitkuttimen vuokraus $x$:ksi tunniksi, kun vuokra on 42 \euro /tunti. Vuokraajan tulee myös ottaa pakollinen 25 euron laitteistovakuutus.
        \item Kuinka monta euroa saa $x$:llä dollarilla, kun 1~EUR vastaa 1,23~USD:a, ja halutaan vaihtaa dollareita euroiksi. Valuutanvaihtaja veloittaa lisäksi palvelumaksun 0,50 euroa.
    \end{enumerate}
    \begin{vastaus}
        \begin{enumerate}[(a)]
            \item $42x + 25$
            \item $\frac{1}{1{,}23}x + 0{,}5$
        \end{enumerate}
    \end{vastaus}
\end{tehtava}



\paragraph*{Hallitse kokonaisuus}

\begin{tehtava}
    Ratkaise yhtälöt.
    \begin{enumerate}[(a)]
        \item $3(x+7)=7x$
        \item $2(3x-1)=-7x $
        \item $3-2x-(4-x)=2 $
    \end{enumerate}
    \begin{vastaus}
        \begin{enumerate}[(a)]
            \item $x = \frac{7}{6} =1\frac{1}{6} $
            \item $x = \frac{2}{13}$
            \item $x = -3$
        \end{enumerate}
    \end{vastaus}
\end{tehtava}

\begin{tehtava}
    Ratkaise yhtälöt.
    \begin{enumerate}[(a)]
        \item $-2\cdot\frac{x-5}{3}-\frac{5}{7}(1-x)=5x+3$
        \item $\frac{4x-5}{3}-\frac{3}{2}(x-8)=-\frac{x+5}{6}$
        \item $3(x-3)+x=4x-9$
    \end{enumerate}
    \begin{vastaus}
        \begin{enumerate}[(a)]
            \item $x = -\frac{1}{13}$
            \item ei ratkaisuja
            \item yhtälö on toteutuu kaikilla reaaliluvuilla
        \end{enumerate}
    \end{vastaus}
\end{tehtava}

%vaatii Pythagoraan lauseen, jota ei vielä ole käsitelty lukiossa.
\begin{tehtava}
    Tässä tehtävässä pitäisi muistaa peruskoulussa käsitelty Pythagoraan lause.
    Suorakulmaisen kolmion sivujen pituuden kateettien pituudet ovat $x+1$ ja $4$. Hypotenuusan pituus $x+3$. Mikä $x$ on?
    \begin{vastaus}
		$x=2$
    \end{vastaus}
\end{tehtava}

%hankala
\begin{tehtava}
    Määritä sekunnin tarkkuudella se ajanhetki, kun kellotaulun minuutti- ja tuntiviisarit ovat päällekkäin ensimmäisen kerran klo 12.00:n jälkeen.
    \begin{vastaus}
		$13.05.27$
    \end{vastaus}
\end{tehtava}

\end{tehtavasivu}
